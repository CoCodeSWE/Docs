\documentclass[./PianoDiProgetto.tex]{subfiles}
\begin{document}
\section{Consuntivo}
\subsection{Consuntivi di periodo}
In questa sezione verranno indicate le spese effettivamente sostenute, divise sia
per ruolo che per persona. Il bilancio potrà risultare:
\begin{itemize}
  \item \textbf{positivo}: se il preventivo supera il consuntivo;
  \item \textbf{in pari}: se il preventivo ed il consuntivo coincidono;
  \item \textbf{in negativo}: se il consuntivo super il preventivo.
\end{itemize}
\subsubsection{\PerAR}

\subsubsection{\PerAD}

  \paragraph{Consuntivo}

  \paragraph{Conclusioni}

\subsubsection{\PerPA}

  \paragraph{Consuntivo}

  \paragraph{Conclusioni}

\subsubsection{\PerPD}

  \paragraph{Consuntivo}

  \paragraph{Conclusioni}

\subsubsection{\PerC}

  \paragraph{Consuntivo}

  \paragraph{Conclusioni}

\subsubsection{\PerV}

  \paragraph{Consuntivo}

  \paragraph{Conclusioni}

\subsection{Consuntivo Finale}

  \subsubsection{Conclusioni}

\end{document}
