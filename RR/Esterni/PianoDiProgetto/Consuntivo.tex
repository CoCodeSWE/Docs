\documentclass[./PianoDiProgetto.tex]{subfiles}
\begin{document}
\section{Consuntivo}
\subsection{Consuntivi di periodo}
In questa sezione verranno indicate le spese effettivamente sostenute, divise sia
per ruolo che per persona. Il bilancio potrà risultare:
\begin{itemize}
  \item \textbf{positivo}: se il preventivo supera il consuntivo;
  \item \textbf{in pari}: se il preventivo ed il consuntivo coincidono;
  \item \textbf{in negativo}: se il consuntivo super il preventivo.
\end{itemize}
\subsubsection{\PerAR}
	\paragraph{Consuntivo}
Le ore di lavoro sostenute in questa fase sono da considerarsi come ore di approfondimento personale, in quanto il gruppo \GRUPPO{} non è ancora stato scelto come fornitore ufficiale per il progetto \PROGETTO.
		
		Tali dati riguardano quindi le ore non rendicontate.

\begin{table}[h]
		\centering
		\begin{tabular}{l * {2}{c}}
			\toprule
			\textbf{Ruolo} & \textbf{Ore} & \textbf{Costo (\euro{})} \\
			\midrule
			Responsabile &	22 & 660,00 \\
			%\midrule
			Amministratore & 57 (+12) & 1.140,00 (+240,00)\\
			%\midrule
			Progettista & 0 & 0,00 \\
			%\midrule
			Analista & 59 (+9) & 1.475,00 (+225,00)\\
			%\midrule
			Programmatore & 0 & 0,00 \\
			%\midrule
			Verificatore & 51 (-13) & 765,00 (-195,00)\\
			\midrule
			\textbf{Totale Preventivo} & 189
 & 4.040,00
 \\		
			\textbf{Totale Consuntivo} & 197 & 4.310,00
 \\
			\midrule
			\textbf{Differenza} & +8 & +270,00 \\
			\bottomrule
		\end{tabular}
		\caption{\PerAR{} - Consuntivo}
		\label{tab:consuntivoA}
		
	\end{table}		
		
  \paragraph{Conclusioni}
		Come si può notare dalla tabella \ref{tab:consuntivoA}, che presenta i dati relativi al consuntivo del periodo AR, è stato necessario investire più tempo del previsto nei ruoli di  \AMM{} e \AN, di conseguenza il bilancio risultante è \textbf{negativo}.
		
		L'attività degli \AMM{} ha richiesto più tempo del previsto in quanto è stato necessario modificare alcune funzioni del software utilizzato per il tracciamento dei requisiti e dei casi d'uso.
		
		L'attività degli \AN{} ha richiesto più tempo del previsto in quanto si è dovuta fare una analisi molto più approfondita rispetto a quella prefissata per una corretta stesura dei requisiti e dei casi d'uso.


\end{document}
