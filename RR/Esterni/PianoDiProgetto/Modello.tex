\documentclass[./PianoDiProgetto.tex]{subfiles}
\begin{document}
  \section{Modello di sviluppo}

  Il modello di sviluppo scelto per il \gl{sistema} è il modello incrementale: il
  \gl{progetto} è stato suddiviso in periodi, i quali sono caratterizzati da
  obbiettivi ed attività differenti. Inoltre nel corso del \gl{progetto} saranno
  fissati diversi incontri col \gl{proponente}, i quali serviranno per ricevere un
  feedback e migliorare il \gl{sistema} di conseguenza.

  \subsection{Periodo AR - Analisi dei Requisiti}

  In questa fase iniziale vengono individuati gli strumenti necessari sia alla
  collaborazione tra i membri del gruppo che alla stesura della documentazione.
  Si individua inoltre il \gl{progetto} da sviluppare e se ne analizzano i requisiti.

  \subsection{Periodo AD - Analisi di Dettaglio}

  In questo periodo i requisiti individuati vengono consolidati ed ampliati. Il
  documento di \textit{Analisi dei Requisiti} viene modificato in base all'esito
  della \RR. Vengono corretti e verificati anche gli altri documenti.

  \subsection{Periodo PA - Progettazione Architetturale}

  L'obbiettivo di questo periodo è la progettazione dell'architettura di alto
  livello del \gl{sistema} e la produzione del documento di \textit{Specifica Tecnica}.

  \subsection{Periodo PD - Progettazione di Dettaglio}

  In questo periodo viene progettato in modo dettagliato il \gl{sistema}, definendo
  in particolare il comportamento e l'interazione tra i vari componenti. Vengono
  inoltre prodotti una documentazione dettagliata di tutte le API ed un piano
  di test di unità.

  \subsection{Periodo C - Codifica}

  Durante questo periodo viene scritto il codice del \gl{sistema}. L'obbiettivo
  di questa fase è quello di consegnare al \gl{proponente} un \gl{prodotto} qualificato.

  \subsection{Periodo V - \gl{Validazione}}

  In questa fase finale del \gl{progetto} viene effettuata la \gl{validazione} del \gl{sistema} ed il
  collaudo dello stesso, in modo da verificare che il \gl{prodotto} soddisfi tutti i requisiti
  dell'\textit{Analise dei Requisiti}. 

\end{document}
