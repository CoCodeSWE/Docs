\documentclass[./PianoDiProgetto.tex]{subfiles}
\begin{document}
\section{Pianificazione}
  Elenchiamo qui di seguito le caratteristiche e le durate di ogni periodo. Nella decisione dei tempi sono stati tenuti in considerazione e si sono cercati di mitigare i rischi relativi alle tempistiche.
	\subsection{\PerAR{} (AR)}
  \textbf{Periodo: dal 2016-12-10 al 2017-01-11}

  Questo periodo comincia con il primo incontro del gruppo e termina con la scadenza della consegna riguardante la \textbf{Revisione dei requisiti}.

  È composto principalmente dalle seguenti attività:
  \begin{enumerate}
		\item \textbf{scelta degli strumeti}: verranno scelti gli strumenti che saranno utilizzati per la stesura dei documenti e per il supporto. La scelta degli strumenti sarà ?incrementale? nel corso del progetto;
		\item \textbf{stesura delle Norme di progetto}: dopo aver scelto i primi strumenti per la produzione di documentazione sarà possibile iniziare la stesura delle \NPdocRR. Questo documento sarà utilizzato indipendentemente dal capitolato che verrà preso in appalto, e sarà steso in modo incrementale via via che nuove norme verrano definite;
		\item \textbf{stesura documentazione}: una volta definiti norme e strumenti per la scrittura di un documento, è possibile iniziare la stesura dei seguanti documenti;
    \begin{itemize}
      \item Analisi dei Requisiti: si redige il documento \ARdocRR. Verrano organizzati degli incontri col proponente, sia prima che durante la stesura di questo documento, per consolidare i requisiti stesi e per chiarire eventuali dubbi sui requisiti da stendere;
      \item Studio di Fattibilità: vengono valutati pro e contro di tutti i capitolati proposti e viene redatto il documento \SFdocRR. Viene quindi scelto il capitolato da sviluppare;
      \item Piano di Progetto: viene steso il documento \PPdocRR, il quale regolerà le attività le attività del gruppo;
      \item Piano di qualifica: si stende il documento \PQdocRR per fissare quali sono gli standard di qualità ed in che modo il gruppo si prefigge di raggiungerli;
      \item Glossario: viene creato il file incrementale "terminiGlossario.txt" e steso in modo automatico il documento \GldocRR.
      \item Lettera di Presentazione: si scrive la lettera in cui si dichiara l'interesse del gruppo a partecipare alla gara di appalto.
      \item Analisi preliminare degli SDK dei principali assistenti virtuali: gli \ANP{} analizzeranno gli SDK dei principali assistenti virtuali, dotati di intelligenza artificiale, presenti sul mercato, e scriveranno il documento esterno \textit{AnalisiSDK v1.0.0} in cui verrà effettuato il confronto tra  essi.
    \end{itemize}
  \end{enumerate}
  \newpage
  \subsubsection{Diagramma di Gantt - \PerAR}
    \begin{figure}[!h]
    \centering
    \includegraphics[width=\textwidth]{images/AR}
    \caption{Gantt - \PerAR}
    \end{figure}

	\subsection{\PerAD{} (AD)}
  \textbf{Periodo: dal 2017-01-24 al 2017-01-31}

  Questo periodo comincia al termine della \PerAR.  le attività da svolgere in questo periodo saranno le seguenti:
  \begin{itemize}
    \item incremento e verifica documenti: tutti i documenti prodotti nel periodo precedente vengono incrementati e corretti in base alle segnalazioni del committente e del proponente;
    \item incremento e verifica requisiti:  gli \ANP provvedono ad individuare nuovi requisiti e a correggere eventuali requisiti segnalati; tutti i documenti saranno aggiornati di conseguenza. Verrà fissato un incontro col proponente per la verifica dei requisiti individuati in questo modo;
  %\newpage
  \subsubsection{Diagramma di Gantt - \PerAD}
    \begin{figure}[!h]
    \centering
    %\includegraphics[width=\textwidth]{images/AD}
    \caption{Gantt - \PerAD}
    \end{figure}

  \subsection{\PerPA{} (PA)}
  \textbf{Periodo: dal 2017-02-01 al 2017-02-22}

  Questo periodo inizia al termine della \PerAD{} e termina con una milestone interna di \textbf{Revisione di Progettazione} minima. Al termine di questa fase verrà fissato un incontro col proponente per presentare l'architettura logica prodotta. Le attività svolte durante questa fase sono:
  \begin{itemize}
    \item incremento dei documenti: dove necessario vengono apportate modifiche ai documenti già stilati, in preparazione alla stesuara della \textit{Specifica Tecnica};
    \item Specifica Tecnica: viene steso il documento di \textit{Specifica Tecnica}, nel quale il \PJ{} descrive le scelte progettuali effettuate ed i design pattern individuati per la realizzazione del prodotto. Viene descritta inoltre l'architettura generale del Software e si effettue il tracciamento dei requisiti;
    \item verifica: viene effettuata la verifica sia dei documenti incrementati sia della \textit{Specifica Tecnica};
  \end{itemize}

    %\item documentazione API: viene prodotta una documentazione dettagliata delle varie API fornite dal sistema;
    %\item piano di test di unità: viene creato il piano dei test di unità.
  \end{itemize}
  %\newpage
  \subsubsection{Diagramma di Gantt - \PerPA}
    \begin{figure}[!h]
    \centering
    %\includegraphics[width=\textwidth]{images/PDR}
    \caption{Gantt - \PerPA}
    \end{figure}

  \subsection{\PerPD{} (PD)}
  \textbf{Periodo: dal 2017-02-23 al 2017-03-06}

  Questo periodo inizia con la fine della \PerPA e termina con la consegna dei documenti per la \textbf{Revisione di Progettazione} massima. Le attività che verranno svolte in questo periodo saranno:
  \begin{itemize}
    \item Definizione di prodotto: viene steso il documento \textit{Definizione di Prodotto}. Esso definisce nel dettaglio la struttura interna del sistema e le relazioni tra i diversi moduli del prodotto; include inoltre una descrizione dettagliata di tutti le API del sistema;
    \item incremento e verifica dei documenti: se ritenuto necessario vengono apportate modifiche ai documenti già stilati. In particolare sarà certamente necessaria l'aggiunta di un piano di test di unità al \textit{Piano di Qualifica}.
  \end{itemize}

  %\newpage
  \subsubsection{Diagramma di Gantt - \PerPD}
    \begin{figure}[!h]
    \centering
    %\includegraphics[width=\textwidth]{images/V}
    \caption{Gantt - \PerPD}
    \end{figure}

  \subsubsection{\PerC{} (C)}
  \textbf{Periodo: dal 2017-03-13 al 2017-04-11}

  Questo periodo inizia in seguito all'esito della \textbf{Revisione di Progettazione} e termina con la consegna del prodotto alla \textbf{Revisione di Qualifica}. le attività svolte saranno:
  \begin{itemize}
    \item codifica: in base a quanto definito nel documento di \textit{Definizione di Prodotto}, i \PRP{} sviluppano il codice del prodotto software. Si divide in codifica obbligatoria, desiderabile ed opzionale in base all'utilità strategica dei requisiti che vengono soddisfatti dal codice prodotto. A seconda del tempo impiegato dalla codifica obbligatoria, il tempo previsto per codifica desiderabile ed opzionale potrebbe essere ridotto parzialmente o totalmente.
    \item test: contemporaneamente all'inizio della codifica del sistema, si inizia la stesura dei test di unità, integrazione e sistema come previsto nel \textit{Piano di Qualifica};
    \item automazione: gli \AMMP{} si occuperanno della predisposizione di sistemi automatizzati per l'esecuzione automatica dei test.
    \item manuale utente: verrà steso il manuale utente, destinato agli utilizzatori finali del sistema.
  \end{itemize}

  %\newpage
  \paragraph{Diagramma di Gantt - \PerC}
    \begin{figure}[!h]
    \centering
    %\includegraphics[width=\textwidth]{images/PDROP}
    \caption{Gantt - \PerC}
    \end{figure}

  \subsection{\PerV (V)}
  \textbf{Periodo: dal 2017-04-18 al 2017-05-08}

  Questo periodo inizia dopo l'esito della \textbf{Revisione di Qualifica} e termina con la consegna della \textbf{Revisione d'Accettazione}. Le attività che verranno scolte saranno:
  \begin{itemize}
    \item incremento e verifica dei documenti: se ritenuto necessario vengono apportate modifiche ai documenti già stilati;
    \item validazione: viene verificato, con l'aiuto del tracciamento, di aver soddisfatto i requisiti presenti nel documento \textit{Analisi dei Requisiti};
    \item esecuzione test: si continua l'esecuzione dei test di unità, integrazione e sistema codificati ed eseguiti nei periodi precedenti;
    \item correzione bug: vengono tracciati e risolti i bug rilevati;
    \item collaudo: si esegue un collaudo completo del sistema creato.
  \end{itemize}

  %\newpage
  \subsubsection{Diagramma di Gantt - \PerV}
    \begin{figure}[!h]
    \centering
    %\includegraphics[width=\textwidth]{images/V}
    \caption{Gantt - \PerV}
    \end{figure}
\end{document}
