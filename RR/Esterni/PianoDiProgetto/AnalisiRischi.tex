\documentclass[PianoDiProgetto.tex]{subfiles}
\begin{document}
\section{Analisi dei rischi}
Al fine di evitare rallentamenti delle fasi di lavoro è stata effettuata una dettagliata analisi dei rischi, in modo da poter evitare le situazioni che portano alla creazione di eventi non pianificati, ove possibile. L'analisi si suddivide in quattro attività:
	\begin{itemize}
		\item Identificazione: individuare quali rischi possono incorrere durante lo sviluppo del progetto, analizzarli e provare a intuire quali saranno le conseguenze se questi si verificano;
		\item Analisi di fase: valutare la probabilità di occorrenza di un rischio e analizzare la criticità delle conseguenze di questo rispetto all' andamento del progetto nella fase in corso. Tutte le fasi sono illustrate nella sezione 5 riguardante la Pianificazione. Il \RESP di progetto è incaricato  di fare una nuova analisi dei rischi ogni qualvolta si cambi fase;
		\item Pianificazione e mitigazione: identificare un metodo di controllo dei rischi così da renderli evitabili, nel qual caso non lo siano, pianificare delle contromisure per ridurre al minimo i danni; 
		\item Controllo: monitorare nel tempo ogni rischio, nel caso si verifichi descriverne il riscontro effettivo e come il team ha reagito per ridurre al minimo i danni. 
	\end{itemize}
I rischi trattati vengono suddivisi in 5 sottosezioni per essere analizzati al meglio:
	\begin{itemize}
		\item livello tecnologico;
		\item livello degli strumenti;
		\item livello del personale;
		\item livello organizzativo;
		\item livello dei requisiti.
	\end{itemize}
Ogni rischio trattato ha una serie di caratteristiche: necessarie per comprenderne la natura:
	\begin{itemize}
		\item nome;
		\item descrizione;
		\item risultati analisi:
			\begin{itemize}
				\item probabilità di occorrenza;
				\item grado di pericolosità;
				\item possibili conseguenze;
				\item riconoscimento;
				\item trattamento;
				\item attuazione nel periodo.
			\end{itemize}
	\end{itemize}
Ogni rischio verrà monitorato nel tempo e ne verrà indicato l’effettivo riscontro nella fase in corso.

\newpage

\begin{table}[]
\centering
\caption{Registro dei rischi}
\label{Analisi dei Rischi}
\begin{tabular}{cc}
\hline
\textbf{Livello}       & \textbf{Tipologia} \\ \hline
Tecnologico   & \begin{tabular}[c]{@{}c@{}}Tecnologie adottate sconosciute\end{tabular}             \\ \hline
              & \begin{tabular}[c]{@{}c@{}}Guasti hardware e malfunzionamenti software\end{tabular} \\ \hline
Organizzativo & \begin{tabular}[c]{@{}c@{}}Valutazioni delle risorse\end{tabular}                   \\ \hline
Personale     & \begin{tabular}[c]{@{}c@{}}Problemi personali dei membri del team\end{tabular}     \\ \hline
              & \begin{tabular}[c]{@{}c@{}}Problemi personali dei membri del team\end{tabular}     \\ \hline
Requisiti     & Mancata comprensione                                                                    \\ \hline
\end{tabular}
\end{table}

\newpage

\subsection{Livello tecnologico}
		\subsubsection{Tecnologie adottate sconosciute}
Descrizione: per lo svolgimento e l' implementazione del progetto, il team dovrà utilizzare una serie di tecnologie che nessun membro ha mai utilizzato.
	\begin{itemize}
		\item Probabilità di occorrenza: media;
		\item Grado di pericolosità: alto;
		\item Possibili conseguenze: l’utilizzo di tecnologie sconosciute richiede tempo per la scelta e l’apprendimento di quest’ultima, il che può portare ad un ritardo sulle date di consegna;
		\item Riconoscimento: il Responsabile deve verificare il grado di preparazione di ogni membro del gruppo relativo alle tecnologie utilizzate;
		\item Trattamento: ogni membro del team deve studiare e approfondire autonomamente i documenti forniti dall' \AMM{} che spiegano come utilizzare propriamente le nuove tecnologie adottate.
		\item Attuazione nel periodo: 
			\begin{itemize}
				\item \ARdoc : il rischio non si è presentato, in quanto le nuove tecnologie non sono ancora state prese in carico.
			\end{itemize}

	\end{itemize}
	
	\subsubsection{Guasti hardware e malfunzionamenti software}
Descrizione: durante tutto il progetto è possibile che si verifichino guasti hardware e/o malfunzionamenti software ai computer usati dal team per sviluppare, in quanto non sono dispositivi professionali; --{non so se devo parlare dei server di zero12}--
	\begin{itemize}
		\item Probabilità di occorrenza: basso;
		\item Grado di pericolosità: basso;
		\item Possibili conseguenze: il malfunzionamento di uno dei dispositivi del team può portare a perdita di dati e di conseguenza perdita di tempo in quanto si va a svolgere nuovamente un lavoro già effettuato;
		\item Riconoscimento: tutti i membri del team mantengono un' elevata attenzione sulle condizioni dei propri dispositivi;
		\item Trattamento: ogni componente del team si impegnerà a fare un backup giornaliero del lavoro effettuato su un dispositivo esterno al computer utilizzato per sviluppare; in caso di rotture hardware ogni membro possiede un altro dispositivo che gli permette di continuare il lavoro.
		\item Attuazione nel periodo: 
			\begin{itemize}
				\item \ARdoc : il rischio non si è presentato.
			\end{itemize}
	
	\end{itemize}

	\subsection{Livello organizzativo}
Descrizione: data la non esperienza con con progetti di questa dimensione il team potrebbe incorrere in stime errate di valutazione delle risorse.
	\begin{itemize}
		\item Probabilità di occorrenza: alta;
		\item Grado di pericolosità: alto;
		\item Possibili conseguenze: un'errata stima delle risorse può portare a spreco di queste o a ritardi nelle date di consegna.
		\item Riconoscimento: il rischio in questo caso è dinamico, per questo si vuole una periodica verifica da parte del \RESP{} così da prendere atto di eventuali ritardi nello sviluppo delle attività;
		\item Trattamento: ogni attività ha un periodo di slack, tale che l'eventuale ritardo di un'attività non condizioni le tempistiche delle altre.
		\item Attuazione nel periodo: 
			\begin{itemize}
				\item \ARdoc : il rischio non si è presentato.
			\end{itemize}
	\end{itemize}
	
	\subsection{Livello personale}
		\subsubsection{Problemi personali dei componenti del team}
Descrizione: ogni membro del team avrà le sue necessità e i suoi impegni personali lungo la durata del progetto. Risulta inevitabile il verificarsi di problemi organizzativi in seguito a sovrapposizioni di tali impegni.

	\begin{itemize}
		\item Probabilità di occorrenza: media;
		\item Grado di pericolosità: alto;
		\item Possibili conseguenze: ritardo nello svolgimento delle attività;
		\item Riconoscimento: per creare un calendario sincronizzato e condiviso tra i membri del gruppo è necessario che vengano notificati al Responsabile in maniera preventiva e tempestiva gli impegni di ognuno. Grazie a questa pratica è possibile ridurre al minimo tale rischio;
		\item Trattamento: nel caso un membro del team abbia un impegno che non gli permetta di proseguire il lavoro, il \RESP{} andrà a modificare la pianificazione prevista in modo da coprire l' assenza creatasi;
		\item Attuazione nel periodo: 
			\begin{itemize}
				\item \ARdoc : i membri si sono impegnati per comunicare anticipatamente gli impegni personali, in questo modo il \RESP{} è riuscito a pianificare al meglio le attività da assegnare. Grazie a questa collaborazione il rischio non si è presentato.
			\end{itemize}
	\end{itemize}
		\subsubsection{Problemi personali tra i componenti del team}
Descrizione: essendo la prima volta che i membri del team collaborano, potrebbero sorgere attriti o squilibri interni che andrebbero a danneggiare il clima lavorativo e porterebbero a ritardi nelle consegne.

	\begin{itemize}
		\item Probabilità di occorrenza: bassa;
		\item Grado di pericolosità: alto;
		\item Possibili conseguenze: ritardo nello svolgimento delle attività;
		\item Riconoscimento: tutti i membri del gruppo devono avere una comunicazione costante con il \RESP{} il quale si occuperà di monitorare i rapporti tra i collaboratori.
		\item Trattamento: nel caso di contrasti tra membri del gruppo, il \RESP{} provvederà ad assegnare a tali membri attività differenti con il minimo contatto (nel limite del possibile).
		\begin{itemize}
				\item \ARdoc : il rischio non si è presentato.
		\end{itemize}
	\end{itemize}
	
	\subsection{Livello dei Requisiti}
		\subsubsection{Incomprensione e scelte non ottimali}
Descrizione: è possibile che alcuni requisiti individuati dagli analisti siano fraintesi, superficiali o errati rispetto alle aspettative del proponente \PROPONENTE{}. Inoltre esiste la probabilità che qualche requisito venga modificato, eliminato o aggiunto durante il corso del progetto.

		\begin{itemize}
		\item Probabilità di occorrenza: alta;
		\item Grado di pericolosità: alto;
		\item Possibili conseguenze: sviluppo di un prodotto non consono alle aspettative del proponente;
		\item Riconoscimento: avere una costante comunicazione con il proponente \PROPONENTE{} durante la fase di \ARdoc{} in modo da chiarire tutte le incomprensioni e assicurare la concordanza sui requisiti del prodotto;
		\item Trattamento: si dovranno effettuare incontri con il \PROPONENTE{} così da poter correggere eventuali errori indicati dal committente durante la revisione.
		\begin{itemize}
				\item \ARdoc : il rischio non si è presentato.
		\end{itemize}
	\end{itemize}	
	
\end{document}










