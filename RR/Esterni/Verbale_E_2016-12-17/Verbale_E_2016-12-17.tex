\documentclass[a4paper,titlepage]{article}

\makeatletter
\def\input@path{{../../../template/}{./img}}
\makeatother

\usepackage{Comandi}
\usepackage{Riferimenti}
\usepackage{Stile}

\def\NOME{Verbale 2016-12-17}
\def\VERSIONE{0.2.1}
\def\DATA{2016-12-17}
\def\REDATTORE{Simeone Pizzi}
\def\VERIFICATORE{Mattia Bottaro}
\def\RESPONSABILE{Simeone Pizzi}
\def\USO{Esterno}
\def\DESTINATARI{\COMMITTENTE \\ & \CARDIN \\ & \GRUPPO \\ & \PROPONENTE} % Se esterno va anche il \gl{proponente}
\def\SOMMARIO{Verbale dell'incontro esterno in data 2016-12-17 per il \gl{capitolato} \quotes{\CAPITOLATO{}}  del gruppo \GRUPPO.}


\begin{document}

\maketitle
\begin{diario}
  \modifica{Simeone Pizzi}{\RESP}{Corretti errori in base a segnalazione verificatori}{2016-12-18}{0.2.1}
  \modifica{Simeone Pizzi}{\RESP}{Corretti errori data ora e durata}{2016-12-17}{0.1.1}
  \modifica{Simeone Pizzi}{\RESP}{Stesura documento e verbalizzazione della riunione}{2016-12-17}{0.0.1}
\end{diario}
\newpage
\tableofcontents

\newpage
\section{Informazioni generali}
\label{sec:Informazioni}

\begin{itemize}
  \item \textbf{Luogo}: GoToMeeting.
  \item \textbf{Data}: 2016-12-17.
  \item \textbf{Orario di inizio}: 10:00.
  \item \textbf{Orario di fine}: 10:45.
  \item \textbf{Durata}: 45m.
  \item \textbf{Oggetto}: discussione sul \gl{capitolato} d'appalto.
  \item \textbf{Partecipanti}: Luca Bertolini, Nicola Tintorri, Simeone Pizzi.
  \item \textbf{Segretario}: Simeone Pizzi.

\end{itemize}
\section{Riassunto della riunione}
\label{sec:RiassuntoRiunione}
 \subsection{Descrizione}
La riunione è avvenuta online tramite l'utilizzo del \gl{software} GoToMeeting. Erano presenti il \gl{proponente} Stefano Dindo, tre componenti del gruppo \GRUPPO{}, tre componenti componenti del gruppo Kern3lP4nic e tre componenti del gruppo anSWEr. Sono state poste diverse domande al \gl{proponente} in merito al \gl{capitolato}  \quotes{\CAPITOLATO} e sono state prese alcune decisioni riguardanti i requisiti del \gl{sistema} e le tecnologie da utilizzare.
 \subsection{Decisioni}
 \begin{itemize}
  \item DE1.1 - Per lo sviluppo verrà utilizzato l'\gl{SDK} dell'assistente virtuale Alexa.
  \item DE1.2 - Saranno previsti 2 tipologie di utenti: ospite e amministratore.
  \item DE1.3 - Il \gl{sistema} dovrà tenere traccia delle interazioni passate.
  \item DE1.4 - Il \gl{sistema} dovrà chiedere all'ospite il nome e chi viene a visitare, in seguito avvertire l'interessato e porre alcune domande aggiuntive.
  \item DE1.5 - Il \gl{sistema} deve permettere all'amministratore di specificare le azioni da compiere in base all'ospite che si presenta.
  \item DE1.5 - \PROPONENTE{} si impegna, per quanto riguarda lo sviluppo, a fornire accesso a macchine virtuali su AWS oppure ad utenti per AWS Lambda in base alle necessità del gruppo.
  \item DE1.6 - \PROPONENTE{} si rende disponibile, previa richiesta da parte dei gruppi, a fornire formazione aggiuntiva riguardo le tecnologie necessarie per lo sviluppo del \gl{sistema}.
 \end{itemize}

\subsection{Tematiche in sospeso}
  \begin{itemize}
  \item SE1.1 - Disponibilità Amazon Echo.
  \item SE1.2 - Salvataggio conversazioni.
  \item SE1.3 - Domande aggiuntive durante l'interazione.
  \item SE1.4 - Interfaccia vocale.
  \end{itemize}
\end{document}
