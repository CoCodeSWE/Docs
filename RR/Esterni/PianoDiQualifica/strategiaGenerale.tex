\documentclass[PianoDiQualifica.tex]{subfiles}

\begin{document}

\section{Visione generale della strategia di gestione della qualità}
	\subsection{Obiettivi qualitativi}
		In questa sezione vengono descritti gli obiettivi di qualità che il gruppo \GRUPPO{} decide di perseguire durante l'intero progetto.
		Ogni obiettivo viene definito in modo quantitativo per permettere al team di valutarne il raggiungimento.
		Vengono quindi fissati dei valori minimi che è obbligatorio superare per soddisfarlo e dei valori ottimali che ne rappresentano il pieno (ma non obbligatorio) conseguimento.
		A tale scopo vengono utilizzati modelli, metriche e standard. \\
		Viene assegnato un codice identificativo ad ogni obiettivo, al fine di semplificarne il tracciamento con la metrica ad esso associata. \\
		Il metodo di denominazione degli obiettivi è descritto in dettaglio nel documento \NPdocRR.
		
		\subsubsection{Qualità di processo}
		Da processi scadenti derivano prodotti scadenti. Quindi, la qualità di processo è un fattore indispensabile per garantire la qualità dei prodotti. Assicurarla, inoltre, permette di:
		\begin{itemize}
			\item favorire l'ottimizzazione delle risorse; 
			\item migliorare la stima dei rischi;
			\item ridurre i costi.
		\end{itemize}
		Desideriamo che ogni processo possegga le seguenti caratteristiche ottimali:
		\begin{itemize}
			\item dovrebbe essere in grado di migliorarsi continuamente:
			\begin{itemize}
					\item le sue performance sono costantemente misurabili;
					\item deve perseguire sempre gli obiettivi quantitativi di miglioramento.
			\end{itemize}
			\item dovrebbe rispettare i tempi indicati nel documento \PPdocRR;
			\item dovrebbe rispettare i costi dichiarati nel documento \PPdocRR;
		\end{itemize}
		Nelle sezioni successive vengono dichiarati gli obiettivi che il gruppo vuole perseguire. Per ognuno di essi, vengono definiti i criteri con cui si effettuano le misurazioni qualitative,
		specificando valori minimi e valori ottimali.

			\paragraph{Miglioramento costante - OPC1}
			Per rendere le performance dei processi costantemente migliorabili e perguire gli obiettivi quantitativi di miglioramento si è deciso di utilizzare il modello CMM.
			Si vuole raggiungere come valore minimo il livello 2 di questa scala, mentre, come valore ottimale, si vuole raggiungere il livello 4. \\
			Riassumendo: \\ \\
			\textbf{Modello utilizzato:} CMM; \\ \\
			\textbf{Soglia di accettabilità:} livello 2 previsto da CMM; \\ \\
			\textbf{Soglia di ottimalità:} livello 4 previsto da CMM. \\ \\
			Per una più dettagliata descrizione del modello CMM consultare l'appendice A. \\
			Per approfondire la scelta delle soglie di accettabilità e ottimalità consultare la metrica alla sezione (scrivere sezione).	
			
			\paragraph{Rispetto della pianificazione - OPC2}
			Per capire se l'attività di un processo rispetta i tempi stabiliti dalla pianificazione all'interno del \PPdocRR{} viene utilizzata la metrica Schedule Variance.
			Si desidera, come soglia minima accettabile, che un processo sia in ritardo non più del 5\% rispetto alla pianificazione. Sarebbe ottimale, invece, non avere ritardi
			rispetto alla pianificazione o, ancora meglio, essere in anticipo.\\
			Riassumendo: \\ \\
			\textbf{Metrica utilizzata:} Schedule Variance; \\ \\
			\textbf{Soglia di accettabilità:} ritardo al massimo del 5\% rispetto alla pianificazione; \\ \\
			\textbf{Soglia di ottimalità:} nessun ritardo (0\%) o in anticipo rispetto alla pianificazione. \\ \\
			Per approfondire la scelta delle soglie di accettabilità e ottimalità consultare la metrica alla sezione (scrivere sezione).	
			
			\paragraph{Rispetto del budget - OPC3}
			Per capire se i costi di un processo rientrano nel budget stabilito dalla pianificazione all'interno del \PPdocRR{} viene utilizzata la metrica Cost Variance.
			Si desidera, come soglia minima accettabile, che un processo non superi il 10\% del budget pianificato. Sarebbe ottimale, invece, non superare i costi pianificati o, ancora meglio,
			spendere meno.\\
			Riassumendo: \\ \\
			\textbf{Metrica utilizzata:} Cost Variance; \\ \\
			\textbf{Soglia di accettabilità:} costi non superiori al 10\% rispetto alla pianificazione; \\ \\
			\textbf{Soglia di ottimalità:}  costi pianificati (0\%) o inferiori. \\ \\
			Per approfondire la scelta delle soglie di accettabilità e ottimalità consultare la metrica alla sezione (scrivere sezione).
			
\end{document}