\documentclass[PianoDiQualifica.tex]{subfiles}

\begin{document}

\section{Visione generale della strategia di gestione della qualità}
	\subsection{Obiettivi qualitativi}
		In questa sezione vengono descritti gli obiettivi di qualità che il gruppo \GRUPPO{} decide di perseguire durante l'intero progetto.
		Ogni obiettivo viene definito in modo quantitativo per permettere al team di valutarne il raggiungimento.
		Vengono quindi fissati dei valori minimi che è obbligatorio superare per soddisfarlo e dei valori ottimali che ne rappresentano il pieno (ma non obbligatorio) conseguimento.
		A tale scopo vengono utilizzati modelli, metriche e standard. \\
		Viene assegnato un codice identificativo ad ogni obiettivo, al fine di semplificarne il tracciamento con la metrica ad esso associata. \\
		Il metodo di denominazione degli obiettivi è descritto in dettaglio nel documento \NPdocRR.
		
		\subsubsection{Qualità di processo}
		Da processi scadenti derivano prodotti scadenti. Quindi, la qualità di processo è un fattore indispensabile per garantire la qualità dei prodotti. Assicurarla, inoltre, permette di:
		\begin{itemize}
			\item favorire l'ottimizzazione delle risorse; 
			\item migliorare la stima dei rischi;
			\item ridurre i costi.
		\end{itemize}
		Desideriamo che ogni processo possegga le seguenti caratteristiche ottimali:
		\begin{itemize}
			\item dovrebbe essere in grado di migliorarsi continuamente:
			\begin{itemize}
					\item le sue performance sono costantemente misurabili;
					\item deve perseguire sempre gli obiettivi quantitativi di miglioramento.
			\end{itemize}
			\item dovrebbe rispettare i tempi indicati nel documento \PPdocRR;
			\item dovrebbe rispettare i costi dichiarati nel documento \PPdocRR;
		\end{itemize}
		Nelle sezioni successive vengono dichiarati gli obiettivi che il gruppo vuole perseguire. Per ognuno di essi, vengono definiti i criteri con cui si effettuano le misurazioni qualitative,
		specificando valori minimi e valori ottimali.

			\paragraph{Miglioramento costante - OPC1}
			Per rendere le performance dei processi costantemente migliorabili e perguire gli obiettivi quantitativi di miglioramento si è deciso di utilizzare il modello CMM.
			Si vuole raggiungere come valore minimo il livello 2 di questa scala, mentre, come valore ottimale, si vuole raggiungere il livello 4. \\
			Riassumendo: \\ \\
			\textbf{Modello utilizzato:} CMM; \\ \\
			\textbf{Soglia di accettabilità:} livello 2 previsto da CMM; \\ \\
			\textbf{Soglia di ottimalità:} livello 4 previsto da CMM. \\ \\
			Per una più dettagliata descrizione del modello CMM consultare l'appendice A. \\
			Per approfondire la scelta delle soglie di accettabilità e ottimalità consultare la metrica alla sezione (scrivere sezione).	
			
			\paragraph{Rispetto della pianificazione - OPC2}
			Per capire se l'attività di un processo rispetta i tempi stabiliti dalla pianificazione all'interno del \PPdocRR{} viene utilizzata la metrica Schedule Variance.
			Si desidera, come soglia minima accettabile, che un processo sia in ritardo non più del 5\% rispetto alla pianificazione. Sarebbe ottimale, invece, non avere ritardi
			rispetto alla pianificazione o, ancora meglio, essere in anticipo.\\
			Riassumendo: \\ \\
			\textbf{Metrica utilizzata:} Schedule Variance; \\ \\
			\textbf{Soglia di accettabilità:} ritardo al massimo del 5\% rispetto alla pianificazione; \\ \\
			\textbf{Soglia di ottimalità:} nessun ritardo (0\%) o in anticipo rispetto alla pianificazione. \\ \\
			Per approfondire la scelta delle soglie di accettabilità e ottimalità consultare la metrica alla sezione (scrivere sezione).	
			
			\paragraph{Rispetto del budget - OPC3}
			Per capire se i costi di un processo rientrano nel budget stabilito dalla pianificazione all'interno del \PPdocRR{} viene utilizzata la metrica Cost Variance.
			Si desidera, come soglia minima accettabile, che un processo non superi il 10\% del budget pianificato. Sarebbe ottimale, invece, non superare i costi pianificati o, ancora meglio,
			spendere meno.\\
			Riassumendo: \\ \\
			\textbf{Metrica utilizzata:} Cost Variance; \\ \\
			\textbf{Soglia di accettabilità:} costi non superiori al 10\% rispetto alla pianificazione; \\ \\
			\textbf{Soglia di ottimalità:}  costi pianificati (0\%) o inferiori. \\ \\
			Per approfondire la scelta delle soglie di accettabilità e ottimalità consultare la metrica alla sezione (scrivere sezione).
			
		\subsubsection{Qualità di prodotto}
		Per garantire la migliore qualità del prodotto è necessario che i processi che lo producono abbiano alta qualità.
		Inoltre, il gruppo \GRUPPO{} cercherà di seguire lo standard ISO/IEC 9126:2001 (vedi appendice ...). \\
		È prevista la realizzazione di due tipologie di prodotto: software e documenti.
		Nelle sezioni successive vengono dichiarati gli obiettivi di qualità di prodotto che il gruppo vuole perseguire, suddivisi per tipo.
		Per ognuno di essi, vengono definiti i criteri con cui si effettuano le misurazioni qualitative, specificando valori minimi e valori ottimali.
		
			\paragraph{Qualità dei documenti}
			Gli obiettivi di qualità riguardanti i documenti prefissati dal gruppo \GRUPPO{} sono i seguenti:
			\begin{itemize}
				\item i documenti devono essere corretti a livello ortografico;
				\item i documenti devono essere corretti a livello concettuale;
				\item i documenti devono essere comprensibili da individui con licenza superiore.
			\end{itemize}
			Verrano ora descritti metriche e criteri utilizzati per garantire le caratteristiche sopra descritte, fissando valori minimi e valori ottimali.
			
				\subparagraph{Leggibilità e comprensibilità - OPDD1}
				Per determinare il grado di leggibilità e comprensibilità del documento, il gruppo ha deciso di utilizzare l'indice Gulpease. Si desidera come soglia minima accettabile un indice
				pari a 40 e, come soglia ottimale, un indice pari a 60. \\
				Riassumendo: \\ \\
				\textbf{Metrica utilizzata:} indice Gulpease; \\ \\
				\textbf{Soglia di accettabilità:} indice maggiore di 40; \\ \\
				\textbf{Soglia di ottimalità:} indice maggiore di 60. \\ \\
				Per approfondire la scelta delle soglie di accettabilità e ottimalità consultare la metrica alla sezione (scrivere sezione).
				
				\subparagraph{Correttezza ortografica - OPDD2}
				Per determinare il grado di correttezza ortografica del documento, il gruppo ha deciso di utilizzare la seguente metrica: percentuale di errori ortografici rinvenuti e non corretti.
				Pertanto, la soglia minima accettabile e la soglia ottimale coincidono e corrispondono a una correzione totale degli errori rinvenuti. \\
				Riassumendo: \\ \\
				\textbf{Metrica utilizzata:} percentuale di errori ortografici rinvenuti e non corretti; \\ \\
				\textbf{Soglia di accettabilità:} tutti gli errori ortografici rinvenuti sono stati corretti (0\%); \\ \\
				\textbf{Soglia di ottimalità:} tutti gli errori ortografici rinvenuti sono stati corretti (0\%). \\ \\
				Per approfondire la scelta delle soglie di accettabilità e ottimalità consultare la metrica alla sezione (scrivere sezione).
				
				\subparagraph{Correttezza concettuale - OPDD3}
				Per determinare il grado di correttezza concettuale del documento, il gruppo ha deciso di utilizzare la seguente metrica: percentuale di errori concettuali rinvenuti e non corretti.
				Si desidera come soglia minima accettabile che non più del 5\% degli errori concettuali rinvenuti non siano stati corretti e, come soglia ottimale, che tutti gli errori
				concettuali rinvenuti siano stati corretti.\\
				Riassumendo: \\ \\
				\textbf{Metrica utilizzata:} percentuale di errori concettuali rinvenuti e non corretti; \\ \\
				\textbf{Soglia di accettabilità:} non più del 5\% degli errori concettuali rinvenuti non sono stati corretti;\\ \\
				\textbf{Soglia di ottimalità:} tutti gli errori concettuali rinvenuti sono stati corretti (0\%). \\ \\
				Per approfondire la scelta delle soglie di accettabilità e ottimalità consultare la metrica alla sezione (scrivere sezione).
				
			\paragraph{Qualità del software}
			Gli obiettivi di qualità riguardanti il software prefissati dal gruppo \GRUPPO{} sono un sottoinsieme di quelli definiti nello standard ISO/IEC 9126:2001:
			\begin{itemize}
				\item il prodotto possiede le funzionalità descritte all’interno dei requisiti obbligatori;
				\item il prodotto possiede le funzionalità descritte all’interno dei requisiti desiderabili;
				\item il codice risulta manutenibile e facilmente comprensibile;
				\item il prodotto è testato in ogni sua parte e in ogni situazione nella quale si può trovare;
				\item il prodotto è robusto e non interrompe l’esecuzione in seguito a situazioni anomale;
				\item il prodotto garantisce un funzionamento senza interruzioni.
			\end{itemize}
			
				\subparagraph{Funzionalità obbligatorie - OPDS1}
				Il prodotto deve possedere tutte le funzionalità descritte nei requisiti obbligatori. Per determinare il numero di requisiti obbligatori soddisfatti viene usata la
				seguente metrica: percetuale di requisiti obbligatori soddisfatti. \\
				Riassumendo: \\ \\
				\textbf{Metrica utilizzata:} percentuale di requisiti obbligatori soddisfatti;\\ \\
				\textbf{Soglia di accettabilità:} tutti i requisiti obbligatori sono soddisfatti (100\%); \\ \\
				\textbf{Soglia di ottimalità:} tutti i requisiti obbligatori sono soddisfatti (100\%). \\ \\
				Per approfondire la scelta delle soglie di accettabilità e ottimalità consultare la metrica alla sezione (scrivere sezione).
				
				\subparagraph{Funzionalità desiderabili - OPDS2}
				Per determinare il numero di requisiti desiderabili soddisfatti viene usata la seguente metrica: percentuale di requisiti desiderabili soddisfatti. Si desidera come soglia
				minima accettabile che x\% dei requisiti desiderabili sia soddisfatto mentre, come soglia ottimale, che tutti i requisiti desiderabili siano soddisfatti. \\
				Riassumendo: \\ \\
				\textbf{Metrica utilizzata:} percentuale di requisiti desiderabili soddisfatti;\\ \\
				\textbf{Soglia di accettabilità:} almeno x\% dei requisiti desiderabili soddisfatti; \\ \\
				\textbf{Soglia di ottimalità:} tutti i requisiti desiderabili sono soddisfatti (100\%). \\ \\
				Per approfondire la scelta delle soglie di accettabilità e ottimalità consultare la metrica alla sezione (scrivere sezione).
				
				\subparagraph{Manutenibilità e Comprensibilità del codice - OPDS3}
				Il grado di manutenibilità e comprensibilità del codice deriva dalla sua complessità e lunghezza. È importante quindi che il prodotto abbia codice manutenibile e
				privo di incomprensioni al suo interno. \\
				Metriche e soglie verrano definite in dettaglio nelle fasi progettuali successive.
				
				\subparagraph{Copertura dei test richiesti - OPDS4}
				Il prodotto deve essere testato in ogni sua parte per garantirne il funzionamento. Vengono considerati solo i test riguardanti le funzionalità descritte nei requisiti.
				Si desidera come soglia minima accettabile che il numero di test passati sia almeno del 80\% mentre, come soglia ottimale, almeno del 90\%. \\
				Riassumendo: \\ \\
				\textbf{Metrica utilizzata:} percentuale di test passati;\\ \\
				\textbf{Soglia di accettabilità:} almeno 80\% dei test passati; \\ \\
				\textbf{Soglia di ottimalità:} almeno 90\% dei test passati. \\ \\
				Per approfondire la scelta delle soglie di accettabilità e ottimalità consultare la metrica alla sezione (scrivere sezione).
				
				\subparagraph{Robustezza - OPDS5}
				Il software deve essere robusto e deve quindi saper gestire situazioni anomale.
				Si desidera come soglia minima accettabile che il numero di situazioni anomale gestite sia almeno del 80\% mentre, come soglia ottimale, almeno del 90\%. \\
				Riassumendo: \\ \\
				\textbf{Metrica utilizzata:} Failure Avoidance;\\ \\
				\textbf{Soglia di accettabilità:} gestite almeno 80\% delle situazioni anomale; \\ \\
				\textbf{Soglia di ottimalità:} gestite almeno 90\% delle situazioni anomale. \\ \\
				Per approfondire la scelta delle soglie di accettabilità e ottimalità consultare la metrica alla sezione (scrivere sezione).
				
				\subparagraph{Funzionamento senza interruzioni - OPDS6}
				Il prodotto deve garantire un funzionamento senza interruzioni.
				Si desidera come soglia minima accettabile che il numero di interruzioni evitate sia almeno del 80\% mentre, come soglia ottimale, almeno del 90\%. \\
				Riassumendo: \\ \\
				\textbf{Metrica utilizzata:} Breakdown Avoidance;\\ \\
				\textbf{Soglia di accettabilità:} evitate almeno 80\% delle interruzioni; \\ \\
				\textbf{Soglia di ottimalità:} evitate almeno 90\% delle interruzioni. \\ \\
				Per approfondire la scelta delle soglie di accettabilità e ottimalità consultare la metrica alla sezione (scrivere sezione).
				
		\subsection{Scadenze temporali}
		Le scadenze che il gruppo \GRUPPO{} ha deciso di rispettare sono riportate nel \PPdocRR{}.
				
\end{document}