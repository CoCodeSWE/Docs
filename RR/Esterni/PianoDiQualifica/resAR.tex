\documentclass[PianoDiQualifica.tex]{subfiles}

\begin{document}

\section{Resoconto delle attività di verifica - fase AR}
All'interno di questa sezione sono riportati gli esiti di tutte le attività di verifica effettuate nell'arco della fase AR, come previsto dal documento \PPdocRR{}. Ove necessario sono state tratte conclusioni sui risultati e su come essi possano essere migliorati.

\subsection{Verifica sui processi}
	\subsubsection{Processo di documentazione}
		\paragraph{Miglioramento costante}
		Per rendere le performance dei processi costantemente migliorabili e perseguire gli obiettivi quantitativi di miglioramento viene utilizzato il modello Capability Maturity Model (\gl{CMM}).\\
		All'inizio della fase i processi si trovavano al livello 1 della scala \gl{CMM}. In seguito, grazie alla stesura del documento \NPdocRR{} sono state definite regole per ogni tipo di documentazione,  strumenti da utilizzare e procedure da seguire. Questo ha permesso un maggiore controllo del processo di documentazione, che ha ottenuto la ripetibilità, proprietà che caratterizza il livello 2 della scala \gl{CMM}. Si può quindi affermare che il processo di documentazione ha raggiunto tale livello. Non si può ancora affermare di aver raggiunto il livello 3 del modello perchè al processo manca ancora la sua caratteristica principale, la proattività.\\
		Secondo le metriche definite il valore raggiunto rappresenta la \textbf{soglia minima accettabile}, ma nelle prossime fasi il gruppo si impegnerà a raggiungere la soglia ottimale (sfruttando \gl{PDCA}).

		\paragraph{Rispetto della pianificazione}
			Per capire se l'attività di un processo rispetta i tempi stabiliti dalla pianificazione all'interno del \PPdocRR{} viene utilizzata la metrica Schedule Variance.
			Si desidera, come soglia minima accettabile, che un processo sia in ritardo non più del 5\% rispetto alla pianificazione. Sarebbe ottimale, invece, non avere ritardi rispetto alla pianificazione o, ancora meglio, essere in anticipo.\\
			Di seguito sono riportati i valori ottenuti calcolando la Schedule Variance sui tempi di stesura di ogni documento nella fase AR:\\
			
			\begin{table}[h]
				\centering
				\begin{tabular}{l c c}
					\hline
					\rule[-0.3cm]{0cm}{0.8cm}
					\textbf{Documento} & \textbf{Schedule Variance} & \textbf{Esito} \\
					\hline
					\rule[0cm]{0cm}{0.4cm}
					\PPdocRR & x & x \\
					\rule[0cm]{0cm}{0.4cm}
					\NPdocRR & x & x \\ 
					\rule[0cm]{0cm}{0.4cm}
					\ARdocRR & x & x \\ 
					\rule[0cm]{0cm}{0.4cm}
					\PQdocRR & x & x \\ 
					\rule[0cm]{0cm}{0.4cm}
					\Gldoc & x & x \\ 
					\rule[0cm]{0cm}{0.4cm}
					analisiSDK(macro) & x & x \\ 
					\hline
				\end{tabular}
				\caption{Esiti del calcolo della Schedule Variance sul processo di documentazione durante la fase AR}
			\end{table}
		\paragraph{Rispetto del budget}
		Per capire se i costi di un processo rientrano nel budget stabilito dalla pianificazione all'interno del 				\PPdocRR{} viene utilizzata la metrica Cost Variance. Si desidera, come soglia minima accettabile, che un 				processo non superi il 10\% del budget pianificato. Sarebbe ottimale, invece, non superare i costi pianificati 			o, ancora meglio, spendere meno.\\
		Di seguito sono riportati i valori ottenuti calcolando la Cost Variance sui tempi di stesura di ogni 				documento nella fase AR:\\
\begin{table}[h]
				\centering
				\begin{tabular}{l c c}
					\hline
					\rule[-0.3cm]{0cm}{0.8cm}
					\textbf{Documento} & \textbf{Cost Variance} & \textbf{Esito} \\
					\hline
					\rule[0cm]{0cm}{0.4cm}
					\PPdocRR & x & x \\
					\rule[0cm]{0cm}{0.4cm}
					\NPdocRR & x & x \\ 
					\rule[0cm]{0cm}{0.4cm}
					\ARdocRR & x & x \\ 
					\rule[0cm]{0cm}{0.4cm}
					\PQdocRR & x & x \\ 
					\rule[0cm]{0cm}{0.4cm}
					\Gldoc & x & x \\ 
					\rule[0cm]{0cm}{0.4cm}
					analisiSDK(macro) & x & x \\ 
					\hline
				\end{tabular}
				\caption{Esiti del calcolo della Cost Variance sul processo di documentazione durante la fase AR}
			\end{table}		
		
	\subsubsection{Processo di verifica}
		\paragraph{Miglioramento costante}
				Per rendere le performance dei processi costantemente migliorabili e perseguire gli obiettivi quantitativi di miglioramento viene utilizzato il modello Capability Maturity Model (\gl{CMM}).\\
		All'inizio della fase i processi si trovavano al livello 1 della scala \gl{CMM}. In seguito, grazie alla stesura del documento \NPdocRR{}  sono state definite regole per ogni tipo di documentazione, strumenti da utilizzare e procedure da seguire, oltre che alla definizione di metriche in questo documento. Questo ha permesso un maggiore controllo del processo di verifica, che ha ottenuto la ripetibilità, proprietà che caratterizza il livello 2 della scala \gl{CMM}. Si può quindi affermare che il processo di documentazione ha raggiunto tale livello. Non si può ancora affermare di aver raggiunto il livello 3 del modello perchè al processo manca ancora la sua caratteristica principale, la proattività.\\
		Secondo le metriche definite il valore raggiunto rappresenta la \textbf{soglia minima accettabile}, ma nelle prossime fasi il gruppo si impegnerà a raggiungere la soglia ottimale (sfruttando \gl{PDCA}).
		\paragraph{Rispetto della pianificazione}
		Per capire se l'attività di un processo rispetta i tempi stabiliti dalla pianificazione all'interno del \PPdocRR{} viene utilizzata la metrica Schedule Variance.
			Si desidera, come soglia minima accettabile, che un processo sia in ritardo non più del 5\% rispetto alla pianificazione. Sarebbe ottimale, invece, non avere ritardi rispetto alla pianificazione o, ancora meglio, essere in anticipo.\\
			Di seguito è riportato il valore ottenuto calcolando la Schedule Variance sul processo di verifica nella fase AR:\\
			
			\begin{table}[h]
				\centering
				\begin{tabular}{l c c}
					\hline
					\rule[-0.3cm]{0cm}{0.8cm}
					\textbf{Processo} & \textbf{Schedule Variance} & \textbf{Esito} \\
					\hline
					\rule[0cm]{0cm}{0.4cm}
					Processo di verifica & x & x \\
					\hline
				\end{tabular}
				\caption{Esiti del calcolo della Schedule Variance sul processo di verifica durante la fase AR}
			\end{table}		
		
			
			
		\paragraph{Rispetto del budget}
		
		Per capire se i costi di un processo rientrano nel budget stabilito dalla pianificazione all'interno del \PPdocRR{} viene utilizzata la metrica Cost Variance. Si desidera, come soglia minima accettabile, che un processo non superi il 10\% del budget pianificato. Sarebbe ottimale, invece, non superare i costi pianificati o, ancora meglio, spendere meno.\\
		Di seguito è riportato il valore ottenuto calcolando la Cost Variance sul processo di verifica nella fase AR:\\
		
		
		
		\begin{table}[h]
				\centering
				\begin{tabular}{l c c}
					\hline
					\rule[-0.3cm]{0cm}{0.8cm}
					\textbf{Processo} & \textbf{Cost Variance} & \textbf{Esito} \\
					\hline
					\rule[0cm]{0cm}{0.4cm}
					Processo di verifica & x & x \\
					\hline
				\end{tabular}
				\caption{Esiti del calcolo della Cost Variance sul processo di verifica durante la fase AR}
			\end{table}		
		
	
\subsection{Verifica sui prodotti}
	\subsubsection{Documenti}
		\paragraph{Leggibilità e comprensibilità}
		Per determinare il grado di leggibilità e comprensibilità del documento, il gruppo ha deciso di utilizzare l'indice Gulpease. Si desidera come soglia minima accettabile un indice
				maggiore o uguale a 40 e, come soglia ottimale, un indice maggiore di 60.\\
				Di seguito sono riportati i valori ottenuti calcolando l'indice Gulpease sui documenti della fase AR:
				
				\begin{table}[h]
				\centering
				\begin{tabular}{l c c}
					\hline
					\rule[-0.3cm]{0cm}{0.8cm}
					\textbf{Documento} & \textbf{Gulpease} & \textbf{Esito} \\
					\hline
					\rule[0cm]{0cm}{0.4cm}
					\PPdocRR & x & x \\
					\rule[0cm]{0cm}{0.4cm}
					\NPdocRR & x & x \\ 
					\rule[0cm]{0cm}{0.4cm}
					\ARdocRR & x & x \\ 
					\rule[0cm]{0cm}{0.4cm}
					\PQdocRR & x & x \\ 
					\rule[0cm]{0cm}{0.4cm}
					\Gldoc & x & x \\ 
					\rule[0cm]{0cm}{0.4cm}
					analisiSDK(macro) & x & x \\ 
					\hline
				\end{tabular}
				\caption{Esiti del calcolo dell'indice Gulpease sui documenti della fase AR}
			\end{table}		
		\paragraph{Correttezza ortografica}
		Per determinare il grado di correttezza ortografica del documento, il gruppo ha deciso di utilizzare la seguente metrica: percentuale di errori ortografici rinvenuti e non corretti.
				Pertanto, la soglia minima accettabile e la soglia ottimale coincidono e corrispondono a una correzione totale degli errori rinvenuti.\\
				Di seguito è riportato il numero di errori ortografici trovati:\\
				
				\begin{table}[h]
				\centering
				\begin{tabular}{l c}
					\hline
					\rule[0cm]{0cm}{0.4cm}
					Errori ortografici & x \\
					\hline
				\end{tabular}
				\caption{Errori ortografici rinvenuti durante la fase AR}
			\end{table}		
			
			Tutti gli errori ortografici rinvenuti sono stati corretti, quindi è stato
raggiunto l’obiettivo \textbf{ottimale}.
				
				
		\paragraph{Correttezza concettuale}
		Per determinare il grado di correttezza concettuale del documento, il gruppo ha deciso di utilizzare la seguente metrica: percentuale di errori concettuali rinvenuti e non corretti.
				Si desidera come soglia minima accettabile che non più del 5\% degli errori concettuali rinvenuti non siano stati corretti e, come soglia ottimale, che tutti gli errori
				concettuali rinvenuti siano stati corretti.\\
				Di seguito è riportato il numero di errori concettuali trovati:\\
				
					\begin{table}[h]
				\centering
				\begin{tabular}{l c}
					\hline
					\rule[0cm]{0cm}{0.4cm}
					Errori concettuali & x \\
					\hline
				\end{tabular}
				\caption{Errori concettuali rinvenuti durante la fase AR}
			\end{table}	
			
			Tutti gli errori concettuali rinvenuti sono stati corretti, quindi è stato
raggiunto l’obiettivo \textbf{ottimale}.

\end{document}