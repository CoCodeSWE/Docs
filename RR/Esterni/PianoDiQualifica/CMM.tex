\documentclass[PianoDiQualifica.tex]{subfiles}

\begin{document}

\hypertarget{CMM_label}{\section{Capability Maturity Model}}
	Il Capability Maturity Model (\gl{CMM}) è stato ideato e introdotto inizialmente dal Dipartimento della Difesa statunitense. Tale modello
	fornisce:
	\begin{itemize}
		\item una base concettuale su cui appoggiarsi per valutare il livello dei processi;
		\item un insieme di best practices consolidate negli anni da esperti e utilizzatori;
		\item un linguaggio comune e una visione condivisa;
		\item un metodo per definire un miglioramento in ambito organizzativo.
	\end{itemize}
	\subsection{Struttura}
	Il \gl{CMM} è costituito dalla seguente struttura:
	\begin{itemize}
		\item \textbf{Livelli di maturità}: sono cinque, dove il più alto (il quinto) è uno stato ideale in cui i processi vengono sistematicamente
		gestiti da una combinazione di processi di ottimizzazione e di miglioramento continuo. 
		\item \textbf{Aree chiave del processo}: identifica una serie di attività correlate che, se svolte collettivamente,
		realizzano un insieme di obiettivi considerati importanti;
		\item \textbf{Obiettivi}: indicano lo scopo, i confini e l'intento di ogni area chiave del processo;
		\item \textbf{Caratteristiche comuni}: includono le pratiche che implementano e regolamentano un'area chiave del processo. Ci sono cinque
		tipologie di caratteristiche comuni:
		\begin{itemize}
			\item impegno nell'operare;
			\item abilità nell'operare;
			\item attività eseguite;
			\item misurazioni ed analisi;
			\item veriche dell'implementazione.
		\end{itemize}
		\item \textbf{pratiche chiave}: descrivono gli elementi dell'infrastruttura e delle pratiche che contribuiscono maggiormente all'implementazione
		e la regolamentazione di un'area.
	\end{itemize}
	
	\subsection{Livelli}
	I livelli di maturità che costituiscono il \gl{CMM} sono:
	\begin{itemize}
		\item \textbf{Primo livello - iniziale (caotico)}: i processi che rientrano in questo livello sono tipicamente privi di documentazione e in uno stato di
		continuo cambiamento. Ad esempio, i processi vengono riadattati alle esigenze degli utenti. Quello che si ottiene è quindi un
		ambiente caotico e instabile per i processi;
		\item \textbf{Secondo livello - ripetibile}:  i processi che rientrano in questo livello sono ripetibili e spesso portano a risultati consistenti.
		Inizia a vedersi una certa disciplina nei processi e nella documentazione ad essi associata;
		\item \textbf{Terzo livello - definito}: i processi che rientrano in questo livello sono definiti e documentati secondo degli standard.
		\item \textbf{Quarto livello - gestito}: i processi che rientrano in questo livello possono essere gestiti dai manager, in base alle esigenze, senza
		perdita di qualità o deviazione dalle specifiche;
		\item \textbf{Quinto livello - ottimizzante}: i processi che rientrano in questo livello sono soggetti ad un continuo miglioramento delle proprie
		performance attraverso cambiamenti incrementali e miglioramenti tecnologici.
	\end{itemize}

\end{document}