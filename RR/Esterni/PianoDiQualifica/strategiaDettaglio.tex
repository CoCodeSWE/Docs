\documentclass[PianoDiQualifica.tex]{subfiles}

\begin{document}

\section{La strategia di gestione della qualità nel dettaglio}

	\subsection{Risorse}
	Per garantire un buon funzionamento del processo di verifica verranno impiegati i seguenti tipi di risorse:
	\begin{itemize}
		\item risorse umane;
		\item risorse hardware;
		\item risorse software.
	\end{itemize}
	
		\subsubsection{Risorse necessarie}
		
			\paragraph{Risorse umane}
			Le risorse umane necessarie al processo di verifica sono i \VERP{} e il \RESP{}. Informazioni più dettagliate sui ruoli sono riportate nelle \NPdocRR{}.
			
			\paragraph{Risorse hardware}
			Per eseguire la verifica, il gruppo dovrà avere a disposizione dei computer con un'adeguata potenza di calcolo in grado di sopportare il carico di lavoro.
			
			\paragraph{Risorse software}
			Le risorse software necessarie alla verifica sono gli strumenti software che eseguono controlli sui documenti e verificano che non violino le \NPdocRR{}.
			Gli strumenti software devono avere le seguenti caratteristiche:
			\begin{itemize}
				\item rilevare eventuali errori ortografici;
				\item costruire e visualizzare in tempo reale il documento scritto in \LaTeX (in modo che sia facile accorgersi di errori nell’utilizzo dei comandi).
			\end{itemize}
			Inoltre è necessario disporre di una piattaforma che raccolga i vari errori incontrati e li segnali ai componenti del gruppo che dovranno occuparsene.
			
		\subsubsection{Risorse disponibili}
		
			\paragraph{Risorse umane}
			Tutti i membri del gruppo sono a disposizione per eseguire operazioni di verifica. Ognuno dei componenti, a turno, ricoprirà il ruolo di \RESP{} o di \VER{} come definito nel \PPdocRR{}.
			
			\paragraph{Risorse hardware}
			Le risorse hardware disponibili sono i vari computer dei componenti del gruppo incaricati di svolgere il ruolo di \RESP{} o \VER{}.
			
			\paragraph{Risorse software}
			Le risorse software disponibili comprendono editor \LaTeX{} con controlli integrati e script per controllare la leggibilità e la complessità dei documenti in riferimento all’indice Gulpease.
			Sarà disponibile anche il sistema di sollevamento delle \gl{issue} offerto dalla piattaforma \gl{GitHub}. Per maggiori informazioni sulla procedura di sollevamento e gestione delle \gl{issue} si veda
			il documento \NPdocRR{}.
			
	\subsection{Misure e metriche}
	
		\subsubsection{Misure}
		Ogni misura effettuata sui processi e sui prodotti deve essre confrontata con una scala. I valori della scala sono:
		\begin{itemize}
			\item \textbf{Negativo}: valore non accettabile, bisogna correggere gli errori presenti ed effettuare ulteriori verifiche;
			\item \textbf{Accettabile}: valore accettabile, l’oggetto sottoposto a verifica ha raggiunto una soglia minima.
			\item \textbf{Ottimale}: valore accettabile, l’oggetto sottoposto a verifica ha raggiunto le massime aspettative del team.
		\end{itemize}
		
		\subsubsection{Metriche per processi}
		Viene assegnato un codice identificativo ad ogni metrica, al fine di semplificarne il tracciamento con l'obiettivo ad essa associato. \\
		Il metodo di denominazione delle metriche è descritto in dettaglio nel documento \NPdocRR{}.
		
			\hypertarget{CMM_m}{\paragraph{Capability Maturity Model - MPC1}}
			Per controllare e verificare la qualità dei processi, vengono adottate le metriche fornite dal modello \gl{CMM}.
			Per ogni fase di lavoro, viene associato un indice che descriverà la qualità della fase.
			L'indice, relativo ad una scala appartenente al \gl{CMM}, può assumere i valori tra 1 (il peggiore) e 5 (il migliore). \\
			Il gruppo ha stabilito i seguenti intervalli:
			\begin{itemize}
				\item il valore 1 è considerato negativo;
				\item i valori 2 e 3 sono considerati accettabili;
				\item i valori 4 e 5 sono considerati ottimali.
			\end{itemize}
			Per approfondimenti sul modello \gl{CMM} consultare l'appendice A.
			
			\hypertarget{Schedule_m}{\paragraph{Schedule Variance - MPC2}}
			La presente metrica indica se le attività di \gl{progetto} sono in anticipo o in ritardo rispetto a quelle pianificate nel \PPdocRR{}. \\
			Viene calcolata come la differenza fra la data reale di fine di un’attività e la data pianificata di fine dell’attività stessa. \\
			Se la Schedule Variance ha un valore maggiore di 0 allora il processo è in ritardo rispetto alla pianificazione. Viceversa, se il valore è minore di 0, allora il processo è
			in anticipo. Se invece il valore è uguale a 0 allora è in linea con quanto pianificato. \\
			Il gruppo ha stabilito i seguenti intervalli:
			\begin{itemize}
				\item un valore maggiore del 5\% rispetto alla pianificazione è considerato negativo;
				\item un valore minore o uguale al 5\% rispetto alla pianificazione è considerato accettabile;
				\item valori minori o uguali a 0 sono considerati ottimali.
			\end{itemize}
			
			\hypertarget{Cost_m}{\paragraph{Cost Variance - MPC3}}
			La presente metrica indica se i costi alla data corrente sono maggiori o minori rispetto a quanto dichiarato nel \PPdocRR{}. \\
			Viene calcolata come la differenza fra consuntivo e preventivo. \\
			Il gruppo ha stabilito i seguenti intervalli:
			\begin{itemize}
				\item un valore maggiore del 10\% delle risorse preventivate per il processo è considerato negativo;
				\item un valore minore o uguale al 10\% delle risorse preventivate per il processo è accettabile;
				\item un valore minore o uguale a 0 è considerato ottimale.
			\end{itemize}
			
		\subsubsection{Metriche per i prodotti}
		Viene assegnato un codice identificativo ad ogni metrica, al fine di semplificarne il tracciamento con l'obiettivo ad essa associato. \\
		Il metodo di denominazione delle metriche è descritto in dettaglio nel documento \NPdocRR{}.
		
			\paragraph{Metriche per i documenti}
			La qualità di un documento dipende soprattutto dai suoi contenuti. La loro qualità, tuttavia, è difficilmente quantificabile allo stato attuale del \gl{progetto} a causa
			dell’esperienza pressoché nulla del gruppo in quest’ambito. Si è deciso dunque di limitarsi a valutare parametri maggiormente oggettivi e soprattutto misurabili automaticamente
			attraverso strumenti software.
			
				\hypertarget{leggi_m}{\subparagraph{Indice di leggibilità - MPDD1}}
				Per valutare la qualità del documento si è deciso di utilizzare l'indice di leggibilità.
				In particolare, è stato considerato l’indice Gulpease, studiato appositamente per la lingua italiana. \\
				Questo particolare indice si basa sulla lunghezza della parola e sulla lunghezza della frase rispetto al numero di lettere. La formula per il suo calcolo è la seguente: \\ \\
				\begin{equation}\textit{Indice Gulpease} = 89 + \frac{300 * \textit{numeroFrasi} + 10 * \textit{numeroLettere}}{\textit{numeroParole}}\end{equation} \\ \\
				Il risultato è compreso tra 0 e 100, dove valori alti indicano leggibilità elevata e viceversa.
				In generale, risulta che testi con un indice:
				\begin{itemize}
					\item inferiore a 80 risultano difficili da leggere per chi ha la licenza elementare;
					\item inferiore a 60 risultano difficili da leggere per chi ha la licenza media;
					\item inferiore a 40 risultano difficili da leggere per chi ha la licenza superiore.
				\end{itemize}
				Poichè la documentazione è rivolta a persone istruite, il gruppo ha stabilito i seguenti intervalli per l'indice:
				\begin{itemize}
					\item valori minori di 40 sono considerati negativi;
					\item valori compresi tra 40 e 60 sono considerati accettabili.
					\item valori maggiori di 60 sono considerati ottimali.
				\end{itemize}
				
				\hypertarget{err_ortografici}{\subparagraph{Errori ortografici rinvenuti e non corretti - MPDD2}}
				Tale metrica è necessaria per capire quanto un documento sia corretto dal punto di vista ortografico. Supponendo che gli strumenti automatici
				siano in grado di trovare tutti gli errori ortografici all’interno di un testo, allora la correttezza ortografica non può che basarsi
				sul numero di errori rinvenuti ma non successivamente corretti. Notare che per errori corretti si intende un errore revisionato manualmente da parte
				di un \VER{}. Le correzioni automatiche, infatti, non sono molto attendibili.
				Il gruppo ha stabilito i seguenti intervalli:
				\begin{itemize}
					\item una percentuale di errori non corretti maggiore allo 0\% è ritenuta negativa;
					\item una percentuale di errori non corretti pari allo 0\% è ritenuta accettabile;
					\item una percentuale di errori non corretti pari allo 0\% è ritenuta ottimale.
				\end{itemize}
				
				\hypertarget{err_concettuali}{\subparagraph{Errori concettuali rinvenuti e non corretti - MPDD3}}
				Tale metrica è necessaria per capire quanto un documento sia corretto dal punto di vista concettuale. Supponendo che, in seguito alle revisioni,
				siano stati individuati tutti gli errori concettuali all’interno di un testo, allora la correttezza concettuale non può che basarsi
				sul numero di errori rinvenuti ma non successivamente corretti. Notare che per errori corretti si intende un errore revisionato manualmente da parte
				di un \VER{}.
				Il gruppo ha stabilito i seguenti intervalli:
				\begin{itemize}
					\item una percentuale di errori non corretti maggiore al 5\% è ritenuta negativa;
					\item una percentuale di errori non corretti minore o uguale al 5\% è ritenuta accettabile;
					\item una percentuale di errori non corretti pari allo 0\% è ritenuta ottimale;
				\end{itemize}
				
			\paragraph{Metriche per il software}
			Il gruppo \GRUPPO{} ha deciso di adottare alcune metriche che hanno il compito di monitorare la qualità interna, la qualità esterna e
			la qualità in uso del software. Tali metriche sono un sottoinsieme di quelle difinite nello standard ISO/IEC 9126:2001. \\
			Ogni metrica scelta viene associata ad una caratteristica di qualità presente all’interno dello standard: \\
			\begin{table}[h]
				\centering
				\begin{tabular}{l c}
					\hline
					\rule[-0.3cm]{0cm}{0.8cm}
					\textbf{Metriche scelte} & \textbf{Caratteristiche di qualità} \\
					\hline
					\rule[0cm]{0cm}{0.4cm}
					MPDS1 - Copertura requisiti obbligatori & Funzionalità \\
					\rule[0cm]{0cm}{0.4cm}
					MPDS2 - Copertura requisiti desiderabili & Funzionalità \\
					\rule[0cm]{0cm}{0.4cm}
					MPDS3 - Copertura requisiti facoltativi & Funzionalità \\
					\rule[0cm]{0cm}{0.4cm}
					MPDS4 - Test passati richiesti & Affidabilità \\
					\rule[0cm]{0cm}{0.4cm}
					MPDS5 - Failure Avoidance & Affidabilità \\
					\rule[0cm]{0cm}{0.4cm}
					MPDS6 - Breakdown Avoidance & Affidabilità \\
					\hline
				\end{tabular}
				\caption{Mappa metriche-caratteristiche}
			\end{table}
			
				\hypertarget{req_obbligatori}{\subparagraph{Copertura requisiti obbligatori - MPDS1}}
				La seguente metrica controlla quanti requisiti obbligatori sono stati soddisfatti. Viene calcolata come rapporto, espresso in percentuale, tra i requisiti obbligatori soddisfatti e
				il numero totale dei requisiti obbligatori. \\
				\begin{equation}\textit{Copertura requisiti obbligatori} = 100 * \frac{\textit{\#requisiti obbligatori soddisfatti}}{\textit{\#requisiti obbligatori totali}}\end{equation}
				Il gruppo ha stabilito i seguenti intervalli:
				\begin{itemize}
					\item una percentuale minore del 100\% è ritenuta negativa;
					\item una percentuale uguale al 100\% è ritenuta accettabile;
					\item una percentuale uguale al 100\% ottimale.
				\end{itemize}
				
				\hypertarget{req_desiderabili}{\subparagraph{Copertura requisiti desiderabili - MPDS2}}
				La seguente metrica controlla quanti requisiti desiderabili sono stati soddisfatti. Viene calcolata come rapporto, espresso in percentuale, tra i requisiti desiderabili soddisfatti e
				il numero totale dei requisiti desiderabili. \\
				\begin{equation}\textit{Copertura requisiti desiderabili} = 100 * \frac{\textit{\#requisiti desiderabili soddisfatti}}{\textit{\#requisiti desiderabili totali}}\end{equation}
				Il gruppo ha stabilito i seguenti intervalli:
				\begin{itemize}
					\item una percentuale minore del 70\% è ritenuta negativa;
					\item una percentuale maggiore o uguale a 70\% è ritenuta accettabile;
					\item una percentuale uguale al 100\% ottimale.
				\end{itemize}
				
							\hypertarget{req_facoltativi}{\subparagraph{Copertura requisiti facoltativi - MPDS3}}
				La seguente metrica controlla quanti requisiti facoltativi sono stati soddisfatti. Viene calcolata come rapporto, espresso in percentuale, tra i requisiti facoltativi soddisfatti e
				il numero totale dei requisiti facoltativi. \\
				\begin{equation}\textit{Copertura requisiti facoltativi} = 100 * \frac{\textit{\#requisiti facoltativi soddisfatti}}{\textit{\#requisiti facoltativi totali}}\end{equation}
				Il gruppo ha stabilito i seguenti intervalli:
				\begin{itemize}
					\item una percentuale maggiore o uguale a 0\% è ritenuta accettabile;
					\item una percentuale uguale al 100\% ottimale.
				\end{itemize}
				
				
				\hypertarget{test_passati}{\subparagraph{Percentuale test passati  - MPDS4}}
				La metrica controlla che il rapporto, espresso in percentuale, tra il numero di test passati e il numero di test totali rientri tra i valori definiti. Questo ci permette
				di valutare se il \gl{prodotto} supera la maggior parte dei test. \\
				\begin{equation}\textit{Test passati richiesti} = 100 * \frac{\textit{\# di test passati}}{\textit{\# di test totali}}\end{equation}
				Il gruppo ha stabilito i seguenti intervalli:
				\begin{itemize}
					\item una percentuale minore del 90\% è ritenuta negativa;
					\item una percentuale maggiore o uguale a 90\% è ritenuta accettabile;
					\item una percentuale uguale a 100\% è ritenuta ottimale;
				\end{itemize}
				
				\hypertarget{failure}{\subparagraph{Failure Avoidance - MPDS5}}
				La metrica controlla che il rapporto, espresso in percentuale, tra il numero di situazioni anomale gestite e il numero di situazioni anomale presentate rientri tra i valori definiti.
				\begin{equation}\textit{Failure Avoidance} = 100 * \frac{\textit{\# situazioni anomale gestite}}{\textit{\# situazioni anomale presentate}}\end{equation}
				Il gruppo ha stabilito i seguenti intervalli:
				\begin{itemize}
					\item una percentuale minore del 80\% è ritenuta negativa;
					\item una percentuale compresa tra 80\% e 90\% è ritenuta accettabile;
					\item una percentuale maggiore di 90\% è ritenuta ottimale.
				\end{itemize}
				
				\hypertarget{breakdown}{\subparagraph{Breakdown Avoidance - MPDS6}}
				La metrica controlla che la percentuale di interruzioni evitate dal \gl{prodotto} rientri tra i valori definiti. Il valore su cui si applicherà la metrica verrà calcolato come il
				complemento delle interruzioni verificate. Questa metrica ci permette di controllare che il \gl{prodotto} lavori senza interruzioni.
				\begin{equation}\textit{Breakdown Avoidance} = 100 * \left ( 1 - \frac{\textit{\# di interruzioni}}{\textit{\# situazioni anomale presentate}} \right ) \end{equation}
				Il gruppo ha stabilito i seguenti intervalli:
				\begin{itemize}
					\item una percentuale minore del 80\% è ritenuta negativa;
					\item una percentuale compresa tra 80\% e 90\% è ritenuta accettabile;
					\item una percentuale maggiore di 90\% è ritenuta ottimale.
				\end{itemize}
				
\end{document}