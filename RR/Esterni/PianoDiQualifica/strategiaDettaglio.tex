\documentclass[PianoDiQualifica.tex]{subfiles}

\begin{document}

\section{La strategia di gestione della qualità nel dettaglio}

	\subsection{Risorse}
	Per garantire un buon funzionamento del processo di verifica verranno impiegati i seguenti tipi di risorse:
	\begin{itemize}
		\item risorse umane;
		\item risorse hardware;
		\item risorse software.
	\end{itemize}
	
		\subsubsection{Risorse necessarie}
		
			\paragraph{Risorse umane}
			Le risorse umane necessarie al processo di verifica sono i \VERP{} e il \RESP{}. Informazioni più dettagliate sui ruoli sono riportate nelle \NPdocRR{}.
			
			\paragraph{Risorse hardware}
			Per eseguire la verifica, il gruppo dovrà avere a disposizione dei computer con un'adeguata potenza di calcolo in grado di sopportare il carico di lavoro.
			
			\paragraph{Risorse software}
			Le risorse software necessarie alla verifica sono gli strumenti software che eseguono controlli sui documenti e verificano che non violino le \NPdocRR{}.
			Gli strumenti software devono avere le seguenti caratteristiche:
			\begin{itemize}
				\item rilevare eventuali errori ortografici;
				\item costruire e visualizzare in tempo reale il documento scritto in \LaTeX (in modo che sia facile accorgersi di errori nell’utilizzo dei comandi).
			\end{itemize}
			Inoltre è necessario disporre di una piattaforma che raccolga i vari errori incontrati e li segnali ai componenti del gruppo che dovranno occuparsene.
			
		\subsubsection{Risorse disponibili}
		
			\paragraph{Risorse umane}
			
			\paragraph{Risorse hardware}
			
			\paragraph{Risorse software}

\end{document}