\documentclass[PianoDiQualifica.tex]{subfiles}

\begin{document}

\hypertarget{ISOIEC}{\section{Standard ISO/IEC 9126}}
Lo Standard ISO/IEC 9126 si suddivide in quattro parti:
\begin{itemize}
\item modello della qualità del software (9126-1);
\item metriche per la qualità esterna (9126-2);
\item metriche per la qualità interna (9126-3);
\item metriche per la qualità in uso (9126-4).
\end{itemize}

Lo standard tratta la qualità del software da tre punti di vista:
\begin{itemize}
\item \textbf{qualità interna:}  è la qualità del software non eseguibile e
fa quindi riferimento alle caratteristiche implementative del software
quali l’architettura e il codice sorgente;
\item \textbf{qualità esterna:} è la qualità del software nel
momento in cui esso viene eseguito e testato in un ambiente di prova;
\item \textbf{qualità in uso:}è la qualità del software dal punto di vista dell’utente che ne fa uso all’interno di uno specifico sistema e contesto.
\end{itemize}
\subsection{Modello della qualità del software}
Nella prima parte (9126-1) vengono descritti i modelli per la qualità esterna, interna
ed in uso.

\subsubsection{Modello della qualità esterna ed interna}

Il modello di qualità esterna ed interna stabilito nella prima parte dello
standard è classificato in sei caratteristiche generali:

\begin{itemize}
\item \textbf{funzionalità:} è la capacità di un software di fornire funzioni in grado di soddisfare specifici bisogni in un determinato contesto;
\item \textbf{affidabilità:} è la capacità di un software di mantenere un certo livello di prestazioni in condizioni specifiche di utilizzo in un intervallo di tempo fissato;
\item \textbf{usabilità:} è la cpacità di un software di essere capito e utilizzato da utenti di un determinato contesto;
\item \textbf{efficienza:} è la capacità di un software di fornire un certo livello di prestazioni con il minor numero di risorse possibile;
\item \textbf{manutenibilità:} è la capacità del software di essere modificato, includendo correzioni, miglioramenti o adattamenti;
\item \textbf{portabilità:} è la capacità del software di essere trasportato da un ambiente di lavoro ad un altro.
\end{itemize}

Tali caratteristiche sono misurabili attraverso delle metriche.

\subsubsection{Modello della qualità in uso}
Gli \gl{attributi} presenti nel modello relativo alla qualità del software in uso
sono classificati in quattro categorie:

\begin{itemize}
\item \textbf{efficacia:} è la capacità del software di permettere agli utenti di raggiungere gli obiettivi specificati con accuratezza e completezza;
\item \textbf{produttività:} è la capacità di permettere all’utente di utilizzare un numero di risorse in relazione all’efficienza raggiunta in uno specifico contesto di utilizzo;
\item \textbf{sicurezza:} è la capacità del software di raggiungere un livello accettabile di rischi  per i dati, le persone, il business, la proprietà o gli ambienti in uno specifico contesto di utilizzo;
\item \textbf{soddisfazione:} è la capacità del software di soddisfare gli utenti in uno specifico contesto di utilizzo.
\end{itemize}

\subsection{Metriche per la qualità del software}
Nelle restanti tre parti (9126-2,9126-3,9126-4) vengono trattate le metriche per la qualità esterna,interna e in uso.

\subsubsection{Metriche per la qualità esterna}
Le metriche esterne misurano i comportamenti del software rilevabili dai test, dall’operatività e dall’osservazione durante la sua esecuzione.
L’esecuzione del software è fatta in un contesto tecnico rilevante.
Le metriche esterne sono scelte in base alle caratteristiche che il \gl{prodotto}
finale dovrà dimostrare durante la sua esecuzione in esercizio.

\subsubsection{Metriche per la qualità interna}
Le metriche interne si applicano al software non eseguibile (come il codice sorgente) e alla documentazione. Le misure effettuate permettono di prevedere il livello di qualità esterna ed in uso del \gl{prodotto} finale poiché gli \gl{attributi} interni influenzano le caratteristiche esterne e quelle in uso.

\subsubsection{Metriche per la qualità in uso}
Le metriche della qualità in uso rappresentano il punto di vista dell'utente e misurano il grado con cui il software permette agli utenti di svolgere le proprie attività con efficacia, produttività, sicurezza e soddisfazione nel contesto operativo previsto.

\end{document}