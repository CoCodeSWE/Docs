% !TeX spellcheck = <none>
\documentclass[a4paper,titlepage]{article}

\makeatletter
\def\input@path{{../../../template/}{./img}}
\makeatother

\usepackage{array}
\usepackage{pifont}
\usepackage{slashbox}
\usepackage{graphicx}
\usepackage[export]{adjustbox}
\usepackage{Comandi}
\usepackage{Riferimenti}
\usepackage{Stile}
\usepackage{subfiles}
\usepackage{tabulary}
\usepackage{tabularx}
\usepackage{booktabs}
\usepackage{eurosym}

\newcolumntype{U}{>{\centering\arraybackslash}m{2cm}}
\newcolumntype{P}{>{\centering\arraybackslash}m{3cm}}
\newcolumntype{Z}{>{\centering\arraybackslash}m{4.2cm}}
\newcommand{\cmark}{\ding{51}}%
\newcommand{\xmark}{\ding{55}}%

\def\NOME{Analisi \gl{SDK} dei principali A.V.}
\def\VERSIONE{1.0.0}
\def\DATA{2017-01-07}
\def\REDATTORE{Luca Bertolini \\ & Simeone Pizzi \\ & Nicola Tintorri}
\def\VERIFICATORE{Mattia Bottaro}
\def\RESPONSABILE{Pier Paolo Tricomi}
\def\USO{Esterno}
\def\DESTINATARI{\COMMITTENTE \\ & \CARDIN \\ & \PROPONENTE}
\def\SOMMARIO{Documento contenente l'analisi degli \gl{SDK} dei principali assistenti virtuali relativo al \gl{prodotto} \PROGETTO{} determinato dal gruppo \GRUPPO{} nel corso della realizzazione del \gl{progetto} \PROGETTO.}

\begin{document}

\maketitle

\newpage
\tableofcontents
\newpage
\listoftables
\newpage

\begin{diario}
	\modifica{Pier Paolo Tricomi}{\AMM}{Approvazione del documento}{2017-01-06}{1.0.0}
	\modifica{Nicola Tintorri}{\AN}{Correzione errori segnalati}{2017-01-06}{0.3.1}
	\modifica{Mattia Bottaro}{\VER}{\gl{Verifica}}{2017-01-06}{0.3.0}
	\modifica{Nicola Tintorri}{\AN}{Stesura sezione Conclusioni}{2017-01-05}{0.2.3}
	\modifica{Luca Bertolini}{\RESP}{Stesura sezione Confronto tra i diversi assistenti virtuali}{2017-01-05}{0.2.2}
	\modifica{Simeone Pizzi}{\AN}{Correzione errori segnalati}{2017-01-05}{0.2.1}
	\modifica{Mattia Bottaro}{\VER}{\gl{Verifica}}{2017-01-04}{0.2.0}
	\modifica{Nicola Tintorri}{\AN}{Stesura sezione Siri}{2017-01-04}{0.1.2}
	\modifica{Simeone Pizzi}{\AN}{Correzione errori segnalati e stesura sezione Alexa}{2017-01-04}{0.1.1}
	\modifica{Mattia Bottaro}{\VER}{\gl{Verifica}}{2017-01-03}{0.1.0}
	\modifica{Luca Bertolini}{\RESP}{Stesura sezioni wit.ai e \gl{API}.AI}{2017-01-03}{0.0.2}
	\modifica{Luca Bertolini}{\RESP}{Stesura struttura documento}{2017-01-02}{0.0.1}
\end{diario}
\newpage

\section{Introduzione}
	\subsection{Scopo del documento}
Questo documento riporta un'analisi eseguita dal gruppo \GRUPPO{} sui principali \gl{SDK} di assistenti virtuali presenti sul mercato. L'analisi in questione ha lo scopo di mettere in mostra i punti di forza e le debolezze degli \gl{SDK} presi in considerazione.
	\subsection{Scopo del \gl{prodotto}}
	\SCOPO{}
	\subsection{Glossario}
	\GLOSSARIO{}
	\subsection{Riferimenti}
		\subsubsection{Normativi}
			\begin{itemize}
				\item \gl{Capitolato} d'appalto C2 - AtAVi: Accoglienza tramite Assistente Virtuale \\	\url{http://www.math.unipd.it/~tullio/IS-1/2016/Progetto/C2.pdf};
				\item \NPdoc.
			\end{itemize}
		\subsubsection{Informativi}
			\begin{itemize}
				\item wit.ai: \\
				\url{https://wit.ai}
				\item \gl{API}.AI: \\
				\url{https://api.ai}
				\item Alexa: \\
				\url{https://developer.amazon.com/alexa-skills-kit} \\
				\url{https://developer.amazon.com/alexa-voice-service}
				\item Siri: \\
				\url{https://developer.apple.com/library/content/documentation/Intents/Conceptual/SiriIntegrationGuide/index.html}
			\end{itemize}
\newpage
\section{wit.ai}
	\subsection{Descrizione}
		\begin{minipage}{0.7\textwidth}\raggedright
			wit.ai è una società nata nell'ottobre del 2013 e acquisita da Facebook Inc. nel 2015. \\
			L'obiettivo di wit.ai è quello di semplificare la creazione di applicazioni che prevedono interazioni testuali o vocali; per farlo viene messa a disposizione degli sviluppatori una piattaforma di linguaggio naturale aperta ed estensibile che ha la peculiarità di apprendere tramite ogni interazione eseguita.
		\end{minipage}
		\hfill
		\noindent\begin{minipage}{0.15\textwidth}
		\includegraphics[scale=0.6]{images/witai.jpg}
		\end{minipage}
		\subsection{Caratteristiche}

			wit.ai mette a disposizione un \gl{SDK} gratuito ed \gl{open source} per il riconoscimento del linguaggio naturale. Questa piattaforma è caratterizzata dall'utilizzo di \textit{Context}, \textit{Intent} ed \textit{Entity} che sono dei costrutti messi a disposizione per tradurre le richieste vocali dell'utente in dati processabili. In particolare il \textit{Context} si utilizza per monitorare lo stato della conversazione tra l'utente e wit.ai.
			La piattaforma non del tutto stabile (crash, freeze temporanei, ecc..) e l'assenza di moduli di integrazione diretti con \gl{API} esterne rendono lo sviluppo su wit.ai non ottimale.

In questa sezione si vogliono elencare le principali caratteristiche del \gl{SDK} in questione:
\begin{itemize}
	\item ambiente \gl{open source} e totalmente gratuito per progetti sia pubblici che privati;
	\item programmatori agevolati a sviluppare grazie ai costrutti \textit{Intent} ed \textit{Entity}; 
	\item compatibilità con molti sistemi operativi, tra cui: \gl{iOS}, \gl{Android}, Windows Phone, Raspberry Pi, Python e C;
	\item \textbf{Context}: è un "object" che lo sviluppatore utilizzerà per monitorare lo stato della conversazione tra l'utente e Wit.ai. La sua principale funzione è quella di trasformare il linguaggio naturale in dati processabili da Wit;	
\end{itemize}	

		
\newpage
\section{API.AI}
	\subsection{Descrizione}
	\begin{minipage}{0.7\textwidth}\raggedright
		\gl{API}.AI è una società nata nell'ottobre del 2010 e acquisita da Google Inc. nel 2016.
		\gl{API}.AI è una piattaforma di conversazione che permette interazioni sofisticate con il linguaggio naturale.
	\end{minipage}
	\hfill
	\noindent\begin{minipage}{0.1\textwidth}
		\includegraphics[scale=0.15]{images/apiai.png}
	\end{minipage}
	\subsection{Caratteristiche}
	\gl{API}.AI fornisce \gl{SDK} per i principali liguaggi di programmazione tra i quali C++, C\#, \gl{Java}, \gl{Node.js}, Javascript e Phyton. Inoltre può essere integrato con Amazon Echo e Microsoft Cortana.\\
	Le applicazioni sviluppate su questa piattaforma sono costuite da \textit{Agent}, i quali si occupano di trasformare il linguaggio naturale in dati processabili.
	Tali \textit{Agent} sono a loro volta costituiti da \textit{Intent}, che hanno il compito di associare la richiesta dell'utente ad una determinata azione del software, ed \textit{Entity}, che sono strumenti per estrarre dal linguaggio naturale i parametri attesi.\\
	
\newpage
	\section{Alexa}
	\subsection{Descrizione}
	\begin{minipage}{0.7\textwidth}\raggedright
		Alexa è stata annunciata per la prima volta nel novembre del 2014 come assistente virtuale dello \textit{smart speaker} Echo, sviluppato da Amazon.com.
		Le funzionalità di Alexa sono ispirate dal computer di bordo delle navicelle interstellari presenti nelle serie TV sci-fi "Star Trek".

	\end{minipage}
	\hfill
	\noindent\begin{minipage}{0.1\textwidth}
		\includegraphics[scale=0.3]{images/alexa.png}
	\end{minipage}
	\subsection{Caratteristiche}
		Amazon.com mette a disposizione gratuitamente Alexa Voice Service (AVS), un servizio che permette di integrare, in qualsiasi dispositivo e applicazione con accesso al web, l'assistente virtuale Alexa. La compagnia inoltre fornisce l'\textit{Alexa Skills Kit} (ASK), un \gl{SDK} che permette di estendere le funzionalità dell'assistente virtuale. Per creare una \textit{Custom Skill} ASK richiede:
		\begin{itemize}
			\item \textbf{Insieme di intenti}: rappresentano le azioni che l' utente può compiere con quella determinata \textit{Skill}; 
			\item \textbf{Insieme di affermazioni d'esempio}: specificano le parole e le frasi che l'utente può pronunciare per invocare gli intenti;
			\item \textbf{Nome della \textit{Skill}}: identifica la \textit{Skill} e viene pronunciato dall'utente per interagire con essa;
			\item \textbf{Servizio \textit{\gl{cloud}-based}}: accetta gli intenti come richieste strutturate ed agisce in base ad essi;
			\item \textbf{Configurazione}: combina gli elementi precedentemente descritti in modo che Alexa possa indirizzare le richieste al servizio creato per la \textit{Skill}.
			
		\end{itemize}
	Amazon.com suggerisce di usare \textit{AWS Lambda} per la creazione di servizi \textit{\gl{cloud}-based}.
		
	
\newpage	
	\section{Siri}
	\subsection{Descrizione}
	
	\begin{minipage}{0.7\textwidth}\raggedright
		Siri nasce come applicazione indipendente per \gl{iOS} resa disponibile tramite il canale commerciale App Store, per poi essere acquisita da \gl{Apple} Inc. nel 2011. È presente nei dispositivi \gl{iOS}, \gl{macOS}, \gl{watchOS} e \gl{tvOS} come assistente virtuale.
	\end{minipage}
	\hfill
	\noindent\begin{minipage}{0.1\textwidth}
		\includegraphics[scale=0.3]{images/siri.jpg}
	\end{minipage}
	\subsection{Caratteristiche}
		Essendo Siri parte integrante di \gl{iOS}, il suo utilizzo è vincolato alla creazione di un'applicazione per tale sistema operativo. Inoltre questa applicazione deve appartenere ad uno degli ambiti prestabiliti da SiriKit i quali sono:
			\begin{itemize}
				\item chiamate \gl{VoIP};
				\item messaggistica;
				\item pagamenti;
				\item foto;
				\item allenamento;
				\item trasporti;
				\item prenotazioni ristoranti;
				\item CarPlay.
			\end{itemize}
		L'assistente virtuale di casa \gl{Apple} processa per intero ogni frase dell'utente e ne ricava le intenzioni. Tali intenzioni, al fine di essere rappresentate in dati processabili, sono tradotte in \textit{Intents}. Una volta acquisiti i dati e aver creato l'\textit{Intent}, nel caso questi non risultino validi, Siri riformula le domande per gli attributi interessati.
\newpage
	\section{Confronto tra i diversi assistenti virtuali}
	
	\begin{table}[h]
		\centering
		\begin{tabulary}{\textwidth}{|Z||U|P|U|U|}
		\hline
		\backslashbox{\textbf{Caratteristica}}{\textbf{A.V.}} & \textbf{wit.ai} & \textbf{\gl{API}.AI} & \textbf{Siri} & \textbf{Alexa} \\
		\hline 
		\hline 
		\rule[-0.2cm]{0cm}{0.6cm} Risveglio vocale & \xmark & \xmark & obbligatorio & \cmark \\
		\hline 
		\rule[-0.2cm]{0cm}{0.6cm}Creazione \textit{Intent} con NLU & \cmark & \cmark & \cmark & \cmark \\
		\hline 
		\rule[-0.2cm]{0cm}{0.6cm}\textit{Intents} personalizzabili & \cmark & \cmark & \xmark & \cmark \\
		\hline 
		\rule[-0.2cm]{0cm}{0.6cm}Entità predefinite & \cmark & \cmark & \cmark & \cmark \\
		\hline 
		\rule[-0.2cm]{0cm}{0.6cm}Ambiti predefiniti conversazione & forniti dalla community & 35+ & 8 & \xmark \\
		\hline 
		Costi & gratis & gratis senza ambiti predefiniti & gratis & \rule[-0.2cm]{0cm}{0.6cm} gratis fino a 1.000.000 di chiamate (per AWS Lambda)\\
		\hline 
		Piattaforme e linguaggi  di prog. supportati & \textit{\gl{Node.js}; Python; Ruby; \gl{HTTP}} & \rule[-0.2cm]{0cm}{0.6cm} \textit{\gl{Android}; \gl{iOS}; \gl{Apple} Watch; \gl{Node.js}; Cordova; Unity; C\#; Xamarin; Windows Phone; Python; \gl{JavaScript}; PHP; Botkit; C++; Mac OS X; HTML; Ruby} & \textit{\gl{iOS}} & qualsiasi linguaggio permetta di accettare richieste \textit{HTTPS}\\
		\hline 
		\rule[-0.2cm]{0cm}{0.6cm}Documentazione \gl{API} esaustiva & \cmark & \cmark & \cmark & \cmark	\\	
		\hline
		\rule[-0.2cm]{0cm}{0.6cm}Gui modificazione \textit{Intents} & \cmark & \cmark & \xmark & \cmark \\
		\hline 
		Autenticazione & OAuth2 & double token & \xmark & \rule[-0.2cm]{0cm}{0.6cm}LWA(\textit{Login with Amazon}) \\
		\hline
		\rule[-0.2cm]{0cm}{0.6cm}Numero di lingue supportate & 15 & 50 & 21 & 2 \\
		\hline
	\end{tabulary}
	\caption{Confronto delle caratteristiche tra i diversi assistenti virtuali}
	\end{table}
\newpage
	\section{Conclusioni}
	Dall'attenta analisi del gruppo \GRUPPO{} è emerso che \textit{SiriKit} non è adeguato a soddisfare le esigenze del \gl{progetto} \PROGETTO{} in quanto gli ambiti forniti sono limitati, le interazioni con l'utente non sono personalizzabili ed il supporto a piattaforme diverse da \gl{iOS} è assente.\\
	I restanti \gl{SDK} non presentano tali limitazioni e quinidi potrebbero essere tutti potenzialmente sfruttati. La scelta del gruppo è ricaduta su \gl{API}.AI per la maggiore stabilità rispetto a wit.ai, per il  maggior numero di lingue disponibili in vista di un supporto futuro ad interazioni con utenti di diversa nazionalità, per la maggiore maturità della piattaforma e per la possibilità di esportare gli \textit{Agents} in formati compatibili con i principali assistenti virtuali presenti sul mercato, tra cui Cortana ed Alexa.
	
	
	\end{document}

