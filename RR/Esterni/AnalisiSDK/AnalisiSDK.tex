% !TeX spellcheck = <none>
\documentclass[a4paper,titlepage]{article}

\makeatletter
\def\input@path{{../../../template/}{./img}}
\makeatother

\usepackage{array}
\usepackage{pifont}
\usepackage{slashbox}
\usepackage{graphicx}
\usepackage[export]{adjustbox}
\usepackage{Comandi}
\usepackage{Riferimenti}
\usepackage{Stile}
\usepackage{subfiles}
\usepackage{tabulary}
\usepackage{tabularx}
\usepackage{booktabs}
\usepackage{eurosym}

\newcolumntype{U}{>{\centering\arraybackslash}m{2cm}}
\newcolumntype{V}{>{\centering\arraybackslash}m{3cm}}
\newcolumntype{Z}{>{\centering\arraybackslash}m{4.2cm}}
\newcommand{\cmark}{\ding{51}}%
\newcommand{\xmark}{\ding{55}}%

\def\NOME{Analisi SDK dei principali A.V.}
\def\VERSIONE{1.0}
\def\DATA{2016-12-28}
\def\REDATTORE{Luca Bertolini \\ & Simeone Pizzi \\ & Nicola Tintorri}
\def\VERIFICATORE{Mattia Bottaro}
\def\RESPONSABILE{Simeone Pizzi}
\def\USO{Esterno}
\def\DESTINATARI{\COMMITTENTE \\ & \CARDIN \\ & \PROPONENTE}
\def\SOMMARIO{Documento contenente l'analisi degli SDK dei principali assistenti virtuali relativo al \gl{prodotto} \PROGETTO{} determinato dal gruppo \GRUPPO{} nel corso della realizzazione del \gl{progetto} \PROGETTO.}

\begin{document}

\maketitle

\newpage
\tableofcontents
\newpage


\section{Introduzione}
	\subsection{Scopo del documento}
Questo documento riporta un analisi eseguita dal gruppo \GRUPPO{} sui principali SDK di assistenti virtuali presenti nel mercato. L'analisi in questione ha lo scopo di mettere in mostra i punti di forza e le debolezze degli SDK presi in considerazione.
	\subsection{Glossario}
	\GLOSSARIO
	\subsection{Riferimenti}
		\subsubsection{Normativi}
\begin{itemize}
\item \gl{Capitolato} d'appalto C2 - AtAVi: Accoglienza tramite Assistente Virtuale \\	\url{http://www.math.unipd.it/~tullio/IS-1/2016/Progetto/C2.pdf};
\item \NPdoc.
\end{itemize}
\newpage
\section{wit.ai}
	\subsection{Descrizione}
		\begin{minipage}{0.7\textwidth}\raggedright
			wit.ai è una società nata nell'ottobre del 2013 e acquisita da Facebook Inc. nel 2015. \\
			L'obiettivo di wit.ai è quello di semplificare la creazione di applicazioni che prevedono interazioni testuali o vocali; per farlo viene messa a disposizione degli sviluppatori una piattaforma di linguaggio naturale aperta ed estensibile che ha la peculiarità di apprendere tramite ogni interazione eseguita.
		\end{minipage}
		\hfill
		\noindent\begin{minipage}{0.15\textwidth}
		\includegraphics[scale=0.6]{images/witai.jpg}
		\end{minipage}
		\subsection{Caratteristiche}
			wit.ai mette a disposizione un SDK gratuito ed open source per il riconoscimento del linguaggio naturale. Questa piattaforma è caratterizzata dall'utilizzo di \textit{Context}, \textit{Intent} ed \textit{Entity} che sono dei costrutti messi a disposizione per tradurre le richieste vocali dell'utente in dati processabili. In particolare il \textit{Context} si utilizza per monitorare lo stato della conversazione tra l'utente e wit.ai.
			La piattaforma non del tutto stabile (crash, freeze temporanei, ecc..) e l'assenza di moduli di integrazione diretti con API esterne rendono lo sviluppo su wit.ai non ottimale.
		
\newpage
\section{api.ai}
	\subsection{Descrizione}
	\begin{minipage}{0.7\textwidth}\raggedright
		api.ai è una società nata nell'ottobre del 2010 e acquisita da Google Inc. nel 2016.
		api.ai è una piattaforma di conversazione che permette interazioni sofisticate con il linguaggio naturale.
	\end{minipage}
	\hfill
	\noindent\begin{minipage}{0.1\textwidth}
		\includegraphics[scale=0.15]{images/apiai.png}
	\end{minipage}
	\subsection{Caratteristiche}
	api.ai fornisce SDK per i principali liguaggi di programmazione tra i quali C++, C\#, Java, Node.js, Javascript e Phyton. Inoltre può essere integrato con Amazon Echo e Microsoft Cortana.\\
	Le applicazioni sviluppate su questa piattaforma sono costuite da \textit{Agent}, i quali si occupano di trasformare in dati processabili il linguaggio naturale.
	Tali \textit{Agent} sono a loro volta costituiti da \textit{Intent}, che hanno il compito di associare la richiesta dell'utente ad una determinata azione del software, ed \textit{Entity}, che sono strumenti per estrarre dal linguaggio naturale i parametri attesi.\\
	
\newpage
	\section{Alexa}
	\subsection{Descrizione}
	\begin{minipage}{0.7\textwidth}\raggedright
		Alexa è stata annunciata per la prima volta nel novembre del 2014 come assistente virtuale dello \textit{smart speaker} Echo, sviluppato da Amazon.com.
		Le funzionalità di Alexa sono ispirate dal computer di bordo delle navicelle interstellari presenti nelle serie TV sci-fi "Star Trek".

	\end{minipage}
	\hfill
	\noindent\begin{minipage}{0.1\textwidth}
		\includegraphics[scale=0.3]{images/alexa.png}
	\end{minipage}
	\subsection{Caratteristiche}
		Amazon.com mette a disposizione gratuitamente Alexa Voice Service (AVS), un servizio che permette di integrare, in qualsiasi dispositivo e applicazione con accesso al web, l'assistente virtuale Alexa. La compagnia inoltre fornisce l'\textit{Alexa Skills Kit} (ASK), un SDK che permette di estendere le funzionalità dell'assistente virtuale. Per creare una \textit{Custom Skill} ASK richiede:
		\begin{itemize}
			\item \textbf{Insieme di intenti}: rappresentano le azioni che l' utente può compiere con quella determinata \textit{Skill}; 
			\item \textbf{Insieme di affermazioni d'esempio}: specificano le parole e le frasi che l'utente può pronunciare per invocare gli intenti;
			\item \textbf{Nome della \textit{Skill}}: identifica la \textit{Skill} e viene pronunciato dall'utente per interagire con essa;
			\item \textbf{Servizio \textit{cloud-based}}: accetta gli intenti come richieste strutturate ed agisce in base ad essi;
			\item \textbf{Configurazione}: combina gli elementi precedentemente descritti in modo che Alexa possa indirizzare le richieste al servizio creato per la \textit{Skill}.
			
		\end{itemize}
	Amazon.com suggerisce di usare \textit{AWS Lamda} per la creazione di servizi \textit{cloud-based}.
		
	
\newpage	
	\section{Siri}
	\subsection{Descrizione}
	
	\begin{minipage}{0.7\textwidth}\raggedright
		Siri nasce come applicazione indipendente per iOS resa disponibile tramite il canale commerciale App Store, per poi essere acquisita da Apple Inc. nel 2011. È presente nei dispositivi iOS, macOS, watchOS e tvOS come assistente virtuale.
	\end{minipage}
	\hfill
	\noindent\begin{minipage}{0.1\textwidth}
		\includegraphics[scale=0.3]{images/siri.jpg}
	\end{minipage}
	\subsection{Caratteristiche}
		Essendo Siri parte integrante di iOS, il suo utilizzo è vincolato alla creazione di un'applicazione per tale sistema operativo. Inoltre questa applicazione deve appartenere ad uno dei ambiti prestabiliti da SiriKit i quali sono:
			\begin{itemize}
				\item chiamate VoIP;
				\item messaggistica;
				\item pagamenti;
				\item foto;
				\item allenamento;
				\item trasporti;
				\item prenotazioni ristoranti;
				\item CarPlay.
			\end{itemize}
		L'assistente virtuale di casa Apple processa per intero ogni frase dell'utente e ne ricava le intenzioni. Tali intenzioni, al fine di essere rappresentate in dati processabili, sono tradotte in \textit{Intents}. Una volta acquisiti i dati e aver creato l'\textit{Intent}, nel caso questi non risultino validi, Siri riformula le domande per gli attributi interessati.
\newpage
	\section{Confronto tra i diversi assistenti virtuali}
	
	\begin{center}
		\begin{tabulary}{\textwidth}{|Z||U|V|U|U|}
		\hline
		\backslashbox{\textbf{Caratteristica}}{\textbf{A.V.}} & \textbf{wit.ai} & \textbf{api.ai} & \textbf{Siri} & \textbf{Alexa} \\
		\hline 
		\hline 
		\rule[-0.2cm]{0cm}{0.6cm} Risveglio vocale & \xmark & \xmark & obbligatorio & \cmark \\
		\hline 
		\rule[-0.2cm]{0cm}{0.6cm}Creazione \textit{Intent} con NLU & \cmark & \cmark & \cmark & \cmark \\
		\hline 
		\rule[-0.2cm]{0cm}{0.6cm}\textit{Intents} personalizzabili & \cmark & \cmark & \xmark & \cmark \\
		\hline 
		\rule[-0.2cm]{0cm}{0.6cm}Entità predefinite & \cmark & \cmark & \cmark & \cmark \\
		\hline 
		\rule[-0.2cm]{0cm}{0.6cm}Ambiti predefiniti conversazione & forniti dalla community & 35+ & 8 & \xmark \\
		\hline 
		Costi & gratis & gratis senza ambiti predefiniti & gratis & \rule[-0.2cm]{0cm}{0.6cm} gratis fino a 1.000.000 di chiamate\\
		\hline 
		Piattaforme e linguaggi  di prog. supportati & \textit{Node.js; Python; Ruby; HTTP} & \rule[-0.2cm]{0cm}{0.6cm} \textit{Android; iOS; Apple Watch; Node.js; Cordova; Unity; C\#; Xamarin; Windows Phone; Python; JavaScript; PHP; Botkit; C++; Mac OS X; HTML; Ruby} & \textit{iOS} & qualsiasi linguaggio permetta di accettare richieste \textit{HTTPS}\\
		\hline 
		\rule[-0.2cm]{0cm}{0.6cm}Documentazione API esaustiva & \cmark & \cmark & \cmark & \cmark	\\	
		\hline
		\rule[-0.2cm]{0cm}{0.6cm}Gui modificazione \textit{Intents} & \cmark & \cmark & \xmark & \cmark \\
		\hline 
		Autenticazione & OAuth2 & double token & \xmark & \rule[-0.2cm]{0cm}{0.6cm}LWA(\textit{Login with Amazon}) \\
		\hline
		\rule[-0.2cm]{0cm}{0.6cm}Numero di lingue supportate & 15 & 50 & 21 & 2 \\
		\hline
	\end{tabulary}
	\end{center}
\newpage
	\section{Conclusioni}
	Dall'attenta analisi del gruppo \GRUPPO{} è emerso che \textit{SiriKit} non è adeguato a soddisfare le esigenze del progetto \PROGETTO{} in quanto gli ambiti forniti sono limitati, le interazioni con l'utente non sono personalizzabili ed il supporto a piattaforme diverse da iOS è assente.\\
	I restanti SDK non presentano tali limitazioni e quinidi potrebbero essere tutti potenzialmente sfruttati. La scelta del gruppo è ricaduta su api.ai per la maggiore stabilità rispetto a wit.ai, per il  maggior numero di lingue disponibili in vista di un supporto futuro ad interazioni con utenti di diversa nazionalità, per la maggiore maturità della piattaforma e per la possibilità di esportare gli \textit{Agents} in formati compatibili con i principali assistenti virtuali presenti sul mercato, tra cui Cortana ed Alexa.
	
	
	\end{document}

