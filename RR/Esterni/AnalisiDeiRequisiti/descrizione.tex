\subsection{Obiettivo del prodotto}
Il \gl{prodotto} vuole essere un assistente virtuale in grado di accogliere un cliente in visita all'ufficio di \PROPONENTE{} e, allo stesso tempo, informare la persona desiderata dell'arrivo dell'ospite. Il \gl{software} dovrà avvalersi di Alexa, assistente virtuale sviluppato da Amazon.
\subsection{Funzioni del prodotto}
Il \gl{prodotto}, dopo aver dato il benvenuto all'ospite, dovrà richiedergli alcune informazioni riguardanti la sua visita all'ufficio di \PROPONENTE. 
I dati di interesse riguardanti l'incontro, che l'assistente ha il compito di raccogliere tramite delle domande mirate, sono: l'identità del visitatore, dell'eventuale azienda di provenienza e della persona desiderata. Inoltre l'assistente virtuale dovrà intrattenere l'ospite fino all'arrivo di un componente del team di \PROPONENTE{}. Il fornitore è libero di scegliere le modalità di intrattenimento dell'ospite.  
\subsection{Caratteristiche degli utenti}
Non sono richieste competenze particolari per poter utilizzare questo \gl{prodotto}, che deve risultare
quindi accessibile ad un'ampia categoria di utenti. Questo sarà garantito dal fatto che l'interazione con l'assistente sarà quasi completamente di carattere vocale.
\subsection{Vincoli generali}
Essendo un applicativo Web dovrà funzionare correttamente su PC, Mac o tablet, senza alcuna limitazione sul \gl{sistema} operativo.\\
Il \gl{browser} che verrà utilizzato dovrà essere compatibile con \gl{JavaScript} e gli standard \gl{HTML5}, \gl{CSS3}.