\documentclass[a4paper,titlepage]{article}

\makeatletter
\def\input@path{{../../../template/}{./img}}
\makeatother

\usepackage{Comandi}
\usepackage{Riferimenti}
\usepackage{Stile}

\def\NOME{Verbale 2016-12-10}
\def\VERSIONE{1.0.0}
\def\DATA{2016-12-10}
\def\REDATTORE{Mattia Bottaro}
\def\VERIFICATORE{Luca Bertolini}
\def\RESPONSABILE{Simeone Pizzi}
\def\USO{Interno}
\def\DESTINATARI{\COMMITTENTE \\ & \CARDIN \\ & \GRUPPO} % Se esterno va anche il \gl{proponente}
\def\SOMMARIO{Verbale della riunione in data 2016-12-10.}


\begin{document}

\maketitle
\begin{diario}
    \modifica{Mattia Bottaro}{\AMM}{Stesura documento e verbalizzazione della riunione}{2016-12-10}{0.0.1}
\end{diario}
\newpage
\tableofcontents

\newpage
\section{Informazioni generali}
\label{sec:Informazioni}

\begin{itemize}
  \item \textbf{Luogo}: Google Hangout.
  \item \textbf{Data}: 2016-12-10.
  \item \textbf{Orario di inizio}: 9:00.
  \item \textbf{Orario di fine}: 12:30.
  \item \textbf{Durata}: 3h 30m.
  \item \textbf{Oggetto}: scelta di nome, logo, \gl{capitolato}, ruoli, strumenti, prime direttive. Chiarimenti sul \gl{prodotto} da realizzare. 
  \item \textbf{Partecipanti}: Andrea Magnan, Luca Bertolini, Mattia Bottaro, Mauro Carlin, Nicola Tintorri, Pier Paolo Tricomi, Simeone Pizzi.
  \item \textbf{Segretario}: Mattia Bottaro.
  
\end{itemize}
\subsection{Segnalazioni}
\begin{itemize}
 \item Mauro Carlin lascia la riunione alle ore 11:50.
\end{itemize}
\section{Riassunto della riunione}
\label{sec:RiassuntoRiunione}
 \subsection{Descrizione}
 Nel corso della riunione interna sono state affrontate le seguenti tematiche: nome del gruppo, logo, scelta \gl{capitolato}, assegnazione dei ruoli, rotazione dei ruoli, strumenti da utilizzare, prime direttive sul lavoro da fare, dubbi su requisiti e specifiche del \gl{software}.
 \subsection{Decisioni}
 \begin{itemize}
  \item DI1.1 - Scelta del nome: proposto il nome " \GRUPPO{} " da Luca Bertolini e approvato all'unanimità. Quest'ultimo si è incaricato di realizzare il logo.
  \item DI1.2 - Scelta \gl{capitolato}: dopo una votazione è stato scelto il \gl{capitolato} C2 (\gl{progetto} \PROGETTO) proposto dall'azienda \PROPONENTE.
  \item DI1.3 - Assegnazione dei ruoli: sono stati assegnati ai membri i ruoli da ricoprire nella fase di partenza:
  \begin{itemize}
  \item \RESP: Simeone Pizzi;
  \item \AMMP: Mauro Carlin, Mattia Bottaro;
  \item \VERP: Luca Bertolini, Mattia Bottaro, Andrea Magnan, Simeone Pizzi;
  \item \ANP: Andrea Magnan, Nicola Tintorri, Pier Paolo Tricomi.
  \end{itemize}
  \item DI1.4 - Strumenti: sono stati decisi solo alcuni strumenti da utilizzare. Per la stesura e formattazione dei documenti si utilizzerà il linguaggio \LaTeX{}{} unito all'editor \gl{Texmaker}. Per il versionamento si utilizzerà \gl{Git} unito al servizio di host \gl{GitHub}. 
  \item DI1.5 - Prime direttive: in base ai ruoli prima decisi, sono stati assegnati i primi compiti quali: formazione di un \gl{template} \LaTeX{}{}, stesura struttura documenti, primi contenuti.
 \end{itemize}

\subsection{Tematiche in sospeso} 
  \begin{itemize}
  \item SI1.1 - Rotazione dei ruoli: l'effettiva rotazione dei ruoli verrà presa in esame durante la stesura del \PPdoc.
  \item SI1.2 - Strumenti: sono stati stabiliti solo alcuni strumenti precedenteme elencati. Quelli mancanti necessari verrano ricercati e studiati da ogni membro del team, per poi essere proposti alla prossima riunione.
  \item SI1.3 - Requisiti e specifiche del \gl{software}: non tutti i requisiti e le specifiche necessarie al \gl{software} sono colte dal team. È necessario chiarire i dubbi alla prima riunione con il \gl{proponente}.
  \end{itemize}
\end{document}
