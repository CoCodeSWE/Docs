\section {Capitolato C3}
	\subsection {Descrizione}
		Sono molti i rischi corsi da un'azienda durante la sua attività, tra i quali, possibili catastrofi naturali. Lo scopo del progetto è quello di ideare un'applicazione web che disegni gli scenari di
		danno che possono colpire un'azienda.
	\subsection {Dominio applicativo}
		L'applicazione è rivolta a tutte le aziende, infatti è interesse di tutte poter conoscere anticipatamente quelli che potrebbero essere i possibili rischi da loro corsi.
	\subsection {Dominio tecnologico}
		Non c'è l'obbligo di usare un particolare stack tecnologico, vengono però suggerite le seguenti tecnologie:
		\begin{itemize}
			\item Slack: per la comunicazione;
			\item Asana: per la gestione dei processi;
			\item Amazon Web Service: per l'archiviazione dei dati;
			\item Bootstrap e JavaScript: per la realizzazione dell'applicazione web.
		\end{itemize}
	\subsection {Valutazione}
		\subsubsection {Aspetti positivi}
			\begin{itemize}
				\item Il progetto nella sua complessità non sembra essere eccessivamente impegnativo;
			\end{itemize}
		\subsubsection {Fattori di rischio}
			\begin{itemize}
				\item Il gruppo ritiene il progetto poco stimolante.
				\item Difficoltà di contatto con il proponente.
			\end{itemize}
	\subsection {Conclusione}
		Tutti i membri del gruppo ritengono il progetto poco stimolante. Inoltre, poichè il progetto è molto semplice, nasce il rischio di concorrere con altri gruppi per la vincita del capitolato.