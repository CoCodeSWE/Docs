\section {Capitolato C4}
	\subsection {Descrizione}
	Il \gl{progetto} prevede la realizzazione di un'applicazione per dispositivi mobili (smartphone e tablet), che agevoli la
lettura da parte di una persona affetta da dislessia.  L'applicazione può consistere in un lettore di \gl{ebook} o in un
client di messaggistica.
	\subsection {Dominio applicativo}
	Il \gl{progetto} fornisce un servizio che ha l'obiettivo di agevolare la lettura alle persone affette da dislessia tramite l'utilizzo di dispositivi mobili. 
	\subsection {Dominio tecnologico}
Il \gl{progetto} prevede che l'applicazione utilizzi il motore di sintesi vocale \gl{Flexible and Adaptive Text To Speech} (FA-TTS), un'applicazione web che
espone le proprie funzionalità mediante interfaccia \gl{HTTP}. Viene inoltre richiesto lo sviluppo su piattaforma \gl{Android}.
	\subsection {Valutazione}
			\subsubsection {Aspetti positivi}
			Hanno attirato l'interesse del gruppo:
				\begin {itemize}
				  \item l'utilizzo del motore FA-TTS per la sintesi vocale;
				  \item la possibilità di realizzare un'applicazione per dispositivi mobili.		
				\end {itemize}
			\subsubsection {Fattori di rischio}
			I possibili fattori di rischio sono:
				\begin {itemize}
					\item nessuna competenza di utilizzo del motore di sintesi vocale FA-TTS e più in generale sulla sintesi vocale.
				\end {itemize}
	\subsection {Conclusioni}
		Nonostante la sintesi vocale sia un argomento innovativo e interessante,la mancanza di conoscenze sull'argomento da parte del gruppo viene ritenuta un problema per la sua realizzazione. Il lato mobile aveva attirato l'attenzione di alcuni membri,ma le scelte possibili sul tipo di applicazione realizzabile avrebbero portato una maggiore difficoltà nell'individuazione dei requisiti desiderati dal \gl{proponente}.