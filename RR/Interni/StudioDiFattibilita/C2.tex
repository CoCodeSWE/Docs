\section {Capitolato scelto}
	\subsection {Descrizione}
		\SCOPO
	\subsection {Dominio applicativo}
Il servizio proposto consiste nell'utilizzo di un assistente virtuale per ricevere i clienti di Zero12. Pertanto il suo dominio applicativo risulta essere l'accoglienza di clienti in un contesto aziendale.
	\subsection {Dominio tecnologico}
		Per realizzare l'oggetto del \gl{capitolato} sono necessarie le seguenti tecnologie:
		\begin {itemize}
			\item \textbf{\gl{Amazon Web Services (AWS)}};
			\item \textbf{Database \gl{NoSQL}, \gl{DynamoDB}} o \textbf{\gl{MongoDB}};
			\item \textbf{\gl{HTML5}, \gl{CSS3}} e \textbf{Javascript} (per l'interfaccia con l'utente);
			\item \textbf{\gl{Slack}} (per il \gl{sistema} di comunicazione);
			\item \textbf{\gl{Node.js}} o \textbf{\gl{Swift}} (come linguaggio di programmazione
 server-side);
 			\item \textbf{\gl{SDK} Alexa} o \textbf{Siri} (come assistente virtuale).

		\end {itemize}
	\subsection {Valutazione}
		\subsubsection {Aspetti positivi}
			 La possibilità di lavorare con le seguenti tecnologie risulta molto interessante per il gruppo:
				 \begin {itemize}
				 	\item \gl{Amazon Web Services};
				 	\item i vari \gl{SDK} degli assistenti virtuali;
				 	\item database \gl{NoSQL}.
				 \end {itemize}
		\subsubsection {Fattori di rischio}
		Eventuali fattori di rischio:
			\begin {itemize}
				\item mancanza di competenza sulle tecnologie necessarie;
				\item la scelta di una specifica tecnologia rispetto ad un'altra potrebbe portare a risultati ben diversi da quelli desiderati dal \gl{proponente}.
			\end {itemize}
	\subsection {Conclusioni}
	Il \gl{progetto} è stato definito molto interessante da parte dei membri del gruppo, che lo ritengono innovativo e un'utile base formativa. Questi fattori hanno fatto sì che la decisione da prendere per la scelta di un \gl{capitolato} si orientasse verso di esso.
