\section {Capitolato C6}
	\subsection {Descrizione}
	Il presente \gl{capitolato} ha per oggetto l’ affidamento della fornitura per la realizzazione
di un \gl{software} di costruzione di diagrammi \gl{UML} con la relativa generazione di codice
\gl{Java} e Javascript tramite tecnologie web.
	\subsection {Dominio applicativo}
	Il \gl{progetto} consiste nel formare un collegamento diretto tra \gl{UML} e codice: permetterà pertanto a chiunque sia in grado di costruire diagrammi \gl{UML} la generazione automatica di codice.
	\subsection {Dominio tecnologico}
	Sono richieste le seguenti tecnologie:
	\begin {itemize}
	\item \textbf{\gl{Java}} o \textbf{\gl{JavaScript}} (come linguaggio del codice che il \gl{software} dovrà generare);
	\item \textbf{\gl{UML}} (come standard per i diagrammi);
	\item \textbf{\gl{Tomcat}} o \textbf{\gl{Node.js}} (come linguaggio server-side);
	\item \textbf{\gl{HTML5}} e \textbf{\gl{CSS3}} (per l'interfaccia client);
	\end {itemize}
	\subsection {Valutazione}
		\subsubsection {Aspetti positivi}
		Gli aspetti considerati positivi sono:
			\begin {itemize}
			 	\item l'idea proposta e le sue specifiche sono state definite in modo chiaro ed esaustivo;
			 	\item possibilità di utilizzo di tecnologie già conosciute dal team.
			\end {itemize}
		\subsubsection {Fattori di rischio}
		I fattori che possono causare rischi sono:
			\begin {itemize}
				\item il gruppo risulta poco interessato al \gl{progetto} e al suo dominio applicativo.
			\end {itemize}
	\subsection {Conclusioni}
	Il team ha deciso di scartare questo \gl{capitolato}, nonostante la sua chiarezza di esposizione e l'uso di tecnologie conosciute, a causa di una mancanza di interesse in esso e nel suo ambito di utilizzo.