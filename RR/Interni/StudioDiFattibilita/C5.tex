\section {Capitolato C5}
	\subsection {Descrizione}
		L'obiettivo del progetto è quello di creare un framework che permetta di sviluppare bolle interattive che dovranno lavorare all'interno di Rocket.chat. \\
		Le bolle, create tramite Monolith, sono di tre tipi:
		\begin{itemize}
			\item "rich media bubble": shortcut per la condivisione di contenuti come, ad esempio, l'animazione di una GIF oppure o l'anteprima di un video di Youtube;
			\item "self-updating bubble": forniscono delle informazioni che possono cambiare nel tempo;
			\item "editing bubble": permettono ai destinatari della bolla di apportare modifiche in modo collaborativo alla bolla stessa. Un esempio di editing bubble sono i sondaggi rapidi.
		\end{itemize}
	\subsection {Dominio applicativo}
		Le bolle interattive introducono nuove caratteristiche a Rocket.chat migliorando l'esperienza degli utenti.
	\subsection {Dominio tecnologico}
		\begin{itemize}
		\item \textbf{JavaScript}: per la realizzazione del framework e della demo;
		\item \textbf{GitHub} or \textbf{Bitbucket}: per pubblicare e versionare il codice sorgente.
		\end{itemize}
	\subsection {Valutazione}
		\subsubsection {Aspetti positivi}
			\begin{itemize}
				\item L’idea proposta e le sue specifiche sono molto chiare ed esaustive.
				\item Il gruppo ritiene interessante lavorare con framework JavaScript.
			\end{itemize}
		\subsubsection {Fattori di rischio}
			\begin{itemize}
				\item Ampia libertà lasciata dal committente al gruppo nella realizzazione dei contenuti delle bolle.		
				\item Il gruppo ritiene il progetto poco interessante.
			\end{itemize}
	\subsection {Conclusioni}
		Il progetto lascia troppa libertà al gruppo e, di conseguenza, il risultato potrebbe non essere esaustivo per il committente. Inoltre, molti membri del gruppo trovano molto più stimolanti altri progetti.
	