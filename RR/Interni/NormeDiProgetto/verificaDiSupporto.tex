\subsubsection{Scopo del processo}
Si occupa di accertare che lo svolgimento del processo in esame non introduca errori nel \gl{prodotto}.
\subsubsection{Aspettative del processo}
Una corretta implementazione di tale processo permette di individuare:
\begin{itemize}
	\item una procedura di \gl{verifica};
	\item i criteri per la \gl{verifica} del \gl{prodotto}.
\end{itemize}
\subsubsection{Attività}
 \paragraph{Analisi statica}
E' una tecnica di analisi del codice sorgente e della documentazione associata, prevalentemente
usata quando il \gl{sistema} non è ancora disponibile e durante tutto l'arco del suo sviluppo. Non
richiede l'esecuzione del \gl{prodotto} \gl{software} in alcuna sua parte. Può essere applicata tramite una
delle seguenti strategie:
\begin{itemize}
	\item \textbf{Walkthrough}: si legge l'intero documento (o codice) in cerca di tutte le possibili anomalie. E' una tecnica onerosa che richiede l'impegno di più persone e per questo deve essere utilizzata solo durante la prima parte del \gl{progetto}, dove non tutti i membri hanno piena padronanza e conoscenza delle \NPdoc e del \PQdoc;
	\item \textbf{Inspection}: questa tecnica dev'essere applicata quando si ha idea della
problematica che si sta cercando; consiste in una lettura mirata del
documento (o del codice), sulla base di una lista degli errori precedentemente
stilata.
\end{itemize}
 \paragraph{Analisi dinamica}
L'attività di analisi dinamica è una tecnica di \gl{verifica} applicabile solamente al \gl{software}. Tale tecnica può essere utilizzata per analizzare l'intero \gl{software} o una
porzione limitata dello stesso. L'attività consiste nell'esecuzione di test automatici realizzati
dal team. Le verifiche devono essere effettuate su un insieme finito di casi, con valori di
ingresso, uno stato iniziale e un esito decidibile. Tutti i test producono risultati automatici
che inviano notifiche sulla tipologia di problema individuato. Ogni test è ripetibile, ossia
applicabile durante l'intero \gl{ciclo di vita} del \gl{software}.
\subsubsection{Metriche}
 Per garantire la qualità del lavoro del team gli \AMMP{} hanno definito delle metriche, riportandole
nel \PQdoc, che devono rispettare la seguente notazione:\\ \\
\centerline{\textbf{M\textbraceleft{}X\textbraceright{}\textbraceleft{}Y\textbraceright{}\textbraceleft{}Z\textbraceright{}}} \\ \\
dove:  
\begin{itemize}
	\item \textbf{X} indica se la metrica si riferisce a prodotti o processi e può assumere
i valori:
	\begin{itemize}
		\item \textbf{PC} per indicare i processi;
		\item \textbf{PD} per indicare i prodotti.
	\end{itemize}
	\item \textbf{Y} presente solo se la metrica è riferita ai prodotti, indica se il termine prodotto si riferisce a documenti o al software e può assumere i seguenti valori:
	\begin{itemize}
		\item \textbf{D} per indicare i documenti;
		\item \textbf{S} per indicare il software;
	\end{itemize}
	\item \textbf{Z} indica il codice univoco della metrica (numero intero incrementale a partire da 1).
\end{itemize}
\subsubsection{Obiettivi}
Per garantire la qualità del lavoro del team, gli \AMMP{} hanno definito degli obiettivi di qualità,
riportandoli nel \PQdoc, che devono rispettare la seguente notazione:\\ \\
\centerline{\textbf{O\textbraceleft{}X\textbraceright{}\textbraceleft{}Y\textbraceright{}\textbraceleft{}Z\textbraceright{}}} \\ \\
dove: 
\begin{itemize}
	\item \textbf{X} indica se l'obiettivo si riferisce a prodotti o processi e può assumere i valori:
	\begin{itemize}
		\item \textbf{PC} per indicare i processi;
		\item \textbf{PD} per indicare i prodotti.
	\end{itemize}
	\item \textbf{Y} presente solo se l'obiettivo è riferito ai prodotti, indica se il termine prodotto si riferisce a documenti o al software e può assumere i seguenti valori:
	\begin{itemize}
		\item \textbf{D} per indicare i documenti;
		\item \textbf{S} per indicare il software;
	\end{itemize}
	\item \textbf{Z} indica il codice univoco dell'obiettivo (numero intero incrementale a partire da 1).
\end{itemize}
\subsubsection{Issue tracking}
L'\gl{issue} tracking è un'attività di supporto per la figura dei \VERP, ai quali permette di tenere traccia, e contemporaneamente segnalare al \RESP, la presenza di potenziali errori in un documento o nel codice sorgente.
 \paragraph{Gestione delle \gl{issue}}
Qualora un \VER{} dovesse riscontrare delle anomalie, la procedura per la segnalazione e gestione del \gl{ticketing} di una \gl{issue} è la seguente:
\begin{enumerate}
	\item il \VER{} dovrà aprire una nuova \gl{issue} assegnandole una label che si riferisca al problema trovato;
	\item il \RESP{} di \gl{progetto} dovrà valutare la \gl{issue}; se la ritiene appropriata assegnerà ai redattori del documento (o ai \PRP) il compito di risolvere la \gl{issue};
	\item una volta risolta, e verificata, la \gl{issue} dovrà essere marcata come conlcusa da parte del \RESP{} o del \VER;
\end{enumerate}
\subsubsection{Strumenti}
\paragraph{Strumenti per l'issue tracking}
Lo strumento utilizzato per l'\gl{issue} tracking è il servizio Issues messo a disposizione da \gl{GitHub}.
\paragraph{Verifica ortografica}
Viene utilizzata la \gl{verifica} in tempo reale dell'ortografia, integrata in TexMaker. Essa marca,
sottolineando in rosso, le parole errate secondo la lingua italiana.
\paragraph{Indice di Gulpease}
Affinché un documento possa superare la fase di approvazione, è necessario che soddisfi il test di leggibilità con un indice Gulpease superiore a 40 punti.