\subsubsection{Scopo}
Lo scopo del processo è produrre il \PPdoc , al fine di pianificare e gestire i ruoli che i membri dovranno assumere.
\subsubsection{Aspettative}
Le aspettative del processo sono:
 \begin{itemize}
  \item produrre il \PPdoc ;
  \item definire i ruoli dei membri del gruppo;
  \item definire il piano per l'esecuzione dei compiti programmati.
 \end{itemize}
\subsubsection{Descrizione}
 
\subsubsection{Ruoli di progetto}
 In ogni momento temporale ogni membro deve ricoprire almeno un ruolo e, durante tutta la durata del \gl{progetto}, ricoprire tutti i ruoli almeno una volta. Per ogni membro, le ore di lavoro devono essere il più possibile equamente distribuite. L'assegnazione e la rotazione dei ruoli sono pianificate nel \PPdocRR.
 \paragraph{Responsabile}
 Il \RESP{} è il rappresentante e il punto di riferimento del gruppo, nonchè colui che si assume le responsabilità delle scelte del gruppo.
 Le responsabilità assunte sono:
 \begin{itemize}
  \item pianificazione e coordinamento delle attività;
  \item analisi e gestione dei rischi;
  \item gestione delle risorse;
  \item approvazione dei documenti;
  \item approvazione dell'offerta economica;
  \item assicurarsi del rispetto delle \NPdoc{} e che vengano rispettate le pianificazioni nel \PPdoc
 \end{itemize}
 \paragraph{Amministratore}
 L'\AMM{} è responsabile dell'efficienza dell'ambiente di lavoro, in particolare si occupa di:
 \begin{itemize}
  \item studiare e fornire strumenti che migliorano l'ambiente di lavoro, automatizzando il lavoro ove possibile;
  \item gestire archiviazione, versionamento e configurazione dei documenti e del \gl{software};
  \item garantire la qualità del \gl{prodotto}, fornendo procedure e strumenti di monitoraggio e segnalazione;
  \item eliminare le difficoltà sulla gestione di processi e risorse.
 \end{itemize}
 \paragraph{Analista}
 L'\AN{} deve identificare e comprendere il domimio del problema. \\
 In particolare si occupa di:
 \begin{itemize}
  \item mappare le richieste del cliente in specifiche per il \gl{prodotto};
  \item catalogare e spiegare specifiche comprensibili nell'\ARdoc{} e nello \SFdoc{}.
 \end{itemize}
 \paragraph{Progettista}
 Il \PJ{} ha forti competenze sullo \gl{stack} tecnologico usato. \\
 In particolare deve: 
 \begin{itemize}
  \item indicare le tecnologie più adatte allo sviluppo del \gl{progetto};
  \item descrivere il funzionamento del \gl{sistema} progettandone l'architettura;
  \item produrre una soluzione fattibile in termini di risorse.
 \end{itemize}
 \paragraph{Programmatore}
 Il \PR{} si occupa della codifica, in particolare:
 \begin{itemize}
  \item implementa le soluzioni indicate dal \PJ ;
  \item scrive codice documentato, versionato e mantenibile nel rispetto delle \NPdoc ;
  \item realizza e fornisce gli strumenti per verificare e validare il \gl{prodotto}.
 \end{itemize}
 \paragraph{Verificatore}
 Il \VER , disponendo di una profonda conoscenza delle \NPdoc , si occupa delle attività di \gl{verifica}. \\
 In particolare deve: 
 \begin{itemize}
  \item controllare il rispetto delle \NPdoc durante ogni attività del \gl{progetto}.
 \end{itemize} 
\subsubsection{Comunicazioni}
 \paragraph{Interne}
 È stato creato un gruppo Telegram, accessibile solo ai membri del team, per effettuare le comunicazioni interne. In caso siano necessaria maggiore interazione, si farà utilizzo di Google Hangout. 
 \paragraph{Esterne}
 È stata creata un'apposita cartella di posta elettronica per mantenere i contatti con il proponente, il committente ed altre eventuali figure esterne.
 La gestione della casella di posta elettronica è compito del \RESP. \\
 L'indirizzo e-mail è il seguente: \EMAIL{} .
\subsubsection{Incontri}
 \paragraph{Interni}
 Ogni membro del team può proporre un incontro interno tramite il bot Telegram "VotePoll", specificando i motivi e l'oggetto dell'incontro. 
 Sarà poi compito del \RESP{} decidere se effettuare l'incontro o meno.\\
  La verbalizzazione degli incontri esterni è compito di uno tra gli \AMMP.
 \paragraph{Esterni} 
 Ogni membro del team può proporre un incontro esterno tramite il Bot Telegram "VotePoll", specificando i motivi e l'oggetto dell'incontro. 
 Sarà poi compito del \RESP{} decidere se organizzare l'incontro o meno. Una volta che il \RESP{} si è accordato con la figura esterna, egli dovrà comunicare gli estremi dell'incontro ai membri del team.\\
 La verbalizzazione degli incontri esterni è compito del \RESP.
\subsubsection{Strumenti di coordinamento}
 \paragraph{Ticketing}
 Il \RESP{} ha il compito di assegnare i task ai membri del team utilizzando l'applicativo web Asana. \\
 Definendo delle milestone, è possibile tenere traccia dello stato di avanzamento del lavoro di ogni task.
 
\subsubsection{Strumenti di versionamento}
 \paragraph{\gl{Repository}}
 Per il versionamento e l'archiviazione dei file, l'\AMM{} ha creato un repository GitHub, il quale è disponibile al seguente indirizzo \url{https://github.com/CoCodeSWE/Docs}. Tutti i membri del gruppo dovranno creare un proprio account GitHub, per poi ricevere i permessi in scrittura sul repository da parte dell'\AMM.
 La gestione del repository è responsabilità degli \AMMP.
 \paragraph{Struttura del \gl{repository}}
 Al fine di mantenere ordine e coerenza tra i file, il repository è così strutturato:
 \begin{itemize}
  \item Docs
   \begin{itemize}
    \item RR
     \begin{itemize}
      \item Esterni: contiene i documenti esterni.
      \item Interni: contiene i documenti interni.
     \end{itemize}
     \item script: contiene gli script utilizzati.
     \item template: contiene i template utilizzati.
    \end{itemize}
   \end{itemize}
 \paragraph{Commit}
 Ogni commit effettuata deve essere accompagnata da un messaggio descrittivo delle modifiche effettuate. L'autore della commit dovrà assicurarsi della correttezza dei file. È sconsigliato committare intere cartelle, al fine di evitare inclusioni di file inutili(ad es: file di compilazione).
 Dovrà inoltre essere segnalata l'eventuale aggiunta di nuovi file.
 \subsubsection{Rischi}
 Il \RESP{} ha il dovere di individuare e monitorare i rischi indicati nel \PPdoc. In caso ne vengano identificati di nuovi, il \RESP{} deve agire nel modo seguente:
 \begin{itemize}
  \item comunicare i nuovi rischi al team;
  \item pianificare una strategia per la gestione dei nuovi rischi;
  \item aggiornare le procedure di gestione dei rischi nel \PPdoc.
 \end{itemize}
\subsubsection{Strumenti}
 \paragraph{Telegram}
 Telegram è un software libero che fornisce un servizio di messaggistica istantanea erogato senza fini di lucro dalla società Telegram LLC. È stato ritenuto più adatto di Whatsapp.
 \paragraph{Google Hangout}
 Hangouts è un software di messaggistica istantanea e di VoIP   sviluppato da Google. È disponibile per le piattaforme mobili Android e iOS e come estensione per il browser web Google Chrome. Inoltre, permette la condivisione degli schermi tra i membri della chiamata. È stato ritenuto più adatto di Skype
 \paragraph{Git}
 Git è un software open-source di controllo versione distribuito utilizzabile dal terminale. Come versione si utilizza la 2.7.4 o superiori.
 \paragraph{GitHub}
 GitHub è un servizio di hosting per progetti software, con il quale è possibile interagire tramite Git. GitHub
offre diversi piani per repository privati sia a pagamento, sia gratuiti, molto utilizzati per lo
sviluppo di progetti open-source. 
 \paragraph{GitHub desktop}
 GitHub Desktop è l’applicativo desktop per contribuire e collaborare ai progetti del corrispon-
dente servizio web GitHub. Esso è disponibile per Windows e MacOS. Per Windows si utilizza la versione 3.3.3 o superiori.
 \paragraph{Asana}
 Asana è un applicativo web e mobile che consente al team di assegnare, tracciare e gestire dei task.
 \paragraph{strumento per i diagrammi di gantt}
 
 