\subsubsection{Scopo}
Lo scopo del processo è produrre il \PPdoc , al fine di pianificare e gestire i ruoli che i membri dovranno assumere.
\subsubsection{Aspettative}
Le aspettative del processo sono:
 \begin{itemize}
  \item produrre il \PPdoc ;
  \item definire i ruoli dei membri del gruppo;
  \item definire il piano per l'esecuzione dei compiti programmati.
 \end{itemize}
\subsubsection{Descrizione}
 
\subsubsection{Ruoli di progetto}
 In ogni momento temporale ogni membro deve ricoprire almeno un ruolo e, durante tutta la durata del \gl{progetto}, ricoprire tutti i ruoli almeno una volta. Per ogni membro, le ore di lavoro devono essere il più possibile equamente distribuite. L'assegnazione e la rotazione dei ruoli sono pianificate nel \PPdocRR.
 \paragraph{Responsabile}
 Il \RESP{} è il rappresentante e il punto di riferimento del gruppo, nonchè colui che si assume le responsabilità delle scelte del gruppo.
 Le responsabilità assunte sono:
 \begin{itemize}
  \item pianificazione e coordinamento delle attività;
  \item analisi e gestione dei rischi;
  \item gestione delle risorse;
  \item approvazione dei documenti;
  \item approvazione dell'offerta economica;
  \item assicurarsi del rispetto delle \NPdoc{} e che vengano rispettate le pianificazioni nel \PPdoc
 \end{itemize}
 \paragraph{Amministratore}
 L'\AMM{} è responsabile dell'efficienza dell'ambiente di lavoro, in particolare si occupa di:
 \begin{itemize}
  \item studiare e fornire strumenti che migliorano l'ambiente di lavoro, automatizzando il lavoro ove possibile;
  \item gestire archiviazione, versionamento e configurazione dei documenti e del \gl{software};
  \item garantire la qualità del \gl{prodotto}, fornendo procedure e strumenti di monioraggio e segnalazione;
  \item eliminare le difficoltà sulla gestione di processi e risorse.
 \end{itemize}
 \paragraph{Analista}
 L'\AN{} deve identificare e comprendere il domimio del problema. \\
 In particolare si occupa di:
 \begin{itemize}
  \item mappare le richieste del cliente in specifiche per il \gl{prodotto};
  \item catalogare e spiegare specifiche comprensibili nell'\ARdoc{} e nello \SFdoc{}.
 \end{itemize}
 \paragraph{Progettista}
 Il \PJ{} ha forti competenze sullo \gl{stack} tecnologico usato. \\
 In particolare deve: 
 \begin{itemize}
  \item indicare le tecnologie più adatte allo sviluppo del \gl{progetto};
  \item descrivere il funzionamento del \gl{sistema} progettandone l'architettura;
  \item produrre una soluzione fattibile in termini di risorse.
 \end{itemize}
 \paragraph{Programmatore}
 Il \PR{} si occupa della codifica, in particolare:
 \begin{itemize}
  \item implementa le soluzioni indicate dal \PJ ;
  \item scrive codice documentato, versionato e mantenibile nel rispetto delle \NPdoc ;
  \item realizza e fornisce gli strumenti per verificare e validare il \gl{prodotto}.
 \end{itemize}
 \paragraph{Verificatore}
 Il \VER , disponendo di una profonda conoscenza delle \NPdoc , si occupa delle attività di \gl{verifica}. \\
 In particolare deve: 
 \begin{itemize}
  \item controllare il rispetto delle \NPdoc durante ogni attività del \gl{progetto}.
 \end{itemize} 
\subsubsection{Comunicazioni}
 \paragraph{Interne}
 \paragraph{Esterne}
\subsubsection{Incontri}
 \paragraph{Interni}
 \paragraph{Esterni} 
\subsubsection{Strumenti di coordinamento}
 \paragraph{Ticketing}
\subsubsection{Strumenti di versionamento}
 \paragraph{\gl{Repository}}
 \paragraph{Struttura del \gl{repository}}
 \paragraph{Commit}
 \paragraph{Rischi} 
\subsubsection{Strumenti}
 \paragraph{\gl{Sistema} Operativo}
 \paragraph{Telegram}
 \paragraph{Git}
 \paragraph{Ecc...}