\subsection{Requisiti Funzionali}
\normalsize
\begin{longtable}{|c|>{\centering}m{7cm}|c|}
\hline
\textbf{Id Requisito} & \textbf{Descrizione} & \textbf{Stato}\\
\hline
\endhead\hypertarget{RFO1}{RFO1} & Il sistema deve permettere all'utente di fornire i propri dati identificativi. & \gl{Capitolato}, UC1\\ \hline
\hypertarget{RFO2}{RFO2} & L'amministratore deve poter accedere alla sezione amministrativa. & Verbale Esterno 1, UC2\\ \hline
\hypertarget{RFO2.1}{RFO2.1} & L'amministratore deve poter gestire le direttive da lui accessibili. & Verbale Esterno 1, UC2.1\\ \hline
\hypertarget{RFO2.1.1}{RFO2.1.1} & L'amministratore deve poter creare una nuova \gl{direttiva}. & Interno, UC2.1.1\\ \hline
\hypertarget{RFO2.1.1.1}{RFO2.1.1.1} & L'amministratore deve poter inserire la funzione di una \gl{direttiva}. & Interno, UC2.1.1.1\\ \hline
\hypertarget{RFO2.1.1.2}{RFO2.1.1.2} & L'amministratore deve poter inserire il nome di una \gl{direttiva}. & Interno, UC2.1.1.2\\ \hline
\hypertarget{RFO2.1.1.3}{RFO2.1.1.3} & L'amministratore deve poter inserire il target di una \gl{direttiva}. & Interno, UC2.1.1.3\\ \hline
\hypertarget{RFD2.1.1.4}{RFD2.1.1.4} & L'amministratore deve poter concedere i privilegi per la \gl{direttiva} ad altri amministratori. & Interno, UC2.1.1.4\\ \hline
\hypertarget{RFO2.1.1.5}{RFO2.1.1.5} & L'amministratore deve poter confermare la creazione di una \gl{direttiva}. & Interno, UC2.1.1.5\\ \hline
\hypertarget{RFO2.1.1.6}{RFO2.1.1.6} & L'amministratore deve poter visualizzare un messaggio d'errore se ha comunicato dei dati nulli o non validi per la creazione di una nuova \gl{direttiva}. & Interno, UC2.1.1.6\\ \hline
\hypertarget{RFO2.1.2}{RFO2.1.2} & L'amministratore deve poter eliminare dal sistema una \gl{direttiva} di cui ha i privilegi. & Interno, UC2.1.2\\ \hline
\hypertarget{RFO2.1.2.1}{RFO2.1.2.1} & L'amministratore deve poter confermare l'eliminazione di una \gl{direttiva}. & Interno, UC2.1.2.1\\ \hline
\hypertarget{RFD2.1.3}{RFD2.1.3} & L'amministratore deve poter modificare una \gl{direttiva} di cui ha i privilegi di modifica. & Interno, UC2.1.3\\ \hline
\hypertarget{RFD2.1.3.1}{RFD2.1.3.1} & L'amministratore deve poter modificare il nome di una \gl{direttiva}. & Interno, UC2.1.3.1\\ \hline
\hypertarget{RFD2.1.3.2}{RFD2.1.3.2} & L'amministratore deve poter modificare i target di una \gl{direttiva}. & Interno, UC2.1.3.2\\ \hline
\hypertarget{RFD2.1.3.3}{RFD2.1.3.3} & L'amministratore deve poter modificare la funzione di una \gl{direttiva}. & Interno, UC2.1.3.3\\ \hline
\hypertarget{RFD2.1.3.4}{RFD2.1.3.4} & L'amministratore deve poter modificare l'abilitazione di una \gl{direttiva}. & Interno, UC2.1.3.4\\ \hline
\hypertarget{RFD2.1.3.5}{RFD2.1.3.5} & L'amministratore deve poter modificare i privilegi degli altri amministratori per la \gl{direttiva}. & Interno, UC2.1.3.5\\ \hline
\hypertarget{RFD2.1.3.5.1}{RFD2.1.3.5.1} & L'amministratore deve poter concedere ad altri amministratori i privilegi per la \gl{direttiva}. & Interno, UC2.1.3.5.1\\ \hline
\hypertarget{RFD2.1.3.5.2}{RFD2.1.3.5.2} & L'amministratore deve poter revocare i privilegi degli altri amministratori per la \gl{direttiva}. & Interno, UC2.1.3.5.2\\ \hline
\hypertarget{RFD2.1.3.6}{RFD2.1.3.6} & L'amministratore deve poter confermare la modifica di una \gl{direttiva}. & Interno, UC2.1.3.6\\ \hline
\hypertarget{RFD2.1.3.7}{RFD2.1.3.7} & L'amministratore deve poter visualizzare un messaggio d'errore se ha comunicato dei dati nulli o non validi per la modifica di una \gl{direttiva}. & Interno, UC2.1.3.7\\ \hline
\hypertarget{RFO2.1.4}{RFO2.1.4} & L'amministratore deve poter visualizzare tutte le direttive da lui accessibili. & Interno, UC2.1.4\\ \hline
\hypertarget{RFO2.1.4.1}{RFO2.1.4.1} & L'amministratore deve poter cercare delle direttive in base al nome. & Interno, UC2.1.4.1\\ \hline
\hypertarget{RFD2.1.4.2}{RFD2.1.4.2} & L'amministratore deve poter cercare delle direttive in base ai target. & Interno, UC2.1.4.2\\ \hline
\hypertarget{RFD2.1.4.3}{RFD2.1.4.3} & L'amministratore deve poter cercare le direttive in base alla loro funzione. & Interno, UC2.1.4.3\\ \hline
\hypertarget{RFD2.1.4.4}{RFD2.1.4.4} & L'amministratore deve poter cercare le direttive in base alla loro abilitazione. & Interno, UC2.1.4.4\\ \hline
\hypertarget{RFO2.2}{RFO2.2} & L'amministratore deve poter gestire le impostazioni del proprio profilo. & Interno, UC2.2\\ \hline
\hypertarget{RFF2.2.1}{RFF2.2.1} & L'amministratore deve poter modificare il nome e cognome del suo profilo. & Interno, UC2.2.1\\ \hline
\hypertarget{RFO2.2.2}{RFO2.2.2} & L'amministratore deve poter modificare la password del suo profilo. & Interno, UC2.2.2\\ \hline
\hypertarget{RFO2.2.2.1}{RFO2.2.2.1} & L'amministratore deve poter inserire la vecchia password. & Interno, UC2.2.2.1\\ \hline
\hypertarget{RFO2.2.2.2}{RFO2.2.2.2} & L'amministratore deve poter inserire la nuova password. & Interno, UC2.2.2.2\\ \hline
\hypertarget{RFO2.2.3}{RFO2.2.3} & L'amministratore deve poter confermare le modifiche al profilo. & Interno, UC2.2.3\\ \hline
\hypertarget{RFO2.2.4}{RFO2.2.4} & L'amministratore deve poter visualizzare un messaggio d'errore se ha comunicato dei dati nulli o non validi per la modifica del profilo d'amministratore. & Interno, UC2.2.4\\ \hline
\hypertarget{RFO3}{RFO3} & L'ospite deve venire accolto dal sistema. & \gl{Capitolato}, UC3\\ \hline
\hypertarget{RFO3.1}{RFO3.1} & L'ospite deve poter comunicare al sistema la persona che desidera incontrare. & \gl{Capitolato}, UC3.1\\ \hline
\hypertarget{RFD3.2}{RFD3.2} & L'ospite deve poter comunicare al sistema particolari necessità  per l'incontro. & \gl{Capitolato}, UC3.2\\ \hline
\hypertarget{RFD3.2.1}{RFD3.2.1} & L'ospite deve poter richiedere un caffè. & \gl{Capitolato}, UC3.2.1\\ \hline
\hypertarget{RFD3.2.2}{RFD3.2.2} & L'ospite deve poter chiedere informazioni riguardanti una particolare stanza. & Verbale Esterno 1, UC3.2.2\\ \hline
\hypertarget{RFD3.2.3}{RFD3.2.3} & L'ospite deve poter richiedere le indicazioni necessarie a raggiungere una particolare stanza. & Interno, UC3.2.3\\ \hline
\hypertarget{RFD3.2.4}{RFD3.2.4} & L'ospite deve poter richiedere particolare materiale per l'incontro. & Interno, UC3.2.4\\ \hline
\hypertarget{RFD3.2.5}{RFD3.2.5} & L'ospite deve poter visualizzare un errore nel caso richieda informazioni su una stanza inesistente. & Interno, UC3.2.5\\ \hline
\hypertarget{RFD3.3}{RFD3.3} & L'ospite deve poter scegliere tra alcuni tipi di intrattenimento forniti dal sistema. & Verbale Esterno 1, UC3.3\\ \hline
\hypertarget{RFF3.3.1}{RFF3.3.1} & L'ospite deve poter rispondere ad alcuni indovinelli fatti dal sistema. & Interno, UC3.3.1\\ \hline
\hypertarget{RFD3.3.2}{RFD3.3.2} & L'ospite deve poter visualizzare curiosità  di vario genere. & Interno, UC3.3.2\\ \hline
\hypertarget{RFF3.3.3}{RFF3.3.3} & L'ospite deve poter visualizzare le ultime notizie riguardanti categorie di vario genere. & Interno, UC3.3.3\\ \hline
\hypertarget{RFF3.3.4}{RFF3.3.4} & L'ospite deve poter essere intrattenuto tramite alcuni giochi forniti dal sistema. & Interno, UC3.3.4\\ \hline
\hypertarget{RFO4}{RFO4} & L'ospite deve poter comunicare il nome dell'azienda di cui fa parte. & Verbale Esterno 1, UC4\\ \hline
\hypertarget{RFO5}{RFO5} & Il sistema, nel caso in cui non riesca ad interpretare la risposta, deve chiedere nuovamente l'informazione all'utente. & Interno, UC5\\ \hline
\hypertarget{RFD6}{RFD6} & Il sistema deve mostrare un opportuno messaggio qualora il tempo trascorso tra due successive interazioni superi un certo limite. & Interno, UC6\\ \hline
\hypertarget{RFO7}{RFO7} & Il sistema deve riconoscere gli ospiti passati e modificare il proprio comportamento in base alle interazioni passate. & \gl{Capitolato}\\ \hline
\hypertarget{RFO7.1}{RFO7.1} & Il sistema deve prevedere metodi di apprendimento per migliorare la comunicazione con gli utenti. & Verbale Esterno 1\\ \hline
\hypertarget{RFO8}{RFO8} & Il sistema deve sollecitare la persona desiderata ed eventualmente avvisare gli altri membri dell'azienda su richiesta dell'ospite. & Verbale Esterno 1\\ \hline
\hypertarget{RFD9}{RFD9} & Il super amministratore deve poter accedere alla sezione dedicata al super amministratore. & Interno, UC7\\ \hline
\hypertarget{RFD9.1}{RFD9.1} & Il super amministratore deve poter gestire gli amministratori del sistema. & Interno, UC7.1\\ \hline
\hypertarget{RFD9.1.1}{RFD9.1.1} & Il super amministratore deve poter creare un nuovo amministratore. & Interno, UC7.1.1\\ \hline
\hypertarget{RFD9.1.1.1}{RFD9.1.1.1} & Il super amministratore deve poter inserire nome e cognome del nuovo amministratore. & Interno, UC7.1.1.1\\ \hline
\hypertarget{RFD9.1.1.2}{RFD9.1.1.2} & Il super amministratore deve poter inserire la password di un nuovo amministratore. & Interno, UC7.1.1.2\\ \hline
\hypertarget{RFD9.1.1.3}{RFD9.1.1.3} & Il super amministratore deve poter confermare i dati inseriti per un nuovo amministratore. & Interno, UC7.1.1.3\\ \hline
\hypertarget{RFD9.1.1.4}{RFD9.1.1.4} & Il super amministratore deve poter visualizzare un messaggio d'errore se ha comunicato dei dati nulli o non validi per la creazione di un nuovo amministratore. & Interno, UC7.1.1.4\\ \hline
\hypertarget{RFD9.1.2}{RFD9.1.2} & Il super amministratore deve poter resettare la password dell'amministratore. & Interno, UC7.1.2\\ \hline
\hypertarget{RFD9.1.2.1}{RFD9.1.2.1} & Il super amministratore deve poter confermare il reset della password di un amministratore. & Interno, UC7.1.2.1\\ \hline
\hypertarget{RFD9.1.3}{RFD9.1.3} & Il super amministratore deve poter eliminare un amministratore dal sistema. & Interno, UC7.1.3\\ \hline
\hypertarget{RFD9.1.3.1}{RFD9.1.3.1} & Il super amministratore deve poter confermare la revoca dei privilegi ad un amministratore. & Interno, UC7.1.3.1\\ \hline
\hypertarget{RFD9.1.3.2}{RFD9.1.3.2} & L'amministratore può visualizzare un messaggio d'errore se ha comunicato dei dati nulli o non validi per l'eliminazione di un amministratore. & Interno, UC7.1.3.2\\ \hline
\hypertarget{RFD9.2}{RFD9.2} & Il super amministratore può accedere ai file \gl{log}. & Interno, UC7.2\\ \hline
\hypertarget{RFO10}{RFO10} & Il sistema deve permettere all'amministratore di definire il comportamento del sistema stesso in base alla persona che sta interagendo con esso. & Verbale Esterno 1\\ \hline
\hypertarget{RFD11}{RFD11} & Il sistema deve notificare l'arrivo di ospiti in modo differente in base alla loro azienda di provenienza. & Verbale Esterno 1\\ \hline
\hypertarget{RFO12}{RFO12} & Il sistema deve memorizzare i dati relativi alle interazioni con gli ospiti. & Verbale Esterno 1\\ \hline
\hypertarget{RFO12.1}{RFO12.1} & Il sistema deve registrare i dati identificativi dell'ospite & Verbale Esterno 1\\ \hline
\hypertarget{RFO12.2}{RFO12.2} & Il sistema deve registrare l'azienda di provenienza dell'ospite. & Verbale Esterno 1\\ \hline
\hypertarget{RFO12.3}{RFO12.3} & Il sistema deve registrare le diverse persone che l'ospite viene a trovare e con quale frequenza & Verbale Esterno 1\\ \hline
\hypertarget{RFD12.4}{RFD12.4} & Il sistema deve registrare i dati relativi alle necessità  dell'ospite. & Interno\\ \hline
\hypertarget{RFD12.5}{RFD12.5} & Il sistema deve registrare dati relativi ai metodi di intrattenimento selezionati dall'ospite. & Interno\\ \hline
\hypertarget{RFF12.6}{RFF12.6} & Il sistema deve registrare i dati relativi all'ora di arrivo dell'ospite. & Interno\\ \hline
\hypertarget{RFF12.7}{RFF12.7} & Il sistema deve registrare dati relativi agli errori verificatisi nell'interazione con l'ospite. & Interno\\ \hline
\hypertarget{RFO13}{RFO13} & L'utente deve poter fornire la propria password di amministrazione per poter accedere all'area di amministrazione. & Interno, UC8\\ \hline

\caption[Requisiti Funzionali]{Requisiti Funzionali}
\label{tabella:req0}
\end{longtable}
\clearpage
\subsection{Requisiti di Qualità}
\normalsize
\begin{longtable}{|c|>{\centering}m{7cm}|c|}
\hline
\textbf{Id Requisito} & \textbf{Descrizione} & \textbf{Stato}\\
\hline
\endhead\hypertarget{RQO1}{RQO1} & Il gruppo deve fare un'analisi preliminare degli \gl{SDK} dei principali assistenti virtuali presenti sul mercato. & \gl{Capitolato}\\ \hline
\hypertarget{RQO2}{RQO2} & Il gruppo deve fornire uno schema design per la base di dati \gl{NoSQL}.
 & \gl{Capitolato}\\ \hline
\hypertarget{RQO3}{RQO3} & Deve essere \gl{prodotto} un piano di test di unità per il sistema. & \gl{Capitolato}\\ \hline
\hypertarget{RQO4}{RQO4} & Deve essere fornita una documentazione dettagliata di tutte le \gl{API}. & \gl{Capitolato}\\ \hline
\hypertarget{RQD5}{RQD5} & Deve essere fornito un manuale utente. & Interno\\ \hline
\hypertarget{RQD5.1}{RQD5.1} & Il manuale utente deve contenere una sezione che spiega come utilizzare l'applicazione. & Interno\\ \hline
\hypertarget{RQD5.2}{RQD5.2} & Il manuale utente deve contenere una sezione che descriva eventuali errori e possibili cause. & Interno\\ \hline
\hypertarget{RQF6}{RQF6} & Deve essere fornito un manuale per l'amministratore. & Interno\\ \hline
\hypertarget{RQF6.1}{RQF6.1} & Il manuale per l'amministratore deve contenere una sezione che spiega come installare l'applicazione. & Interno\\ \hline
\hypertarget{RQF6.2}{RQF6.2} & Il manuale per l'amministratore deve contenere una sezione che spiega come utilizzare le funzioni da amministratore. & Interno\\ \hline
\hypertarget{RQF6.3}{RQF6.3} & Il manuale per l'amministratore deve contenere una sezione che descriva eventuali errori e possibili cause. & Interno\\ \hline

\caption[Requisiti di Qualità]{Requisiti di Qualità}
\label{tabella:req2}
\end{longtable}
\clearpage
\subsection{Requisiti di Vincolo}
\normalsize
\begin{longtable}{|c|>{\centering}m{7cm}|c|}
\hline
\textbf{Id Requisito} & \textbf{Descrizione} & \textbf{Stato}\\
\hline
\endhead\hypertarget{RVO1}{RVO1} & Il sistema deve interagire con i membri dell'azienda mediante \gl{Slack}. & \gl{Capitolato}\\ \hline
\hypertarget{RVO2}{RVO2} & Il sistema deve essere sviluppato utilizzando AWS, con lambda function o server dedicato. & \gl{Capitolato}\\ \hline
\hypertarget{RVO3}{RVO3} & Il sistema deve utilizzare un database \gl{NoSQL} per la memorizzazione dei dati. & \gl{Capitolato}\\ \hline
\hypertarget{RVO4}{RVO4} & Il sistema deve presentare un'interfaccia web. & \gl{Capitolato}\\ \hline
\hypertarget{RVO5}{RVO5} & L'interazione col sistema deve essere principalmente vocale. & \gl{Capitolato}\\ \hline
\hypertarget{RVF6}{RVF6} & Il sistema deve permettere all'amministratore di interagire con il sistema anche tramite l'invio di messaggi con \gl{Slack}. & Interno\\ \hline
\hypertarget{RVO7}{RVO7} & L'assistente virtuale deve essere sviluppato in lingua inglese. & \gl{Capitolato}\\ \hline
\hypertarget{RVO8}{RVO8} & L'interfaccia web deve funzionare su PC, Mac e Tablet (\gl{Android}/iOs). & Verbale Esterno 1\\ \hline
\hypertarget{RVD9}{RVD9} & Il sistema deve permettere l'interazione in altre lingue. & Interno\\ \hline

\caption[Requisiti di Vincolo]{Requisiti di Vincolo}
\label{tabella:req3}
\end{longtable}
\clearpage
