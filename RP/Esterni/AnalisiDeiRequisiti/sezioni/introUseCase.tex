\documentclass[AdR.tex]{subfiles}
\begin{document}
Vengono elencati tutti i \gl{casi d'uso} rilevati dall'analisi del \gl{capitolato} C2 e dalle riunioni con \PROPONENTE. \\
Ogni caso d'uso ha un proprio codice identificativo  che rispetta il seguente formalismo:\\ \\
\centerline{UC\textbraceleft{Codice}\textbraceright{}}
\\ \\dove:
\begin{itemize}
	\item \textbf{Codice}: indica il codice identificativo del requisito, è univoco e deve essere identificato in forma gerarchica.
\end{itemize}

%gerarchia attori
\subsection{Gerarchia degli attori}
\begin{figure}[h]
  \centering
  \includegraphics[scale=0.4]{images/Attori.png}
  \caption{Gerarchia degli attori}
\end{figure}
Dopo un'attenta analisi da parte del gruppo gli attori che sono stati individuati sono: 
\begin{itemize}
\item \textbf{Utente}: un utente del sistema, il quale non è ancora stato identificato nè come amministratore nè come ospite;
\item \textbf{Ospite}: ospite esterno, il quale vuole incontrare un membro dell'azienda;
\item \textbf{Amministratore}: membro dell'azienda che dispone dei privilegi di amministrazione, i quali gli permettono di modificare il comportamento del sistema;
\item \textbf{Super Amministratore}: generalizzazione di amministratore, può creare e modificare altri amministratori e dispone di privilegi globali;
\end{itemize}
\newpage
\subsection{UCG: Scenario principale}
\label{UCG}
\begin{figure}[h]
\centering
\includegraphics[width=\textwidth,height=\textheight,keepaspectratio]{images/UseCaseUCG.png}
\caption{UCG: Scenario principale}
\end{figure}
\begin{longtable}{l|p{10cm}}
\rowcolor[gray]{0.8} \multicolumn{2}{c}{} \\
\rowcolor[gray]{0.8} \multicolumn{2}{c}{\textbf{UCG - Scenario principale}} \\
\rowcolor[gray]{0.8} \multicolumn{2}{c}{} \\
\hline
&\\
\textbf{Attori} & Utente, Amministratore, Super Amministratore, Ospite;\\[7pt]
\textbf{Descrizione} & Il sistema deve offrire tutte le funzionalità previste per ogni tipo di attore.\\[7pt]
\textbf{Precondizione} & Il sistema è avviato.\\[7pt]
\textbf{Postcondizione} & Gli attori hanno usufruito delle funzionalità offerte dal sistema.\\[7pt]
\textbf{Scenario principale} & L'attore usufruisce delle funzionalità a lui offerte dal sistema.\\[7pt]
\textbf{Scenari alternativi} & 
\begin{enumerate}
 \item Nel caso in cui il sistema non riesca ad interpretare le informazioni fornite dall'utente, comunica l'errore chiedendo nuovamente l'informazione.
 \item Nel caso in cui l'utente non interagisca con il sistema nel tempo stabilito, il sistema deve visualizzare un messaggio d'errore adeguato.
\end{enumerate}\\[7pt]\hline
\end{longtable}

\newpage
\subsection{UC0: Funzionalità sistema}
\label{UC0}
\begin{figure}[h]
\centering
\includegraphics[width=\textwidth,height=\textheight,keepaspectratio]{images/UseCase.png}
\caption{UC0: Funzionalità sistema}
\end{figure}
\begin{longtable}{l|p{10cm}}
\rowcolor[gray]{0.8} \multicolumn{2}{c}{} \\
\rowcolor[gray]{0.8} \multicolumn{2}{c}{\textbf{UC0 - Funzionalità sistema}} \\
\rowcolor[gray]{0.8} \multicolumn{2}{c}{} \\
\hline
&\\
\textbf{Attori} & Utente, Amministratore, Super Amministratore, Ospite;\\[7pt]
\textbf{Descrizione} & Un utente deve poter fornire il proprio nome e cognome e l'azienda di provenienza. Nel caso in cui l'utente sia riconosciuto dal sistema come possibile amministratore (o super amministratore) deve poter fornire la propria password di amministrazione per accedere alle funzionalità a lui dedicate. Nel caso in cui si tratti di un ospite in visita, il sistema deve accoglierlo.\\[7pt]
\textbf{Precondizione} & Il sistema è avviato e pronto ad interagire con l'attore.\\[7pt]
\textbf{Postcondizione} & L'attore ha usufruito delle funzionalità desiderate offerte dal sistema. \\[7pt]
\textbf{Scenario principale} &
\begin{enumerate}
 \item L'utente può fornire il proprio nome e cognome;
 \item L'utente può fornire la propria azienda di provenienza;
 \item L'utente può fornire la propria password di amministrazione per accedere alle funzionalità di amministratore;
 \item L'ospite deve poter essere accolto tramite le funzionalità offerte dal sistema;
 \item L'amministratore può usufruire delle funzionalità di amministratore;
 \item Il super amministratore, oltre a quelle di amministratore, può usufruire di altre funzionalità dedicate al super amministratore.
\end{enumerate}
\\[7pt]\hline
\end{longtable}
\end{document}