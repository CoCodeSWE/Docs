\section{Introduzione}
	\subsection{Scopo del documento}
	Questo documento contiene la pianificazione delle attività che saranno svolte dai membri del gruppo \GRUPPO{} per realizzare il \gl{progetto} \PROGETTO. In particolare, questo documento contiene:
	\begin{itemize}
		\item analisi e trattamento dei rischi;
		\item il preventivo delle risorse necessarie allo svolgimento del \gl{progetto};
		\item il consuntivo delle attività finora svolte.
	\end{itemize}	
	\subsection{Scopo del \gl{prodotto}}
		\SCOPO
	\subsection{Glossario}
		\GLOSSARIO
	\subsection{Riferimenti}
	\subsubsection{Riferimenti Normativi}
		\begin{itemize}
			\item \gl{Capitolato} d'appalto C2 - AtAVi: Accoglienza tramite Assistente Virtuale \\
			\url{http://www.math.unipd.it/~tullio/IS-1/2016/Progetto/C2.pdf};
			\item Rappresentazione date \\
			\url{https://en.wikipedia.org/wiki/ISO_8601};
			\item Composizione processo di sviluppo \\
			\url{https://en.wikipedia.org/wiki/ISO/IEC_12207}.

	\end{itemize}
	    \subsubsection{Riferimenti Informativi}
	    	\begin{itemize}
	    		\item \textbf{Ingegneria del \gl{Software} - Ian Sommerville - Decima edizione}:
	    		\begin{itemize}
	    			\item Parte 4 - \gl{Software} management.
	    		\end{itemize}
	    		\item \textbf{Slides corso Ingegneria del \gl{Software}}: \\ \url{http://www.math.unipd.it/~tullio/IS-1/2016/};
	    	\end{itemize}
	    
	    \subsection{Ciclo di Vita}
	    Il modello di \gl{ciclo di vita} scelto è il \gl{modello incrementale}. Esso prevede che:
	    \begin{itemize}
	    	\item l'analisi e la progettazione architetturale costituiscano una base solida: i requisiti e l'architettura del \gl{sistema} sono identificati e fissati definitivamente e sono essenziali per la pianificazione dei cicli incrementali;
	    	\item la progettazione di dettaglio, la codifica e i test vengono ripetuti più volte, permettendo sia il miglioramento di parti del \gl{sistema} già esistenti che l'aggiunta di nuove funzionalità.
	    \end{itemize}
	    I vantaggi attesi dalla scelta di tale modello sono i seguenti:
	    \begin{itemize}
	    	\item i requisiti utenti sono realizzati in base all'importanza strategica, ovvero vengono soddisfatti per primi quelli di maggiore rilevanza;
	    	\item ogni incremento può produrre valore e riduce il rischio di fallimento, in quando esso consolida ulteriormente la base ed eventualmente ne aumenta la qualità;
	    	\item esecuzione più dettagliata dei test che quindi risulteranno essere maggiormente esaustivi;
	    	\item rilasci multipli e in successione, che inizialmente punteranno a soddisfare i requisiti di primaria importanza mentre successivamente andranno ad adempiere a possibili funzionalità aggiuntive ed opzionali. Grazie a questi prototipi il \gl{proponente} avrà la possibilità di fornire una prima valutazione del lavoro nel periodo di produzione.
	    \end{itemize}
	    \subsection{Scadenze}
	    Il gruppo \AUTORE{} ha deciso di rispettare le seguenti scadenze:
	    \begin{itemize}
			\item \RR{}: 2017-01-11;
			\item \RP{}\ped{max}: 2017-03-06;
			\item \RQ{}: 2017-04-11;
			\item \RA{}: 2017-05-08;
	    \end{itemize}