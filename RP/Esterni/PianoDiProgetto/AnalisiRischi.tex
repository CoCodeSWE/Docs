\documentclass[PianoDiProgetto.tex]{subfiles}
\renewcommand{\arraystretch}{1.5}

\begin{document}
\section{Analisi dei rischi}
Al fine di evitare rallentamenti delle fasi di lavoro è stata effettuata una dettagliata analisi dei rischi, in modo da poter evitare le situazioni che portano alla creazione di eventi non pianificati, ove possibile. L'analisi si suddivide in quattro attività:
	\begin{itemize}
		\item \textbf{Identificazione}: individuare quali rischi possono incorrere durante lo sviluppo del \gl{progetto}, analizzarli e provare a intuire quali saranno le conseguenze se questi si verificano;
		\item \textbf{Analisi di fase}: valutare la probabilità di occorrenza di un rischio e analizzare la criticità delle conseguenze di questo rispetto all'andamento del \gl{progetto} nella fase in corso. Tutte le fasi sono illustrate nella sezione 5 riguardante la Pianificazione. Il \RESP{} di \gl{progetto} è incaricato  di fare una nuova analisi dei rischi ogni qualvolta si cambi fase;
		\item \textbf{Pianificazione e mitigazione}: identificare un metodo di controllo dei rischi così da renderli evitabili, nel qual caso non lo siano, pianificare delle contromisure per ridurre al minimo i danni; 
		\item \textbf{Controllo}: monitorare nel tempo ogni rischio, e nel caso si verifichi descriverne il riscontro effettivo e come il team ha reagito per ridurre al minimo i danni. 
	\end{itemize}
I rischi trattati vengono suddivisi in 5 sottosezioni per essere analizzati al meglio:
	\begin{itemize}
		\item livello tecnologico;
		\item livello degli strumenti;
		\item livello del personale;
		\item livello organizzativo;
		\item livello dei requisiti.
	\end{itemize}
Ogni rischio trattato ha una serie di caratteristiche necessarie per comprenderne la natura:
	\begin{itemize}
		\item nome;
		\item descrizione;
		\item risultati analisi:
			\begin{itemize}
				\item probabilità di occorrenza;
				\item grado di pericolosità;
				\item possibili conseguenze;
				\item riconoscimento;
				\item trattamento;
				\item attualizzazione nel periodo.
			\end{itemize}
	\end{itemize}
Ogni rischio verrà monitorato nel tempo e ne verrà indicato l'effettivo riscontro nella fase in corso.

\newpage

\begin{table}[h]
\centering
\label{Analisi dei Rischi}
\begin{tabular}{cc}
\hline
\textbf{Livello}       & \textbf{Tipologia} \\ \hline
Tecnologico   & \begin{tabular}[cl]{ll} Tecnologie adottate sconosciute \ref{sec:tas}
 \end{tabular}\\  \cline{2-2}
              & \begin{tabular}[cl]{ll} Guasti hardware e malfunzionamenti \gl{software} \ref{sec:ghs}\end{tabular} \\ \hline
Organizzativo & \begin{tabular}[cl]{ll} Valutazioni delle risorse \ref{sec:lo}\end{tabular}                   \\ \hline
Personale     & \begin{tabular}[cl]{ll} Problemi personali dei membri del team \ref{sec:ppdct}\end{tabular}     \\ \cline{2-2}
              & \begin{tabular}[cl]{ll} Problemi personali tra i membri del team \ref{sec:pptc}\end{tabular}     \\ \hline
Requisiti     &   Mancata comprensione \ref{sec:lr}\\ \hline 
\end{tabular}
\caption{Registro dei rischi}
\end{table}


\subsection{Livello tecnologico}
		\subsubsection{Tecnologie adottate sconosciute}
			\label{sec:tas}
	\begin{table}[h]
		\begin{center}
			\begin{tabular}{ | c | p{10cm} |}
				\hline
					\textbf{Descrizione}& per lo svolgimento e l'implementazione del \gl{progetto}, il team dovrà utilizzare una serie di tecnologie che nessun membro ha mai utilizzato; \\ \hline
				\textbf{Probabilità di occorrenza} & media  \\ \hline
				\textbf{Grado di pericolosità} & alto \\ \hline
				\textbf{Possibili conseguenze} & l'utilizzo di tecnologie sconosciute richiede tempo per la scelta e l'apprendimento, il che può portare ad un ritardo sulle date di consegna \\ \hline
				\textbf{Riconoscimento} & il \RESP{} deve verificare il grado di preparazione di ogni membro del gruppo relativo alle tecnologie utilizzate; \\ \hline
				\textbf{Trattamento} &  ogni membro del team deve studiare e approfondire autonomamente i documenti forniti dall'\AMM{} che spiegano come utilizzare propriamente le nuove tecnologie adottate; \\ \hline
				\textbf{Attualizzazione nel periodo AR} & \begin{itemize} 
				\item \PerAR: il rischio non si è presentato, in quanto le nuove tecnologie non sono ancora state prese in carico;
				\item \PerAD:
				\item \PerPA:
				\item \PerPD: 
				\end{itemize} \\
				\hline
				
			\end{tabular}
		\caption{Tecnologie adottate sconosciute}
		\end{center}	
	\end{table}

\clearpage	
	
	\subsubsection{Guasti hardware e malfunzionamenti \gl{software}}
		\label{sec:ghs}

		\begin{table}[h]
		\begin{center}
			\begin{tabular}{ | c | p{10cm} |}
				\hline
		 \textbf{Descrizione} & durante tutto il \gl{progetto} è possibile che si verifichino guasti hardware e/o malfunzionamenti \gl{software} ai computer usati dal team per sviluppare \\ \hline
		 \textbf{Probabilità di occorrenza} & basso \\ \hline
		 \textbf{Grado di pericolosità} & medio \\ \hline
	 \textbf{Possibili conseguenze} & il malfunzionamento di uno dei dispositivi del team può portare a perdita di dati e di conseguenza perdita di tempo in quanto si va a svolgere nuovamente un lavoro già effettuato \\ \hline
		 \textbf{Riconoscimento} & tutti i membri del team devono mantenere un'elevata attenzione sulle condizioni dei propri dispositivi \\ \hline
		 \textbf{Trattamento} & ogni componente del team si impegnerà a fare un backup giornaliero del lavoro effettuato su un dispositivo esterno al computer utilizzato per sviluppare; in caso di rotture hardware ogni membro possiede un altro dispositivo che gli permette di continuare il lavoro \\ \hline
		 \textbf{Attuazione nel periodo} & 
			\begin{itemize}
				\item \PerAR: il rischio non si è presentato;
				\item \PerAD: il rischio non si è presentato;
				\item \PerPA: il rischio non si è presentato;
				\item \PerPD: il rischio non si è presentato;
			\end{itemize} \\
			\hline
			
			\end{tabular}
		\caption{Guasti hardware e malfunzionamenti \gl{software}}
		\end{center}	
	\end{table}

\clearpage		
	\subsection{Livello organizzativo}
	\subsubsection{Valutazione delle risorse}
		\label{sec:lo}
		
		
\begin{table}[h]
		\begin{center}
			\begin{tabular}{ | c | p{10cm} |}
				\hline		
		
		 \textbf{Descrizione} & data la mancanza di esperienza con progetti di questa dimensione il team potrebbe incorrere in stime errate di valutazione delle risorse \\ \hline
		 \textbf{Probabilità di occorrenza} & alta \\ \hline
		 \textbf{Grado di pericolosità} & alto \\ \hline
		 \textbf{Possibili conseguenze} & un'errata stima delle risorse può portare ad uno spreco di queste o a ritardi nelle date di consegna \\ \hline
		 \textbf{Riconoscimento} & il rischio in questo caso è dinamico, per questo è necessario controllare lo sviluppo delle attività di progettazione periodicamente, tramite  \gl{verifica} da parte del \RESP{}, così da prendere atto di eventuali ritardi \\ \hline
		 \textbf{Trattamento} & ogni attività ha un periodo di slack, tale che l'eventuale ritardo di un'attività non condizioni le tempistiche delle altre \\ \hline
		 \textbf{Attuazione nel periodo} &
			\begin{itemize}
				\item \PerAR : il rischio non si è presentato;
				\item \PerAD:
				\item \PerPA:
				\item \PerPD:
			\end{itemize}
			\\ \hline
	
		\end{tabular}
		\caption{Valutazione delle risorse}
		\end{center}	
	\end{table}	
	
	\clearpage	
	\subsection{Livello personale}
		\subsubsection{Problemi personali dei componenti del team}
			\label{sec:ppdct}
			
\begin{table}[h]
		\begin{center}
			\begin{tabular}{ | c | p{10cm} |}
				\hline				
			

		 \textbf{Descrizione} & ogni membro del team avrà le sue necessità e i suoi impegni personali lungo la durata del \gl{progetto}. Risulta inevitabile il verificarsi di problemi organizzativi in seguito a sovrapposizioni di tali impegni \\ \hline
		 \textbf{Probabilità di occorrenza} & media \\ \hline
		 \textbf{Grado di pericolosità} & alto \\ \hline
		 \textbf{Possibili conseguenze} & ritardo nello svolgimento delle attività \\ \hline
		 \textbf{Riconoscimento} &  per creare un calendario sincronizzato e condiviso tra i membri del gruppo è necessario che vengano notificati al \RESP{} in maniera preventiva e tempestiva gli impegni di ognuno. Grazie a questa pratica è possibile ridurre al minimo tale rischio \\ \hline
		 \textbf{Trattamento} & nel caso un membro del team abbia un impegno che non gli permetta di proseguire il lavoro, il \RESP{} andrà a modificare la pianificazione prevista in modo da coprire l'assenza creatasi \\ \hline
		 \textbf{Attuazione nel periodo}& 
			\begin{itemize}
				\item \PerAR : i membri si sono impegnati per comunicare anticipatamente gli impegni personali, in questo modo il \RESP{} è riuscito a pianificare al meglio le attività da assegnare. Grazie a questa collaborazione il rischio non si è presentato;
				\item \PerAD: il rischio non si è presentato;
				\item \PerPA: il rischio non si è presentato;
				\item \PerPD: il rischio non si è presentato;
			\end{itemize}
			\\ \hline	
\end{tabular}
		\caption{Problemi personali dei componenti del team}
		\end{center}	
	\end{table}		
	
\clearpage	
	
		\subsubsection{Problemi personali tra i componenti del team}
			\label{sec:pptc}
\begin{table}[h]
		\begin{center}
			\begin{tabular}{ | c | p{10cm} |}
				\hline				
			

		 \textbf{Descrizione} & essendo la prima volta che i membri del team collaborano, potrebbero sorgere attriti o squilibri interni che andrebbero a danneggiare il clima lavorativo e porterebbero a ritardi nelle consegne \\ \hline
		 \textbf{Probabilità di occorrenza} & bassa \\ \hline
		 \textbf{Grado di pericolosità} & alto \\ \hline
		 \textbf{Possibili conseguenze} & ritardo nello svolgimento delle attività \\ \hline
		 \textbf{Riconoscimento} & tutti i membri del gruppo devono avere una comunicazione costante con il \RESP{} il quale si occuperà di monitorare i rapporti tra i collaboratori \\ \hline
		 \textbf{Trattamento} & nel caso di contrasti tra membri del gruppo, il \RESP{} provvederà ad assegnare a tali membri attività differenti con il minimo contatto (nel limite del possibile) \\ \hline
		 \textbf{Attuazione nel periodo} &
		\begin{itemize}
				\item \PerAR : il rischio non si è presentato;
				\item \PerAD: il rischio non si è presentato;
				\item \PerPA: il rischio non si è presentato;
				\item \PerPD: il rischio non si è presentato;
		\end{itemize}
		\\ \hline
	
\end{tabular}
		\caption{Problemi personali tra componenti del team}
		\end{center}	
	\end{table}		

\clearpage		
	\subsection{Livello dei Requisiti}
		\subsubsection{Incomprensione e scelte non ottimali}
			\label{sec:lr}
\begin{table}[h]
		\begin{center}
			\begin{tabular}{ | c | p{10cm} |}
				\hline				
			

		 \textbf{Descrizione} & è possibile che alcuni requisiti individuati dagli \ANP{} siano fraintesi, superficiali o errati rispetto alle aspettative del \gl{proponente} \PROPONENTE{}. Inoltre esiste la probabilità che qualche requisito venga modificato, eliminato o aggiunto durante il corso del \gl{progetto} \\ \hline
		 \textbf{Probabilità di occorrenza} & alta \\ \hline
		 \textbf{Grado di pericolosità} & alto \\ \hline
		 \textbf{Possibili conseguenze} & sviluppo di un \gl{prodotto} non consono alle aspettative del \gl{proponente} \\ \hline
		 \textbf{Riconoscimento} & avere una costante comunicazione con il \gl{proponente} \PROPONENTE{} durante la fase di \ARdoc{} in modo da chiarire tutte le incomprensioni e assicurare la concordanza sui requisiti del \gl{prodotto} \\ \hline
		 \textbf{Trattamento} & si dovranno effettuare incontri con il \gl{proponente} \PROPONENTE{} così da poter correggere eventuali errori indicati dal committente durante la revisione \\ \hline
		 \textbf{Attuazione nel periodo}&
		\begin{itemize}
				\item \PerAR : il rischio non si è presentato;
				\item \PerAD: il rischio si è presentato in minima parte, in quanto l'esito della \RR{} ha portato a modifiche dei requisiti, accettate in seguito dal proponente;
				\item \PerPA: il rischio non si è presentato;
				\item \PerPD: il rischio non si è presentato;
		\end{itemize}
		\\ \hline

\end{tabular}
		\caption{Incomprensione e scelte non ottimali}
		\end{center}	
	\end{table}	
	
\end{document}










