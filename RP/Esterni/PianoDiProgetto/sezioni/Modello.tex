\documentclass[./PianoDiProgetto.tex]{subfiles}
\begin{document}
  \section{Modello di sviluppo}

  Il modello di sviluppo scelto per il sistema è il modello incrementale: il
  progetto è stato suddiviso in periodi, i quali sono caratterizzati da
  obbiettivi ed attività differenti. Inoltre nel corso del progetto saranno
  fissati diversi incontri col proponente, i quali serviranno per ricevere un
  feedback e migliorare il sistema di conseguenza.

  \subsection{Periodo AR - Analisi dei Requisiti}

  In questa fase iniziale vengono individuati gli strumenti necessari sia alla
  collaborazione tra i membri del gruppo che alla stesura della documentazione.
  Si individua inoltre il progetto da sviluppare e se ne analizzano i requisiti.

  \subsection{Periodo AD - Analisi di Dettaglio}

  In questo periodo i requisiti individuati vengono consolidati ed ampliati. Il
  documento di \ARdoc viene modificato in base all'esito
  della \RR. Vengono corretti e verificati anche gli altri documenti.

  \subsection{Periodo PA - Progettazione Architetturale}

  L'obbiettivo di questo periodo è la progettazione dell'architettura di alto
  livello del sistema e la produzione del documento di \textit{Specifica Tecnica}.

  \subsection{Periodo PD - Progettazione di Dettaglio iniziale}

  In questo periodo inizia la progettazione dettagliata del sistema, nella quale si definiscono
  in particolare il comportamento e l'interazione tra i vari componenti. Vengono
  inoltre prodotti una documentazione dettagliata di tutte le API ed un piano
  di test di unità.

  \subsection{Periodo C - Codifica e progettazione di dettaglio}

  Durante questo periodo viene scritto il codice del sistema e viene incrementata la progettazione di dettaglio. L'obbiettivo di questa fase è quello di consegnare al proponente un prodotto qualificato.

  \subsection{Periodo V - Validazione}

  In questa fase finale del progetto viene effettuata la validazione del sistema ed il
  collaudo dello stesso, in modo da verificare che il prodotto soddisfi tutti i requisiti
  dell'\ARdoc. 

\end{document}
