\section{Struttura del database DynamoDB}
In questa appendice descriviamo la struttura del database DynamoDB, come richiesto dal proponente \PROPONENTE. \\
Sebbene DynamoDB richieda di specificare solamente la primary key (partition key oppure partition key e sort key) per la creazione di una tabella, abbiamo deciso di riportare una struttura più descrittiva all'interno del documento. La descrizione di ciascuna tabella riporta le informazioni relative alla chiave primaria e agli attributi che ogni item possiede quando viene inserito all'interno del DB.
\\ \\La tabella \textbf{Conversation} descrive l'intera interazione tra un utente del sistema e l'assistente virtuale, dove "messages" memorizza tutti i messaggi che vengono scambiati tra i due attori.
\begin{lstlisting}[language=json,firstnumber=1]
TableName: "Conversation",
KeySchema: 
[
  {
    AttributeName: "guest_id",
    KeyType: "HASH"
  },
  {
    AttributeName: "session_id",
    KeyType: "RANGE"	
  }
],
AttributeDefinitions: 
[
  "guest_id": {"S"},
  "session_id": {"S"},
  "messages":
  {
    "L": 
    [
      {
        "sender": {"S"},
        "text": {"S"},
        "timestamp": {"S"}
      }
    ]
  }
]
\end{lstlisting}
\newpage
La tabella \textbf{Guest} descrive gli ospiti che hanno fatto visita all'azienda.
\begin{lstlisting}[language=json,firstnumber=1]
TableName: "Guest",
KeySchema: 
[
  {
    AttributeName: "company",
    KeyType: "HASH"
  },
  {
    AttributeName: "name",
    KeyType: "RANGE"	
  }
],
AttributeDefinitions: 
[
  "company": {"S"},
  "name": {"S"},
  "conversations": 
  {
    "L":
    [
      {
        "session_id": {"S"}
      }
    ]
  },
  "met": {"SS"}
]
\end{lstlisting}
\newpage
La tabella \textbf{Rule} descrive le direttive che gli amministratori hanno inserito per modificare il comportamento dell'assistente virtuale. Ogni direttiva contiene un array di "target", ossia il gruppo di persone a cui deve essere applicata.
\begin{lstlisting}[language=json,firstnumber=1]
TableName: "Rule",
KeySchema: 
[
  {
    AttributeName: "id",
    KeyType: "HASH"
  }
],
AttributeDefinitions: 
[
  "id": {"N"},
  "name": {"S"},
  "targets": 
  {
    "L": 
    [
      {
        "company": {"S"},
        "member": {"S"},
        "name": {"S"}
      }
    ]
  },
  "task":
  {
    "M":
    {
      "params": {"M"},
      "task": {"S"}
    }
  },
  "enabled": {"BOOL"}
]
\end{lstlisting}

La tabella \textbf{Task} descrive il compito assegnato ad ogni direttiva.
\begin{lstlisting}[language=json,firstnumber=1]
TableName: "Task",
KeySchema: 
[
  {
    AttributeName: "type",
    KeyType: "HASH"	
  }
],
AttributeDefinitions: 
[
  "type": {"S"}
]
\end{lstlisting}
\newpage
La tabella \textbf{User} descrive gli utenti registrati dal sistema. 
\begin{lstlisting}[language=json,firstnumber=1]
TableName: "User",
KeySchema: 
[
  {
    AttributeName: "username",
    KeyType: "HASH"
  }	
],
AttributeDefinitions: 
[
  "username": {"S"},
  "name": {"S"},
  "password": {"S"},
  "slack_channel": {"S"},
  "sr_id": {"S"}
]
\end{lstlisting}

La tabella \textbf{Agent} descrive gli agenti di api.ai che fanno riferimento alle varie applicazioni.
\begin{lstlisting}[language=json,firstnumber=1]
TableName: "Agent",
KeySchema:
[
  {
    AttributeName: "token",
    KeyType: "HASH"
  }	
],
AttributeDefinitions: 
[
  "token": {"S"},
  "name": {"S"},
  "lang": {"S"}
]
\end{lstlisting}