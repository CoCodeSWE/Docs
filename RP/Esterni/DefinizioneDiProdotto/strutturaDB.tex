\section{Struttura del database DynamoDB}
In questa appendice descriviamo la struttura del database DynamoDB, come richiesto dal proponente \PROPONENTE.
\\ \\La tabella \textbf{Conversation} descrive l'intera interazione tra un utente del sistema e l'assistente virtuale, dove "messages" memorizza tutti i messaggi che vengono scambiati tra i due attori.
\begin{lstlisting}[language=json,firstnumber=1]
TableName: "Conversation",
KeySchema: [
{
	AttributeName: "guest_id",
	KeyType: "HASH"
},
{
	AttributeName: "session_id",
	KeyType: "RANGE"	
}
],
"guest_id": {"S"},
"session_id": {"S"},
"messages":
{
	M:
	{
		"sender": {"S"},
		"text": {"S"},
		"timestamp": {"S"}
	}
}
\end{lstlisting}

La tabella \textbf{Guest} descrive gli ospiti che hanno fatto visita all'azienda.
\begin{lstlisting}[language=json,firstnumber=1]
TableName: "Guest",
KeySchema: [
{
	AttributeName: "company",
	KeyType: "HASH"
},
{
	AttributeName: "name",
	KeyType: "RANGE"	
}
],
"company": {"S"},
"name": {"S"},
"conversations": [{"Conversation"}],
"met": [{"S"}]
\end{lstlisting}
\newpage
La tabella \textbf{Rule} descrive le direttive che gli amministratori hanno inserito per modificare il comportamento dell'assistente virtuale. Ogni direttiva contiene un array di "target", ossia il gruppo di persone a cui deve essere applicata.
\begin{lstlisting}[language=json,firstnumber=1]
TableName: "Rule",
KeySchema: [
{
	AttributeName: "id",
	KeyType: "HASH"
}
],
"id": {"N"},
"name": {"S"},
"targets": {
	M:
	{
		"company": {"S"},
       	"member": {"S"},
       	"name": {"S"}
	}
},
"task":
{
	M:
	{
		"params": {"B"},
       	"task": {"S"}
	}
},
"enabled": {"BOOL"}
\end{lstlisting}

La tabella \textbf{Task} descrive il compito assegnato ad ogni direttiva.
\begin{lstlisting}[language=json,firstnumber=1]
TableName: "Task",
KeySchema: [
{
	AttributeName: "type",
	KeyType: "HASH"	
}
],
"type": {"S"}
\end{lstlisting}

La tabella \textbf{User} descrive gli utenti registrati dal sistema. 
\begin{lstlisting}[language=json,firstnumber=1]
TableName: "User",
KeySchema: [
{
	AttributeName: "username",
	KeyType: "HASH"
}	
],
"username": {"S"},
"name": {"S"},
"password": {"S"},
"slack_channel": {"S"},
"sr_id": {"S"}
\end{lstlisting}

La tabella \textbf{Agent} descrive gli agenti di api.ai che fanno riferimento alle varie applicazioni.
\begin{lstlisting}[language=json,firstnumber=1]
TableName: "Agent",
KeySchema: [
{
	AttributeName: "token",
	KeyType: "HASH"
}	
],
"token": {"S"},
"name": {"S"},
"lang": {"S"}
\end{lstlisting}