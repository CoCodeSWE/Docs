\section{Design Patterns}
  \subsection{Architetturali}
    \subsubsection{Architettura a microservizi}
      \begin{itemize}
       \item \textbf{Scopo:}l'architettura a microservizi è un approccio allo sviluppo di una singola applicazione come insieme di piccoli servizi, ciascuno dei quali viene eseguito da un proprio processo e comunica con un meccanismo snello, spesso una HTTP API;
       \item \textbf{Vantaggi:}
	      \begin{itemize}
	       \item ogni microservizio è relativamente piccolo, quindi più semplice da implementare e da capire per gli sviluppatori;
	       \item ogni microservizio è indipendente dagli altri; è quindi possibile distribuire nuove versioni più frequentemente e isolare i possibili errori.
	      \end{itemize}

       \item \textbf{Svantaggi:}
	\begin{itemize}
	 \item l'architettura risulta maggiormente complessa perchè risulta essere un sistema distribuito;
	 \item la gestione di più microservizi potrebbe risultare in un carico di lavoro maggiore rispetto ad una sua versione monolitica.
	\end{itemize}
       \item \textbf{Utilizzo:}
      \end{itemize}
      
    \subsubsection{Architettura event-driven}
      \begin{itemize}
       \item \textbf{Scopo:}anche se non è un vero e proprio pattern, l'architettura event-driven è un particolare tipo di architettura asicrona per sistemi distribuiti basata sugli eventi.
	\item \textbf{Vantaggi:}
	  \begin{itemize}
	   \item per definizione, questo tipo di architettura è particolarmente adatto ad ambienti di tipo asicrono basati sugli eventi, come l'interazione con gli utenti.
	  \end{itemize}
	\item \textbf{Svantaggi:}
	  \begin{itemize}
	   \item i sistemi che utilizzano tale architettura sono spesso distribuiti: ciò comporta un maggiore livello di complessità.
	  \end{itemize}
	\item \textbf{Utilizzo:}
	\end{itemize}
     
     \subsubsection{Client-side discovery}
      \begin{itemize}
       \item \textbf{Scopo:}all'interno di un'architettura a microservizi, questi ultimi si trovano spesso in posizioni non fissate in quanto decise dinamicamente. Un metodo per la loro localizzazione consiste nel pattern Client-side discovery, che consiste nella richiesta della posizione di uno specifico microservizio da parte del client ad un registro, che conosce le posizioni di tutte le istanze dei servizi. 
	\item \textbf{Vantaggi:}
	  \begin{itemize}
	   \item permette di allocare dinamicamente diverse istanze di diversi servizi.
	  \end{itemize}
	\item \textbf{Svantaggi:}
	  \begin{itemize}
	   \item crea dipendenze tra il registro e il client.
	  \end{itemize}
	\item \textbf{Utilizzo:}
	\end{itemize}

    
   \subsubsection{Data Access Object}
      \begin{itemize}
       \item \textbf{Scopo:}
	\item \textbf{Vantaggi:}
	  \begin{itemize}
	   \item 
	  \end{itemize}
	\item \textbf{Svantaggi:}
	  \begin{itemize}
	   \item .
	  \end{itemize}
	\item \textbf{Utilizzo:}
	\end{itemize}
	
    \subsubsection{Dependency Injection?}
     \begin{itemize}
       \item \textbf{Scopo:}
	\item \textbf{Vantaggi:}
	  \begin{itemize}
	   \item 
	  \end{itemize}
	\item \textbf{Svantaggi:}
	  \begin{itemize}
	   \item .
	  \end{itemize}
	\item \textbf{Utilizzo:}
	\end{itemize}
  
  \subsection{Strutturali}
  
    \subsubsection{Facade}
      \begin{itemize}
       \item \textbf{Scopo:}
	\item \textbf{Vantaggi:}
	  \begin{itemize}
	   \item 
	  \end{itemize}
	\item \textbf{Svantaggi:}
	  \begin{itemize}
	   \item .
	  \end{itemize}
	\item \textbf{Utilizzo:}
	\end{itemize}
	
    \subsubsection{Adapter}
      \begin{itemize}
       \item \textbf{Scopo:}
	\item \textbf{Vantaggi:}
	  \begin{itemize}
	   \item 
	  \end{itemize}
	\item \textbf{Svantaggi:}
	  \begin{itemize}
	   \item .
	  \end{itemize}
	\item \textbf{Utilizzo:}
	\end{itemize}
	
  \subsection{Creazionali}
  
    \subsubsection{Singleton}
      \begin{itemize}
       \item \textbf{Scopo:}
	\item \textbf{Vantaggi:}
	  \begin{itemize}
	   \item 
	  \end{itemize}
	\item \textbf{Svantaggi:}
	  \begin{itemize}
	   \item .
	  \end{itemize}
	\item \textbf{Utilizzo:}
	\end{itemize}
	
  \subsection{Comportamentali}
  
    \subsubsection{Observer}
      \begin{itemize}
       \item \textbf{Scopo:}
	\item \textbf{Vantaggi:}
	  \begin{itemize}
	   \item 
	  \end{itemize}
	\item \textbf{Svantaggi:}
	  \begin{itemize}
	   \item .
	  \end{itemize}
	\item \textbf{Utilizzo:}
	\end{itemize}