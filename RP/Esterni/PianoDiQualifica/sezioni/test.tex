\documentclass[PianoDiQualifica.tex]{subfiles}

\begin{document}

\section{Test}
Al fine di produrre software di qualità, il gruppo ha strutturato dei test atti a verificare che le funzionalità del software prodotto corrispondano alle attese.
Tali test sono ottenuti dall'applicazione delle tecniche di analisi dinamica descritte nel documento \NPdocRP{}. Inoltre, devono possedere le seguenti caratteristiche:
\begin{itemize}
	\item devono essere ripetibili al fine di fornire informazioni utili per poter eseguire operazioni di correzione, ove sia necessario;
	\item devono essere tracciabili al fine di classificare le informazioni ottenute per garantire una più facile consultazione;
\end{itemize}
Le tipologie di test che verranno eseguiti sono:
\begin{itemize}
\item \textbf{Test di validazione:} test che hanno lo scopo di verificare che tutte le funzionalità richieste dal proponente siano soddisfatte. A questo scopo, attraverso una serie di
azioni, si andrà a simulare il comportamento generale del software e dell'utente che interagisce con esso;
\item \textbf{Test di unità: } test che  hanno lo scopo di verificare il corretto funzionamento delle unità. Le unità, individuate durante la fase di progettazione, sono le
		più piccole parti del sistema dotate di funzionamento proprio. Questo si traduce nel verificare metodi e classi scritte dai \PRP{};
\item \textbf{Test di integrazione: } test che hanno lo scopo di verificare il corretto funzionamento delle varie componenti. In particolare, l'obiettivo è quello di testare le varie componenti prodotte dall'unione delle unità. Nel determinarli, è stato scelto l'approccio top-down, in maniera tale da sottoporre per prime le componenti di livello più alto ai test, integrandole fin da subito. Così facendo anche la logica di alto livello e il flusso di dati vengono sottoposti a test fin da subito; sarà perciò necessario simulare le componenti di livello più basso con degli stub. Una volta codificate, le componenti di più basso livello dovranno a loro volta essere
integrate e testate. In questo modo, i difetti rilevati dai test saranno spesso attribuiti all'ultima componente aggiunta. In \DPdoc{} sono descritte come le componenti devono interagire tra loro;
\item \textbf{Test di sistema: }test che hanno lo scopo di verificare il corretto funzionamento del prodotto software. Inoltre verranno verificate la sua robustezza in presenza di
		possibili malfunzionamenti e il suo comportamento di fronte a possibili violazioni;
\item \textbf{Test di regressione: } test che hanno lo scopo di verificare che una modifica dell'implementazione del software non ne comprometta la qualità. Consistono nella ripetizione di test di unità o integrazione sul componente modificato.			
\end{itemize}
	\paragraph{Test di Validazione}
I test di \gl{validazione} saranno identificati secondo quanto riportato nel documento \NPdoc{}.
\normalsize
\begin{longtable}{|c|>{}m{8cm}|c|}
\hline
\textbf{Id Test} & \textbf{Descrizione} & \textbf{Stato}\\
\hline
\endhead
\hypertarget{TVFO1}{TVFO1} & L'utente deve \gl{verifica}re che il \gl{sistema} riesca a riconoscerlo come ospite o possibile amministratore. All'utente viene richiesto di:
\begin{itemize}
\item fornire nome e cognome;
\end{itemize} & \textit{Non Implementato}\\ \hline
\hypertarget{TVFO1.1.2}{TVFO1.1.2} & L'utente deve verificare che il sistema ne permetta l'accesso all'area amministrativa tramite l'assistente virtuale. All'utente viene richiesto di:
\begin{itemize}
\item comunicare i propri dati identificativi;
\item verificare che il sistema riconosca l'utente come un possibile amministratore non autenticato;
\item comunicare l'intento di volersi autenticare come amministratore;
\item comunicare la frase per lo \gl{Speaker Recognition};
\item verificare l'accesso all'area amministrativa.
\end{itemize} & \textit{Non Implementato}\\ \hline
\hypertarget{TVFO2.1}{TVFO2.1} & L'utente deve verificare che il sistema permetta la creazione di una nuova \gl{direttiva}. All'utente viene richiesto di:
\begin{itemize}
\item autenticarsi come amministratore;
\item comunicare l'intento di voler creare una nuova direttiva;
\item inserire il nome della direttiva;
\item inserire la funzione della direttiva;
\item inserire il target della direttiva;
\item confermare la creazione della direttiva;
\item verificare che la direttiva sia stata creata correttamente.
\end{itemize}
 & \textit{Non Implementato}\\ \hline
\hypertarget{TVFO2.1.1.6}{TVFO2.1.1.6} & L'utente deve verificare che, durante la creazione di una direttiva, l'inserimento di dati non validi (funzione o target della direttiva inesistenti) comporti la visualizzazione di un messaggio d'errore. All'utente viene richiesto di:
\begin{itemize}
\item autenticarsi come amministratore;
\item comunicare l'intento di voler creare una nuova direttiva;
\item inserire il nome della direttiva;
\item inserire una funzione per la direttiva che non sia valida (inesistente);
\item inserire un target per la direttiva che non sia valido (inesistente);
\item verificare la comparsa di un messaggio d'errore.
\end{itemize} & \textit{Non Implementato}\\ \hline
\hypertarget{TVFO2.1.2}{TVFO2.1.2} & L'utente deve verificare che il sistema permetta l'eliminazione di una direttiva. All'utente viene richiesto di:
\begin{itemize}
\item autenticarsi come amministratore;
\item comunicare l'intento di voler eliminare una direttiva;
\item comunicare il nome della direttiva da eliminare;
\item confermare l'eliminazione della direttiva;
\item verificare che la direttiva sia stata eliminata correttamente.
\end{itemize}
 & \textit{Non Implementato}\\ \hline
\hypertarget{TVFO2.1.4}{TVFO2.1.4} & L'utente deve verificare che il sistema permetta la visualizzazione di una direttiva. All'utente viene richiesto di:
\begin{itemize}
\item autenticarsi come amministratore;
\item comunicare l'intento di voler visualizzare una direttiva;
\item verificare che il sistema permetta la visualizzazione di nome, funzione, target,funzionalità e abilitazione della direttiva.
\end{itemize} & \textit{Non Implementato}\\ \hline
\hypertarget{TVFO2.2}{TVFO2.2} & L'utente deve verificare che il sistema permetta la modifica dei dati del proprio profilo. All'utente viene richiesto di:
\begin{itemize}
\item autenticarsi come amministratore;
\item comunicare l'intento di voler modificare il proprio profilo;
\item comunicare nome e cognome;
\item confermare la modifica;
\item verificare che la modifica sia stata effettuata;
\end{itemize}
 & \textit{Non Implementato}\\ \hline
\hypertarget{TVFO3.1}{TVFO3.1} & L'utente deve verificare che sia possibile comunicare al sistema la persona che si desidera incontrare. All'utente viene richiesto di:
\begin{itemize}
\item aver comunicato i propri dati al sistema ed essere riconosciuti come ospiti;
\item comunicare la persona che si desidera raggiungere;
\item verificare che il sistema abbia capito le informazioni comunicate;
\end{itemize} & \textit{Non Implementato}\\ \hline
\hypertarget{TVFO5}{TVFO5} & L'utente deve verificare che, nel caso in cui il sistema nel caso non riesca ad interpretare la risposta, chieda nuovamente l'informazione all'utente. All'utente viene richiesto di:
\begin{itemize}
\item comunicare al sistema qualcosa che non può essere interpretato da esso;
\item verificare che il sistema richieda nuovamente l'informazione.
\end{itemize} & \textit{Non Implementato}\\ \hline
\hypertarget{TVFO7}{TVFO7} & L'utente deve verificare che, nel caso in cui esso sia già stato un ospite in passato, il sistema lo riconosca. All'utente viene richiesto di:
\begin{itemize}
\item aver comunicato i propri dati al sistema ed essere riconosciuti come ospiti;
\item verificare che il sistema riconosca l'utente come qualcuno che è già stato un ospite in passato.
\end{itemize}
 & \textit{Non Implementato}\\ \hline
\caption[Test di Validazione]{Test di Validazione}
\label{tabella:test0}
\end{longtable}
\clearpage

\paragraph{Test di Sistema}
I test di sistema saranno identificati secondo quanto riportato nel documento \NPdoc{}.
\normalsize
\begin{longtable}{|c|>{}m{8cm}|c|}
\hline
\textbf{Id Test} & \textbf{Descrizione} & \textbf{Stato}\\
\hline
\endhead
\hypertarget{TSFO1}{TSFO1} & Il sistema deve poter riconoscere un utente come ospite. & \textit{Non Implementato}\\ \hline
\hypertarget{TSFO1.1.2.1}{TSFO1.1.2.1} & Il sistema deve permettere all'utente di autenticarsi come amministratore tramite frase di riconoscimento. & \textit{Non Implementato}\\ \hline
\hypertarget{TSFO2.1.1}{TSFO2.1.1} & Il sistema deve permettere all'amministratore di poter creare una nuova direttiva. & \textit{Non Implementato}\\ \hline
\hypertarget{TSFO2.1.2}{TSFO2.1.2} & Il sistema deve permettere all'amministratore di poter eliminare una direttiva di cui ha i privilegi. & \textit{Non Implementato}\\ \hline
\hypertarget{TSFO2.1.4}{TSFO2.1.4} & Il sistema deve permettere all'amministratore di poter visualizzare le \gl{direttive} di cui ha i privilegi. & \textit{Non Implementato}\\ \hline
\hypertarget{TSFO2.2.1}{TSFO2.2.1} & L'amministratore deve poter modificare il nome e cognome del suo profilo. & \textit{Non Implementato}\\ \hline
\hypertarget{TSFO3.1}{TSFO3.1} & Il sistema deve permettere all'ospite di richiedere la persona desiderata. & \textit{Non Implementato}\\ \hline
\hypertarget{TSFO5}{TSFO5} & Il sistema deve richiedere nuovamente le informazioni nel caso in cui non fossero state comprese. & \textit{Non Implementato}\\ \hline
\hypertarget{TSFO7}{TSFO7} & Il sistema deve essere in grado di riconoscere ospiti già stati in visita all'azienda. In questo caso, l'assistente virtuale deve poter prevedere le sue necessità. & \textit{Non Implementato}\\ \hline
\hypertarget{TSFO8}{TSFO8} & Il sistema deve sollecitare la persona desiderata o eventualmente avvisare gli altri membri dell'azienda su richiesta dell'ospite. & \textit{Non Implementato}\\ \hline
\hypertarget{TSFO13}{TSFO13} & Il sistema deve comunicare nell'opportuno canale di \gl{Slack} le informazioni raccolte durante l'interazione con l'ospite. & \textit{Non Implementato}\\ \hline
\hypertarget{TSVO1.1}{TSVO1.1} & Vogliamo testare che il \gl{software} funzioni correttamente in un PC con sistema operativo Windows 7 o superiore.
 & \textit{Non Implementato}\\ \hline
\hypertarget{TSVO4}{TSVO4} & Le pagine HTML devono essere validate. & \textit{Non Implementato}\\ \hline
\hypertarget{TSVO5}{TSVO5} & I fogli di stile CSS devono essere validati. & \textit{Non Implementato}\\ \hline
\hypertarget{TSVO10}{TSVO10} & Vogliamo testare che il software funzioni correttamente con il \gl{browser} Google Chrome versione 53 o superiore. & \textit{Non Implementato}\\ \hline
\caption[Test di Sistema]{Test di Sistema}
\label{tabella:test1}
\end{longtable}
\clearpage

\paragraph{Test di Integrazione}
I test di integrazione saranno identificati secondo quanto riportato nel documento \NPdoc{}.
\normalsize
\begin{longtable}{|c|>{}m{8cm}|c|}
\hline
\textbf{Id Test} & \textbf{Descrizione} & \textbf{Stato}\\
\hline
\endhead
\hypertarget{TI1}{TI1} & Vogliamo verificare che \file{Recorder}, \file{Logic}, \file{Utility}, \file{TTS}, \file{ConversationApp} e \file{ApplicationManager} interagiscano correttamente fra loro. & \textit{Non Implementato}\\ \hline
\hypertarget{TI2}{TI2} & Vogliamo verificare che \file{APIGateway}, \file{STT}, \file{VirtualAssistant},  \file{Users}, \file{Guests}, \file{Rules}, \file{Members}, \file{Conversations} e \file{Events} interagiscano correttamente tra di loro. Inoltre, vogliamo verificare che interagiscano correttamente con i servizi e librerie esterne AWS, Speaker Recognition, \gl{Speech to text} IBM Watson, api.ai, Slack e \file{WebAPI}. & \textcolor{green}{\textit{Superato}}\\ \hline
\hypertarget{TI3}{TI3} & Vogliamo verificare che le seguenti classi, contenute in \file{Client::ApplicationManager}, interagiscano tra loro correttamente: \file{ApplicationManagerObserver}, \file{ApplicationRegistryClient}, \file{ApplicationRegistryLocalClient}, \file{ApplicationLocalRegistry}, \file{Manager}, \file{State}, \file{Application}, \file{ApplicationPackage}. & \textcolor{green}{\textit{Superato}}\\ \hline
\hypertarget{TI4}{TI4} & Vogliamo verificare che le seguenti classi, contenute in \file{Client::Logic}, interagiscano tra loro correttamente: \file{DataArrivedSubject}, \file{DataArrivedObservable}, \file{Logic}, \file{HttpError}, \file{HttpPromise}, \file{LogicObserver}. & \textit{Non Implementato}\\ \hline
\hypertarget{TI5}{TI5} & Vogliamo verificare che le seguenti classi, contenute in \file{Client::Recorder}, interagiscano tra loro correttamente: \file{Recorder}, \file{RecorderWorker}, \file{RecorderMsg}, \file{RecorderWorkerMsg}, \file{RecorderWorkerConfig}, \file{RecorderConfig}, \file{SpeechEndSubject}, \file{SpeechEndObservable}. & \textcolor{green}{\textit{Superato}}\\ \hline
\hypertarget{TI6}{TI6} & Vogliamo verificare che le seguenti classi, contenute in \file{Client::TTS}, interagiscano tra loro correttamente: \file{TTSConfig}, \file{Player}, \file{PlayerObserver}. & \textcolor{green}{\textit{Superato}}\\ \hline
\hypertarget{TI7}{TI7} & Vogliamo verificare che le seguenti classi, contenute in \file{Client::Utility}, interagiscano tra loro correttamente: \file{BoolSubject}, \file{BoolObservable}, \file{BoolObserver}. & \textcolor{green}{\textit{Superato}}\\ \hline
\hypertarget{TI8}{TI8} & Vogliamo verificare che le seguenti classi, contenute in \file{Back-end::APIGateway}, interagiscano tra loro correttamente: \file{VocalAPI}, \file{Enrollement}. & \textcolor{green}{\textit{Superato}}\\ \hline
\hypertarget{TI9}{TI9} & Vogliamo verificare che le seguenti classi, contenute in \file{Back-end::Users}, interagiscano tra loro correttamente: \file{UsersDAODynamoDB}, \file{User}, \file{UsersService}. & \textcolor{green}{\textit{Superato}}\\ \hline
\hypertarget{TI10}{TI10} & Vogliamo verificare che le seguenti classi, contenute in \file{Back-end::Rules}, interagiscano tra loro correttamente: \file{Rule}, \file{RulesDAODynamoDB}, \file{RuleTarget}, \file{RuleTaskInstance}, \file{RulesService}, \file{TasksDAODynamoDB}, \file{Task}. & \textcolor{green}{\textit{Superato}}\\ \hline
\hypertarget{TI11}{TI11} & Vogliamo verificare che le seguenti classi, contenute in \file{Back-end::VirtualAssistant}, interagiscano tra loro correttamente: \file{VAService}, \file{ApiAIVAAdapter}, \file{VAQuery}, \file{Agent}, \file{AgentDAODynamoDB}, \file{VAEventObject}, \file{Fulfillment}, \file{MsgObject}, \file{ButtonObject}. & \textcolor{green}{\textit{Superato}}\\ \hline
\hypertarget{TI12}{TI12} & Vogliamo verificare che le seguenti classi, contenute in \file{Back-end::Member}, interagiscano tra loro correttamente: \file{MembersSlackDAO}, \file{Member}. & \textcolor{green}{\textit{Superato}}\\ \hline
\hypertarget{TI13}{TI13} & Vogliamo verificare che le seguenti classi, contenute in \file{Back-end::Guests}, interagiscano tra loro correttamente: \file{Guest}, \file{GuestDAODynamoDB}. & \textcolor{green}{\textit{Superato}}\\ \hline
\hypertarget{TI14}{TI14} & Vogliamo verificare che le seguenti classi, contenute in \file{Back-end::Conversations}, interagiscano tra loro correttamente: \file{ConversationDAODynamoDB}, \file{Conversation}, \file{ConversationMsg}.
 & \textcolor{green}{\textit{Superato}}\\ \hline
\hypertarget{TI15}{TI15} & Vogliamo verificare che le seguenti classi, contenute in \file{Back-end::Events}, interagiscano tra loro correttamente: \file{SNSRecord}, \file{SNSMessage}.
 & \textcolor{green}{\textit{Superato}}\\ \hline
\hypertarget{TI16}{TI16} & Vogliamo verificare che le seguenti classi, contenute in \file{Back-end::Notifications}, interagiscano tra loro correttamente: \file{NotificationChannel}, \file{Purpose}, \file{Topic}, \file{NotificationMessage}, \file{Attachment}, \file{Action}, \file{ConfirmationFields}. & \textcolor{green}{\textit{Superato}}\\ \hline
\hypertarget{TI17}{TI17} & Vogliamo verificare che le seguenti classi, contenute in \file{Back-end::Utility}, interagiscano tra loro correttamente: \file{WebhookRequest}, \file{ProcessingResult}, \file{LamdaIdEvent}, \file{PathIdParam}. & \textcolor{green}{\textit{Superato}}\\ \hline
\hypertarget{TI18}{TI18} & Vogliamo verificare che le seguenti classi, contenute in \file{Client::ConversationApp}, interagiscano tra loro correttamente: \file{ConversationApp}, \file{ConversationActionObserver}, \file{ConversationActionObservable}, \file{ConversaionActionSubject}, \file{ConversationAction}, \file{ConversationDispatcher}, \file{ConversationView}, \file{MessageStore}. & \textcolor{green}{\textit{Superato}}\\ \hline
\caption[Test di Integrazione]{Test di Integrazione}
\label{tabella:test2}
\end{longtable}
\clearpage

\paragraph{Test di Unità}
I test di unita saranno identificati secondo quanto riportato nel documento \NPdoc{}.
\normalsize
\begin{longtable}{|c|>{}m{8cm}|c|}
\hline
\textbf{Id Test} & \textbf{Descrizione} & \textbf{Stato}\\
\hline
\endhead
\hypertarget{TU1}{TU1} & Vogliamo testare che il metodo imposti il campo \file{status} della risposta a 200 e che il campo \file{speech} sia uguale al campo \file{fulfillment.speech} del corpo della richiesta, in caso il token sia presente e valido. & \textcolor{green}{\textit{Superato}}\\ \hline
\hypertarget{TU2}{TU2} & Vogliamo testare che il metodo imposti il campo \file{status} della risposta a 403 in caso di mancata autenticazione, ovvero token assente o non valido. & \textcolor{green}{\textit{Superato}}\\ \hline
\hypertarget{TU3}{TU3} & Vogliamo testare che il metodo sollevi un'eccezione alla sua chiamata. & \textcolor{green}{\textit{Superato}}\\ \hline
\hypertarget{TU4}{TU4} & Vogliamo testare che il metodo accetti un parametro di tipo \file{Agent} senza generare eccezioni. & \textcolor{green}{\textit{Superato}}\\ \hline
\hypertarget{TU5}{TU5} & Vogliamo testare che il metodo sollevi un'eccezione nel caso in cui il parametro non sia di tipo \file{Agent}. & \textcolor{green}{\textit{Superato}}\\ \hline
\hypertarget{TU6}{TU6} & Vogliamo testare che, se la chiamata al servizio di STT non va a buon fine, venga chiamato il metodo \file{succeed} del \file{context}, con un parametro \file{LambdaResponse} avente \file{statusCode} pari a 500. & \textcolor{green}{\textit{Superato}}\\ \hline
\hypertarget{TU7}{TU7} & Vogliamo testare che, se lo \file{status} della risposta ricevuta dall'assistente virtuale è diverso da 200, venga chiamato il metodo \file{succeed} di \file{context} con un oggetto di tipo \file{LambdaResponse} come parametro, avente il campo \file{statusCode} uguale a quello ricevuto e corpo del messaggio "Errore nel contattare l'assistente virtuale". & \textcolor{green}{\textit{Superato}}\\ \hline
\hypertarget{TU8}{TU8} & Vogliamo testare che, se l'action del body della risposta è uguale a \file{"rule.add"}, venga chiamato il metodo privato \file{addRule}. & \textcolor{green}{\textit{Superato}}\\ \hline
\hypertarget{TU9}{TU9} & Vogliamo testare che, se l'action del body della risposta è uguale a \file{"user.add"}, venga chiamato il metodo privato \file{addUser}. & \textcolor{green}{\textit{Superato}}\\ \hline
\hypertarget{TU10}{TU10} & Vogliamo testare che, se l'action del body della risposta è uguale a \file{"user.addEnrollment"}, venga chiamato il metodo privato \file{addUserEnrollment}. & \textcolor{green}{\textit{Superato}}\\ \hline
\hypertarget{TU11}{TU11} & Vogliamo testare che, se l'action del body della risposta è uguale a \file{"rule.get"}, venga chiamato il metodo privato \file{getRule}. & \textcolor{green}{\textit{Superato}}\\ \hline
\hypertarget{TU12}{TU12} & Vogliamo testare che, se l'action del body della risposta è uguale a \file{"rule.getList"}, venga chiamato il metodo privato \file{getRuleList}. & \textcolor{green}{\textit{Superato}}\\ \hline
\hypertarget{TU13}{TU13} & Vogliamo testare che, se l'action del body della risposta è uguale a \file{"user.get"}, venga chiamato il metodo privato \file{getUser}. & \textcolor{green}{\textit{Superato}}\\ \hline
\hypertarget{TU14}{TU14} & Vogliamo testare che, se l'action del body della risposta è uguale a \file{"user.login"}, venga chiamato il metodo privato \file{loginUser}. & \textcolor{green}{\textit{Superato}}\\ \hline
\hypertarget{TU15}{TU15} & Vogliamo testare che, se l'action del body della risposta è uguale a \file{"rule.remove"}, venga chiamato il metodo privato \file{removeRule}. & \textcolor{green}{\textit{Superato}}\\ \hline
\hypertarget{TU16}{TU16} & Vogliamo testare che, se l'action del body della risposta è uguale a \file{"user.remove"}, venga chiamato il metodo privato \file{removeUser}. & \textcolor{green}{\textit{Superato}}\\ \hline
\hypertarget{TU17}{TU17} & Vogliamo testare che, se l'action del body della risposta è uguale a \file{"user.resetEnrollment"}, venga chiamato il metodo privato \file{resetUserEnrollment}. & \textcolor{green}{\textit{Superato}}\\ \hline
\hypertarget{TU18}{TU18} & Vogliamo testare che, se l'action del body della risposta è uguale a \file{"rule.update"}, venga chiamato il metodo privato \file{updateRule}. & \textcolor{green}{\textit{Superato}}\\ \hline
\hypertarget{TU19}{TU19} & Vogliamo testare che, se l'action del body della risposta è uguale a \file{"user.update"}, venga chiamato il metodo privato \file{updateUser}. & \textcolor{green}{\textit{Superato}}\\ \hline
\hypertarget{TU20}{TU20} & Vogliamo testare che, se durante la chiamata al metodo privato \file{addRule} si verifica un errore, venga chiamato il metodo \file{succeed} del \file{context} con un parametro \file{LambdaResponse} il quale campo \file{statusCode} è impostato ad un valore uguale a quello restituito dal microservizio \file{Rules}. & \textcolor{green}{\textit{Superato}}\\ \hline
\hypertarget{TU21}{TU21} & Vogliamo testare che, se durante la chiamata al metodo privato \file{addUser} si verifica un errore, venga chiamato il metodo \file{succeed} del \file{context} con un parametro \file{LambdaResponse} il quale campo \file{statusCode} è impostato ad un valore uguale a quello restituito dal \gl{microservizio} \file{Users}. & \textcolor{green}{\textit{Superato}}\\ \hline
\hypertarget{TU22}{TU22} & Vogliamo testare che, se durante la chiamata al metodo privato \file{addUserEnrollment} si verifica un errore, venga chiamato il metodo \file{succeed} del \file{context} con un parametro \file{LambdaResponse} il quale campo \file{statusCode} è impostato ad un valore uguale a quello restituito dal microservizio \file{Users}. & \textcolor{green}{\textit{Superato}}\\ \hline
\hypertarget{TU23}{TU23} & Vogliamo testare che, se durante la chiamata al metodo privato \file{getRule} si verifica un errore, venga chiamato il metodo \file{succeed} del \file{context} con un parametro \file{LambdaResponse} il quale campo \file{statusCode} è impostato ad un valore uguale a quello restituito dal microservizio \file{Rules}. & \textcolor{green}{\textit{Superato}}\\ \hline
\hypertarget{TU24}{TU24} & Vogliamo testare che, se durante la chiamata al metodo privato \file{getRuleList} si verifica un errore, venga chiamato il metodo \file{succeed} del \file{context} con un parametro \file{LambdaResponse} il quale campo \file{statusCode} è impostato ad un valore uguale a quello restituito dal microservizio \file{Rules}. & \textcolor{green}{\textit{Superato}}\\ \hline
\hypertarget{TU25}{TU25} & Vogliamo testare che, se durante la chiamata al metodo privato \file{getUser} si verifica un errore, venga chiamato il metodo \file{succeed} del \file{context} con un parametro \file{LambdaResponse} il quale campo \file{statusCode} è impostato ad un valore uguale a quello restituito dal microservizio \file{Users}. & \textcolor{green}{\textit{Superato}}\\ \hline
\hypertarget{TU26}{TU26} & Vogliamo testare che, se durante la chiamata al metodo privato \file{getUserList} si verifica un errore, venga chiamato il metodo \file{succeed} del \file{context} con un parametro \file{LambdaResponse} il quale campo \file{statusCode} è impostato ad un valore uguale a quello restituito dal microservizio \file{Users}. & \textcolor{green}{\textit{Superato}}\\ \hline
\hypertarget{TU27}{TU27} & Vogliamo testare che, se durante la chiamata al metodo privato \file{loginUser} si verifica un errore, venga chiamato il metodo \file{succeed} del \file{context} con un parametro \file{LambdaResponse} il quale campo \file{statusCode} è impostato ad un valore uguale a quello restituito dal microservizio \file{Users}. & \textcolor{green}{\textit{Superato}}\\ \hline
\hypertarget{TU28}{TU28} & Vogliamo testare che, se durante la chiamata al metodo privato \file{removeRule} si verifica un errore, venga chiamato il metodo \file{succeed} del \file{context} con un parametro \file{LambdaResponse} il quale campo \file{statusCode} è impostato ad un valore uguale a quello restituito dal microservizio \file{Rules}. & \textcolor{green}{\textit{Superato}}\\ \hline
\hypertarget{TU29}{TU29} & Vogliamo testare che, se durante la chiamata al metodo privato \file{removeUser} si verifica un errore, venga chiamato il metodo \file{succeed} del \file{context} con un parametro \file{LambdaResponse} il quale campo \file{statusCode} è impostato ad un valore uguale a quello restituito dal microservizio \file{Users}. & \textcolor{green}{\textit{Superato}}\\ \hline
\hypertarget{TU30}{TU30} & Vogliamo testare che, se durante la chiamata al metodo privato \file{resetUserEnrollment} si verifica un errore, venga chiamato il metodo \file{succeed} del \file{context} con un parametro \file{LambdaResponse} il quale campo \file{statusCode} è impostato ad un valore uguale a quello restituito dal microservizio \file{Users}. & \textcolor{green}{\textit{Superato}}\\ \hline
\hypertarget{TU31}{TU31} & Vogliamo testare che, se durante la chiamata al metodo privato \file{updateRule} si verifica un errore, venga chiamato il metodo \file{succeed} del \file{context} con un parametro \file{LambdaResponse} il quale campo \file{statusCode} è impostato ad un valore uguale a quello restituito dal microservizio \file{Rules}. & \textcolor{green}{\textit{Superato}}\\ \hline
\hypertarget{TU32}{TU32} & Vogliamo testare che, se durante la chiamata al metodo privato \file{updateUser} si verifica un errore, venga chiamato il metodo \file{succeed} del \file{context} con un parametro \file{LambdaResponse} il quale campo \file{statusCode} è impostato ad un valore uguale a quello restituito dal microservizio \file{Users}. & \textcolor{green}{\textit{Superato}}\\ \hline
\hypertarget{TU33}{TU33} & Vogliamo testare che, se la risposta ricevuta dalla chiamata al microservizio \file{Rules} ha uno status code diverso da 200, il metodo sollevi un'eccezione di tipo \file{Exception} con campo \file{code} pari allo status code della risposta. & \textcolor{green}{\textit{Superato}}\\ \hline
\hypertarget{TU34}{TU34} & Vogliamo testare che, se la risposta ricevuta dalla chiamata al microservizio \file{Users} ha uno status code diverso da 200, il metodo sollevi un'eccezione di tipo \file{Exception} con campo \file{code} pari allo status code della risposta. & \textcolor{green}{\textit{Superato}}\\ \hline
\hypertarget{TU35}{TU35} & Vogliamo testare che, se la risposta ricevuta dalla chiamata al microservizio \file{Users} ha uno status code diverso da 200, il metodo sollevi un'eccezione di tipo \file{Exception} con campo \file{code} pari allo status code della risposta. & \textcolor{green}{\textit{Superato}}\\ \hline
\hypertarget{TU36}{TU36} & Vogliamo testare che, se la risposta ricevuta dalla chiamata al microservizio \file{Rules} ha uno status code diverso da 200, il metodo sollevi un'eccezione di tipo \file{Exception} con campo \file{code} pari allo status code della risposta. & \textcolor{green}{\textit{Superato}}\\ \hline
\hypertarget{TU37}{TU37} & Vogliamo testare che, se la risposta ricevuta dalla chiamata al microservizio \file{Rules} ha uno status code diverso da 200, il metodo sollevi un'eccezione di tipo \file{Exception} con campo \file{code} pari allo status code della risposta. & \textcolor{green}{\textit{Superato}}\\ \hline
\hypertarget{TU38}{TU38} & Vogliamo testare che, se la risposta ricevuta dalla chiamata al microservizio \file{Users} ha uno status code diverso da 200, il metodo sollevi un'eccezione di tipo \file{Exception} con campo \file{code} pari allo status code della risposta. & \textcolor{green}{\textit{Superato}}\\ \hline
\hypertarget{TU39}{TU39} & Vogliamo testare che, se la risposta ricevuta dalla chiamata al microservizio \file{Users} ha uno status code diverso da 200, il metodo sollevi un'eccezione di tipo \file{Exception} con campo \file{code} pari allo status code della risposta. & \textcolor{green}{\textit{Superato}}\\ \hline
\hypertarget{TU40}{TU40} & Vogliamo testare che, se la risposta ricevuta dalla chiamata al microservizio \file{Users} ha uno status code diverso da 200, il metodo sollevi un'eccezione di tipo \file{Exception} con campo \file{code} pari allo status code della risposta. & \textcolor{green}{\textit{Superato}}\\ \hline
\hypertarget{TU41}{TU41} & Vogliamo testare che, se la risposta ricevuta dalla chiamata al microservizio \file{Rules} ha uno status code diverso da 200, il metodo sollevi un'eccezione di tipo \file{Exception} con campo \file{code} pari allo status code della risposta. & \textcolor{green}{\textit{Superato}}\\ \hline
\hypertarget{TU42}{TU42} & Vogliamo testare che, se la risposta ricevuta dalla chiamata al microservizio \file{Users} ha uno status code diverso da 200, il metodo sollevi un'eccezione di tipo \file{Exception} con campo \file{code} pari allo status code della risposta. & \textcolor{green}{\textit{Superato}}\\ \hline
\hypertarget{TU43}{TU43} & Vogliamo testare che, se la risposta ricevuta dalla chiamata al microservizio \file{Users} ha uno status code diverso da 200, il metodo sollevi un'eccezione di tipo \file{Exception} con campo \file{code} pari allo status code della risposta. & \textcolor{green}{\textit{Superato}}\\ \hline
\hypertarget{TU44}{TU44} & Vogliamo testare che, se la risposta ricevuta dalla chiamata al microservizio \file{Rules} ha uno status code diverso da 200, il metodo sollevi un'eccezione di tipo \file{Exception} con campo \file{code} pari allo status code della risposta. & \textcolor{green}{\textit{Superato}}\\ \hline
\hypertarget{TU45}{TU45} & Vogliamo testare che, se la risposta ricevuta dalla chiamata al microservizio \file{Users} ha uno status code diverso da 200, il metodo sollevi un'eccezione di tipo \file{Exception} con campo \file{code} pari allo status code della risposta. & \textcolor{green}{\textit{Superato}}\\ \hline
\hypertarget{TU46}{TU46} & Vogliamo dimostrare che, se la chiamata al metodo \file{sns.publish} genera un errore, venga chiamato il metodo \file{succeed} del \file{context} con un parametro \file{LambdaResponse} avente campo \file{statusCode} pari allo \file{status} dell'errore. & \textcolor{green}{\textit{Superato}}\\ \hline
\hypertarget{TU47}{TU47} & Vogliamo testare che, se lo status code della risposta di un microservizio è pari a 200 e l'action contenuta nel suo body non corrisponde a nessuna action supportata dal back-end, il metodo rielabori la risposta e la inoltri. & \textcolor{green}{\textit{Superato}}\\ \hline
\hypertarget{TU48}{TU48} & Vogliamo testare che il metodo accetti un parametro di tipo \file{Conversation} senza generare eccezioni. & \textcolor{green}{\textit{Superato}}\\ \hline
\hypertarget{TU49}{TU49} & Vogliamo testare che il metodo sollevi un'eccezione nel caso in cui il parametro non sia di tipo \file{Conversation}. & \textcolor{green}{\textit{Superato}}\\ \hline
\hypertarget{TU50}{TU50} & Vogliamo testare che, se il metodo aggiunge correttamente una conversazione, l'\file{Observable} notifichi l'\file{Observer} iscritto richiamando una sola volta il metodo \file{complete}.  & \textcolor{green}{\textit{Superato}}\\ \hline
\hypertarget{TU51}{TU51} & Vogliamo testare che, se la conversazione non viene aggiunta a causa di un errore, l'\file{Observable} notifichi l'\file{Observer} iscritto richiamando il metodo \file{error}.  & \textcolor{green}{\textit{Superato}}\\ \hline
\hypertarget{TU52}{TU52} & Vogliamo testare che, se il metodo aggiunge correttamente un messaggio ad una conversazione, l'\file{Observable} notifichi l'\file{Observer} iscritto richiamando una sola volta il metodo \file{complete}.  & \textcolor{green}{\textit{Superato}}\\ \hline
\hypertarget{TU53}{TU53} & Vogliamo testare che, se il messaggio non viene aggiunto alla conversazione a causa di un errore, l'\file{Observable} notifichi l'\file{Observer} iscritto richiamando il metodo \file{error}.  & \textcolor{green}{\textit{Superato}}\\ \hline
\hypertarget{TU54}{TU54} & Vogliamo testare che, nel caso in cui il metodo ottenga la conversazione, l'\file{Observable} invii tale \file{Conversation} all'\file{Observer} iscritto tramite il metodo \file{next} e lo notifichi richiamando una sola volta il metodo \file{complete}.  & \textcolor{green}{\textit{Superato}}\\ \hline
\hypertarget{TU55}{TU55} & Vogliamo testare che, se si verifica un errore nell’ottenere la conversazione, l'\file{Observable} notifichi l'\file{Observer} iscritto richiamando il metodo \file{error}.  & \textcolor{green}{\textit{Superato}}\\ \hline
\hypertarget{TU56}{TU56} & Vogliamo testare che l'\file{Observable} notifichi l'\file{Observer} con il metodo \file{complete} solo dopo aver inviato tutti i blocchi di \file{Conversation} presenti nel database tramite il metodo \file{next}.  & \textcolor{green}{\textit{Superato}}\\ \hline
\hypertarget{TU57}{TU57} & Vogliamo testare che, se si verifica un errore nell’ottenere la lista delle conversazioni, l'\file{Observable} notifichi l'\file{Observer} iscritto richiamando il metodo \file{error}.  & \textcolor{green}{\textit{Superato}}\\ \hline
\hypertarget{TU58}{TU58} & Vogliamo testare che, se il metodo elimina correttamente una conversazione, l'\file{Observable} notifichi l'\file{Observer} iscritto richiamando una sola volta il metodo \file{complete}. & \textcolor{green}{\textit{Superato}}\\ \hline
\hypertarget{TU59}{TU59} & Vogliamo testare che, se la conversazione non viene eliminata a causa di un errore, l'\file{Observable} notifichi l'\file{Observer} iscritto richiamando il metodo \file{error}. & \textcolor{green}{\textit{Superato}}\\ \hline
\hypertarget{TU60}{TU60} & Vogliamo testare che il metodo accetti un parametro di tipo \file{Guest} senza generare eccezioni. & \textcolor{green}{\textit{Superato}}\\ \hline
\hypertarget{TU61}{TU61} & Vogliamo testare che il metodo sollevi un'eccezione nel caso in cui il parametro non sia di tipo \file{Guest}. & \textcolor{green}{\textit{Superato}}\\ \hline
\hypertarget{TU62}{TU62} & Vogliamo testare che, se il metodo aggiunge correttamente un ospite, l'\file{Observable} notifichi l'\file{Observer} iscritto richiamando una sola volta il metodo \file{complete}. & \textcolor{green}{\textit{Superato}}\\ \hline
\hypertarget{TU63}{TU63} & Vogliamo testare che, se un ospite non viene aggiunto a causa di un errore, l'\file{Observable} notifichi l'\file{Observer} iscritto richiamando il metodo \file{error}. & \textcolor{green}{\textit{Superato}}\\ \hline
\hypertarget{TU64}{TU64} & Vogliamo testare che, nel caso in cui il metodo ottenga un ospite, l'\file{Observable} invii tale \file{Guest} all'\file{Observer} iscritto tramite il metodo \file{next} e lo notifichi richiamando una sola volta il metodo \file{complete}. & \textcolor{green}{\textit{Superato}}\\ \hline
\hypertarget{TU65}{TU65} & Vogliamo testare che, se si verifica un errore nell’ottenere un ospite, l'\file{Observable} notifichi l'\file{Observer} iscritto richiamando il metodo \file{error}. & \textcolor{green}{\textit{Superato}}\\ \hline
\hypertarget{TU66}{TU66} & Vogliamo testare che l'\file{Observable} notifichi l'\file{Observer} con il metodo \file{complete} solo dopo aver inviato tutti i blocchi di \file{Guest} presenti nel database tramite il metodo \file{next}. & \textcolor{green}{\textit{Superato}}\\ \hline
\hypertarget{TU67}{TU67} & Vogliamo testare che, se si verifica un errore nell'ottenere la lista degli ospiti, l'\file{Observable} notifichi l'\file{Observer} iscritto richiamando il metodo \file{error}. & \textcolor{green}{\textit{Superato}}\\ \hline
\hypertarget{TU68}{TU68} & Vogliamo testare che, se il metodo elimina correttamente l'ospite, l'\file{Observable} notifichi l'\file{Observer} iscritto richiamando una sola volta il metodo \file{complete}. & \textcolor{green}{\textit{Superato}}\\ \hline
\hypertarget{TU69}{TU69} & Vogliamo testare che, se l’ospite non viene eliminato a causa di un errore, l'\file{Observable} notifichi l'\file{Observer} iscritto richiamando il metodo \file{error}. & \textcolor{green}{\textit{Superato}}\\ \hline
\hypertarget{TU70}{TU70} & Vogliamo testare che, se il metodo aggiorna correttamente l'ospite, l'\file{Observable} notifichi l'\file{Observer} iscritto richiamando una sola volta il metodo \file{complete}. & \textcolor{green}{\textit{Superato}}\\ \hline
\hypertarget{TU71}{TU71} & Vogliamo testare che, se l’ospite non viene eliminato a causa di un errore, l'\file{Observable} notifichi l'\file{Observer} iscritto richiamando il metodo \file{error}. & \textcolor{green}{\textit{Superato}}\\ \hline
\hypertarget{TU72}{TU72} & Vogliamo testare che il metodo accetti un parametro di tipo \file{Member} senza generare eccezioni. & \textcolor{green}{\textit{Superato}}\\ \hline
\hypertarget{TU73}{TU73} & Vogliamo testare che il metodo sollevi un'eccezione nel caso in cui il parametro non sia di tipo \file{Member}. & \textcolor{green}{\textit{Superato}}\\ \hline
\hypertarget{TU74}{TU74} & Vogliamo testare che, nel caso in cui il metodo ottenga il membro dell’azienda, l'\file{Observable} invii tale \file{Member} all'\file{Observer} iscritto tramite il metodo \file{next} e lo notifichi richiamando una sola volta il metodo \file{complete}. & \textcolor{green}{\textit{Superato}}\\ \hline
\hypertarget{TU75}{TU75} & Vogliamo testare che, se si verifica un errore nell’ottenere il membro dell’azienda, l'\file{Observable} notifichi l'\file{Observer} iscritto richiamando il metodo \file{error}. & \textcolor{green}{\textit{Superato}}\\ \hline
\hypertarget{TU76}{TU76} & Vogliamo testare che l'\file{Observable} notifichi l'\file{Observer} con il metodo \file{complete} solo dopo aver inviato tutti i blocchi di \file{Member} presenti nel database tramite il metodo \file{next}. & \textcolor{green}{\textit{Superato}}\\ \hline
\hypertarget{TU77}{TU77} & Vogliamo testare che, se si verifica un errore nell’ottenere la lista dei membri dell’azienda, l'\file{Observable} notifichi l'\file{Observer} iscritto richiamando il metodo \file{error}. & \textcolor{green}{\textit{Superato}}\\ \hline
\hypertarget{TU78}{TU78} & Vogliamo testare che, anche se viene passato un \file{Member} corretto, il metodo ritorni un \file{ErrorObservable} che notifica l'\file{Observer} richiamando il suo metodo \file{error}. Infatti non è possibile inserire un membro perciò il metodo fallisce sempre. & \textcolor{green}{\textit{Superato}}\\ \hline
\hypertarget{TU79}{TU79} & Vogliamo testare che, anche se viene passato l'username di un \file{Member}, il metodo ritorni un \file{ErrorObservable} che notifica l'\file{Observer} richiamando il suo metodo \file{error}. Infatti non è possibile rimuovere un membro perciò il metodo fallisce sempre. & \textcolor{green}{\textit{Superato}}\\ \hline
\hypertarget{TU80}{TU80} & Vogliamo testare che, se si verifica un errore nella richiesta delle informazioni sui canali a Slack, venga chiamato il metodo \file{succeed} del \file{context} con un parametro \file{LambdaResponse} il quale campo \file{statusCode} è impostato a 500.
 & \textcolor{green}{\textit{Superato}}\\ \hline
\hypertarget{TU81}{TU81} & Nel caso in cui non si verifichino errori, il campo statusCode della risposta deve essere impostato a 200 ed il corpo della risposta deve contenere la lista dei canali Slack (utenti, canali pubblici e gruppi privati) in formato \gl{JSON}. & \textcolor{green}{\textit{Superato}}\\ \hline
\hypertarget{TU82}{TU82} & Vogliamo testare che il metodo imposti il campo \file{statusCode} della risposta a 200 e il campo \file{body} sia vuoto. & \textcolor{green}{\textit{Superato}}\\ \hline
\hypertarget{TU83}{TU83} & Vogliamo testare che, se si verifica un errore, venga chiamato il metodo \file{succeed} del \file{context} con un parametro \file{LambdaResponse} il quale campo \file{statusCode} è impostato a 500. & \textcolor{green}{\textit{Superato}}\\ \hline
\hypertarget{TU84}{TU84} & Vogliamo testare che alla chiamata del metodo venga chiamata la funzione di \gl{callback} \file{complete\_cb}. & \textcolor{green}{\textit{Superato}}\\ \hline
\hypertarget{TU85}{TU85} & Vogliamo testare che alla chiamata del metodo venga chiamata la funzione di callback \file{error\_cb}, passandole come parametro l'errore ricevuto. & \textcolor{green}{\textit{Superato}}\\ \hline
\hypertarget{TU86}{TU86} & Vogliamo testare che alla chiamata del metodo venga chiamata la funzione di callback \file{next\_cb}, passandole come parametro i dati ricevuti. & \textcolor{green}{\textit{Superato}}\\ \hline
\hypertarget{TU87}{TU87} & Vogliamo testare che il metodo accetti un parametro di tipo \file{Rule} senza generare eccezioni. & \textcolor{green}{\textit{Superato}}\\ \hline
\hypertarget{TU88}{TU88} & Vogliamo testare che il metodo sollevi un'eccezione nel caso in cui il parametro non sia di tipo \file{Rule}. & \textcolor{green}{\textit{Superato}}\\ \hline
\hypertarget{TU89}{TU89} & Vogliamo testare che, se il metodo aggiunge correttamente una direttiva, l'\file{Observable} notifichi l'\file{Observer} iscritto richiamando una sola volta il metodo \file{complete}. & \textcolor{green}{\textit{Superato}}\\ \hline
\hypertarget{TU90}{TU90} & Vogliamo testare che, se la direttiva non viene aggiunta a causa di un errore, l'\file{Observable} notifichi l'\file{Observer} iscritto richiamando il metodo \file{error}. & \textcolor{green}{\textit{Superato}}\\ \hline
\hypertarget{TU91}{TU91} & Vogliamo testare che, nel caso in cui il metodo ottenga una direttiva, l'\file{Observable} invii tale \file{Rule} all'\file{Observer} iscritto tramite il metodo \file{next} e lo notifichi richiamando una sola volta il metodo \file{complete}. & \textcolor{green}{\textit{Superato}}\\ \hline
\hypertarget{TU92}{TU92} & Vogliamo testare che, se si verifica un errore nell’ottenere una direttiva, l'\file{Observable} notifichi l'\file{Observer} iscritto richiamando il metodo \file{error}. & \textcolor{green}{\textit{Superato}}\\ \hline
\hypertarget{TU93}{TU93} & Vogliamo testare che l'\file{Observable} notifichi l'\file{Observer} con il metodo \file{complete} solo dopo aver inviato tutti i blocchi di \file{Rule} presenti nel database tramite il metodo \file{next}. & \textcolor{green}{\textit{Superato}}\\ \hline
\hypertarget{TU94}{TU94} & Vogliamo testare che, se si verifica un errore nell’ottenere la lista delle direttive, l'\file{Observable} notifichi l'\file{Observer} iscritto richiamando il metodo \file{error}. & \textcolor{green}{\textit{Superato}}\\ \hline
\hypertarget{TU95}{TU95} & Vogliamo testare che, se il metodo elimina correttamente la direttiva, l'\file{Observable} notifichi l'\file{Observer} iscritto richiamando una sola volta il metodo \file{complete}. & \textcolor{green}{\textit{Superato}}\\ \hline
\hypertarget{TU96}{TU96} & Vogliamo testare che, se la direttiva non viene eliminata a causa di un errore, l'\file{Observable} notifichi l'\file{Observer} iscritto richiamando il metodo \file{error}. & \textcolor{green}{\textit{Superato}}\\ \hline
\hypertarget{TU97}{TU97} & Vogliamo testare che, se il metodo aggiorna correttamente la direttiva, l'\file{Observable} notifichi l'\file{Observer} iscritto richiamando una sola volta il metodo \file{complete}. & \textcolor{green}{\textit{Superato}}\\ \hline
\hypertarget{TU98}{TU98} & Vogliamo testare che, se la direttiva non viene aggiornata a causa di un errore, l'\file{Observable} notifichi l'\file{Observer} iscritto richiamando il metodo \file{error}. & \textcolor{green}{\textit{Superato}}\\ \hline
\hypertarget{TU99}{TU99} & Vogliamo testare che, se il metodo aggiunge correttamente la funzione di una direttiva, l'\file{Observable} notifichi l'\file{Observer} iscritto richiamando una sola volta il metodo \file{complete}. & \textcolor{green}{\textit{Superato}}\\ \hline
\hypertarget{TU100}{TU100} & Vogliamo testare che, se la funziona di una direttiva non viene aggiunta a causa di un errore, l'\file{Observable} notifichi l'\file{Observer} iscritto richiamando il metodo \file{error}. & \textcolor{green}{\textit{Superato}}\\ \hline
\hypertarget{TU101}{TU101} & Vogliamo testare che, nel caso in cui il metodo ottenga la funzione di una direttiva, l'\file{Observable} invii tale \file{Task} all'\file{Observer} iscritto tramite il metodo \file{next} e lo notifichi richiamando una sola volta il metodo \file{complete}. & \textcolor{green}{\textit{Superato}}\\ \hline
\hypertarget{TU102}{TU102} & Vogliamo testare che, se si verifica un errore nell’ottenere una funzione, l'\file{Observable} notifichi l'\file{Observer} iscritto richiamando il metodo \file{error}. & \textcolor{green}{\textit{Superato}}\\ \hline
\hypertarget{TU103}{TU103} & Vogliamo testare che l'\file{Observable} notifichi l'\file{Observer} con il metodo \file{complete} solo dopo aver inviato tutti i blocchi di \file{Task} presenti nel database tramite il metodo \file{next}. & \textcolor{green}{\textit{Superato}}\\ \hline
\hypertarget{TU104}{TU104} & Vogliamo testare che, se si verifica un errore nell'ottenere la lista delle funzioni, l'\file{Observable} notifichi l'\file{Observer} iscritto richiamando il metodo \file{error}. & \textcolor{green}{\textit{Superato}}\\ \hline
\hypertarget{TU105}{TU105} & Vogliamo testare che, se il metodo elimina correttamente la funzione di una direttiva, l'\file{Observable} notifichi l'\file{Observer} iscritto richiamando una sola volta il metodo \file{complete}. & \textcolor{green}{\textit{Superato}}\\ \hline
\hypertarget{TU106}{TU106} & Vogliamo testare che, se la funzione di una direttiva non viene eliminata a causa di un errore, l'\file{Observable} notifichi l'\file{Observer} iscritto richiamando il metodo \file{error}. & \textcolor{green}{\textit{Superato}}\\ \hline
\hypertarget{TU107}{TU107} & Vogliamo testare che, se il metodo aggiorna correttamente la funzione di una direttiva, l'\file{Observable} notifichi l'\file{Observer} iscritto richiamando una sola volta il metodo \file{complete}. & \textcolor{green}{\textit{Superato}}\\ \hline
\hypertarget{TU108}{TU108} & Vogliamo testare che, se la funzione di una direttiva non viene aggiornata a causa di un errore, l'\file{Observable} notifichi l'\file{Observer} iscritto richiamando il metodo \file{error}. & \textcolor{green}{\textit{Superato}}\\ \hline
\hypertarget{TU109}{TU109} & Vogliamo testare che, se la chiamata al metodo \file{stt.recognize} fallisce, venga chiamato il metodo \file{rejected} della Promise con un parametro \file{Exception} avente campo \file{code} 500. & \textcolor{green}{\textit{Superato}}\\ \hline
\hypertarget{TU110}{TU110} & Vogliamo testare che il metodo accetti un parametro di tipo \file{Task} senza generare eccezioni. & \textcolor{green}{\textit{Superato}}\\ \hline
\hypertarget{TU111}{TU111} & Vogliamo testare che il metodo sollevi un'eccezione nel caso in cui il parametro non sia di tipo \file{Task}. & \textcolor{green}{\textit{Superato}}\\ \hline
\hypertarget{TU112}{TU112} & Vogliamo testare che il metodo accetti un parametro di tipo \file{User} senza generare eccezioni. & \textcolor{green}{\textit{Superato}}\\ \hline
\hypertarget{TU113}{TU113} & Vogliamo testare che il metodo sollevi un'eccezione nel caso in cui il parametro non sia di tipo \file{User}. & \textcolor{green}{\textit{Superato}}\\ \hline
\hypertarget{TU114}{TU114} & Vogliamo testare che, se il metodo aggiunge correttamente un utente, l'\file{Observable} notifichi l'\file{Observer} iscritto richiamando una sola volta il metodo \file{complete}. & \textcolor{green}{\textit{Superato}}\\ \hline
\hypertarget{TU115}{TU115} & Vogliamo testare che, se l’utente non viene aggiunto a causa di un errore, l'\file{Observable} notifichi l'\file{Observer} iscritto richiamando il metodo \file{error}. & \textcolor{green}{\textit{Superato}}\\ \hline
\hypertarget{TU116}{TU116} & Vogliamo testare che, nel caso in cui il metodo ottenga un utente a partire dall'\file{username}, l'\file{Observable} invii tale \file{User} all'\file{Observer} iscritto tramite il metodo \file{next} e lo notifichi richiamando una sola volta il metodo \file{complete}. & \textcolor{green}{\textit{Superato}}\\ \hline
\hypertarget{TU117}{TU117} & Vogliamo testare che, se si verifica un errore nell’ottenere un utente a partire dall'\file{username}, l'\file{Observable} notifichi l'\file{Observer} iscritto richiamando il metodo \file{error}. & \textcolor{green}{\textit{Superato}}\\ \hline
\hypertarget{TU118}{TU118} & Vogliamo testare che l'\file{Observable} notifichi l'\file{Observer} con il metodo \file{complete} solo dopo aver inviato tutti i blocchi di \file{User} presenti nel database tramite il metodo \file{next}. & \textcolor{green}{\textit{Superato}}\\ \hline
\hypertarget{TU119}{TU119} & Vogliamo testare che, se si verifica un errore nell’ottenere la lista degli utenti, l'\file{Observable} notifichi l'\file{Observer} iscritto richiamando il metodo \file{error}. & \textcolor{green}{\textit{Superato}}\\ \hline
\hypertarget{TU120}{TU120} & Vogliamo testare che, se il metodo elimina correttamente l’utente, l'\file{Observable} notifichi l'\file{Observer} iscritto richiamando una sola volta il metodo \file{complete}.
 & \textcolor{green}{\textit{Superato}}\\ \hline
\hypertarget{TU121}{TU121} & Vogliamo testare che, se l’utente non viene eliminato a causa di un errore, l'\file{Observable} notifichi l'\file{Observer} iscritto richiamando il metodo \file{error}.
 & \textcolor{green}{\textit{Superato}}\\ \hline
\hypertarget{TU122}{TU122} & Vogliamo testare che, se il metodo aggiorna correttamente l’utente, l'\file{Observable} notifichi l'\file{Observer} iscritto richiamando una sola volta il metodo \file{complete}.
 & \textcolor{green}{\textit{Superato}}\\ \hline
\hypertarget{TU123}{TU123} & Vogliamo testare che, se l’utente non viene aggiornato a causa di un errore, l'\file{Observable} notifichi l'\file{Observer} iscritto richiamando il metodo \file{error}.
 & \textcolor{green}{\textit{Superato}}\\ \hline
\hypertarget{TU124}{TU124} & Vogliamo testare che, se la chiamata al servizio di Speaker Recognition per aggiungere un \file{Enrollment} ritorna uno \file{statusCode} diverso da 200, l’\file{ErrorObservable} notifichi l'\file{ErrorObserver} chiamando il suo metodo \file{error}.
 & \textcolor{green}{\textit{Superato}}\\ \hline
\hypertarget{TU125}{TU125} & Vogliamo testare che, se la chiamata al servizio di Speaker Recognition per creare uno \file{User} ritorna uno \file{statusCode} diverso da 200, \file{StringObservable} notifichi \file{StringObserver} chiamando il suo metodo \file{error}. & \textcolor{green}{\textit{Superato}}\\ \hline
\hypertarget{TU126}{TU126} & Vogliamo testare che, se la chiamata al servizio di Speaker Recognition per eliminare uno \file{User} ritorna uno \file{statusCode} diverso da 200, l’\file{ErrorObservable} notifichi l'\file{ErrorObserver} chiamando il suo metodo \file{error}.
 & \textcolor{green}{\textit{Superato}}\\ \hline
\hypertarget{TU127}{TU127} & Vogliamo testare che, se la chiamata al servizio di Speaker Recognition per effettuare il login ritorna uno \file{statusCode} diverso da 200, l’\file{ErrorObservable} notifichi l'\file{ErrorObserver} chiamando il suo metodo \file{error}.
 & \textcolor{green}{\textit{Superato}}\\ \hline
\hypertarget{TU128}{TU128} & Vogliamo testare che, se la chiamata al servizio di Speaker Recognition per ottenere la lista degli \file{User} ritorna uno \file{statusCode} diverso da 200, \file{SRUserObservable} notifichi \file{SRUserObserver} chiamando il suo metodo \file{error}. & \textcolor{green}{\textit{Superato}}\\ \hline
\hypertarget{TU129}{TU129} & Vogliamo testare che, se la chiamata al servizio di Speaker Recognition per ottenere uno \file{User} ritorna uno \file{statusCode} diverso da 200, \file{SRUserObservable} notifichi \file{SRUserObserver} chiamando il suo metodo \file{error}. & \textcolor{green}{\textit{Superato}}\\ \hline
\hypertarget{TU130}{TU130} & Vogliamo testare che, se la chiamata al servizio di Speaker Recognition per resettare un \file{Enrollment} ritorna uno \file{statusCode} diverso da 200, l’\file{ErrorObservable} notifichi l'\file{ErrorObserver} chiamando il suo metodo \file{error}. & \textcolor{green}{\textit{Superato}}\\ \hline
\hypertarget{TU131}{TU131} & Vogliamo testare che, se il metodo aggiunge correttamente un agente di api.ai, l'\file{Observable} notifichi l'\file{Observer} iscritto richiamando una sola volta il metodo \file{complete}. & \textcolor{green}{\textit{Superato}}\\ \hline
\hypertarget{TU132}{TU132} & Vogliamo testare che, se l’agente non viene aggiunto a causa di un errore, l'\file{Observable} notifichi l'\file{Observer} iscritto richiamando il metodo \file{error}. & \textcolor{green}{\textit{Superato}}\\ \hline
\hypertarget{TU133}{TU133} & Vogliamo testare che, nel caso in cui il metodo ottenga un agente di api.ai, l'\file{Observable} invii tale \file{Agent} all'\file{Observer} iscritto tramite il metodo \file{next} e lo notifichi richiamando una sola volta il metodo \file{complete}. & \textcolor{green}{\textit{Superato}}\\ \hline
\hypertarget{TU134}{TU134} & Vogliamo testare che, se si verifica un errore nell’ottenere un agente, l'\file{Observable} notifichi l'\file{Observer} iscritto richiamando il metodo \file{error}. & \textcolor{green}{\textit{Superato}}\\ \hline
\hypertarget{TU135}{TU135} & Vogliamo testare che l'\file{Observable} notifichi l'\file{Observer} con il metodo \file{complete} solo dopo aver inviato tutti i blocchi di \file{Agent} presenti nel database tramite il metodo \file{next}. & \textcolor{green}{\textit{Superato}}\\ \hline
\hypertarget{TU136}{TU136} & Vogliamo testare che, se si verifica un errore nell’ottenere la lista degli agenti, l'\file{Observable} notifichi l'\file{Observer} iscritto richiamando il metodo \file{error}. & \textcolor{green}{\textit{Superato}}\\ \hline
\hypertarget{TU137}{TU137} & Vogliamo testare che, se il metodo elimina correttamente l’agente, l'\file{Observable} notifichi l'\file{Observer} iscritto richiamando una sola volta il metodo \file{complete}. & \textcolor{green}{\textit{Superato}}\\ \hline
\hypertarget{TU138}{TU138} & Vogliamo testare che, se l’agente non viene eliminato a causa di un errore, l'\file{Observable} notifichi l'\file{Observer} iscritto richiamando il metodo \file{error}. & \textcolor{green}{\textit{Superato}}\\ \hline
\hypertarget{TU139}{TU139} & Vogliamo testare che, se il metodo aggiorna correttamente l’agente di api.ai, l'\file{Observable} notifichi l'\file{Observer} iscritto richiamando una sola volta il metodo \file{complete}. & \textcolor{green}{\textit{Superato}}\\ \hline
\hypertarget{TU140}{TU140} & Vogliamo testare che, se l’agente non viene aggiornato a causa di un errore, l'\file{Observable} notifichi l'\file{Observer} iscritto richiamando il metodo \file{error}. & \textcolor{green}{\textit{Superato}}\\ \hline
\hypertarget{TU141}{TU141} & Vogliamo testare che, se la chiamata al metodo viene fatta con un parametro non atteso, venga chiamato il metodo \file{succeed} del \file{context} con un parametro \file{LambdaResponse} avente campo \file{statusCode} pari a 400. & \textcolor{green}{\textit{Superato}}\\ \hline
\hypertarget{TU142}{TU142} & Vogliamo testare che, se la chiamata al metodo genera un errore del microservizio, venga chiamato il metodo \file{succeed} del \file{context} con un parametro \file{LambdaResponse} avente campo \file{statusCode} pari a 500. & \textcolor{green}{\textit{Superato}}\\ \hline
\hypertarget{TU143}{TU143} & Vogliamo testare che, se la chiamata al metodo va a buon fine, venga chiamato il metodo \file{succeed} del \file{context} con un parametro \file{LambdaResponse} avente campo \file{statusCode} pari a 200. & \textcolor{green}{\textit{Superato}}\\ \hline
\hypertarget{TU144}{TU144} & Vogliamo testare che, se la chiamata al metodo genera un errore del microservizio, venga chiamato il metodo \file{succeed} del \file{context} con un parametro \file{LambdaResponse} avente campo \file{statusCode} pari a 500. & \textcolor{green}{\textit{Superato}}\\ \hline
\hypertarget{TU145}{TU145} & Vogliamo testare che, se la chiamata al metodo va a buon fine, venga chiamato il metodo \file{succeed} del \file{context} con un parametro \file{LambdaResponse} avente campo \file{statusCode} pari a 200. & \textcolor{green}{\textit{Superato}}\\ \hline
\hypertarget{TU146}{TU146} & Vogliamo testare che, se la chiamata al metodo va a buon fine, venga chiamato il metodo \file{succeed} del \file{context} con un parametro \file{LambdaResponse} avente campo \file{statusCode} pari a 200 e campo \file{body} contenente la \file{Rule} cercata. & \textcolor{green}{\textit{Superato}}\\ \hline
\hypertarget{TU147}{TU147} & Vogliamo testare che, se la chiamata al metodo genera un errore del microservizio, venga chiamato il metodo \file{succeed} del \file{context} con un parametro \file{LambdaResponse} avente campo \file{statusCode} pari a 500. & \textcolor{green}{\textit{Superato}}\\ \hline
\hypertarget{TU148}{TU148} & Vogliamo testare che, se la chiamata al metodo va a buon fine, venga chiamato il metodo \file{succeed} del \file{context} con un parametro \file{LambdaResponse} avente campo \file{statusCode} pari a 200 e campo \file{body} contenente la lista delle \file{Rule}. & \textcolor{green}{\textit{Superato}}\\ \hline
\hypertarget{TU149}{TU149} & Vogliamo testare che, se la chiamata al metodo genera un errore del microservizio, venga chiamato il metodo \file{succeed} del \file{context} con un parametro \file{LambdaResponse} avente campo \file{statusCode} pari a 500. & \textcolor{green}{\textit{Superato}}\\ \hline
\hypertarget{TU152}{TU152} & Vogliamo testare che, se la chiamata al metodo genera un errore del microservizio, venga chiamato il metodo \file{succeed} del \file{context} con un parametro \file{LambdaResponse} avente campo \file{statusCode} pari a 500. & \textcolor{green}{\textit{Superato}}\\ \hline
\hypertarget{TU153}{TU153} & Vogliamo testare che, se la chiamata al metodo va a buon fine, venga chiamato il metodo \file{succeed} del \file{context} con un parametro \file{LambdaResponse} avente campo \file{statusCode} pari a 200. & \textcolor{green}{\textit{Superato}}\\ \hline
\hypertarget{TU154}{TU154} & Vogliamo testare che, se la chiamata al metodo viene fatta con un parametro non atteso, venga chiamato il metodo \file{succeed} del \file{context} con un parametro \file{LambdaResponse} avente campo \file{statusCode} pari a 400. & \textcolor{green}{\textit{Superato}}\\ \hline
\hypertarget{TU155}{TU155} & Vogliamo testare che, se la chiamata al metodo genera un errore del microservizio, venga chiamato il metodo \file{succeed} del \file{context} con un parametro \file{LambdaResponse} avente campo \file{statusCode} pari a 500. & \textcolor{green}{\textit{Superato}}\\ \hline
\hypertarget{TU156}{TU156} & Vogliamo testare che, se la chiamata al metodo va a buon fine, venga chiamato il metodo \file{succeed} del \file{context} con un parametro \file{LambdaResponse} avente campo \file{statusCode} pari a 200.
 & \textcolor{green}{\textit{Superato}}\\ \hline
\hypertarget{TU157}{TU157} & Vogliamo testare che, se la chiamata al metodo viene fatta con un parametro non atteso, venga chiamato il metodo \file{succeed} del \file{context} con un parametro \file{LambdaResponse} avente campo \file{statusCode} pari a 400.
 & \textcolor{green}{\textit{Superato}}\\ \hline
\hypertarget{TU158}{TU158} & Vogliamo testare che, se la chiamata al metodo genera un errore del microservizio, venga chiamato il metodo \file{succeed} del \file{context} con un parametro \file{LambdaResponse} avente campo \file{statusCode} pari a 500.
 & \textcolor{green}{\textit{Superato}}\\ \hline
\hypertarget{TU159}{TU159} & Vogliamo testare che, se la chiamata al metodo va a buon fine, venga chiamato il metodo \file{succeed} del \file{context} con un parametro \file{LambdaResponse} avente campo \file{statusCode} pari a 200 e campo \file{body} contenente l’\file{User} cercato. & \textcolor{green}{\textit{Superato}}\\ \hline
\hypertarget{TU160}{TU160} & Vogliamo testare che, se la chiamata al metodo genera un errore del microservizio, venga chiamato il metodo \file{succeed} del \file{context} con un parametro \file{LambdaResponse} avente campo \file{statusCode} pari a 500. & \textcolor{green}{\textit{Superato}}\\ \hline
\hypertarget{TU161}{TU161} & Vogliamo testare che, se la chiamata al metodo va a buon fine, venga chiamato il metodo \file{succeed} del \file{context} con un parametro \file{LambdaResponse} avente campo \file{statusCode} pari a 200 e campo \file{body} contenente la lista degli \file{User}. & \textcolor{green}{\textit{Superato}}\\ \hline
\hypertarget{TU162}{TU162} & Vogliamo testare che, se la chiamata al metodo genera un errore del microservizio, venga chiamato il metodo \file{succeed} del \file{context} con un parametro \file{LambdaResponse} avente campo \file{statusCode} pari a 500. & \textcolor{green}{\textit{Superato}}\\ \hline
\hypertarget{TU163}{TU163} & Vogliamo testare che, se la chiamata al metodo va a buon fine, venga chiamato il metodo \file{succeed} del \file{context} con un parametro \file{LambdaResponse} avente campo \file{statusCode} pari a 200. & \textcolor{green}{\textit{Superato}}\\ \hline
\hypertarget{TU164}{TU164} & Vogliamo testare che, se la chiamata al metodo genera un errore del microservizio, venga chiamato il metodo \file{succeed} del \file{context} con un parametro \file{LambdaResponse} avente campo \file{statusCode} pari a 500. & \textcolor{green}{\textit{Superato}}\\ \hline
\hypertarget{TU165}{TU165} & Vogliamo testare che, se la chiamata al metodo va a buon fine, venga chiamato il metodo \file{succeed} del \file{context} con un parametro \file{LambdaResponse} avente campo \file{statusCode} pari a 200. & \textcolor{green}{\textit{Superato}}\\ \hline
\hypertarget{TU166}{TU166} & Vogliamo testare che, se la chiamata al metodo viene fatta con un parametro non atteso, venga chiamato il metodo \file{succeed} del \file{context} con un parametro \file{LambdaResponse} avente campo \file{statusCode} pari a 400. & \textcolor{green}{\textit{Superato}}\\ \hline
\hypertarget{TU167}{TU167} & Vogliamo testare che, se la chiamata al metodo genera un errore del microservizio, venga chiamato il metodo \file{succeed} del \file{context} con un parametro \file{LambdaResponse} avente campo \file{statusCode} pari a 500. & \textcolor{green}{\textit{Superato}}\\ \hline
\hypertarget{TU168}{TU168} & Vogliamo verificare che, se la chiamata al microservizio \file{Rules} genera un errore, venga chiamata la funzione di callback con un solo parametro diverso da null. & \textcolor{green}{\textit{Superato}}\\ \hline
\hypertarget{TU169}{TU169} & Vogliamo testare che, se la chiamata al microservizio \file{Notification} genera un errore, venga chiamata la funzione di callback con un solo parametro diverso da null. & \textcolor{green}{\textit{Superato}}\\ \hline
\hypertarget{TU170}{TU170} & Vogliamo testare che, se la chiamata ai metodi di \file{GuestsDAO} genera un errore, viene chiamata la funzione di callback con un solo parametro diverso da null. & \textcolor{green}{\textit{Superato}}\\ \hline
\hypertarget{TU171}{TU171} & Vogliamo testare che, se la chiamata ai metodi di \file{ConversationsDAO} genera un errore, viene chiamata la funzione di callback con un solo parametro diverso da null.
 & \textcolor{green}{\textit{Superato}}\\ \hline
\hypertarget{TU172}{TU172} & Vogliamo testare che, se le chiamate ai microservizi e le chiamate ai DAO non generano alcun errore, viene chiamata la funzione di callback con due parametri, il primo uguale a null e il secondo contenente la risposta. & \textcolor{green}{\textit{Superato}}\\ \hline
\hypertarget{TU173}{TU173} & Vogliamo testare che, se la richiesta \gl{HTTP} genera un errore, venga chiamato il metodo \file{reject} della Promise. & \textcolor{green}{\textit{Superato}}\\ \hline
\hypertarget{TU174}{TU174} & Vogliamo testare che, se la richiesta HTTP va a buon fine, venga chiamato il metodo \file{fulfill} della Promise. & \textcolor{green}{\textit{Superato}}\\ \hline
\hypertarget{TU175}{TU175} & Se la chiamata al modulo \file{VAModule} genera un'errore, lo status code della risposta deve essere uguale al codice di errore ricevuto. & \textcolor{green}{\textit{Superato}}\\ \hline
\hypertarget{TU176}{TU176} & Se la richiesta HTTP ad api.ai va a buon fine allora lo status code della risposta deve essere uguale a 200. & \textcolor{green}{\textit{Superato}}\\ \hline
\hypertarget{TU177}{TU177} & L'\file{ApplicationPackage} passato come parametro al metodo register dev'essere inserito con name uguale al parametro passato. L'\file{ApplicationPackage} inserito dev'essere ritornato tramite il metodo query. & \textcolor{green}{\textit{Superato}}\\ \hline
\hypertarget{TU178}{TU178} & L'\file{ApplicationPackage} passato come parametro al metodo register dev'essere inserito con name uguale al parametro passato. L'\file{ApplicationPackage} inserito dev'essere eliminato tramite il metodo remove. & \textcolor{green}{\textit{Superato}}\\ \hline
\hypertarget{TU179}{TU179} & Vogliamo testare che, nel caso in cui l'interrogazione del \file{ApplicationLocalRegistry} vada a buon fine, l'\file{Observable} restituito deve chiamare il metodo \file{next} dell'\file{observer} iscritto con i dati ottenuti dall'interrogazione, ed in seguito il metodo \file{complete} un'unica volta. & \textcolor{green}{\textit{Superato}}\\ \hline
\hypertarget{TU180}{TU180} & Vogliamo testare che, nel caso in cui venga aggiunto correttamente l’\file{ApplicationPackage} passato come parametro, l'\file{Observable} chiami la \file{complete} dell'\file{Observer} iscritto. & \textcolor{green}{\textit{Superato}}\\ \hline
\hypertarget{TU181}{TU181} & Vogliamo testare che, alla chiamata del metodo, l’\file{Observable} notifichi tutti gli \file{Observer} iscritti passando loro un oggetto composto dai parametri con cui il metodo è stato chiamato. & \textcolor{green}{\textit{Superato}}\\ \hline
\hypertarget{TU182}{TU182} & Vogliamo testare che l’oggetto ritornato dalla funzione sia effettivamente un \file{ReactElement}. & \textcolor{green}{\textit{Superato}}\\ \hline
\hypertarget{TU183}{TU183} & Vogliamo testare che, se l’applicazione è presente all'interno di \file{State}, non venga interrogato il Client. & \textcolor{green}{\textit{Superato}}\\ \hline
\hypertarget{TU184}{TU184} & Vogliamo testare che, se l’applicazione non è presente all'interno di \file{State}, venga interrogato il Client per ottenerla e la vecchia applicazione viene salvata nello \file{State}. & \textcolor{green}{\textit{Superato}}\\ \hline
\hypertarget{TU185}{TU185} & Vogliamo testare che venga chiamato \file{appendChild} sul parametro passato al metodo per poter mostrare l’interfaccia utente. & \textcolor{green}{\textit{Superato}}\\ \hline
\hypertarget{TU186}{TU186} & Vogliamo testare che, se \file{action.cmd} è uguale a “clear”, venga chiamato il metodo \file{onClear} e vengano notificati gli \file{Observer} iscritti all’\file{Observable}. & \textcolor{green}{\textit{Superato}}\\ \hline
\hypertarget{TU187}{TU187} & Vogliamo testare che, se \file{action.cmd} è uguale a “displayMsgs”, venga chiamato il metodo \file{onDisplayMsgs} e vengano notificati gli \file{Observer} iscritti all’\file{Observable}. & \textcolor{green}{\textit{Superato}}\\ \hline
\hypertarget{TU188}{TU188} & Vogliamo testare che, se \file{action.cmd} è uguale a “msgReceived”, venga chiamato il metodo \file{onMsgReceived} e vengano notificati gli \file{Observer} iscritti all’\file{Observable}. & \textcolor{green}{\textit{Superato}}\\ \hline
\hypertarget{TU189}{TU189} & Vogliamo testare che, se \file{action.cmd} è uguale a “msgSent”, venga chiamato il metodo \file{onMsgSent} e vengano notificati gli \file{Observer} iscritti all’\file{Observable}. & \textcolor{green}{\textit{Superato}}\\ \hline
\hypertarget{TU190}{TU190} & Vogliamo testare che, se \file{action.cmd} non corrisponde a nessuna delle action prestabilite, non vengano notificati gli \file{Observer} e non venga sollevata alcuna eccezione. & \textcolor{green}{\textit{Superato}}\\ \hline
\hypertarget{TU191}{TU191} & Il metodo aggiunge correttamente l’\file{Application} passata come parametro e restituisce l’\file{Application} a partire dal suo nome passato come parametro.
 & \textcolor{green}{\textit{Superato}}\\ \hline
\hypertarget{TU192}{TU192} & Vogliamo testare che richiami il metodo \file{dispatcher.dispatch} inoltrandogli i parametri ricevuti. & \textcolor{green}{\textit{Superato}}\\ \hline

\hypertarget{TU194}{TU194} & Vogliamo testare che se la richiesta va a buon fine, venga chiamato il metodo then. & \textit{Non Implementato}\\ \hline
\hypertarget{TU195}{TU195} & Vogliamo tesatare che se la richiesta fallisce, venga chiamato il metodo catch. & \textit{Non Implementato}\\ \hline
\hypertarget{TU196}{TU196} & Vogliamo testare che, se la promessa viene soddisfatta (fulfill), venga chiamato il metodo \file{next} del \file{subject} che si occupa di notificare gli \file{Observer} iscritti. & \textit{Non Implementato}\\ \hline
\hypertarget{TU197}{TU197} & Vogliamo testare che, se la promessa viene respinta (reject), venga chiamato il metodo \file{error} del \file{subject} che si occupa di notificare tale errore agli \file{Observer} iscritti. & \textit{Non Implementato}\\ \hline
\hypertarget{TU198}{TU198} & Vogliamo testare che, una volta chiamato il metodo \file{start}, venga inviata una serie di oggetti \file{RecorderMsg} a \file{RecorderWorker} con campo \file{command} uguale a “record” e che questa serie di messaggi venga interrotta alla chiamata del metodo \file{stop}. & \textcolor{green}{\textit{Superato}}\\ \hline
\hypertarget{TU199}{TU199} & Vogliamo testare che, nel caso in cui la \file{Rule} richiesta non sia disponibile, il metodo chiami \file{context.success} con una LambdaResponse il cui campo status è impostato a 404. & \textcolor{green}{\textit{Superato}}\\ \hline
\hypertarget{TU200}{TU200} & Vogliamo testare che, nel caso in cui la \file{Rule} richiesta non sia disponibile, il metodo chiami \file{context.success} con una LambdaResponse il cui campo status è impostato a 404. & \textcolor{green}{\textit{Superato}}\\ \hline
\hypertarget{TU201}{TU201} & Vogliamo testare che, nel caso in cui l'\file{User} richiesto non sia disponibile, il metodo chiami \file{context.success} con una LambdaResponse il cui campo status è impostato a 404. & \textcolor{green}{\textit{Superato}}\\ \hline
\hypertarget{TU202}{TU202} & Vogliamo testare che, nel caso in cui l'\file{User} richiesto non sia disponibile, il metodo chiami \file{context.success} con una LambdaResponse il cui campo status è impostato a 404. & \textcolor{green}{\textit{Superato}}\\ \hline
\hypertarget{TU203}{TU203} & Vogliamo testare che, se l'utente viene riconosciuto come possibile amministratore del sistema, il campo \file{name} del \file{context} della risposta sia uguale ad "admin". & \textcolor{green}{\textit{Superato}}\\ \hline
\hypertarget{TU204}{TU204} & Vogliamo testare che, se l'utente viene riconosciuto come ospite che ha avuto interazioni passate con il sistema, il \file{name} del \file{context} della risposta sia uguale a "welcome". & \textcolor{green}{\textit{Superato}}\\ \hline
\hypertarget{TU207}{TU207} & Vogliamo testare che, se si verifica un errore nella richiesta delle informazioni sugli utenti a Slack, venga chiamato il metodo \file{succeed} del \file{context} con un parametro \file{LambdaResponse} il quale campo \file{statusCode} è impostato a 500.
 & \textcolor{green}{\textit{Superato}}\\ \hline
\hypertarget{TU208}{TU208} & Vogliamo testare che, se si verifica un errore nella richiesta delle informazioni sui gruppi a Slack, venga chiamato il metodo \file{succeed} del \file{context} con un parametro \file{LambdaResponse} il quale campo \file{statusCode} è impostato a 500.
 & \textcolor{green}{\textit{Superato}}\\ \hline
\hypertarget{TU209}{TU209} & Vogliamo testare che alla chiamata del metodo non venga chiamata la funzione di callback \file{complete\_cb}, se l'\file{observer} è in pausa. & \textcolor{green}{\textit{Superato}}\\ \hline
\hypertarget{TU210}{TU210} & Vogliamo testare che complete\_cb sia chiamato dopo che l'observer è stato ripreso se, mentre era in pausa, il metodo è stato chiamato. & \textcolor{green}{\textit{Superato}}\\ \hline
\hypertarget{TU211}{TU211} & Vogliamo testare che non venga chiamata la funzione di callback \file{error\_cb}, se l'observer è in pausa & \textcolor{green}{\textit{Superato}}\\ \hline
\hypertarget{TU212}{TU212} & Vogliamo testare che \file{error\_cb} sia chiamato dopo che l\'observer è stato ripreso se, mentre era in pausa, il metodo è stato chiamato.
 & \textcolor{green}{\textit{Superato}}\\ \hline
\hypertarget{TU213}{TU213} & Vogliamo testare che non venga chiamata la funzione di callback \file{next\_cb}, se l'observer è in pausa.
 & \textcolor{green}{\textit{Superato}}\\ \hline
\hypertarget{TU214}{TU214} & Vogliamo testare che \file{next\_cb} sia chiamato dopo che l'observer è stato ripreso se, mentre era in pausa, il metodo è stato chiamato.
 & \textcolor{green}{\textit{Superato}}\\ \hline
\hypertarget{TU215}{TU215} & Vogliamo testare che, se viene passato un oggetto la cui chiave primaria è uguale a quella di un oggetto già presente nel database, venga chiamato il metodo \file{succeed} del \file{context} con un parametro \file{LambdaResponse} avente campo \file{statusCode} pari a 409. & \textcolor{green}{\textit{Superato}}\\ \hline
\hypertarget{TU216}{TU216} & Vogliamo testare che, se viene passato un oggetto la cui chiave primaria è uguale a quella di un oggetto già presente nel database, venga chiamato il metodo \file{succeed} del \file{context} con un parametro \file{LambdaResponse} avente campo \file{statusCode} pari a 409. & \textcolor{green}{\textit{Superato}}\\ \hline
\hypertarget{TU217}{TU217} & Vogliamo testare che, se la chiamata al metodo va a buon fine, venga chiamato il metodo \file{succeed} del \file{context} con un parametro \file{LambdaResponse} avente campo \file{statusCode} pari a 200 e campo \file{body} contenente la lista degli \file{Task}. & \textcolor{green}{\textit{Superato}}\\ \hline
\hypertarget{TU218}{TU218} & Nel caso in cui si provi ad inserire un Package parziale, l'Observable deve chiamare il metodo error dell'Observer iscritto & \textcolor{green}{\textit{Superato}}\\ \hline
\hypertarget{TU219}{TU219} & Vogliamo testare che, anche se viene passato un \file{Member} corretto, il metodo ritorni un \file{ErrorObservable} che notifica l'\file{Observer} richiamando il suo metodo \file{error}. Infatti non è possibile aggiornare un membro perciò il metodo fallisce sempre. & \textcolor{green}{\textit{Superato}}\\ \hline
\hypertarget{TU220}{TU220} & Vogliamo testare che, se il body della richiesta non rispetta la struttura attesa, venga chiamato il metodo \file{succeed} del \file{context} con un parametro \file{LambdaResponse} il quale campo \file{statusCode} è impostato a 400. & \textcolor{green}{\textit{Superato}}\\ \hline
\hypertarget{TU221}{TU221} & Nel caso in cui il metodo venga chiamato con un parametro \file{queryStringParameters}, l'\file{Observable} restituito deve chiamare il metodo next dell'\file{Observer} iscritto con i dati filtrati ottenuti dall'interrogazione, ed in seguito il metodo complete un'unica volta. & \textcolor{green}{\textit{Superato}}\\ \hline
\hypertarget{TU222}{TU222} & Nel caso in cui il metodo venga chiamato con un oggetto di tipo \file{queryStringParameters} avente un attributo, l'\file{Observable} restituito deve chiamare il metodo next dell'\file{Observer} iscritto con i dati filtrati ottenuti dall'interrogazione, ed in seguito il metodo complete un'unica volta. & \textcolor{green}{\textit{Superato}}\\ \hline
\hypertarget{TU223}{TU223} & Nel caso in cui il metodo venga chiamato con un oggetto di tipo \file{queryStringParameters} avente due attributi, l'\file{Observable} restituito deve chiamare il metodo next dell'\file{Observer} iscritto con i dati filtrati ottenuti dall'interrogazione, ed in seguito il metodo complete un'unica volta. & \textcolor{green}{\textit{Superato}}\\ \hline
\caption[Test di Unità]{Test di Unità}
\label{tabella:test3}
\end{longtable}
\clearpage

	\subsection{Tracciamento Test di Validazione-Requisiti}
\normalsize
\begin{longtable}{|>{\centering}m{5cm}|m{5cm}<{\centering}|}
\hline 
\textbf{Test} & \textbf{Requisito}\\
\hline
\endhead
\hyperlink{TVFO1}{TVFO1} & RFO1\\ \hline
\hyperlink{TVFO1.1.2}{TVFO1.1.2} & RFO1.1.2\\ \hline
\hyperlink{TVFO2.1}{TVFO2.1} & RFO2.1\\ \hline
\hyperlink{TVFO2.1.1.6}{TVFO2.1.1.6} & RFO2.1.1.6\\ \hline
\hyperlink{TVFO2.1.2}{TVFO2.1.2} & RFO2.1.2\\ \hline
\hyperlink{TVFD2.1.3}{TVFD2.1.3} & RFD2.1.3\\ \hline
\hyperlink{TVFD2.1.3.7}{TVFD2.1.3.7} & RFD2.1.3.7\\ \hline
\hyperlink{TVFO2.1.4}{TVFO2.1.4} & RFO2.1.4\\ \hline
\hyperlink{TVFO2.2}{TVFO2.2} & RFO2.2\\ \hline
\hyperlink{TVFO3.1}{TVFO3.1} & RFO3.1\\ \hline
\hyperlink{TVFD3.2}{TVFD3.2} & RFD3.2\\ \hline
\hyperlink{TVFD3.3}{TVFD3.3} & RFD3.3\\ \hline
\hyperlink{TVFO5}{TVFO5} & RFO5\\ \hline
\hyperlink{TVFO7}{TVFO7} & RFO7\\ \hline
\hyperlink{TVFO8}{TVFO8} & RFO8\\ \hline
\hyperlink{TVFD9.1.1}{TVFD9.1.1} & RFD9.1.1\\ \hline
\hyperlink{TVFD9.1.1.4}{TVFD9.1.1.4} & RFD9.1.1.4\\ \hline
\hyperlink{TVFD9.1.3}{TVFD9.1.3} & RFD9.1.3\\ \hline
\hyperlink{TVFD9.1.3.2}{TVFD9.1.3.2} & RFD9.1.3.2\\ \hline
\caption[Tracciamento Test di Validazione-Requisiti]{Tracciamento Test di Validazione-Requisiti}
\label{tabella:tv-requi}
\end{longtable}
\clearpage

	\subsection{Tracciamento Componenti-Test di Integrazione}
\normalsize
\begin{longtable}{|>{\centering}m{9cm}|m{3cm}<{\centering}|}
\hline 
\textbf{Componente} & \textbf{Test}\\
\hline
\endhead
\texttt{Back-end} & \hyperlink{TI2}{TI2}\\ \hline
\texttt{Back-end::APIGateway} & \hyperlink{TI8}{TI8}\\ \hline
\texttt{Back-end::Conversations} & \hyperlink{TI14}{TI14}\\ \hline
\texttt{Back-end::Events} & \hyperlink{TI15}{TI15}\\ \hline
\texttt{Back-end::Guests} & \hyperlink{TI13}{TI13}\\ \hline
\texttt{Back-end::Members} & \hyperlink{TI12}{TI12}\\ \hline
\texttt{Back-end::Notifications} & \hyperlink{TI16}{TI16}\\ \hline
\texttt{Back-end::Rules} & \hyperlink{TI10}{TI10}\\ \hline
\texttt{Back-end::Users} & \hyperlink{TI9}{TI9}\\ \hline
\texttt{Back-end::Utility} & \hyperlink{TI17}{TI17}\\ \hline
\texttt{Back-end::VirtualAssistant} & \hyperlink{TI11}{TI11}\\ \hline
\texttt{Client} & \hyperlink{TI1}{TI1}\\ \hline
\texttt{Client::ApplicationManager} & \hyperlink{TI3}{TI3}\\ \hline
\texttt{Client::ConversationApp} & \hyperlink{TI18}{TI18}\\ \hline
\texttt{Client::Logic} & \hyperlink{TI4}{TI4}\\ \hline
\texttt{Client::Recorder} & \hyperlink{TI5}{TI5}\\ \hline
\texttt{Client::TTS} & \hyperlink{TI6}{TI6}\\ \hline
\texttt{Client::Utility} & \hyperlink{TI7}{TI7}\\ \hline
\caption[Tracciamento Componenti-Test di Integrazione]{Tracciamento Componenti-Test di Integrazione}
\label{tabella:pkg-ti}
\end{longtable}
\clearpage

	\subsection{Tracciamento Metodi-Test di Unità}
\normalsize
\begin{longtable}{|>{\centering}m{12cm}|m{1cm}<{\centering}|}
\hline 
\textbf{Metodo} & \textbf{Test}\\
\hline
\endhead\texttt{Back-end::AdministrationWebhookService::-\linebreak webhook()} & \hyperlink{TU1}{TU1}\\ & \hyperlink{TU2}{TU2}\\ \hline
\texttt{Back-end::APIGateway::VocalAPI::-\linebreak addRule()} & \hyperlink{TU33}{TU33}\\ \hline
\texttt{Back-end::APIGateway::VocalAPI::-\linebreak addUser()} & \hyperlink{TU34}{TU34}\\ \hline
\texttt{Back-end::APIGateway::VocalAPI::-\linebreak addUserEnrollment()} & \hyperlink{TU35}{TU35}\\ \hline
\texttt{Back-end::APIGateway::VocalAPI::-\linebreak getRule()} & \hyperlink{TU36}{TU36}\\ \hline
\texttt{Back-end::APIGateway::VocalAPI::-\linebreak getRuleList()} & \hyperlink{TU37}{TU37}\\ \hline
\texttt{Back-end::APIGateway::VocalAPI::-\linebreak getUser()} & \hyperlink{TU38}{TU38}\\ \hline
\texttt{Back-end::APIGateway::VocalAPI::-\linebreak getUserList()} & \hyperlink{TU39}{TU39}\\ \hline
\texttt{Back-end::APIGateway::VocalAPI::-\linebreak loginUser()} & \hyperlink{TU40}{TU40}\\ \hline
\texttt{Back-end::APIGateway::VocalAPI::-\linebreak queryLambda()} & \hyperlink{TU6}{TU6}\\ & \hyperlink{TU7}{TU7}\\ & \hyperlink{TU8}{TU8}\\ & \hyperlink{TU9}{TU9}\\ & \hyperlink{TU10}{TU10}\\ & \hyperlink{TU11}{TU11}\\ & \hyperlink{TU12}{TU12}\\ & \hyperlink{TU13}{TU13}\\ & \hyperlink{TU14}{TU14}\\ & \hyperlink{TU15}{TU15}\\ & \hyperlink{TU16}{TU16}\\ & \hyperlink{TU17}{TU17}\\ & \hyperlink{TU18}{TU18}\\ & \hyperlink{TU19}{TU19}\\ & \hyperlink{TU20}{TU20}\\ & \hyperlink{TU21}{TU21}\\ & \hyperlink{TU22}{TU22}\\ & \hyperlink{TU23}{TU23}\\ & \hyperlink{TU24}{TU24}\\ & \hyperlink{TU25}{TU25}\\ & \hyperlink{TU26}{TU26}\\ & \hyperlink{TU27}{TU27}\\ & \hyperlink{TU28}{TU28}\\ & \hyperlink{TU29}{TU29}\\ & \hyperlink{TU30}{TU30}\\ & \hyperlink{TU31}{TU31}\\ & \hyperlink{TU32}{TU32}\\ & \hyperlink{TU46}{TU46}\\ & \hyperlink{TU47}{TU47}\\ \hline
\texttt{Back-end::APIGateway::VocalAPI::-\linebreak removeRule()} & \hyperlink{TU41}{TU41}\\ \hline
\texttt{Back-end::APIGateway::VocalAPI::-\linebreak removeUser()} & \hyperlink{TU42}{TU42}\\ \hline
\texttt{Back-end::APIGateway::VocalAPI::-\linebreak resetUserEnrollment()} & \hyperlink{TU43}{TU43}\\ \hline
\texttt{Back-end::APIGateway::VocalAPI::-\linebreak updateRule()} & \hyperlink{TU44}{TU44}\\ \hline
\texttt{Back-end::APIGateway::VocalAPI::-\linebreak updateUser()} & \hyperlink{TU45}{TU45}\\ \hline
\texttt{Back-end::Conversations::-\linebreak <<interface>> ConversationsDAO::removeConversation()} & \hyperlink{TU59}{TU59}\\ \hline
\texttt{Back-end::Conversations::ConversationObserver::-\linebreak next()} & \hyperlink{TU48}{TU48}\\ & \hyperlink{TU49}{TU49}\\ \hline
\texttt{Back-end::Conversations::ConversationsDAODynamoDB::-\linebreak addConversation()} & \hyperlink{TU50}{TU50}\\ & \hyperlink{TU51}{TU51}\\ \hline
\texttt{Back-end::Conversations::ConversationsDAODynamoDB::-\linebreak addMessage()} & \hyperlink{TU52}{TU52}\\ & \hyperlink{TU53}{TU53}\\ \hline
\texttt{Back-end::Conversations::ConversationsDAODynamoDB::-\linebreak getConversation()} & \hyperlink{TU54}{TU54}\\ & \hyperlink{TU55}{TU55}\\ \hline
\texttt{Back-end::Conversations::ConversationsDAODynamoDB::-\linebreak getConversationList()} & \hyperlink{TU56}{TU56}\\ & \hyperlink{TU57}{TU57}\\ \hline
\texttt{Back-end::Conversations::ConversationsDAODynamoDB::-\linebreak removeConversation()} & \hyperlink{TU58}{TU58}\\ \hline
\texttt{Back-end::Events::VAMessageListener::-\linebreak onMessage()} & \hyperlink{TU177}{TU177}\\ & \hyperlink{TU178}{TU178}\\ & \hyperlink{TU179}{TU179}\\ & \hyperlink{TU180}{TU180}\\ & \hyperlink{TU181}{TU181}\\ \hline
\texttt{Back-end::Guests::GuestObserver::next()} & \hyperlink{TU60}{TU60}\\ & \hyperlink{TU61}{TU61}\\ \hline
\texttt{Back-end::Guests::GuestsDAODynamoDB::-\linebreak addGuest()} & \hyperlink{TU62}{TU62}\\ & \hyperlink{TU63}{TU63}\\ \hline
\texttt{Back-end::Guests::GuestsDAODynamoDB::-\linebreak getGuest()} & \hyperlink{TU64}{TU64}\\ & \hyperlink{TU65}{TU65}\\ \hline
\texttt{Back-end::Guests::GuestsDAODynamoDB::-\linebreak getGuestList()} & \hyperlink{TU66}{TU66}\\ & \hyperlink{TU67}{TU67}\\ \hline
\texttt{Back-end::Guests::GuestsDAODynamoDB::-\linebreak removeGuest()} & \hyperlink{TU68}{TU68}\\ & \hyperlink{TU69}{TU69}\\ \hline
\texttt{Back-end::Guests::GuestsDAODynamoDB::-\linebreak updateGuest()} & \hyperlink{TU70}{TU70}\\ & \hyperlink{TU71}{TU71}\\ \hline
\texttt{Back-end::Members::MemberObserver::-\linebreak next()} & \hyperlink{TU72}{TU72}\\ & \hyperlink{TU73}{TU73}\\ \hline
\texttt{Back-end::Members::MembersDAOSlack::-\linebreak addMember()} & \hyperlink{TU78}{TU78}\\ \hline
\texttt{Back-end::Members::MembersDAOSlack::-\linebreak getMember()} & \hyperlink{TU74}{TU74}\\ & \hyperlink{TU75}{TU75}\\ \hline
\texttt{Back-end::Members::MembersDAOSlack::-\linebreak getMemberList()} & \hyperlink{TU76}{TU76}\\ & \hyperlink{TU77}{TU77}\\ \hline
\texttt{Back-end::Members::MembersDAOSlack::-\linebreak removeMember()} & \hyperlink{TU79}{TU79}\\ \hline
\texttt{Back-end::Notifications::NotificationService::-\linebreak getChannelList()} & \hyperlink{TU80}{TU80}\\ & \hyperlink{TU81}{TU81}\\ \hline
\texttt{Back-end::Notifications::NotificationService::-\linebreak sendMsg()} & \hyperlink{TU82}{TU82}\\ & \hyperlink{TU83}{TU83}\\ \hline
\texttt{Back-end::Rules::RuleObserver::next()} & \hyperlink{TU87}{TU87}\\ & \hyperlink{TU88}{TU88}\\ \hline
\texttt{Back-end::Rules::RulesDAODynamoDB::-\linebreak addRule()} & \hyperlink{TU89}{TU89}\\ & \hyperlink{TU90}{TU90}\\ \hline
\texttt{Back-end::Rules::RulesDAODynamoDB::-\linebreak getRule()} & \hyperlink{TU91}{TU91}\\ & \hyperlink{TU92}{TU92}\\ \hline
\texttt{Back-end::Rules::RulesDAODynamoDB::-\linebreak getRuleList()} & \hyperlink{TU93}{TU93}\\ & \hyperlink{TU94}{TU94}\\ \hline
\texttt{Back-end::Rules::RulesDAODynamoDB::-\linebreak removeRule()} & \hyperlink{TU95}{TU95}\\ & \hyperlink{TU96}{TU96}\\ \hline
\texttt{Back-end::Rules::RulesDAODynamoDB::-\linebreak updateRule()} & \hyperlink{TU97}{TU97}\\ & \hyperlink{TU98}{TU98}\\ \hline
\texttt{Back-end::Rules::RulesService::-\linebreak addRule()} & \hyperlink{TU141}{TU141}\\ & \hyperlink{TU142}{TU142}\\ & \hyperlink{TU143}{TU143}\\ \hline
\texttt{Back-end::Rules::RulesService::-\linebreak deleteRule()} & \hyperlink{TU144}{TU144}\\ & \hyperlink{TU145}{TU145}\\ & \hyperlink{TU146}{TU146}\\ & \hyperlink{TU213}{TU213}\\ \hline
\texttt{Back-end::Rules::RulesService::-\linebreak getRule()} & \hyperlink{TU147}{TU147}\\ & \hyperlink{TU148}{TU148}\\ & \hyperlink{TU149}{TU149}\\ & \hyperlink{TU214}{TU214}\\ \hline
\texttt{Back-end::Rules::RulesService::-\linebreak getRuleList()} & \hyperlink{TU150}{TU150}\\ & \hyperlink{TU151}{TU151}\\ & \hyperlink{TU152}{TU152}\\ \hline
\texttt{Back-end::Rules::RulesService::-\linebreak getTaskList()} & \hyperlink{TU155}{TU155}\\ \hline
\texttt{Back-end::Rules::RulesService::-\linebreak queryRule()} & \hyperlink{TU156}{TU156}\\ & \hyperlink{TU157}{TU157}\\ & \hyperlink{TU158}{TU158}\\ \hline
\texttt{Back-end::Rules::RulesService::-\linebreak updateRule()} & \hyperlink{TU159}{TU159}\\ & \hyperlink{TU160}{TU160}\\ & \hyperlink{TU161}{TU161}\\ \hline
\texttt{Back-end::Rules::TaskObserver::next()} & \hyperlink{TU110}{TU110}\\ & \hyperlink{TU111}{TU111}\\ \hline
\texttt{Back-end::Rules::TasksDAODynamoDB::-\linebreak addTask()} & \hyperlink{TU99}{TU99}\\ & \hyperlink{TU100}{TU100}\\ \hline
\texttt{Back-end::Rules::TasksDAODynamoDB::-\linebreak getTask()} & \hyperlink{TU101}{TU101}\\ & \hyperlink{TU102}{TU102}\\ \hline
\texttt{Back-end::Rules::TasksDAODynamoDB::-\linebreak getTaskList()} & \hyperlink{TU103}{TU103}\\ & \hyperlink{TU104}{TU104}\\ \hline
\texttt{Back-end::Rules::TasksDAODynamoDB::-\linebreak removeTask()} & \hyperlink{TU105}{TU105}\\ & \hyperlink{TU106}{TU106}\\ \hline
\texttt{Back-end::Rules::TasksDAODynamoDB::-\linebreak updateTask()} & \hyperlink{TU107}{TU107}\\ & \hyperlink{TU108}{TU108}\\ \hline
\texttt{Back-end::STT::STTWatsonAdapter::-\linebreak speechToText()} & \hyperlink{TU109}{TU109}\\ \hline
\texttt{Back-end::Users::<<interface>>VocalLoginModule::-\linebreak addEnrollment()} & \hyperlink{TU124}{TU124}\\ \hline
\texttt{Back-end::Users::<<interface>>VocalLoginModule::-\linebreak createUser()} & \hyperlink{TU125}{TU125}\\ \hline
\texttt{Back-end::Users::<<interface>>VocalLoginModule::-\linebreak deleteUser()} & \hyperlink{TU126}{TU126}\\ \hline
\texttt{Back-end::Users::<<interface>>VocalLoginModule::-\linebreak doLogin()} & \hyperlink{TU127}{TU127}\\ \hline
\texttt{Back-end::Users::<<interface>>VocalLoginModule::-\linebreak resetEnrollments()} & \hyperlink{TU130}{TU130}\\ \hline
\texttt{Back-end::Users::UserObserver::next()} & \hyperlink{TU112}{TU112}\\ & \hyperlink{TU113}{TU113}\\ \hline
\texttt{Back-end::Users::UsersDAODynamoDB::-\linebreak addUser()} & \hyperlink{TU114}{TU114}\\ & \hyperlink{TU115}{TU115}\\ \hline
\texttt{Back-end::Users::UsersDAODynamoDB::-\linebreak getUser()} & \hyperlink{TU116}{TU116}\\ & \hyperlink{TU117}{TU117}\\ \hline
\texttt{Back-end::Users::UsersDAODynamoDB::-\linebreak getUserList()} & \hyperlink{TU118}{TU118}\\ & \hyperlink{TU119}{TU119}\\ \hline
\texttt{Back-end::Users::UsersDAODynamoDB::-\linebreak removeUser()} & \hyperlink{TU120}{TU120}\\ & \hyperlink{TU121}{TU121}\\ \hline
\texttt{Back-end::Users::UsersDAODynamoDB::-\linebreak updateUser()} & \hyperlink{TU122}{TU122}\\ & \hyperlink{TU123}{TU123}\\ \hline
\texttt{Back-end::Users::UsersService::-\linebreak addUser()} & \hyperlink{TU162}{TU162}\\ & \hyperlink{TU163}{TU163}\\ & \hyperlink{TU164}{TU164}\\ \hline
\texttt{Back-end::Users::UsersService::-\linebreak deleteUser()} & \hyperlink{TU171}{TU171}\\ & \hyperlink{TU172}{TU172}\\ & \hyperlink{TU173}{TU173}\\ & \hyperlink{TU216}{TU216}\\ \hline
\texttt{Back-end::Users::UsersService::-\linebreak getUser()} & \hyperlink{TU165}{TU165}\\ & \hyperlink{TU166}{TU166}\\ & \hyperlink{TU167}{TU167}\\ & \hyperlink{TU215}{TU215}\\ \hline
\texttt{Back-end::Users::UsersService::-\linebreak getUserList()} & \hyperlink{TU168}{TU168}\\ & \hyperlink{TU169}{TU169}\\ & \hyperlink{TU170}{TU170}\\ \hline
\texttt{Back-end::Users::UsersService::-\linebreak updateUser()} & \hyperlink{TU174}{TU174}\\ & \hyperlink{TU175}{TU175}\\ & \hyperlink{TU176}{TU176}\\ \hline
\texttt{Back-end::Users::VocalLoginMicrosoftModule::-\linebreak getList()} & \hyperlink{TU128}{TU128}\\ \hline
\texttt{Back-end::Users::VocalLoginMicrosoftModule::-\linebreak getUser()} & \hyperlink{TU129}{TU129}\\ \hline
\texttt{Back-end::VirtualAssistant::-\linebreak <<interface>> AgentsDAO::getAgentsList()} & \hyperlink{TU135}{TU135}\\ & \hyperlink{TU136}{TU136}\\ \hline
\texttt{Back-end::VirtualAssistant::-\linebreak <<interface>> AgentsDAO::removeAgent()} & \hyperlink{TU137}{TU137}\\ & \hyperlink{TU138}{TU138}\\ \hline
\texttt{Back-end::VirtualAssistant::-\linebreak <<interface>> AgentsDAO::updateAgent()} & \hyperlink{TU139}{TU139}\\ & \hyperlink{TU140}{TU140}\\ \hline
\texttt{Back-end::VirtualAssistant::AgentObserver::-\linebreak next()} & \hyperlink{TU4}{TU4}\\ & \hyperlink{TU5}{TU5}\\ \hline
\texttt{Back-end::VirtualAssistant::AgentsDAODynamoDB::-\linebreak addAgent()} & \hyperlink{TU131}{TU131}\\ & \hyperlink{TU132}{TU132}\\ \hline
\texttt{Back-end::VirtualAssistant::AgentsDAODynamoDB::-\linebreak getAgent()} & \hyperlink{TU133}{TU133}\\ & \hyperlink{TU134}{TU134}\\ \hline
\texttt{Back-end::VirtualAssistant::ApiAiVAAdapter::-\linebreak query()} & \hyperlink{TU182}{TU182}\\ & \hyperlink{TU183}{TU183}\\ \hline
\texttt{Back-end::VirtualAssistant::VAService::-\linebreak query()} & \hyperlink{TU184}{TU184}\\ & \hyperlink{TU185}{TU185}\\ & \hyperlink{TU186}{TU186}\\ \hline
\texttt{Client::ApplicationManager::-\linebreak ApplicationLocalRegistry::query()} & \hyperlink{TU189}{TU189}\\ \hline
\texttt{Client::ApplicationManager::-\linebreak ApplicationLocalRegistry::register()} & \hyperlink{TU188}{TU188}\\ \hline
\texttt{Client::ApplicationManager::-\linebreak ApplicationLocalRegistry::remove()} & \hyperlink{TU190}{TU190}\\ \hline
\texttt{Client::ApplicationManager::-\linebreak ApplicationRegistryLocalClient::query()} & \hyperlink{TU191}{TU191}\\ \hline
\texttt{Client::ApplicationManager::-\linebreak ApplicationRegistryLocalClient::register()} & \hyperlink{TU192}{TU192}\\ \hline
\texttt{Client::ApplicationManager::Manager::-\linebreak runApplication()} & \hyperlink{TU195}{TU195}\\ & \hyperlink{TU196}{TU196}\\ \hline
\texttt{Client::ApplicationManager::Manager::-\linebreak setFrame()} & \hyperlink{TU197}{TU197}\\ \hline
\texttt{Client::ApplicationManager::State::-\linebreak addApp()} & \hyperlink{TU203}{TU203}\\ \hline
\texttt{Client::ApplicationManager::State::-\linebreak getApp()} & \hyperlink{TU204}{TU204}\\ \hline
\texttt{Client::ConversationApp::ConversationApp::-\linebreak runCmd()} & \hyperlink{TU205}{TU205}\\ & \hyperlink{TU206}{TU206}\\ \hline
\texttt{Client::ConversationApp::ConversationDispatcher::-\linebreak dispatch()} & \hyperlink{TU193}{TU193}\\ \hline
\texttt{Client::ConversationApp::ConversationView::-\linebreak render()} & \hyperlink{TU194}{TU194}\\ \hline
\texttt{Client::ConversationApp::MessageStore::-\linebreak onCmd()} & \hyperlink{TU198}{TU198}\\ & \hyperlink{TU199}{TU199}\\ & \hyperlink{TU200}{TU200}\\ & \hyperlink{TU201}{TU201}\\ & \hyperlink{TU202}{TU202}\\ \hline
\texttt{Client::Logic::HttpPromise::then()} & \hyperlink{TU208}{TU208}\\ & \hyperlink{TU209}{TU209}\\ \hline
\texttt{Client::Logic::Logic::sendData()} & \hyperlink{TU210}{TU210}\\ & \hyperlink{TU211}{TU211}\\ \hline
\texttt{Client::Recorder::Recorder::start()} & \hyperlink{TU212}{TU212}\\ \hline
\texttt{Client::Recorder::Recorder::stop()} & \hyperlink{TU212}{TU212}\\ \hline
\texttt{Libs::ErrorObserver::next()} & \hyperlink{TU3}{TU3}\\ \hline
\texttt{Libs::ObserverAdapter::pause()} & \hyperlink{TU187}{TU187}\\ \hline

\caption[Tracciamento Metodi-Test di Unità]{Tracciamento Metodi-Test di Unità}
\label{tabella:met-tu}
\end{longtable}
\clearpage

	\subsection{Tracciamento Requisiti-Test di Sistema}
\normalsize
\begin{longtable}{|>{\centering}m{5cm}|m{5cm}<{\centering}|}
\hline 
\textbf{Requisito} & \textbf{Test}\\
\hline
\endhead
RFO1 & \hyperlink{TSFO1}{TSFO1}\\ \hline
RFO1.1.2.1 & \hyperlink{TSFO1.1.2.1}{TSFO1.1.2.1}\\ \hline
RFO2.1.1 & \hyperlink{TSFO2.1.1}{TSFO2.1.1}\\ \hline
RFO2.1.2 & \hyperlink{TSFO2.1.2}{TSFO2.1.2}\\ \hline
RFD2.1.3 & \hyperlink{TSFD2.1.3}{TSFD2.1.3}\\ \hline
RFO2.1.4 & \hyperlink{TSFO2.1.4}{TSFO2.1.4}\\ \hline
RFO2.2.1 & \hyperlink{TSFO2.2.1}{TSFO2.2.1}\\ \hline
RFO3.1 & \hyperlink{TSFO3.1}{TSFO3.1}\\ \hline
RFD3.2 & \hyperlink{TSFD3.2}{TSFD3.2}\\ \hline
RFD3.3 & \hyperlink{TSFD3.3}{TSFD3.3}\\ \hline
RFO5 & \hyperlink{TSFO5}{TSFO5}\\ \hline
RFO7 & \hyperlink{TSFO7}{TSFO7}\\ \hline
RFO8 & \hyperlink{TSFO8}{TSFO8}\\ \hline
RFD9.1.1 & \hyperlink{TSFD9.1.1}{TSFD9.1.1}\\ \hline
RFD9.1.3 & \hyperlink{TSFD9.1.3}{TSFD9.1.3}\\ \hline
RFD11 & \hyperlink{TSFD11}{TSFD11}\\ \hline
RFO12 & \hyperlink{TSFO12}{TSFO12}\\ \hline
RFO13 & \hyperlink{TSFO13}{TSFO13}\\ \hline
RVO1.1 & \hyperlink{TSVO1.1}{TSVO1.1}\\ \hline
RVO4 & \hyperlink{TSVO4}{TSVO4}\\ \hline
RVO5 & \hyperlink{TSVO5}{TSVO5}\\ \hline
RVO10 & \hyperlink{TSVO10}{TSVO10}\\ \hline
RVD11 & \hyperlink{TSVD11}{TSVD11}\\ \hline
RVD13 & \hyperlink{TSVD13}{TSVD13}\\ \hline
\caption[Tracciamento Requisiti-Test di Sistema]{Tracciamento Requisiti-Test di Sistema}
\label{tabella:requi-tv}
\end{longtable}
\clearpage

	\subsection{Tracciamento Requisiti-Test di Validazione}
\normalsize
\begin{longtable}{|>{\centering}m{5cm}|m{5cm}<{\centering}|}
\hline 
\textbf{Requisito} & \textbf{Test}\\
\hline
\endhead
RFO1 & \hyperlink{TVFO1}{TVFO1}\\ \hline
RFO1.1.2 & \hyperlink{TVFO1.1.2}{TVFO1.1.2}\\ \hline
RFO2.1 & \hyperlink{TVFO2.1}{TVFO2.1}\\ \hline
RFO2.1.1.6 & \hyperlink{TVFO2.1.1.6}{TVFO2.1.1.6}\\ \hline
RFO2.1.2 & \hyperlink{TVFO2.1.2}{TVFO2.1.2}\\ \hline
RFO2.1.4 & \hyperlink{TVFO2.1.4}{TVFO2.1.4}\\ \hline
RFO2.2 & \hyperlink{TVFO2.2}{TVFO2.2}\\ \hline
RFO3.1 & \hyperlink{TVFO3.1}{TVFO3.1}\\ \hline
RFO5 & \hyperlink{TVFO5}{TVFO5}\\ \hline
RFO7 & \hyperlink{TVFO7}{TVFO7}\\ \hline
\caption[Tracciamento Requisiti-Test di Validazione]{Tracciamento Requisiti-Test di Validazione}
\label{tabella:requi-tv}
\end{longtable}
\clearpage

	\subsection{Tracciamento Test di Integrazione-Componenti}
\normalsize
\begin{longtable}{|>{\centering}m{3cm}|m{9cm}<{\centering}|}
\hline 
\textbf{Test} & \textbf{Componente}\\
\hline
\endhead
\hyperlink{TI1}{TI1} & \texttt{Client}\\ \hline
\hyperlink{TI2}{TI2} & \texttt{Back-end}\\ \hline
\hyperlink{TI3}{TI3} & \texttt{Client::ApplicationManager}\\ \hline
\hyperlink{TI4}{TI4} & \texttt{Client::Logic}\\ \hline
\hyperlink{TI5}{TI5} & \texttt{Client::Recorder}\\ \hline
\hyperlink{TI6}{TI6} & \texttt{Client::TTS}\\ \hline
\hyperlink{TI7}{TI7} & \texttt{Client::Utility}\\ \hline
\hyperlink{TI8}{TI8} & \texttt{Back-end::APIGateway}\\ \hline
\hyperlink{TI9}{TI9} & \texttt{Back-end::Users}\\ \hline
\hyperlink{TI10}{TI10} & \texttt{Back-end::Rules}\\ \hline
\hyperlink{TI11}{TI11} & \texttt{Back-end::VirtualAssistant}\\ \hline
\hyperlink{TI12}{TI12} & \texttt{Back-end::Members}\\ \hline
\hyperlink{TI13}{TI13} & \texttt{Back-end::Guests}\\ \hline
\hyperlink{TI14}{TI14} & \texttt{Back-end::Conversations}\\ \hline
\hyperlink{TI15}{TI15} & \texttt{Back-end::Events}\\ \hline
\hyperlink{TI16}{TI16} & \texttt{Back-end::Notifications}\\ \hline
\hyperlink{TI17}{TI17} & \texttt{Back-end::Utility}\\ \hline
\hyperlink{TI18}{TI18} & \texttt{Client::ConversationApp}\\ \hline
\caption[Tracciamento Test di Integrazione-Componenti]{Tracciamento Test di Integrazione-Componenti}
\label{tabella:ts-requi}
\end{longtable}
\clearpage

	\subsection{Tracciamento Test di Sistema-Requisiti}
\normalsize
\begin{longtable}{|>{\centering}m{5cm}|m{5cm}<{\centering}|}
\hline 
\textbf{Test} & \textbf{Requisito}\\
\hline
\endhead
\hyperlink{TSFO1}{TSFO1} & RFO1\\ \hline
\hyperlink{TSFO1.1.2.1}{TSFO1.1.2.1} & RFO1.1.2.1\\ \hline
\hyperlink{TSFO2.1.1}{TSFO2.1.1} & RFO2.1.1\\ \hline
\hyperlink{TSFO2.1.2}{TSFO2.1.2} & RFO2.1.2\\ \hline
\hyperlink{TSFO2.1.4}{TSFO2.1.4} & RFO2.1.4\\ \hline
\hyperlink{TSFO2.2.1}{TSFO2.2.1} & RFO2.2.1\\ \hline
\hyperlink{TSFO3.1}{TSFO3.1} & RFO3.1\\ \hline
\hyperlink{TSFO5}{TSFO5} & RFO5\\ \hline
\hyperlink{TSFO7}{TSFO7} & RFO7\\ \hline
\hyperlink{TSFO8}{TSFO8} & RFO8\\ \hline
\hyperlink{TSFO13}{TSFO13} & RFO13\\ \hline
\hyperlink{TSVO1.1}{TSVO1.1} & RVO1.1\\ \hline
\hyperlink{TSVO4}{TSVO4} & RVO4\\ \hline
\hyperlink{TSVO5}{TSVO5} & RVO5\\ \hline
\hyperlink{TSVO10}{TSVO10} & RVO10\\ \hline
\caption[Tracciamento Test di Sistema-Requisiti]{Tracciamento Test di Sistema-Requisiti}
\label{tabella:ts-requi}
\end{longtable}
\clearpage

	\subsection{Tracciamento Test di Unità-Metodi}
\normalsize
\begin{longtable}{|>{\centering}m{1cm}|m{12cm}<{\centering}|}
\hline
\textbf{Test} & \textbf{Metodi}\\
\hline
\endhead
\hyperlink{TU1}{TU1} & \texttt{Back-end::AdministrationWebhookService::-\linebreak webhook()}\\ \hline

\hyperlink{TU2}{TU2} & \texttt{Back-end::AdministrationWebhookService::-\linebreak webhook()}\\ \hline

\hyperlink{TU3}{TU3} & \texttt{Libs::ErrorObserver::next()}\\ \hline

\hyperlink{TU4}{TU4} & \texttt{Back-end::VirtualAssistant::AgentObserver::-\linebreak next()}\\ \hline

\hyperlink{TU5}{TU5} & \texttt{Back-end::VirtualAssistant::AgentObserver::-\linebreak next()}\\ \hline

\hyperlink{TU6}{TU6} & \texttt{Back-end::APIGateway::VocalAPI::-\linebreak queryLambda()}\\ \hline

\hyperlink{TU7}{TU7} & \texttt{Back-end::APIGateway::VocalAPI::-\linebreak queryLambda()}\\ \hline

\hyperlink{TU8}{TU8} & \texttt{Back-end::APIGateway::VocalAPI::-\linebreak queryLambda()}\\ \hline

\hyperlink{TU9}{TU9} & \texttt{Back-end::APIGateway::VocalAPI::-\linebreak queryLambda()}\\ \hline

\hyperlink{TU10}{TU10} & \texttt{Back-end::APIGateway::VocalAPI::-\linebreak queryLambda()}\\ \hline

\hyperlink{TU11}{TU11} & \texttt{Back-end::APIGateway::VocalAPI::-\linebreak queryLambda()}\\ \hline

\hyperlink{TU12}{TU12} & \texttt{Back-end::APIGateway::VocalAPI::-\linebreak queryLambda()}\\ \hline

\hyperlink{TU13}{TU13} & \texttt{Back-end::APIGateway::VocalAPI::-\linebreak queryLambda()}\\ \hline

\hyperlink{TU14}{TU14} & \texttt{Back-end::APIGateway::VocalAPI::-\linebreak queryLambda()}\\ \hline

\hyperlink{TU15}{TU15} & \texttt{Back-end::APIGateway::VocalAPI::-\linebreak queryLambda()}\\ \hline

\hyperlink{TU16}{TU16} & \texttt{Back-end::APIGateway::VocalAPI::-\linebreak queryLambda()}\\ \hline

\hyperlink{TU17}{TU17} & \texttt{Back-end::APIGateway::VocalAPI::-\linebreak queryLambda()}\\ \hline

\hyperlink{TU18}{TU18} & \texttt{Back-end::APIGateway::VocalAPI::-\linebreak queryLambda()}\\ \hline

\hyperlink{TU19}{TU19} & \texttt{Back-end::APIGateway::VocalAPI::-\linebreak queryLambda()}\\ \hline

\hyperlink{TU20}{TU20} & \texttt{Back-end::APIGateway::VocalAPI::-\linebreak queryLambda()}\\ \hline

\hyperlink{TU21}{TU21} & \texttt{Back-end::APIGateway::VocalAPI::-\linebreak queryLambda()}\\ \hline

\hyperlink{TU22}{TU22} & \texttt{Back-end::APIGateway::VocalAPI::-\linebreak queryLambda()}\\ \hline

\hyperlink{TU23}{TU23} & \texttt{Back-end::APIGateway::VocalAPI::-\linebreak queryLambda()}\\ \hline

\hyperlink{TU24}{TU24} & \texttt{Back-end::APIGateway::VocalAPI::-\linebreak queryLambda()}\\ \hline

\hyperlink{TU25}{TU25} & \texttt{Back-end::APIGateway::VocalAPI::-\linebreak queryLambda()}\\ \hline

\hyperlink{TU26}{TU26} & \texttt{Back-end::APIGateway::VocalAPI::-\linebreak queryLambda()}\\ \hline

\hyperlink{TU27}{TU27} & \texttt{Back-end::APIGateway::VocalAPI::-\linebreak queryLambda()}\\ \hline

\hyperlink{TU28}{TU28} & \texttt{Back-end::APIGateway::VocalAPI::-\linebreak queryLambda()}\\ \hline

\hyperlink{TU29}{TU29} & \texttt{Back-end::APIGateway::VocalAPI::-\linebreak queryLambda()}\\ \hline

\hyperlink{TU30}{TU30} & \texttt{Back-end::APIGateway::VocalAPI::-\linebreak queryLambda()}\\ \hline

\hyperlink{TU31}{TU31} & \texttt{Back-end::APIGateway::VocalAPI::-\linebreak queryLambda()}\\ \hline

\hyperlink{TU32}{TU32} & \texttt{Back-end::APIGateway::VocalAPI::-\linebreak queryLambda()}\\ \hline

\hyperlink{TU33}{TU33} & \texttt{Back-end::APIGateway::VocalAPI::-\linebreak addRule()}\\ \hline

\hyperlink{TU34}{TU34} & \texttt{Back-end::APIGateway::VocalAPI::-\linebreak addUser()}\\ \hline

\hyperlink{TU35}{TU35} & \texttt{Back-end::APIGateway::VocalAPI::-\linebreak addUserEnrollment()}\\ \hline

\hyperlink{TU36}{TU36} & \texttt{Back-end::APIGateway::VocalAPI::-\linebreak getRule()}\\ \hline

\hyperlink{TU37}{TU37} & \texttt{Back-end::APIGateway::VocalAPI::-\linebreak getRuleList()}\\ \hline

\hyperlink{TU38}{TU38} & \texttt{Back-end::APIGateway::VocalAPI::-\linebreak getUser()}\\ \hline

\hyperlink{TU39}{TU39} & \texttt{Back-end::APIGateway::VocalAPI::-\linebreak getUserList()}\\ \hline

\hyperlink{TU40}{TU40} & \texttt{Back-end::APIGateway::VocalAPI::-\linebreak loginUser()}\\ \hline

\hyperlink{TU41}{TU41} & \texttt{Back-end::APIGateway::VocalAPI::-\linebreak removeRule()}\\ \hline

\hyperlink{TU42}{TU42} & \texttt{Back-end::APIGateway::VocalAPI::-\linebreak removeUser()}\\ \hline

\hyperlink{TU43}{TU43} & \texttt{Back-end::APIGateway::VocalAPI::-\linebreak resetUserEnrollment()}\\ \hline

\hyperlink{TU44}{TU44} & \texttt{Back-end::APIGateway::VocalAPI::-\linebreak updateRule()}\\ \hline

\hyperlink{TU45}{TU45} & \texttt{Back-end::APIGateway::VocalAPI::-\linebreak updateUser()}\\ \hline

\hyperlink{TU46}{TU46} & \texttt{Back-end::APIGateway::VocalAPI::-\linebreak queryLambda()}\\ \hline

\hyperlink{TU47}{TU47} & \texttt{Back-end::APIGateway::VocalAPI::-\linebreak queryLambda()}\\ \hline

\hyperlink{TU48}{TU48} & \texttt{Back-end::Conversations::ConversationObserver::-\linebreak next()}\\ \hline

\hyperlink{TU49}{TU49} & \texttt{Back-end::Conversations::ConversationObserver::-\linebreak next()}\\ \hline

\hyperlink{TU50}{TU50} & \texttt{Back-end::Conversations::ConversationsDAODynamoDB::-\linebreak addConversation()}\\ \hline

\hyperlink{TU51}{TU51} & \texttt{Back-end::Conversations::ConversationsDAODynamoDB::-\linebreak addConversation()}\\ \hline

\hyperlink{TU52}{TU52} & \texttt{Back-end::Conversations::ConversationsDAODynamoDB::-\linebreak addMessage()}\\ \hline

\hyperlink{TU53}{TU53} & \texttt{Back-end::Conversations::ConversationsDAODynamoDB::-\linebreak addMessage()}\\ \hline

\hyperlink{TU54}{TU54} & \texttt{Back-end::Conversations::ConversationsDAODynamoDB::-\linebreak getConversation()}\\ \hline

\hyperlink{TU55}{TU55} & \texttt{Back-end::Conversations::ConversationsDAODynamoDB::-\linebreak getConversation()}\\ \hline

\hyperlink{TU56}{TU56} & \texttt{Back-end::Conversations::ConversationsDAODynamoDB::-\linebreak getConversationList()}\\ \hline

\hyperlink{TU57}{TU57} & \texttt{Back-end::Conversations::ConversationsDAODynamoDB::-\linebreak getConversationList()}\\ \hline

\hyperlink{TU58}{TU58} & \texttt{Back-end::Conversations::ConversationsDAODynamoDB::-\linebreak removeConversation()}\\ \hline

\hyperlink{TU59}{TU59} & \texttt{Back-end::Conversations::ConversationsDAODynamoDB::-\linebreak removeConversation()}\\ \hline

\hyperlink{TU60}{TU60} & \texttt{Back-end::Guests::GuestObserver::next()}\\ \hline

\hyperlink{TU61}{TU61} & \texttt{Back-end::Guests::GuestObserver::next()}\\ \hline

\hyperlink{TU62}{TU62} & \texttt{Back-end::Guests::GuestsDAODynamoDB::-\linebreak addGuest()}\\ \hline

\hyperlink{TU63}{TU63} & \texttt{Back-end::Guests::GuestsDAODynamoDB::-\linebreak addGuest()}\\ \hline

\hyperlink{TU64}{TU64} & \texttt{Back-end::Guests::GuestsDAODynamoDB::-\linebreak getGuest()}\\ \hline

\hyperlink{TU65}{TU65} & \texttt{Back-end::Guests::GuestsDAODynamoDB::-\linebreak getGuest()}\\ \hline

\hyperlink{TU66}{TU66} & \texttt{Back-end::Guests::GuestsDAODynamoDB::-\linebreak getGuestList()}\\ \hline

\hyperlink{TU67}{TU67} & \texttt{Back-end::Guests::GuestsDAODynamoDB::-\linebreak getGuestList()}\\ \hline

\hyperlink{TU68}{TU68} & \texttt{Back-end::Guests::GuestsDAODynamoDB::-\linebreak removeGuest()}\\ \hline

\hyperlink{TU69}{TU69} & \texttt{Back-end::Guests::GuestsDAODynamoDB::-\linebreak removeGuest()}\\ \hline

\hyperlink{TU70}{TU70} & \texttt{Back-end::Guests::GuestsDAODynamoDB::-\linebreak updateGuest()}\\ \hline

\hyperlink{TU71}{TU71} & \texttt{Back-end::Guests::GuestsDAODynamoDB::-\linebreak updateGuest()}\\ \hline

\hyperlink{TU72}{TU72} & \texttt{Back-end::Members::MemberObserver::-\linebreak next()}\\ \hline

\hyperlink{TU73}{TU73} & \texttt{Back-end::Members::MemberObserver::-\linebreak next()}\\ \hline

\hyperlink{TU74}{TU74} & \texttt{Back-end::Members::MembersDAOSlack::-\linebreak getMember()}\\ \hline

\hyperlink{TU75}{TU75} & \texttt{Back-end::Members::MembersDAOSlack::-\linebreak getMember()}\\ \hline

\hyperlink{TU76}{TU76} & \texttt{Back-end::Members::MembersDAOSlack::-\linebreak getMemberList()}\\ \hline

\hyperlink{TU77}{TU77} & \texttt{Back-end::Members::MembersDAOSlack::-\linebreak getMemberList()}\\ \hline

\hyperlink{TU78}{TU78} & \texttt{Back-end::Members::MembersDAOSlack::-\linebreak addMember()}\\ \hline

\hyperlink{TU79}{TU79} & \texttt{Back-end::Members::MembersDAOSlack::-\linebreak removeMember()}\\ \hline

\hyperlink{TU80}{TU80} & \texttt{Back-end::Notifications::NotificationService::-\linebreak getChannelList()}\\ \hline

\hyperlink{TU81}{TU81} & \texttt{Back-end::Notifications::NotificationService::-\linebreak getChannelList()}\\ \hline

\hyperlink{TU82}{TU82} & \texttt{Back-end::Notifications::NotificationService::-\linebreak sendMsg()}\\ \hline

\hyperlink{TU83}{TU83} & \texttt{Back-end::Notifications::NotificationService::-\linebreak sendMsg()}\\ \hline

\hyperlink{TU84}{TU84} & \texttt{Libs::ObserverAdapter::complete()}\\ \hline

\hyperlink{TU85}{TU85} & \texttt{Libs::ObserverAdapter::error()}\\ \hline

\hyperlink{TU86}{TU86} & \texttt{Libs::ObserverAdapter::next()}\\ \hline

\hyperlink{TU87}{TU87} & \texttt{Back-end::Rules::RuleObserver::next()}\\ \hline

\hyperlink{TU88}{TU88} & \texttt{Back-end::Rules::RuleObserver::next()}\\ \hline

\hyperlink{TU89}{TU89} & \texttt{Back-end::Rules::RulesDAODynamoDB::-\linebreak addRule()}\\ \hline

\hyperlink{TU90}{TU90} & \texttt{Back-end::Rules::RulesDAODynamoDB::-\linebreak addRule()}\\ \hline

\hyperlink{TU91}{TU91} & \texttt{Back-end::Rules::RulesDAODynamoDB::-\linebreak getRule()}\\ \hline

\hyperlink{TU92}{TU92} & \texttt{Back-end::Rules::RulesDAODynamoDB::-\linebreak getRule()}\\ \hline

\hyperlink{TU93}{TU93} & \texttt{Back-end::Rules::RulesDAODynamoDB::-\linebreak getRuleList()}\\ \hline

\hyperlink{TU94}{TU94} & \texttt{Back-end::Rules::RulesDAODynamoDB::-\linebreak getRuleList()}\\ \hline

\hyperlink{TU95}{TU95} & \texttt{Back-end::Rules::RulesDAODynamoDB::-\linebreak removeRule()}\\ \hline

\hyperlink{TU96}{TU96} & \texttt{Back-end::Rules::RulesDAODynamoDB::-\linebreak removeRule()}\\ \hline

\hyperlink{TU97}{TU97} & \texttt{Back-end::Rules::RulesDAODynamoDB::-\linebreak updateRule()}\\ \hline

\hyperlink{TU98}{TU98} & \texttt{Back-end::Rules::RulesDAODynamoDB::-\linebreak updateRule()}\\ \hline

\hyperlink{TU99}{TU99} & \texttt{Back-end::Rules::TasksDAODynamoDB::-\linebreak addTask()}\\ \hline

\hyperlink{TU100}{TU100} & \texttt{Back-end::Rules::TasksDAODynamoDB::-\linebreak addTask()}\\ \hline

\hyperlink{TU101}{TU101} & \texttt{Back-end::Rules::TasksDAODynamoDB::-\linebreak getTask()}\\ \hline

\hyperlink{TU102}{TU102} & \texttt{Back-end::Rules::TasksDAODynamoDB::-\linebreak getTask()}\\ \hline

\hyperlink{TU103}{TU103} & \texttt{Back-end::Rules::TasksDAODynamoDB::-\linebreak getTaskList()}\\ \hline

\hyperlink{TU104}{TU104} & \texttt{Back-end::Rules::TasksDAODynamoDB::-\linebreak getTaskList()}\\ \hline

\hyperlink{TU105}{TU105} & \texttt{Back-end::Rules::TasksDAODynamoDB::-\linebreak removeTask()}\\ \hline

\hyperlink{TU106}{TU106} & \texttt{Back-end::Rules::TasksDAODynamoDB::-\linebreak removeTask()}\\ \hline

\hyperlink{TU107}{TU107} & \texttt{Back-end::Rules::TasksDAODynamoDB::-\linebreak updateTask()}\\ \hline

\hyperlink{TU108}{TU108} & \texttt{Back-end::Rules::TasksDAODynamoDB::-\linebreak updateTask()}\\ \hline

\hyperlink{TU109}{TU109} & \texttt{Back-end::STT::STTWatsonAdapter::-\linebreak speechToText()}\\ \hline

\hyperlink{TU110}{TU110} & \texttt{Back-end::Rules::TaskObserver::next()}\\ \hline

\hyperlink{TU111}{TU111} & \texttt{Back-end::Rules::TaskObserver::next()}\\ \hline

\hyperlink{TU112}{TU112} & \texttt{Back-end::Users::UserObserver::next()}\\ \hline

\hyperlink{TU113}{TU113} & \texttt{Back-end::Users::UserObserver::next()}\\ \hline

\hyperlink{TU114}{TU114} & \texttt{Back-end::Users::UsersDAODynamoDB::-\linebreak addUser()}\\ \hline

\hyperlink{TU115}{TU115} & \texttt{Back-end::Users::UsersDAODynamoDB::-\linebreak addUser()}\\ \hline

\hyperlink{TU116}{TU116} & \texttt{Back-end::Users::UsersDAODynamoDB::-\linebreak getUser()}\\ \hline

\hyperlink{TU117}{TU117} & \texttt{Back-end::Users::UsersDAODynamoDB::-\linebreak getUser()}\\ \hline

\hyperlink{TU118}{TU118} & \texttt{Back-end::Users::UsersDAODynamoDB::-\linebreak getUserList()}\\ \hline

\hyperlink{TU119}{TU119} & \texttt{Back-end::Users::UsersDAODynamoDB::-\linebreak getUserList()}\\ \hline

\hyperlink{TU120}{TU120} & \texttt{Back-end::Users::UsersDAODynamoDB::-\linebreak removeUser()}\\ \hline

\hyperlink{TU121}{TU121} & \texttt{Back-end::Users::UsersDAODynamoDB::-\linebreak removeUser()}\\ \hline

\hyperlink{TU122}{TU122} & \texttt{Back-end::Users::UsersDAODynamoDB::-\linebreak updateUser()}\\ \hline

\hyperlink{TU123}{TU123} & \texttt{Back-end::Users::UsersDAODynamoDB::-\linebreak updateUser()}\\ \hline

\hyperlink{TU124}{TU124} & \texttt{Back-end::Users::VocalLoginMicrosoftModule::-\linebreak addEnrollment()}\\ \hline

\hyperlink{TU125}{TU125} & \texttt{Back-end::Users::VocalLoginMicrosoftModule::-\linebreak createUser()}\\ \hline

\hyperlink{TU126}{TU126} & \texttt{Back-end::Users::VocalLoginMicrosoftModule::-\linebreak deleteUser()}\\ \hline

\hyperlink{TU127}{TU127} & \texttt{Back-end::Users::VocalLoginMicrosoftModule::-\linebreak doLogin()}\\ \hline

\hyperlink{TU128}{TU128} & \texttt{Back-end::Users::VocalLoginMicrosoftModule::-\linebreak getList()}\\ \hline

\hyperlink{TU129}{TU129} & \texttt{Back-end::Users::VocalLoginMicrosoftModule::-\linebreak getUser()}\\ \hline

\hyperlink{TU130}{TU130} & \texttt{Back-end::Users::VocalLoginMicrosoftModule::-\linebreak resetEnrollments()}\\ \hline

\hyperlink{TU131}{TU131} & \texttt{Back-end::VirtualAssistant::AgentsDAODynamoDB::-\linebreak addAgent()}\\ \hline

\hyperlink{TU132}{TU132} & \texttt{Back-end::VirtualAssistant::AgentsDAODynamoDB::-\linebreak addAgent()}\\ \hline

\hyperlink{TU133}{TU133} & \texttt{Back-end::VirtualAssistant::AgentsDAODynamoDB::-\linebreak getAgent()}\\ \hline

\hyperlink{TU134}{TU134} & \texttt{Back-end::VirtualAssistant::AgentsDAODynamoDB::-\linebreak getAgent()}\\ \hline

\hyperlink{TU135}{TU135} & \texttt{Back-end::VirtualAssistant::-\linebreak <<interface>> AgentsDAO::getAgentsList()}\\ \hline

\hyperlink{TU136}{TU136} & \texttt{Back-end::VirtualAssistant::-\linebreak <<interface>> AgentsDAO::getAgentsList()}\\ \hline

\hyperlink{TU137}{TU137} & \texttt{Back-end::VirtualAssistant::-\linebreak <<interface>> AgentsDAO::removeAgent()}\\ \hline

\hyperlink{TU138}{TU138} & \texttt{Back-end::VirtualAssistant::-\linebreak <<interface>> AgentsDAO::removeAgent()}\\ \hline

\hyperlink{TU139}{TU139} & \texttt{Back-end::VirtualAssistant::-\linebreak <<interface>> AgentsDAO::updateAgent()}\\ \hline

\hyperlink{TU140}{TU140} & \texttt{Back-end::VirtualAssistant::-\linebreak <<interface>> AgentsDAO::updateAgent()}\\ \hline

\hyperlink{TU141}{TU141} & \texttt{Back-end::Rules::RulesService::-\linebreak addRule()}\\ \hline

\hyperlink{TU142}{TU142} & \texttt{Back-end::Rules::RulesService::-\linebreak addRule()}\\ \hline

\hyperlink{TU143}{TU143} & \texttt{Back-end::Rules::RulesService::-\linebreak addRule()}\\ \hline

\hyperlink{TU144}{TU144} & \texttt{Back-end::Rules::RulesService::-\linebreak deleteRule()}\\ \hline

\hyperlink{TU145}{TU145} & \texttt{Back-end::Rules::RulesService::-\linebreak deleteRule()}\\ \hline

\hyperlink{TU146}{TU146} & \texttt{Back-end::Rules::RulesService::-\linebreak getRule()}\\ \hline

\hyperlink{TU147}{TU147} & \texttt{Back-end::Rules::RulesService::-\linebreak getRule()}\\ \hline

\hyperlink{TU148}{TU148} & \texttt{Back-end::Rules::RulesService::-\linebreak getRuleList()}\\ \hline

\hyperlink{TU149}{TU149} & \texttt{Back-end::Rules::RulesService::-\linebreak getRuleList()}\\ \hline

\hyperlink{TU152}{TU152} & \texttt{Back-end::Rules::RulesService::-\linebreak getTaskList()}\\ \hline

\hyperlink{TU153}{TU153} & \texttt{Back-end::Rules::RulesService::-\linebreak updateRule()}\\ \hline

\hyperlink{TU154}{TU154} & \texttt{Back-end::Rules::RulesService::-\linebreak updateRule()}\\ \hline

\hyperlink{TU155}{TU155} & \texttt{Back-end::Rules::RulesService::-\linebreak updateRule()}\\ \hline

\hyperlink{TU156}{TU156} & \texttt{Back-end::Users::UsersService::-\linebreak addUser()}\\ \hline

\hyperlink{TU157}{TU157} & \texttt{Back-end::Users::UsersService::-\linebreak addUser()}\\ \hline

\hyperlink{TU158}{TU158} & \texttt{Back-end::Users::UsersService::-\linebreak addUser()}\\ \hline

\hyperlink{TU159}{TU159} & \texttt{Back-end::Users::UsersService::-\linebreak getUser()}\\ \hline

\hyperlink{TU160}{TU160} & \texttt{Back-end::Users::UsersService::-\linebreak getUser()}\\ \hline

\hyperlink{TU161}{TU161} & \texttt{Back-end::Users::UsersService::-\linebreak getUserList()}\\ \hline

\hyperlink{TU162}{TU162} & \texttt{Back-end::Users::UsersService::-\linebreak getUserList()}\\ \hline

\hyperlink{TU163}{TU163} & \texttt{Back-end::Users::UsersService::-\linebreak deleteUser()}\\ \hline

\hyperlink{TU164}{TU164} & \texttt{Back-end::Users::UsersService::-\linebreak deleteUser()}\\ \hline

\hyperlink{TU165}{TU165} & \texttt{Back-end::Users::UsersService::-\linebreak updateUser()}\\ \hline

\hyperlink{TU166}{TU166} & \texttt{Back-end::Users::UsersService::-\linebreak updateUser()}\\ \hline

\hyperlink{TU167}{TU167} & \texttt{Back-end::Users::UsersService::-\linebreak updateUser()}\\ \hline

\hyperlink{TU168}{TU168} & \texttt{Back-end::Events::VAMessageListener::-\linebreak onMessage()}\\ \hline

\hyperlink{TU169}{TU169} & \texttt{Back-end::Events::VAMessageListener::-\linebreak onMessage()}\\ \hline

\hyperlink{TU170}{TU170} & \texttt{Back-end::Events::VAMessageListener::-\linebreak onMessage()}\\ \hline

\hyperlink{TU171}{TU171} & \texttt{Back-end::Events::VAMessageListener::-\linebreak onMessage()}\\ \hline

\hyperlink{TU172}{TU172} & \texttt{Back-end::Events::VAMessageListener::-\linebreak onMessage()}\\ \hline

\hyperlink{TU173}{TU173} & \texttt{Back-end::VirtualAssistant::ApiAiVAAdapter::-\linebreak query()}\\ \hline

\hyperlink{TU174}{TU174} & \texttt{Back-end::VirtualAssistant::ApiAiVAAdapter::-\linebreak query()}\\ \hline

\hyperlink{TU175}{TU175} & \texttt{Back-end::VirtualAssistant::VAService::-\linebreak query()}\\ \hline

\hyperlink{TU176}{TU176} & \texttt{Back-end::VirtualAssistant::VAService::-\linebreak query()}\\ \hline

\hyperlink{TU177}{TU177} & \texttt{Client::ApplicationManager::-\linebreak ApplicationLocalRegistry::query()}\\ & \texttt{Client::ApplicationManager::-\linebreak ApplicationLocalRegistry::register()}\\ \hline

\hyperlink{TU178}{TU178} & \texttt{Client::ApplicationManager::-\linebreak ApplicationLocalRegistry::register()}\\ & \texttt{Client::ApplicationManager::-\linebreak ApplicationLocalRegistry::remove()}\\ \hline

\hyperlink{TU179}{TU179} & \texttt{Client::ApplicationManager::-\linebreak ApplicationRegistryLocalClient::query()}\\ \hline

\hyperlink{TU180}{TU180} & \texttt{Client::ApplicationManager::-\linebreak ApplicationRegistryLocalClient::register()}\\ \hline

\hyperlink{TU181}{TU181} & \texttt{Client::ConversationApp::ConversationDispatcher::-\linebreak dispatch()}\\ \hline

\hyperlink{TU182}{TU182} & \texttt{Client::ConversationApp::ConversationView::-\linebreak render()}\\ \hline

\hyperlink{TU183}{TU183} & \texttt{Client::ApplicationManager::Manager::-\linebreak runApplication()}\\ \hline

\hyperlink{TU184}{TU184} & \texttt{Client::ApplicationManager::Manager::-\linebreak runApplication()}\\ \hline

\hyperlink{TU185}{TU185} & \texttt{Client::ApplicationManager::Manager::-\linebreak setFrame()}\\ \hline

\hyperlink{TU186}{TU186} & \texttt{Client::ConversationApp::MessageStore::-\linebreak onCmd()}\\ \hline

\hyperlink{TU187}{TU187} & \texttt{Client::ConversationApp::MessageStore::-\linebreak onCmd()}\\ \hline

\hyperlink{TU188}{TU188} & \texttt{Client::ConversationApp::MessageStore::-\linebreak onCmd()}\\ \hline

\hyperlink{TU189}{TU189} & \texttt{Client::ConversationApp::MessageStore::-\linebreak onCmd()}\\ \hline

\hyperlink{TU190}{TU190} & \texttt{Client::ConversationApp::MessageStore::-\linebreak onCmd()}\\ \hline

\hyperlink{TU191}{TU191} & \texttt{Client::ApplicationManager::State::-\linebreak addApp()}\\ & \texttt{Client::ApplicationManager::State::-\linebreak getApp()}\\ \hline

\hyperlink{TU192}{TU192} & \texttt{Client::ConversationApp::ConversationApp::-\linebreak runCmd()}\\ \hline

\hyperlink{TU194}{TU194} & \texttt{Client::Logic::HttpPromise::then()}\\ \hline

\hyperlink{TU195}{TU195} & \texttt{Client::Logic::HttpPromise::then()}\\ \hline

\hyperlink{TU196}{TU196} & \texttt{Client::Logic::Logic::sendData()}\\ \hline

\hyperlink{TU197}{TU197} & \texttt{Client::Logic::Logic::sendData()}\\ \hline

\hyperlink{TU198}{TU198} & \texttt{Client::Recorder::Recorder::start()}\\ & \texttt{Client::Recorder::Recorder::stop()}\\ \hline

\hyperlink{TU199}{TU199} & \texttt{Back-end::Rules::RulesService::-\linebreak deleteRule()}\\ \hline

\hyperlink{TU200}{TU200} & \texttt{Back-end::Rules::RulesService::-\linebreak getRule()}\\ \hline

\hyperlink{TU201}{TU201} & \texttt{Back-end::Users::UsersService::-\linebreak getUser()}\\ \hline

\hyperlink{TU202}{TU202} & \texttt{Back-end::Users::UsersService::-\linebreak deleteUser()}\\ \hline

\hyperlink{TU203}{TU203} & \texttt{Back-end::ConversationWebhookService::-\linebreak webhook()}\\ \hline

\hyperlink{TU204}{TU204} & \texttt{Back-end::ConversationWebhookService::-\linebreak webhook()}\\ \hline

\hyperlink{TU207}{TU207} & \texttt{Back-end::Notifications::NotificationService::-\linebreak getChannelList()}\\ \hline

\hyperlink{TU208}{TU208} & \texttt{Back-end::Notifications::NotificationService::-\linebreak getChannelList()}\\ \hline

\hyperlink{TU209}{TU209} & \texttt{Libs::ObserverAdapter::complete()}\\ \hline

\hyperlink{TU210}{TU210} & \texttt{Libs::ObserverAdapter::complete()}\\ \hline

\hyperlink{TU211}{TU211} & \texttt{Libs::ObserverAdapter::error()}\\ \hline

\hyperlink{TU212}{TU212} & \texttt{Libs::ObserverAdapter::error()}\\ \hline

\hyperlink{TU213}{TU213} & \texttt{Libs::ObserverAdapter::next()}\\ \hline

\hyperlink{TU214}{TU214} & \texttt{Libs::ObserverAdapter::next()}\\ \hline

\hyperlink{TU215}{TU215} & \texttt{Back-end::Users::UsersService::-\linebreak addUser()}\\ \hline

\hyperlink{TU216}{TU216} & \texttt{Back-end::Rules::RulesService::-\linebreak addRule()}\\ \hline

\hyperlink{TU217}{TU217} & \texttt{Back-end::Rules::RulesService::-\linebreak getTaskList()}\\ \hline

\hyperlink{TU218}{TU218} & \texttt{Client::ApplicationManager::-\linebreak ApplicationRegistryLocalClient::register()}\\ \hline

\hyperlink{TU219}{TU219} & \texttt{Back-end::Members::MembersDAOSlack::-\linebreak updateMember()}\\ \hline

\hyperlink{TU220}{TU220} & \texttt{Back-end::Notifications::NotificationService::-\linebreak sendMsg()}\\ \hline

\hyperlink{TU221}{TU221} & \texttt{Back-end::Rules::RulesDAODynamoDB::-\linebreak getRuleList()}\\ \hline

\hyperlink{TU222}{TU222} & \texttt{Back-end::Users::UsersDAODynamoDB::-\linebreak getUserList()}\\ \hline

\hyperlink{TU223}{TU223} & \texttt{Back-end::Users::UsersDAODynamoDB::-\linebreak getUserList()}\\ \hline

\caption[Tracciamento Test di Unità-Metodi]{Tracciamento Test di Unità-Metodi}
\label{tabella:tu-met}
\end{longtable}
\clearpage

	\subsection{Tracciamento Test di Validazione-Requisiti}
\normalsize
\begin{longtable}{|>{\centering}m{5cm}|m{5cm}<{\centering}|}
\hline 
\textbf{Test} & \textbf{Requisito}\\
\hline
\endhead
\hyperlink{TVFO1}{TVFO1} & RFO1\\ \hline
\hyperlink{TVFO1.1.2}{TVFO1.1.2} & RFO1.1.2\\ \hline
\hyperlink{TVFO2.1}{TVFO2.1} & RFO2.1\\ \hline
\hyperlink{TVFO2.1.1.6}{TVFO2.1.1.6} & RFO2.1.1.6\\ \hline
\hyperlink{TVFO2.1.2}{TVFO2.1.2} & RFO2.1.2\\ \hline
\hyperlink{TVFD2.1.3}{TVFD2.1.3} & RFD2.1.3\\ \hline
\hyperlink{TVFD2.1.3.7}{TVFD2.1.3.7} & RFD2.1.3.7\\ \hline
\hyperlink{TVFO2.1.4}{TVFO2.1.4} & RFO2.1.4\\ \hline
\hyperlink{TVFO2.2}{TVFO2.2} & RFO2.2\\ \hline
\hyperlink{TVFO3.1}{TVFO3.1} & RFO3.1\\ \hline
\hyperlink{TVFD3.2}{TVFD3.2} & RFD3.2\\ \hline
\hyperlink{TVFD3.3}{TVFD3.3} & RFD3.3\\ \hline
\hyperlink{TVFO5}{TVFO5} & RFO5\\ \hline
\hyperlink{TVFO7}{TVFO7} & RFO7\\ \hline
\hyperlink{TVFO8}{TVFO8} & RFO8\\ \hline
\hyperlink{TVFD9.1.1}{TVFD9.1.1} & RFD9.1.1\\ \hline
\hyperlink{TVFD9.1.1.4}{TVFD9.1.1.4} & RFD9.1.1.4\\ \hline
\hyperlink{TVFD9.1.3}{TVFD9.1.3} & RFD9.1.3\\ \hline
\hyperlink{TVFD9.1.3.2}{TVFD9.1.3.2} & RFD9.1.3.2\\ \hline
\caption[Tracciamento Test di Validazione-Requisiti]{Tracciamento Test di Validazione-Requisiti}
\label{tabella:tv-requi}
\end{longtable}
\clearpage

\end{document}

