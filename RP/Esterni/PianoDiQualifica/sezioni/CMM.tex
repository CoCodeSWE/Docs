\documentclass[PdQ.tex]{subfiles}

\begin{document}

\hypertarget{CMM}{\section{Capability Maturity Model}}
	Il Capability Maturity Model (CMM) è stato ideato e introdotto inizialmente dal Dipartimento della Difesa statunitense, per poi essere acquisito, sviluppato e sponsorizzato dalla SEI (\gl{Software} Engineering Institute). Tale modello
assume che la qualità del software dipende decisamente dal processo utilizzato per il suo sviluppo e per la successiva manutenzione, e consiste nell'applicare le migliori tecniche di gestione dei processi e del miglioramento della qualità. Si basa su:
	\begin{itemize}
		\item linee guida comuni per lo sviluppo e la manutenzione del software;
		\item struttura per la valutazione consistente dei livelli raggiunti.
	\end{itemize}
	
	\subsection{Scopo}
	Lo scopo principale dell'adozione del modello in esame è quello di migliorare i processi di sviluppo del software in ottica di:
	\begin{itemize}
		\item miglioramento della qualità del software prodotto;
		\item aumento della produttività dell'organizzazione di sviluppo;
		\item riduzione dei tempi di sviluppo.
	\end{itemize}
	\subsection{Struttura}
	Il CMM è costituito dalla seguente struttura:
	\begin{itemize}
		\item \textbf{Livelli di maturità}: Il modello definisce cinque livelli di maturità crescente del processo di sviluppo del software. Il più alto (il quinto) è uno stato ideale in cui i processi vengono sistematicamente
		gestiti da una combinazione di processi di ottimizzazione e di miglioramento continuo. 
		\item \textbf{Aree chiave del processo}: identifica una serie di attività correlate che, se svolte collettivamente,
		realizzano un insieme di obiettivi considerati importanti;
		\item \textbf{Obiettivi}: indicano lo scopo, i confini e l'intento di ogni area chiave del processo;
		\item \textbf{Caratteristiche comuni}: includono le pratiche che implementano e regolamentano un'area chiave del processo. Ci sono cinque
		tipologie di caratteristiche comuni:
		\begin{itemize}
			\item impegno nell'operare;
			\item abilità nell'operare;
			\item attività eseguite;
			\item misurazioni ed analisi;
			\item veriche dell'implementazione.
		\end{itemize}
		\item \textbf{pratiche chiave}: descrivono gli elementi dell'infrastruttura e delle pratiche che contribuiscono maggiormente all'implementazione
		e la regolamentazione di un'area.
	\end{itemize}
	
	\subsection{Livelli}
	I livelli di maturità che costituiscono il CMM sono:
	\begin{itemize}
		\item \textbf{Primo livello - iniziale (caotico)}: i processi che rientrano in questo livello sono disorganizzati. Il non essere sufficientemente definiti e documentati non permette loro di essere riutilizzati;
		\item \textbf{Secondo livello - ripetibile}:  
		sono stabiliti processi base di gestione per tracciare i costi, la schedulazione delle attività e le funzionalità sviluppate. Il processo è stabilito per essere ripetibile.
		\item \textbf{Terzo livello - definito}: Il processo di sviluppo software, sia per la parte di gestione che per quella di sviluppo tecnico, è definito, documentato e standardizzato per il riutilizzo. 
		\item \textbf{Quarto livello - gestito}: un organizzazione monitora e controlla i propri processi attraverso analisi e Data collection.
		\item \textbf{Quinto livello - ottimizzante}: i processi che rientrano in questo livello sono soggetti ad un continuo miglioramento delle proprie
		performance attraverso cambiamenti incrementali e miglioramenti tecnologici.
	\end{itemize}

\end{document}