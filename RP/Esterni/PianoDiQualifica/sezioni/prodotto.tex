\documentclass[PdQ.tex]{subfiles}

\begin{document}

\section{Qualità di prodotto}
	È prevista la realizzazione di due tipologie di prodotto: documenti e \gl{software}. Per ciascuno di essi sono stati individuati
	degli obiettivi di qualità e i rispettivi intervalli di accettabilità e ottimalità. \\
	Per quantificare gli obiettivi di qualità vengono utilizzate delle metriche, descritte nel documento \NPdocRP{}. \\
	Viene assegnato un codice identificativo ad ogni obiettivo al fine di semplificarne il tracciamento. Il metodo di denominazione
	è descritto nel documento \NPdocRP{}.
	
	\subsection{Documenti}
		I documenti prodotti dal gruppo \GRUPPO{} si dividono in interni ed esterni. I primi definiscono le strategie, gli strumenti e il metodo di lavoro (ways of working)
		del gruppo affinchè i membri realizzino prodotti simili tra loro secondo delle regole definite. I secondi definiscono tutto ciò che riguarda il software prodotto,
		partendo dalla progettazione fino a giungere all'analisi dei requisiti e la definizione di prodotto. \\
		Poichè i documenti interni devono essere letti e compresi da tutti i membri del gruppo e quelli esterni devono essere comprensibili affinchè \gl{proponente} e committente
		siano informati correttamente, il gruppo ha deciso di perseguire le strategie e gli obiettivi di qualità definiti di seguito.
	
		\subsubsection{Strategie}
		Tutti i documenti prodotti devono avere le seguenti caratteristiche:
		\begin{itemize}
			\item devono essere comprensibili da utenti con almeno la licenza superiore;
			\item i termini con significato ambiguo o poco chiaro dovranno essere inseriti nel \Gldoc{};
			\item saranno sempre aggiornati e allineati allo stato attuale del processo di sviluppo;
			\item dovranno essere dotati di numero di versione e diario delle modifiche.
		\end{itemize}
		
		\subsubsection{Obiettivi di qualità - OPDD1}
			\paragraph{Leggibilità e comprensibilità}
			Indica il livello di leggibilità e comprensibilità del documento. Maggiore è il valore dell'indice maggiore sarà la leggibilità del documento.
			\begin{itemize}
				\item \textbf{Metrica utilizzata}: indice Gulpease (MPDD1);
				\item \textbf{Soglia di accettabilità}: 40 - 100;
				\item \textbf{Soglia di ottimalità}: 60 - 100.
			\end{itemize}
					
	\subsection{Software}
		Sono state individuate dallo standard ISO/IEC 9126:2001 le principali caratteristiche che il software deve soddisfare.
		Sulla base di esse sono state definiti gli obiettivi di qualità e i relativi intervalli.
		
		\subsubsection{Functionality}
		È la capacità del prodotto software di fornire le funzionalità definite nei requisiti individuati nel documento \ARdocRP{}.
		
			\paragraph{Strategie}
			Il software deve soddisfare le seguenti caratteristiche:
			\begin{itemize}
				\item \textbf{Suitability}: fornire un appropriato insieme di funzionalità in base alle richieste dell'utente;
				\item \textbf{Accuracy}: fornire i corretti risultati con un adeguato grado di precisione;
				\item \textbf{Security}: proteggere le informazioni e i dati affinchè solo gli utenti autorizzati possano modificarli e/o leggerli.
			\end{itemize}
			
			\paragraph{Obiettivi di qualità}
				\subparagraph{Implementazione funzionale - OPDS1}
				Indica quanti requisiti funzionali sono stati implementati.
				\begin{itemize}
					\item \textbf{Metrica utilizzata}: completezza dell'implementazione funzionale (MPDS1);
					\item \textbf{Soglia di accettabilità}: 100\%;
					\item \textbf{Soglia di ottimalità}: 100\%.
				\end{itemize}
				
				\subparagraph{Accuratezza rispetto alle attese - OPDS2}
				Indica quanti risultati sono concordi alle attese.
				\begin{itemize}
					\item \textbf{Metrica utilizzata}: percentuale risultati concordi alle attese (MPDS2);
					\item \textbf{Soglia di accettabilità}: 90\% - 100\%;
					\item \textbf{Soglia di ottimalità}: 100\%.
				\end{itemize}
				
				\subparagraph{Controllo degli accessi - OPDS3}
				Indica quante operazioni illegali non sono state bloccate. Valori grandi indicano un \gl{sistema} poco sicuro e facilmenete violabile.
				Valori bassi sono d'obbligo per poter garantire la sicurezza dei dati.
				\begin{itemize}
					\item \textbf{Metrica utilizzata}: percentuale operazioni illegali non bloccate (MPDS3);
					\item \textbf{Soglia di accettabilità}: 0\% - 10\%;
					\item \textbf{Soglia di ottimalità}: 0\%.
				\end{itemize}
		
		\subsubsection{Reliability}
		È la capacità del prodotto software di svolgere correttamente le sue funzioni in qualunque situazione, anche in caso di situazioni anomale.
		
			\paragraph{Strategie}
			Il software deve soddisfare le seguenti caratteristiche:
			\begin{itemize}
				\item \textbf{Maturity}: evitare che si verifichino malfunzionamenti, operazioni illegali e risultati errati in seguito ad errori;
				\item \textbf{Fault tolerance}: mantenere un certo livello di performance nonostante siano presenti errori e guasti o come conseguenza di un uso scorretto dell'applicativo.
			\end{itemize}
			
			\paragraph{Obiettivi di qualità}
				\subparagraph{Densità di failure - OPDS4}
				Indica quante operazioni di testing sono concluse in failure.
				\begin{itemize}
					\item \textbf{Metrica utilizzata}: percentuale failure su test (MPDS4);
					\item \textbf{Soglia di accettabilità}: 0\% - 10\%;
					\item \textbf{Soglia di ottimalità}: 0\%.
				\end{itemize}
				
				\subparagraph{Blocco di operazioni non corrette - OPDS5}
				Indica quante funzionalità sono in grado di gestire correttamente gli errori che potrebbero verificarsi. Un valore alto è sinonimo di robustezza.
				\begin{itemize}
					\item \textbf{Metrica utilizzata}: numero di failure evitati (MPDS5);
					\item \textbf{Soglia di accettabilità}: 90\% - 100\%;
					\item \textbf{Soglia di ottimalità}: 100\%;
				\end{itemize}
		
		\subsubsection{Usability}
		È la capacità del prodotto software di essere capito, compreso e usato dall'utente.
		
			\paragraph{Strategie}
			Il software deve soddisfare le seguenti caratteristiche:
			\begin{itemize}
				\item \textbf{Understandability}: permettere all'utente di capire il grado di conformità del software e il suo dominio applicativo;
				\item \textbf{Learnability}: permettere all'utente di capire come utilizzarlo;
				\item \textbf{Operability}: permettere all'utente di utilizzarlo e controllarlo;
				\item \textbf{Attractiveness}: essere piacevole all'utente che lo utilizza.
			\end{itemize}
			
			\paragraph{Obiettivi di qualità}
			
				\subparagraph{Comprensibilità delle funzioni offerte - OPDS6}
				Indica quante funzionalità sono state comprese immediatamente dall'utente senza la consultazione del manuale. L'assistente virtuale dovrebbe essere
				il più user friendly possibile. Infatti, l'interazione dell'utente dovrebbe essere il più fluida possibile senza dover consultare il manuale
				ad ogni operazione da eseguire.
				\begin{itemize}
					\item \textbf{Metrica utilizzata}: percentuale delle funzionalità comprese (MPDS6);
					\item \textbf{Soglia di accettabilità}: 85\% - 100\%;
					\item \textbf{Soglia di ottimalità}: 95\% - 100\%.
				\end{itemize}
			
				\subparagraph{Consistenza operazionale in uso - OPDS7}
				Indica quante funzionalità rispettano le aspettative dell'utente.
				\begin{itemize}
					\item \textbf{Metrica utilizzata}: percentuale di funzionalità conformi alle aspettative (MPDS7);
					\item \textbf{Soglia di accettabilità}: 80\% - 100\%;
					\item \textbf{Soglia di ottimalità}: 90\% - 100\%.
				\end{itemize}
		
		\subsubsection{Efficiency}
		È la capacità del prodotto software di fornire le proprie funzioni in modo appropriato, in relazione alla quantità di risorse utilizzate.
		
			\paragraph{Strategie}
			Il software deve soddisfare le seguenti caratteristiche:
			\begin{itemize}
				\item \textbf{Time behaviour}: svolgere le proprie funzioni in tempi adeguati;
				\item \textbf{Resource utilisation}: eseguire le proprie funzioni utilizzando un'appropriata quantità di risorse.			
			\end{itemize}
			
			\paragraph{Obiettivi di qualità}
			
				\subparagraph{Tempo di risposta - OPDS8}
				Indica il periodo temporale medio trascorso tra la richiesta al software di una determinata funzionalità e la risposta all'utente.
				\begin{itemize}
					\item \textbf{Metrica utilizzata}: tempo medio di risposta (MPDS8);
					\item \textbf{Soglia di accettabilità}: 0 - 20 secondi;
					\item \textbf{Soglia di ottimalità}: 0 - 7 secondi;
				\end{itemize}
		
		\subsubsection{Maintainability}
		È la capacità del prodotto software di essere modificato, ovvero corretto, migliorato o adattato in base a cambiamenti negli ambienti, nei requisiti o
		nelle specifiche funzionali.
		
			\paragraph{Strategie}
			Il software deve soddisfare le seguenti caratteristiche:
			\begin{itemize}
				\item \textbf{Analysability}: poter essere analizzato per poter individuare cause di errori e/o parti da modificare;
				\item \textbf{Changeability}: permettere cambiamenti in alcune sue parti;
				\item \textbf{Stability}: evitare comportamenti indesiderati in seguito a modifiche;
				\item \textbf{Testability}: permettere l'esecuzione di test per validare le modifiche effettuate.
			\end{itemize}
			
			\paragraph{Obiettivi di qualità}
			
				\subparagraph{Capacità analisi di failure - OPDS9}
				Indica il numero di failure di cui sono state individuate le cause.  Maggiore è il valore più facile sarà individuare gli errori nel software 
				e determinare delle soluzioni per correggerli.
				\begin{itemize}
					\item \textbf{Metrica utilizzata}: percentuale di failure con cause individuate (MPDS9)
					\item \textbf{Soglia di accettabilità}: 65\% - 100\%;
					\item \textbf{Soglia di ottimalità}: 85\% - 100\%;
				\end{itemize}
				
				\subparagraph{Impatto delle modifiche - OPDS10}
				Indica il numero di modifiche effettuate in risposta alle failure che ne hanno introdotte di nuove. Valori alti indicano l'individuazione di
				soluzioni scandenti per la risoluzione delle failure.
				\begin{itemize}
					\item \textbf{Metrica utilizzata}: percentuale di failure introdotte con modifiche (MPDS10);
					\item \textbf{Soglia di accettabilità}: 0\% - 20\%;
					\item \textbf{Soglia di ottimalità}: 0\% - 10\%.
				\end{itemize}
				
\end{document}