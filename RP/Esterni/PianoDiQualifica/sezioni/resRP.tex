\documentclass[PdQ.tex]{subfiles}

\begin{document}

\section{Resoconto delle attività di verifica - RP}
All'interno di questa sezione sono riportati gli esiti di tutte le attività di verifica effettuate sui documenti da consegnare per la \RP{}. Ove necessario sono state tratte conclusioni sui risultati e su come essi possano essere migliorati.

\subsection{Qualità di processo}
		\subsubsection{Miglioramento continuo tramite CMM}
	
		All'inizio del periodo i processi si trovavano al livello 2 della scala CMM. In seguito, grazie alla riorganizzazione del documento \NPdocRP{} e alla maggiore esperienza dei membri del gruppo i processi e la loro organizzazione sono migliorati. Questo ci ha permesso di raggiungere il livello 3 della scala CMM. 

		\paragraph{Soddisfacimento obiettivi di qualità}
			Di seguito sono riportati i valori ottenuti utilizzando le metriche definite sui seguenti obiettivi di qualità:
			\begin{table}[h]
				\centering
				\begin{tabular}{l c c}
					\hline
					\rule[-0.3cm]{0cm}{0.8cm}
					\textbf{Obiettivo} & \textbf{Valore} & \textbf{Esito} \\
					\hline
					\rule[0cm]{0cm}{0.4cm}
					Disponibilità PragmaDB - OPC1 & 99\% & ottimale \\
					\rule[0cm]{0cm}{0.4cm}
					Rispetto dei tempi - OPC2 & 2\% & ottimale \\
					\rule[0cm]{0cm}{0.4cm}
					Rispetto dei costi - OPC3 & 13\% & ottimale\\ 
					\rule[0cm]{0cm}{0.4cm}
					Rischi non preventivati - OPC4 & 0 & ottimale\\ 
					\rule[0cm]{0cm}{0.4cm}
					Structural Fan-In - OPC8 & 26.51 & ottimale\\ 
					\rule[0cm]{0cm}{0.4cm}
					Structural Fan-Out - OPC9 & 0.75 & ottimale\\ 
					\rule[0cm]{0cm}{0.4cm}
					Numero di metodi per classe - OPC10 & 2.13 & ottimale\\ 
					\rule[0cm]{0cm}{0.4cm}
					Numero di parametri per metodo - OPC11 & 1.03 & ottimale\\ 
					\hline
				\end{tabular}
				\caption{Esiti del calcolo delle metriche sui processi}
			\end{table}
		

\newpage		
\subsection{Qualità di prodotto}
	\subsubsection{Documenti}
		\paragraph{Leggibilità e comprensibilità - OPDD1}
				Di seguito sono riportati i valori ottenuti calcolando l'indice Gulpease sui documenti:
				\begin{table}[h]
				\centering
				\begin{tabular}{l c c}
					\hline
					\rule[-0.3cm]{0cm}{0.8cm}
					\textbf{Documento} & \textbf{Gulpease} & \textbf{Esito} \\
					\hline
					\rule[0cm]{0cm}{0.4cm}
					\PPdocRP & 50 & accettabile \\
					\rule[0cm]{0cm}{0.4cm}
					\NPdocRP &  60 & ottimale \\ 
					\rule[0cm]{0cm}{0.4cm}
					\ARdocRP & 65 & ottimale \\ 
					\rule[0cm]{0cm}{0.4cm}
					\PQdocRP & 63 & ottimale \\ 
					\rule[0cm]{0cm}{0.4cm}
					\Gldoc & 48 & accettabile\\ 
					\rule[0cm]{0cm}{0.4cm}
					\DPdoc & 61 & ottimale\\ 
					\rule[0cm]{0cm}{0.4cm}
					\textit{Verbale esterno 2017-02-07} & 56 & accettabile\\ 
					\rule[0cm]{0cm}{0.4cm}
					\textit{Verbale esterno 2017-02-28} & 60 & ottimale\\ 
					\rule[0cm]{0cm}{0.4cm}
					\textit{Verbale interno 2017-01-24} & 58 & accettabile\\ 
					\rule[0cm]{0cm}{0.4cm}
					\textit{Verbale interno 2017-03-05} & 54& accettabile\\ 
					
					\hline
				\end{tabular}
				\caption{Esiti del calcolo dell'indice Gulpease sui documenti}
			\end{table}		

\end{document}