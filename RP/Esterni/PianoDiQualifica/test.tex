\documentclass[PianoDiQualifica.tex]{subfiles}

\begin{document}

\section{Specifica dei test}
Al fine di produrre software di qualità, il gruppo ha strutturato dei test atti a verificare che le funzionalità del software prodotto corrispondano alle attese.
Tali test sono ottenuti dall'applicazione delle tecniche di analisi dinamica descritte nel documento \NPdocRP{}. Inoltre, devono possedere le seguenti caratteristiche:
\begin{itemize}
	\item devono essere ripetibili al fine di fornire informazioni utili per poter eseguire operazioni di correzione, ove sia necessario;
	\item devono essere tracciabili al fine di classificare le informazioni ottenute per garantire una più facile consultazione;
\end{itemize}

	\subsection{Tipi di test}
		Sono stati individuati quattro livelli di testing, ovvero:
		\begin{itemize}
			\item \textbf{Test di unità [TU]}: consiste nell'attività di testing delle unità software, ovvero le più piccole parti dotate di funzionamento proprio prodotte dai \PRP{}.
			Questo si traduce solitamente nel verificare i metodi e le classi scritte;
			\item \textbf{Test di integrazione [TI]}: consiste nell'attività di testing delle componenti del sistema. Nel dettaglio, l'obiettivo è quello di testare i vari package
			prodotti dall'unione delle unità;
			\item \textbf{Test di sistema [TS]}: consiste nell'attività di testing del sistema. In particolare, l'obiettivo è quello di verificare che il comportamento del sistema,
			nella sua interezza, sia corretto;
			\item \textbf{Test di validazione [TV]}: consiste nel verificare che il prodotto software soddisfi le richieste del proponente.
		\end{itemize}
	
		\subsubsection{Test di validazione}
		I test di validazione hanno lo scopo di verificare che tutte le funzionalità richieste dal proponente siano soddisfatte. A questo scopo, attraverso una serie di
		azioni, si andrà a simulare il comportamento generale del software e dell'utente che interagisce con esso. \\
		I test di validazione saranno organizzati nel modo seguente:
		\begin{center} TV\{TipoRequisito\}\{Importanza\}\{Codice\} \end{center}
		dove:
		\begin{itemize}
			\item TipoRequisito può assumere uno dei seguenti valori:
			\begin{itemize}
				\item F: per i requisiti funzionali;
				\item Q: per i requisiti di qualità;
				\item V: per i requisiti di vincolo;
			\end{itemize}
			\item Importanza può assumere uno dei seguenti valori:
			\begin{itemize}
				\item O: per i requisiti obbligatori;
				\item D: per i requisiti desiderabili;
				\item F: per i requisiti facoltativi;
			\end{itemize}
			\item Codice assume un valore numerico univoco che identifica il singolo requisito.
		\end{itemize}
		
		\subsubsection{Test di unità}
		
		\subsubsection{Test di integrazione}
		
		\subsubsection{Test di sistema}

\end{document}
