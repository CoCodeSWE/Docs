\documentclass[PianoDiQualifica.tex]{subfiles}

\begin{document}

\section{Test}
Al fine di produrre software di qualità, il gruppo ha strutturato dei test atti a verificare che le funzionalità del software prodotto corrispondano alle attese.
Tali test sono ottenuti dall'applicazione delle tecniche di analisi dinamica descritte nel documento \NPdocRP{}. Inoltre, devono possedere le seguenti caratteristiche:
\begin{itemize}
	\item devono essere ripetibili al fine di fornire informazioni utili per poter eseguire operazioni di correzione, ove sia necessario;
	\item devono essere tracciabili al fine di classificare le informazioni ottenute per garantire una più facile consultazione;
\end{itemize}

	\subsection{Test di validazione}
		I test di validazione hanno lo scopo di verificare che tutte le funzionalità richieste dal proponente siano soddisfatte. A questo scopo, attraverso una serie di
		azioni, si andrà a simulare il comportamento generale del software e dell'utente che interagisce con esso. \\
		I test di validazione saranno identificati secondo quanto riportato nelle \NPdocRP{}.
	
	\subsection{Test di unità}
		I test di unità hanno lo scopo di verificare il corretto funzionamento delle unità. Le unità, individuate duarante la fase di progettazione, sono le
		più piccole parti del sistema dotate di funzionamento proprio. Questo si traduce nel verificare metodi e classi scritte dai \PRP{}. \\
		I test di unità saranno identificati secondo quanto riportato nelle \NPdocRP{}.
	
	\subsection{Test di integrazione}
		I test di integrazione hanno lo scopo di verificare il corretto funzionamento delle varie componenti. In particolare, l'obiettivo è quello di testare i vari
		package prodotti dall'unione delle unità.
		I test di integrazione saranno identificati secondo quanto riportato nelle \NPdocRP{}.
	
	\subsection{Test di sistema}
		I test di sistema hanno lo scopo di verificare il corretto funzionamento del prodotto software. Inoltre verranno verificate la sua robustezza in presenza di
		possibili malfunzionamenti e il suo comportamento di fronte a possibili violazioni (sicurezza). \\
		I test di sistema saranno identificati secondo quanto riportato nelle \NPdocRP{}.

\end{document}


