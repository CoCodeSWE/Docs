\documentclass[PianoDiQualifica.tex]{subfiles}

\begin{document}

\section{Introduzione}
	\subsection{Scopo del documento}
		Il documento ha lo scopo di definire gli obiettivi di qualità e le strategie che il gruppo \GRUPPO{}
		adotterà per raggiungerli. Verrà inoltre illustrato come il gruppo affronterà le varie fasi di verifica
		per poter garantire il miglior risultato qualitativo possibile.
	\subsection{Scopo del \gl{prodotto}}
		\SCOPO
	\subsection{Glossario}
		\GLOSSARIO
	\subsection{Riferimenti}
		\subsubsection{Normativi}
			\begin{itemize}
				\item Norme di \gl{progetto}: \NPdocRR{};
				\item Standard ISO/IEC 9126:2001: \\
				\url{https://en.wikipedia.org/wiki/ISO/IEC_9126};
				\item Capability Maturity Model (\gl{CMM}): \\
				\url{https://en.wikipedia.org/wiki/Capability_Maturity_Model};
				\item Plan-Do-Check-Act (\gl{PDCA}): \\
				\url{https://en.wikipedia.org/wiki/PDCA}.
			\end{itemize}
		\subsubsection{Informativi}
			\begin{itemize}
				\item Piano di \gl{progetto}: \PPdocRR{};
				\item Slide del corso di Ingegneria del software - Qualità del software : \\
				\url{http://www.math.unipd.it/~tullio/IS-1/2016/Dispense/L10.pdf};
				\item Slide del corso di Ingegneria del software - Qualità di processo : \\
				\url{http://www.math.unipd.it/~tullio/IS-1/2016/Dispense/L11.pdf};
				\item Indice Gulpease: \\
				\url{https://it.wikipedia.org/wiki/Indice_Gulpease}.
			\end{itemize}
			
\end{document}