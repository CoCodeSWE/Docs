\lettera{A}

\parola{Adapter}{Design pattern utilizzato nella programmazione orientata agli oggetti col fine di far comunicare tra loro interfacce di classi differenti. Per un maggiore approfondimento si rimanda all'appendice del file \DPdoc{}.}

\parola{Amazon Echo}{È uno smart speaker sviluppato da Amazon.}

\parola{Amazon Simple Notification Services}{Amazon SNS è un servizio di notifiche push rapido, flessibile e completamente gestito che consente di inviare messaggi individuali o collettivi a un numero elevato di destinatari. È sviluppato da Amazon.}

\parola{Amazon Web Services}{Collezione di servizi forniti tramite server remoto dall'azienda Amazon.}

\parola{Android}{Sistema operativo open source per dispositivi mobili sviluppato da Google.}

\parola{API}{Acronimo di application programming interface, è l'insieme di procedure disponibili al programmatore di solito raggruppate a formare un set di strumenti specifici per l'espletamento di un determinato compito all'interno di un certo programma.}

\parola{API Gateway}{API che agisce come porta d'entrata attraverso la quale le applicazioni possono accedere a dati, logica di business o funzionalità dei servizi di back-end.}

\parola{Apple}{Azienda statunitense che produce sistemi operativi, computer e dispositivi multimediali con sede a Cupertino, nello stato della California.}

\parola{Apple Watch}{È un orologio dotato di funzionalità per dispositivi mobili prodotto da \gl{Apple}.}

\parola{Asana}{Applicazione mobile e web creata per aiutare i team a tracciare il proprio lavoro. Il gruppo \GRUPPO{} l'ha utilizzata nel corso dello sviluppo del progetto \PROGETTO{}.}

\parola{Astah}{Strumento per la modellazione di diagrammi \gl{UML}. Fra le varie funzionalità, questo software permette di creare: diagrammi delle classi, diagrammi dei casi d’uso, diagrammi delle attività, diagrammi di sequenza e diagrammi dei componenti.}

\parola{AWS Lambda}{AWS Lambda è una piattaforma di calcolo senza server guidata dagli eventi fornita da Amazon.}
