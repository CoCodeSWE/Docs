\lettera{U}

\parola{UML}{\textit{Unified Modelling Language}, è un linguaggio di modellazione
e specifica basato sul paradigma object-oriented. Attraverso tale linguaggio è possibile definire i diagrammi dei casi d’uso, delle attività e delle
classi.}

\parola{Unità}{Nella documentazione del progetto \PROGETTO, per unità si intende il minimo componente di software dotato di funzionamento autonomo.}

\parola{Unity}{È un motore per videogiochi disponibile per diverse piattaforme.}

\parola{Unix}{È un sistema operativo portabile per computer inizialmente sviluppato da un gruppo di ricerca dei laboratori AT\&T e Bell Laboratories.}

\parola{Unix Timestamp}{Sistema che rappresenta il tempo come secondi trascorsi dalla mezzanotte (UTC) del 1º gennaio 1970 (detta epoca).}

\parola{URI}{Acronimo di Uniform Resource Identifier, è una stringa che identifica univocamente una risorsa generica che può essere un indirizzo Web, un documento, un'immagine, un file, un servizio, un indirizzo di posta elettronica e molto altro.}

\parola{URL}{Acronimo di  Uniform Resource Locator, è una sequenza di caratteri che identifica univocamente l'indirizzo di una risorsa in Internet.}