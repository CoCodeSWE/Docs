\documentclass[a4paper,titlepage]{article}

\makeatletter
\def\input@path{{../../../template/}{./img}}
\makeatother

\usepackage{Comandi}
\usepackage{Riferimenti}
\usepackage{Stile}

\def\NOME{Verbale 2017-02-07}
\def\VERSIONE{1.0.0}
\def\DATA{2017-02-07}
\def\REDATTORE{Mattia Bottaro}
\def\VERIFICATORE{Simeone Pizzi}
\def\RESPONSABILE{Mattia Bottaro}
\def\USO{Esterno}
\def\DESTINATARI{\COMMITTENTE \\ & \CARDIN \\ & \GRUPPO}
\def\SOMMARIO{Verbale dell'incontro interno in data 2017-02-28 per il \gl{capitolato} \quotes{\CAPITOLATO{}}  del gruppo \GRUPPO.}


\begin{document}

\maketitle
\begin{diario}
    \modifica{Pier Paolo Tricomi}{\RESP}{Stesura documento e verbalizzazione della riunione}{2017-02-28}{0.0.1}
\end{diario}
\newpage
\tableofcontents

\newpage
\section{Informazioni generali}
\label{sec:Informazioni}

\begin{itemize}
  \item \textbf{Luogo}: aula 1C150, Torre Archimede, Padova.
  \item \textbf{Data}: 2017-02-28.
  \item \textbf{Orario di inizio}: 10:00.
  \item \textbf{Orario di fine}: 10:50.
  \item \textbf{Durata}: 50m.
  \item \textbf{Oggetto}: discussione sull'analisi dei requisiti e sulle strategie di progettazione.
  \item \textbf{Partecipanti}: Luca Bertolini, Mattia Bottaro, Mauro Carlin, Andrea Magnan, Simeone Pizzi, Nicola Tintorri, Pier Paolo Tricomi.
  \item \textbf{Segretario}: Pier Paolo Tricomi.

\end{itemize}
\section{Riassunto della riunione}
\label{sec:RiassuntoRiunione}
 \subsection{Descrizione}
La riunione è avvenuta presso Torre Archimede a Padova, in particolare nell'aula 1BC45. Erano presenti il \gl{proponente} Stefano Dindo, accompagnato da Francesco Meneguzzo, quattro componenti del gruppo \GRUPPO{}, sei componenti del gruppo Kern3lP4nic e tre componenti del gruppo anSWEr. Sono state poste diverse domande al proponente in merito all'analisi dei requisiti consegnata e ad alcune strategie da adottare nel periodo di progettazione.
 \subsection{Decisioni}
 \begin{itemize}
  \item DE1.1 - Il proponente ha espresso una forte preferenza per un'interfaccia amministrativa vocale piuttosto che grafica, ma questa scelta è lasciata ad ogni singolo gruppo. 
  \item DE1.2 - Il proponente ha assicurato l'attivazione di seminari formativi su NodeJS e AWS, probabilmente dal 2017-03-07.
  \item DE1.3 - Il proponente suggerisce un pattern "personalizzato" per modellare il sistema da progettare. Tuttavia, dopo consulenza con il docente Riccardo Cardin, si e` deciso di non seguire questa strategia ed adottare pattern esistenti.
  \item DE1.4 - Il proponente ha confermato che Amazon Echo non sara` disponibile.
  \item DE1.5 - Il proponente preferisce che la notifica di un ospite in visita all'azienda sia comunicata su un canale Slack utilizzato da piu` persone piuttosto che avvertirne una sola.
  \item DE1.6 - Il proponente suggerisce di adottare un'architettura serverless facendo utilizzo di AWS Lambda.
 \end{itemize}

\subsection{Tematiche in sospeso}
  \begin{itemize}
  \item SE1.1 - Periodo e luogo nei quali verrano svolti i seminari.
  \end{itemize}
\end{document}
