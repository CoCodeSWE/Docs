\section {Capitolato C3}
	\subsection {Descrizione}
		Sono molti i rischi corsi da un'azienda durante la sua attività, tra i quali possibili catastrofi naturali. Lo scopo del \gl{progetto} è quello di realizzare un'applicazione web che disegni gli scenari di
		danno che possono colpire un'azienda.
	\subsection {Dominio applicativo}
		L'applicazione è rivolta a tutte le aziende, infatti è interesse di tutte poter conoscere anticipatamente quelli che potrebbero essere i possibili rischi da loro corsi.
	\subsection {Dominio tecnologico}
		Non c'è l'obbligo di usare un particolare \gl{stack} tecnologico, vengono però suggerite le seguenti tecnologie:
		\begin{itemize}
			\item \textbf{\gl{Slack}} (per la comunicazione);
			\item \textbf{\gl{Asana}} (per la gestione dei processi);
			\item \textbf{\gl{Amazon Web Services}} (per l'archiviazione dei dati);
			\item \textbf{\gl{Bootstrap}} e \textbf{\gl{JavaScript}} (per la realizzazione dell'applicazione web).
		\end{itemize}
	\subsection {Valutazione}
		\subsubsection {Aspetti positivi}
		Gli aspetti ritenuti positivi di questo \gl{progetto} sono:
			\begin{itemize}
				\item il \gl{progetto} nella sua complessità non sembra essere eccessivamente impegnativo.
			\end{itemize}
		\subsubsection {Fattori di rischio}
		Gli eventuali fattori di rischio consistono in:
			\begin{itemize}
				\item il gruppo ritiene il \gl{progetto} poco stimolante;
				\item difficoltà di contatto con il \gl{proponente}.
			\end{itemize}
	\subsection {Conclusioni}
		Tutti i membri del gruppo ritengono il \gl{progetto} poco stimolante. Inoltre la mancata presenza fisica del \gl{proponente} alla presentazione del \gl{capitolato} ha scoraggiato il gruppo che ha deciso di scartare il \gl{progetto}.