\section {Capitolato C6}
	\subsection {Descrizione}
	Il presente capitolato ha per oggetto l’ affidamento della fornitura per la realizzazione
di un software di costruzione di diagrammi UML con la relativa generazione di codice
Java e Javascript tramite tecnologie web.
	\subsection {Dominio applicativo}
	Il progetto consiste nel formare un collegamento diretto tra UML e codice: permetterà pertanto a chiunque sia in grado di costruire diagrammi UML la generazione automatica di codice.
	\subsection {Dominio tecnologico}
	Sono richieste le seguenti tecnologie:
	\begin {itemize}
	\item \textbf{Java} o \textbf{JavaScript} (come linguaggio del codice che il software dovrà generare);
	\item \textbf{UML} (come standard per i diagrammi);
	\item \textbf{Tomcat} o \textbf{Node.js} (come linguaggio server-side);
	\item \textbf{HTML5} e \textbf{CSS3} (per l'interfaccia client);
	\end {itemize}
	\subsection {Valutazione}
		\subsubsection {Aspetti positivi}
		Gli aspetti considerati positivi sono:
			\begin {itemize}
			 	\item l'idea proposta e le sue specifiche sono state definite in modo chiaro ed esaustivo;
			 	\item possibilità di utilizzo di tecnologie già conosciute dal team.
			\end {itemize}
		\subsubsection {Fattori di rischio}
		I fattori che possono causare rischi sono:
			\begin {itemize}
				\item il gruppo risulta poco interessato al progetto e al suo dominio applicativo.
			\end {itemize}
	\subsection {Conclusioni}
	Il team ha deciso di scartare questo capitolato, nonostante la sua chiarezza di esposizione e l'uso di tecnologie conosciute, a causa di una mancanza di interesse in esso e nel suo ambito di utilizzo.