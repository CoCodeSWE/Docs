\section {Capitolato C1}
	\subsection {Descrizione}
		Il \gl{progetto} prevede la realizzazione di un \gl{API} market per microservizi. Viene richiesta un'applicazione web che permetta di:
		\begin {itemize}
			\item registrare le \gl{API} di un \gl{microservizio} attraverso la registrazione e la pubblicazione della sua interfaccia;
			\item consultare le \gl{API} e la documentazione relativa;
			\item visualizzare i dati tecnici d'uso delle singole \gl{API};
			\item associare diverse \gl{API} key d'uso per ogni \gl{API} registrata nell'applicativo e per ogni utente che ne faccia richiesta;
			\item monitorare l'utilizzo delle \gl{API} da parte di utenti esterni in base alle \gl{API} key rilasciate;
			\item bloccare le chiamate di utenti in possesso di \gl{API} key scadute e/o non valide.
		\end {itemize}
		Nel caso in cui più gruppi decidessero di affrontare il \gl{progetto}, allora l'applicazione web deve anche permettere di:
		\begin {itemize}
			\item definire diverse \gl{policy} di vendita delle singole \gl{API} in base a tempo, dati scambiati, numero di chiamate, ecc.;
			\item definire un \gl{Service Level Agreement} (SLA) per ciascuna \gl{API} rilasciata al netto dei ritardi introdotti dall'\gl{API} \gl{Gateway};
			\item monitorare il rispetto delle SLA definite al momento del rilascio;
			\item gestire una moneta virtuale per la compravendita delle \gl{API} da parte degli utenti;
			\item confrontare i dati tecnici delle \gl{API} tra loro;
			\item avere una gestione social degli utenti i quali possono votare e commentare le \gl{API}.
		\end {itemize}
	\subsection {Dominio applicativo} 
		Per ItalianaSoftware "i microservizi rivoluzioneranno ogni aspetto riguardante la produzione di \gl{software}". L'applicazione web è quindi rivolta a tutti gli utenti che usano o che useranno 
		un'architettura a microservizi per la realizzazione dei propri \gl{software}.
	\subsection {Dominio tecnologico}
		Il \gl{capitolato} prevede l'utilizzo delle seguenti tecnologie:
		\begin {itemize}
			\item \textbf{\gl{HTML5}, \gl{CSS3}, \gl{JavaScript}} (per realizzare le componenti web);
			\item \textbf{DB \gl{NoSQL}} o \textbf{DB \gl{SQL}} (per l'archiviazione dei dati);
			\item \textbf{\gl{Jolie}} (per la rappresentazione delle interfacce e per la creazione dell'\gl{API} \gl{Gateway}).
		\end {itemize}
	\subsection {Valutazione}
		\subsubsection {Aspetti positivi}
		Gli aspetti positivi del \gl{progetto} sono:
			\begin{itemize}
				\item molto interessante è l'apprendimento di \gl{Jolie} e di un nuovo paradigma di programmazione, quello orientato ai microservizi.
				\item il gruppo ritiene di avere buona conoscenza di \gl{HTML5}, \gl{CSS3} e \gl{JavaScript}, tecnologie che dovranno essere usate per la realizzazione dell'applicazione web.
			\end{itemize}
		\subsubsection {Fattori di rischio}
		I possibili fattori di rischio sono:
			\begin{itemize}
				\item collaborazione con un altro gruppo per la realizzazione dell'applicazione web;
				\item mancate competenze sull'argomento.
			\end{itemize}
	\subsection {Conclusioni}
		Il \gl{progetto} è stato valutato positivamente dal gruppo in quanto ritiene l'idea e le tecnologie per implementarla molto innovative. La maggioranza del gruppo ha mostrato però maggiore interesse
		per il \gl{capitolato} C2 e questo ha portato a scartare questo \gl{progetto}.
