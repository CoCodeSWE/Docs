\subsubsection{Scopo del processo}
Si occupa di accertare che lo svolgimento del processo in esame non introduca errori nel \gl{prodotto}.
\subsubsection{Aspettative del processo}
Una corretta implementazione di tale processo permette di individuare:
\begin{itemize}
	\item una procedura di \gl{verifica};
	\item i criteri per la verifica del prodotto.
\end{itemize}
\subsubsection{Documenti}
Il \RESP{} ha il compito di avviare la fase di verifica, assegnando i \gl{task} ai \VERP{}. Essi dovranno controllare che:
\begin{itemize}
	\item la sintassi sia corretta, facendo utilizzo degli strumenti automatici preposti e della tecnica walkthrough;
	\item i periodi non siano troppo lunghi, facendo utilizzo degli strumenti automatici preposti e della tecnica walkthrough;
	\item la struttura sia completa, non complicata e coerente con il contenuto del documento in esame.
\end{itemize}

\subsubsection{Diagrammi UML}
I \VERP{} devono controllare tutti i diagrammi UML prodotti, sia che
venga rispettato lo standard UML, sia che siano corretti semanticamente.
\paragraph{Diagrammi dei casi d'uso}
Per tutti i diagrammi d'uso, i \VERP{} dovranno controllare che:
\begin{itemize}
	\item rispettino lo standard UML;
	\item rappresentino ciò che dovrebbero modellare, facendo attenzione a inclusione, estensioni e generalizzazione;
	\item gli attori correlati siano corretti;
	\item non ci siano difetti grafici.
\end{itemize}
\paragraph{Diagrammi di sequenza}
Per tutti i diagrammi di sequenza, i \VERP{} dovranno controllare che:
\begin{itemize}
	\item rispettino lo standard UML;
	\item rappresentino esattamente la sequenza di azioni che il diagramma deve mostrare;
	\item non ci siano difetti grafici.
\end{itemize}
\paragraph{Diagramma dei package}
Per tutti i diagrammi dei package, i \VERP{} dovranno controllare che:
\begin{itemize}
	\item rispettino lo standard UML;
	\item contengano tutte le classi e sotto-package definiti nella fase di progettazione;
	\item non ci siano difetti grafici.
\end{itemize}
\subsubsection{Analisi}
\paragraph{Analisi statica}
E' una tecnica di analisi del codice sorgente e della documentazione associata, prevalentemente
usata quando il \gl{sistema} non è ancora disponibile e durante tutto l'arco del suo sviluppo. Non
richiede l'esecuzione del prodotto \gl{software} in alcuna sua parte. Può essere applicata tramite una
delle seguenti strategie:
\begin{itemize}
	\item \textbf{Walkthrough}: si legge l'intero documento (o codice) in cerca di tutte le possibili anomalie. E' una tecnica onerosa che richiede l'impegno di più persone e per questo deve essere utilizzata solo durante la prima parte del \gl{progetto}, dove non tutti i membri hanno piena padronanza e conoscenza delle \NPdoc e del \PQdoc;
	\item \textbf{Inspection}: questa tecnica dev'essere applicata quando si ha idea della
	problematica che si sta cercando; consiste in una lettura mirata del
	documento (o del codice), sulla base di una lista degli errori precedentemente
	stilata.
\end{itemize}
\paragraph{Analisi dinamica}
L'attività di analisi dinamica è una tecnica di verifica applicabile solamente al software. Tale tecnica può essere utilizzata per analizzare l'intero software o una
porzione limitata dello stesso. L'attività consiste nell'esecuzione di test automatici realizzati
dal \gl{team}. Le verifiche devono essere effettuate su un insieme finito di casi, con valori di
ingresso, uno stato iniziale e un esito decidibile. Tutti i test producono risultati automatici
che inviano notifiche sulla tipologia di problema individuato. Ogni test è ripetibile, ossia applicabile durante l'intero \gl{ciclo di vita} del software.

\subsubsection{Test}
\paragraph{Test di unità}
L'obiettivo primario del test di unità consiste nell'isolare la parte più piccola di software testabile
nell'applicazione, chiamata unità, dal resto del codice per stabilire se funziona esattamente come previsto. Ogni singola unità viene sottoposta a test prima di essere integrata in moduli per
l'esecuzione del test delle interfacce tra i diversi moduli.\\
I test di unità saranno organizzati nel modo seguente: \\
\begin{center}TU\{Codice\}\end{center}
dove:
\begin{itemize}
	\item Codice assume un valore numerico univoco che identifica il singolo requisito.
\end{itemize}
\paragraph{Test di integrazione}
Il test di integrazione rappresenta l'estensione logica del test di unità. Il più semplice di questo
tipo di test consiste nella combinazione di due unità, già sottoposte a test, in un solo componente
e nel test dell'interfaccia presente tra le due. Il concetto che è alla base di questo approccio
consiste nell'esecuzione del test delle combinazioni di parti ed, eventualmente, nell'espansione
del processo al test dei moduli di un gruppo con quelli di altri gruppi. Alla fine, tutti i moduli
che compongono un processo vengono sottoposti al test contemporaneamente.\\
I test di integrazione saranno organizzati nel modo seguente: \\
\begin{center}TI\{Codice\}\end{center}
dove:
\begin{itemize}
	\item Codice assume un valore numerico univoco che identifica il singolo requisito.
\end{itemize}
\paragraph{Test di sistema}
Il test di sistema rappresenta la \gl{validazione} del prodotto software finale, ovvero quando lo si
ritiene giunto ad una versione definitiva. Viene quindi verificato il completo soddisfacimento dei
requisiti da parte del prodotto.\\
I test di sistema saranno organizzati nel modo seguente: \\
\begin{center}TS\{TipoRequisito\}\{Importanza\}\{Codice\}\end{center}
dove:
\begin{itemize}
	\item TipoRequisito può assumere uno dei seguenti valori:
	\begin{itemize}
		\item F: per i requisiti funzionali;
		\item Q: per i requisiti di qualità;
		\item V: per i requisiti di vincolo;
	\end{itemize}
	\item Importanza può assumere uno dei seguenti valori:
	\begin{itemize}
		\item O: per i requisiti obbligatori;
		\item D: per i requisiti desiderabili;
		\item F: per i requisiti facoltativi;
	\end{itemize}
	\item Codice assume un valore numerico univoco che identifica il singolo requisito.
\end{itemize}
\paragraph{Test di validazione}
Il test di validazione rappresenta il collaudo del prodotto in presenza del \gl{proponente}. Al
superamento di tale collaudo segue il rilascio ufficiale del prodotto sviluppato.\\
I test di validazione saranno organizzati nel modo seguente: \\
\begin{center} TV\{TipoRequisito\}\{Importanza\}\{Codice\} \end{center}
dove:
\begin{itemize}
	\item TipoRequisito può assumere uno dei seguenti valori:
	\begin{itemize}
		\item F: per i requisiti funzionali;
		\item Q: per i requisiti di qualità;
		\item V: per i requisiti di vincolo;
	\end{itemize}
	\item Importanza può assumere uno dei seguenti valori:
	\begin{itemize}
		\item O: per i requisiti obbligatori;
		\item D: per i requisiti desiderabili;
		\item F: per i requisiti facoltativi;
	\end{itemize}
	\item Codice assume un valore numerico univoco che identifica il singolo requisito.
\end{itemize}

\subsubsection{Procedure}
\paragraph{Issue tracking}
L'\gl{issue} tracking è un'attività di supporto per la figura dei \VERP, ai quali permette di tenere traccia, e contemporaneamente segnalare al \RESP, la presenza di potenziali errori in un documento o nel codice sorgente.
 \subparagraph{Gestione delle issue}
Qualora un \VER{} dovesse riscontrare delle anomalie, la procedura per la segnalazione e gestione del \gl{ticketing} di una issue è la seguente:
\begin{enumerate}
	\item il \VER{} dovrà aprire una nuova issue assegnandole una label che si riferisca al problema trovato;
	\item il \RESP{} di progetto dovrà valutare la issue; se la ritiene appropriata assegnerà ai redattori del documento (o ai \PRP) il compito di risolvere la issue;
	\item una volta risolta, e verificata, la issue dovrà essere marcata come conclusa da parte del \RESP{} o del \VER;
\end{enumerate}
\subsubsection{Strumenti}
\paragraph{Strumenti per l'issue tracking}
Lo strumento utilizzato per l'issue tracking è il servizio Issues messo a disposizione da \gl{GitHub}.
\paragraph{Verifica ortografica}
Viene utilizzata la verifica in tempo reale dell'ortografia, integrata in TexMaker. Essa marca,
sottolineando in rosso, le parole errate secondo la lingua italiana.
\paragraph{Indice di Gulpease}
Affinché un documento possa superare la fase di approvazione, è necessario che soddisfi il test di leggibilità con un indice Gulpease superiore a 40 punti.
 \subsection{Validazione}
 \subsubsection{Scopo}
 Lo scopo di questo processo è verificare il prodotto ottenuto sia coerente rispetto gli obiettivi prefissati.
 \subsubsection{Aspettative}
 Le aspettative di questo processo sono che:
 \begin{itemize}
 	\item il prodotto finale rispetti i parametri di qualità imposti;
 	\item il prodotto finale sia conforme rispetto le aspettative;
 	\item il prodotto finale sia corretto.
 \end{itemize}
 \subsubsection{Descrizione}
 Le responsabilità sono così distribuite:
 \begin{itemize}
 	\item i \VERP{} hanno il compito di eseguire i test, tracciandone i risultati;
 	\item il \RESP{} revisiona i risultati dei test, decidendo se ritenerli accettabili o meno. Inoltre, si assume la responsabilità con il committente della conformità del prodotto finale rispetto le aspettative.
 \end{itemize}
 \subsubsection{Procedure}
 La procedura di validazione è composta dai seguenti punti:
 \begin{itemize}
 	\item i \VERP{} effettuano test manuali sul prodotto finale, tracciandone i risultati;
 	\item il \RESP{} esamina i risultati e decide se ritenerli buoni o se ripetere i test;
 	\item se accettati, il \RESP{} dovrà consegnare i risultati al proponente.
 \end{itemize}
 \newpage

\subsection{Qualità}
\subsubsection{Notazione}
\paragraph{Metriche}
Per garantire la qualità del lavoro del team gli \AMMP{} hanno definito delle metriche, riportandole
nel \PQdoc, che devono rispettare la seguente notazione:\\ \\
\centerline{\textbf{M\textbraceleft{}X\textbraceright{}\textbraceleft{}Y\textbraceright{}\textbraceleft{}Z\textbraceright{}}} \\ \\
dove:
\begin{itemize}
	\item \textbf{X} indica se la metrica si riferisce a prodotti o processi e può assumere
	i valori:
	\begin{itemize}
		\item \textbf{PC} per indicare i processi;
		\item \textbf{PD} per indicare i prodotti.
	\end{itemize}
	\item \textbf{Y} presente solo se la metrica è riferita ai prodotti, indica se il termine prodotto si riferisce a documenti o al software e può assumere i seguenti valori:
	\begin{itemize}
		\item \textbf{D} per indicare i documenti;
		\item \textbf{S} per indicare il software;
	\end{itemize}
	\item \textbf{Z} indica il codice univoco della metrica (numero intero incrementale a partire da 1).
\end{itemize}
\paragraph{Obiettivi}
Per garantire la qualità del lavoro del team, gli \AMMP{} hanno definito degli obiettivi di qualità,
riportandoli nel \PQdoc, che devono rispettare la seguente notazione:\\ \\
\centerline{\textbf{O\textbraceleft{}X\textbraceright{}\textbraceleft{}Y\textbraceright{}\textbraceleft{}Z\textbraceright{}}} \\ \\
dove:
\begin{itemize}
	\item \textbf{X} indica se l'obiettivo si riferisce a prodotti o processi e può assumere i valori:
	\begin{itemize}
		\item \textbf{PC} per indicare i processi;
		\item \textbf{PD} per indicare i prodotti.
	\end{itemize}
	\item \textbf{Y} presente solo se l'obiettivo è riferito ai prodotti, indica se il termine prodotto si riferisce a documenti o al software e può assumere i seguenti valori:
	\begin{itemize}
		\item \textbf{D} per indicare i documenti;
		\item \textbf{S} per indicare il software;
	\end{itemize}
	\item \textbf{Z} indica il codice univoco dell'obiettivo (numero intero incrementale a partire da 1).
\end{itemize}
\subsubsection{Definizione metriche}
Di seguito sono definite le metriche utilizzate nel documento \PQdocRP{}. Ad ogni metrica è stata assegnato un codice identificativo per facilitare il tracciamento.
\paragraph{Qualità di processo}
\subparagraph{Percentuale di accessi avvenuti correttamente a PragmaDB - MPC1}
Indica il numero  di accessi avvenuti correttamente espresso in percentuale.\\
\begin{equation*}
\text{Percentuale accessi} = \frac{\text{Accessi avvenuti}}{\text{Richieste d'accesso}} * 100
\end{equation*}
\subparagraph{Schedule Variance - MPC2}
Indica se si è in linea, in anticipo o in ritardo rispetto ai tempi pianificati.\\
\begin{equation*}
\text{Schedule Variance = TP - TR}
\end{equation*}
\begin{itemize}
	\item \textbf{TP} è il tempo pianificato per terminare un attività;
	\item \textbf{TR} è il tempo reale che è stato impiegato.
\end{itemize}
\subparagraph{Cost Variance in percentuale - MPC3}
Indica se alla data corrente i costi corrispondono alla pianificazione, espressi in percentuale.
\begin{equation*}
\text{Cost Variance} = \frac{CP - CR}{CP} * 100
\end{equation*}
\begin{itemize}
	\item \textbf{CP} sono i costi pianificati per la data corrente;
	\item \textbf{CR} sono i costi reali sostenuti.
\end{itemize}
\subparagraph{Indice dei rischi non preventivati - MPC4}
É un indice che viene incrementato ogniqualvolta si manifesta un rischio non individuato nell'attività di analisi dei rischi.
\subparagraph{Numero di requisiti obbligatori soddisfatti - MPC5}
Indica la percentuale di requisiti obbligatori soddisfatti del prodotto.
\begin{equation*}
Numero requisiti = \frac{\text{requisiti obbligatori soddisfatti}}{\text{requisiti obbligatori totali}} * 100
\end{equation*}
\subparagraph{Numero di requisiti desiderabili soddisfatti - MPC6}
Indica la percentuale di requisiti desiderabili soddisfatti del prodotto.
\begin{equation*}
\text{Numero requisiti} = \frac{\text{requisiti desiderabili soddisfatti}}{\text{requisiti desiderabili totali}} * 100
\end{equation*}
\subparagraph{Numero di requisiti opzionali soddisfatti - MPC7}
Indica la percentuale di requisiti opzionali soddisfatti del prodotto.
\begin{equation*}
\text{Numero requisiti} = \frac{\text{requisiti opzionali soddisfatti}}{\text{requisiti opzionalii totali}} * 100
\end{equation*}
\subparagraph{SF-IN - MPC8}
Indice numerico che incrementa nel momento in cui viene individuato un modulo che, durante la sua esecuzione, chiama il modulo in oggetto.
\subparagraph{SF-OUT - MPC9}
Indice numerico che incrementa nel momento in cui viene individuato un modulo utilizzato dal modulo in oggetto durante la sua esecuzione.
\subparagraph{Numero di metodi per classe - MPC10}
Indica il numero di metodi definiti in ogni classe.
\subparagraph{Numero di parametri per metodo - MPC11}
Indica il numero di parametri definiti in ogni metodo.
\subparagraph{Indice di complessità ciclomatica - MPC12}
Dato un grafo non fortemente connesso che rappresenta una sezione di codice del software, indica il numero di cammini linearmente indipendenti.
\begin{equation*}
\text{Complessità ciclomatica} = E - N + 2P
\end{equation*}
\begin{itemize}
	\item \textbf{E} è il numero di archi del grafo;
	\item \textbf{N} è il numero di nodi del grafo;
	\item \textbf{P} è il numero di componenti connesse.
\end{itemize}
\subparagraph{Numero di livelli di annidamento - MPC13}
Indica il numero di procedure e funzioni annidate, ovvero richiamate all'interno di altre procedure o funzioni.
\subparagraph{Percentuale di linee di commento per linee di codice - MPC14}
Indica la percentuale di linee di commento rispetto alle linee di codice.
\begin{equation*}
\text{Percentuale linee di commento} = \frac{\text{Linee di commento}}{\text{Linee totali di codice}} * 100
\end{equation*}
\subparagraph{Indice di manutenibilità - MPC15}
Indica quanto sarà semplice mantenere il codice prodotto.
\begin{equation*}
\text{Manutenibilità} =\frac{(171 - 5,2 * \ln(V) - 0.23 * (\text{Complessita ciclomatica}) - 16.2 * \ln(\text{Linee di codice})*100)}{171}
\end{equation*}
dove 

\begin{align*}
\text{V} = & (\text{Numero totale di operatori} + \text{Numero totale di operandi})* \\ 
& \log_2 (\text{Numero di operatori distinti}+\text{Numero di operandi distinti})
\end{align*}


\subparagraph{Percentuale di componenti integrate nel sistema - MPC16}
Indica la percentuale di componenti attualmente implementate e correttamente integrate nel sistema.
\begin{equation*}
\text{Componenti integrate} = \frac{\text{Numero componenti integrate}}{\text{Numero componenti totali progettate}} * 100
\end{equation*}
\subparagraph{Percentuale di test di unità eseguiti - MPC17}
Indica la percentuale di test di unità eseguiti.
\begin{equation*}
\text{Test di unità eseguiti} = \frac{\text{Numero test di unita eseguiti}}{\text{Numero test di unita pianificati}} * 100
\end{equation*}
\subparagraph{Percentuale di test di integrazione eseguiti - MPC18}
Indica la percentuale di test di integrazione eseguiti.
\begin{equation*}
\text{Test di integrazione eseguiti} = \frac{\text{Numero test di integrazione eseguiti}}{\text{Numero test di integrazione pianificati}} * 100
\end{equation*}
\subparagraph{Percentuale di test di sistema eseguiti - MPC19}
Indica la percentuale di test di sistema eseguiti.
\begin{equation*}
\text{Test di sistema eseguiti} = \frac{\text{Numero test di sistema eseguiti}}{\text{Numero test di sistema pianificati}} * 100
\end{equation*}
\subparagraph{Percentuale di test di validazione eseguiti - MPC20}
Indica la percentuale di test di validazione eseguiti.
\begin{equation*}
\text{Test di validazione eseguiti} = \frac{\text{Numero test di validazione eseguiti}}{\text{Numero test di validazione pianificati}} * 100
\end{equation*}
\subparagraph{Percentuale dei test superati - MPC21}
Indica la percentuale di test superati.
\begin{equation*}
\text{Test superati} = \frac{\text{Numero test superati}}{\text{Numero test eseguiti}} * 100
\end{equation*}
\subparagraph{Percentuale di rami decisionali percorsi - MPC22}
Indica la percentuale di rami decisionali percorsi dai test utilizzati.
\begin{equation*}
\text{Rami decisionali percorsi} = \frac{\text{Numero rami decisionali percorsi}}{\text{Numero rami decisionali totali}} * 100
\end{equation*}
\subparagraph{Numero di funzioni chiamate nei test - MPC23}
Indica il numero di funzioni chiamate nei test utilizzati.
\begin{equation*}
\text{Numero di funzioni chiamate} = \frac{\text{Funzioni chiamate}}{\text{Numero di funzioni totali}} * 100
\end{equation*}

\subparagraph{Numero di istruzioni nei test - MPC24}
Indica il numero di istruzioni eseguite nei test utilizzati.

\begin{equation*}
\text{Numero di istruzioni} = \frac{\text{Istruzioni eseguite}}{\text{Istruzioni totali}} * 100
\end{equation*}


\paragraph{Qualità di prodotto}
\subparagraph{Indice Gulpease - MPDD1}
Permette di calcolare i livello di leggibilità e comprensibilià del documento.

\begin{equation*}\textit{Indice Gulpease} = 89 + \frac{300 * \textit{A} + 10 * B}{C}\end{equation*} \\
\begin{itemize}
	\item \textbf{A} è il numero totale di frasi;
	\item \textbf{B} è il numero totale di lettere
	\item \textbf{C} è il numero totale di parole;
\end{itemize}
\subparagraph{Completezza dell’implementazione funzionale - MPDS1}
Permette di calcolare i livello di leggibilità e comprensibilià del documento.

\begin{equation*}
C=(1-\frac{FM}{FI})\cdot 100
\end{equation*}
\begin{itemize}
	\item \textbf{FM} è il numero di funzionalità mancanti nell'implementazione;
	\item \textbf{FI} è il numero di funzionalità individuate nell'attività di analisi;
\end{itemize}
\subparagraph{Percentuale di risultati concordi alle attese - MPDS2}
Calcola quanti risultati sono concordi alle attese.

\begin{equation*} RC = (1-\frac{N_{RD}}{N_{TE}}) \cdot 100 \end{equation*}
\begin{itemize}
	\item \textbf{RD} è il numero di test che producono risultati discordanti rispetto alle attese;
	\item \textbf{TE}  è il numero di test-case eseguiti;
\end{itemize}
\subparagraph{Percentuale di operazioni illegali non bloccate - MPDS3}
Calcola quante operazioni illegali non sono state bloccate.

\begin{equation*}I=\frac{N_{IE}}{N_{II}} \cdot 100\end{equation*}
\begin{itemize}
	\item \textbf{IE} è il numero di operazioni illegali effettuabili dai test;
	\item \textbf{II} è il numero di operazioni illegali individuate;
\end{itemize}
\subparagraph{Percentuale failure su test-case - MPDS4}
Calcola la percentuale di operazioni di testing che si sono concluse in failure.

\begin{equation*}F=\frac{N_{FR}}{N_{TE}} \cdot 100 \end{equation*}
\begin{itemize}
	\item \textbf{FR} è il numero di failure rilevati durante l'attività di testing;
	\item \textbf{TE} è il numero di test-case eseguiti;
\end{itemize}
\subparagraph{Numero di failure evitati - MPDS5}
Calcola la percentuale di funzionalità in grado di gestire correttamente i fault che potrebbero verificarsi.

\begin{equation*}B=\frac{N_{FE}}{N_{ON}} \cdot 100 \end{equation*}
\begin{itemize}
	\item \textbf{FE} è il numero di failure evitati durante i test effettuati;
	\item \textbf{ON} è il numero di test-case eseguiti che prevedono l'esecuzione di operazioni non corrette, causa di possibili failure;
\end{itemize}
\subparagraph{Percentuale delle funzionalità comprese  - MPDS6}
Calcola la percentuale di operazioni comprese in modo immediato dall'utente, senza la consultazione del manuale.

\begin{equation*}C=\frac{N_{FC}}{N_{FO}} \cdot 100\end{equation*}
\begin{itemize}
	\item \textbf{FC} è il numero di funzionalità comprese in modo immediato dall'utente durante l'attività di testing del prodotto;
	\item \textbf{FO} è il numero di funzionalità offerte dal sistema;
\end{itemize}
\subparagraph{Percentuale di funzionalità conformi alle aspettative - MPDS7}
Calcola la percentuale di funzionalità offerte all'utente che rispettano le sue aspettative riguardo al comportamento del software.

\begin{equation*}C=(1-\frac{N_{MFI}}{N_{MFO}}) \cdot 100\end{equation*}
\begin{itemize}
	\item \textbf{MFI} è il numero di messaggi e funzionalità che non rispettano le aspettative dell'utente;
	\item \textbf{MFO} è il numero di messaggi e funzionalità offerti dal sistema;
\end{itemize}
\subparagraph{Tempo medio di risposta - MPDS8}
Calcola il periodo temporale medio trascorso tra la richiesta al software di una determinata funzionalità e la risposta all’utente.

\begin{equation*}T_{RISP} = \frac{\sum_{i=1}^{n} T_{i}}{n}\end{equation*}
\begin{itemize}
	\item \textbf{$T_{RISP}$} espresso in \textit{secondi};
	\item \textbf{$T_{i}$} è il tempo intercorso fra la richiesta $i$ di una funzionalità ed il completamento delle operazioni necessarie a restituire un risultato a tale richiesta;
\end{itemize}
\subparagraph{Percentuale di failure con cause individuate - MPDS9}
Calcola la percentuale di failure di cui sono state individuate le cause.

\begin{equation*}I=\frac{N_{FI}}{N_{FR}} \cdot 100 \end{equation*}
\begin{itemize}
	\item \textbf{FI}  è il numero di failure delle quali sono state individuate le cause;
	\item \textbf{FR} è il numero di failure rilevate;
\end{itemize}
\subparagraph{percentuale di failure introdotte con modifiche - MPDS10}
Calcola la percentuale di modifiche effettuate in risposta a failure che hanno portato all'introduzione di nuove failure in altre componenti del sistema.

\begin{equation*}I=\frac{N_{FRF}}{N_{FR}} \cdot 100 \end{equation*}
\begin{itemize}
	\item \textbf{FRF}  è il numero di failure risolte con l'introduzione di nuove failure;
	\item \textbf{FR} è il numero di failure risolte;
\end{itemize}

\subsubsection{Procedure}
\paragraph{Calcolo dell'indice di Gulpease}
Affinché un documento possa superare la fase di approvazione, è necessario che soddisfi il test di leggibilità con un indice Gulpease superiore a 40 punti. Per valutare questa metrica di qualità, è necessario seguire la seguente procedura:
\begin{itemize}
	\item dirigersi da terminale in \GulScript{};
	\item dare il comando \file{php gulpease.php};
	\item visualizzare il risultato sul terminale.
\end{itemize}
\paragraph{Controllo ortografico}
Per verificare la correttezza ortografica è necessario seguire la seguente procedura:
\begin{itemize}
	\item aprire \gl{Texmaker};
	\item aprire il documento interessato nel formato .tex;
	\item dal menù a tendina "Modifica", selezionare la voce "verifica ortografia".
\end{itemize}
Per rendere ciò possibile, è necessario installare il pacchetto relativo dizionario italiano per Texmaker.
\paragraph{Resoconto stato metriche}
Per visualizzare il resoconto corrente dello stato di ciò che le metriche indicano, è necessario seguire la seguente procedura:
\begin{itemize}
	\item effettuare l'accesso in PragmaDB;
	\item selezionare la voce "Metriche".
\end{itemize}
\paragraph{Resoconto stato test di unità}
Per visualizzare il resoconto corrente dello stato dei test di unità è necessario seguire la seguente procedura:
\begin{itemize}
	\item entrare nella cartella \texttt{/responsoTest} del repository contenente il codice;
	\item aprire il file corrispondente al test eseguito, identificato con data e ora dell'esecuzione.
\end{itemize}
\paragraph{Resoconto stato build}
Per visualizzare il resoconto corrente dello stato di una delle build eseguite, è necessario seguire la seguente procedura:
\begin{itemize}
	\item entrare nella cartella \texttt{/statoBuild} del repository contenente il codice;
	\item aprire il file corrispondente al numero della build sulla quale si vogliono informazioni.
\end{itemize}
\paragraph{Eseguire test di unità in locale}
Per eseguire un test di unità dal proprio computer in locale, è necessario seguire la seguente procedura:
\begin{itemize}
	\item aprire il terminale;
	\item spostarsi nella cartella contenente il test di unità;
	\item scrivere nel terminale \texttt{mocha nometest.js}.
\end{itemize}

\subsubsection{Strumenti}
\paragraph{Script per il calcolo dell'indice di Gulpease}
In \GulScript{} si trova lo script che calcola l'indice di Gulpease per ogni documento.
\paragraph{Controllo ortografico}
Per il controllo ortografico dei documenti si farà utilizzo dello strumento integrato in Texmaker.\\
Per poterlo utilizzare, è necessario disporre del pacchetto per il dizionario italiano.
\paragraph{Integrazione continua - Jenkins}
Jenkins è uno strumento \gl{open source} di continuous integration, scritto in linguaggio \gl{Java} che fornisce dei servizi di integrazione continua per lo sviluppo del software. Viene eseguito lato server all'interno di un server web che supporta la tecnologia Servlet e quindi può essere utilizzato da remoto all'interno di un Web \gl{browser}.\\
Questo strumento permetterà al gruppo di automatizzare i test di unità, che verranno eseguiti ad ogni push eseguita da un programmatore, ed il processo di build e deployment del prodotto.
\subparagraph{Codifica ed esecuzione test di unità - Mocha}
Mocha è un \gl{framework} che permette di creare ed eseguire test sia per \gl{Node.js} che per il browser web.
\subparagraph{Calcolo della copertura dei test - Istanbul}
Istanbul è uno strumento che calcola la copertura dei test rispetto a statement, righe, funzioni e branch di un determinato progetto.
\subparagraph{Automatizzazione test client-side - PhantomJS}
PhantomJS è un headless browser, cioè di un tool che consente la manipolazione via \gl{JavaScript} del \gl{DOM}, di CSS, \gl{JSON}, Ajax altre tecnologie Web client-side da riga di comando, senza alcun rendering a video. Utilizzeremo questo framework per automatizzare il processo di test dell'interfaccia grafica client-side.
\subparagraph{Framework per il testing di applicazioni web - Selenium}
Selenium è un framework che permette di creare test per applicazioni web.

\paragraph{Requisiti obbligatori soddisfatti}
Lo strumento scelto per il calcolo del valore di questa metrica è PragmaDB, il quale permette di tracciare i requisiti ed associarli su use case e fonti.
\paragraph{Requisiti accettati soddisfatti}
Lo strumento scelto per il calcolo del valore di questa metrica è PragmaDB, il quale permette di tracciare i requisiti ed associarli su use case e fonti.
\paragraph{Requisiti non accettati soddisfatti}
Lo strumento scelto per il calcolo del valore di questa metrica è PragmaDB, il quale permette di tracciare i requisiti ed associarli su use case e fonti.
\paragraph{Requisiti obbligatori soddisfatti}
Lo strumento scelto per il calcolo del valore di questa metrica è PragmaDB, il quale permette di tracciare i requisiti ed associarli su use case e fonti.
\paragraph{Structural Fan-In}
Lo strumento scelto per il calcolo del valore di questa metrica è PragmaDB, il quale permette di tracciare quanti moduli durante la loro esecuzione utilizzano un determinato modulo.
\paragraph{Structural Fan-Out}
Lo strumento scelto per il calcolo del valore di questa metrica è PragmaDB, il quale permette di tracciare quanti moduli vengono utilizzate durante l'esecuzione di un determinato modulo.
\paragraph{Metodi per classe}
Lo strumento scelto per il calcolo del valore di questa metrica è PragmaDB, il quale permette di tracciare il numero di metodi per ogni classe.
\paragraph{Parametri per metodo}
Lo strumento scelto per il calcolo del valore di questa metrica è PragmaDB, il quale permette di tracciare il numero di parametri per ogni metodo.
\paragraph{Componenti integrate}
Lo strumento scelto per il calcolo del valore di questa metrica è PragmaDB, il quale permette di tracciare le componenti che sono attualmente implementate e integrate nel sistema.
\paragraph{Test di unità eseguiti}
Lo strumento scelto per il calcolo del valore di questa metrica è PragmaDB, il quale permette di tracciare i test di unità eseguiti.
\paragraph{Test di integrazione eseguiti}
Lo strumento scelto per il calcolo del valore di questa metrica è PragmaDB, il quale permette di tracciare i test di integrazione eseguiti.
\paragraph{Test di sistema eseguiti}
Lo strumento scelto per il calcolo del valore di questa metrica è PragmaDB, il quale permette di tracciare i test di sistema eseguiti.
\paragraph{Test di validazione eseguiti}
Lo strumento scelto per il calcolo del valore di questa metrica è PragmaDB, il quale permette di tracciare i test di validazione eseguiti.
\paragraph{Test superati}
Lo strumento scelto per il calcolo del valore di questa metrica è PragmaDB, il quale permette di tracciare i test superati.
\paragraph{Completezza implementazione funzionale}
Lo strumento scelto per il calcolo del valore di questa metrica è PragmaDB, il quale permette di tracciare il numero di requisiti funzionali implementati.
\paragraph{Densità di failure}
Lo strumento scelto per il calcolo del valore di questa metrica è PragmaDB, il quale permette di tracciare le operazioni di testing che sono concluse in failure.

