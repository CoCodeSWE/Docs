\subsubsection{Scopo}
Include le attività e i compiti svolti per creare il \gl{prodotto}.
\subsubsection{Aspettative}
Le aspettative della corretta implementazione del processo sono:
\begin{itemize}
		\item realizzare un \gl{prodotto} finale conforme alle richieste del \gl{proponente} e che soddisfi le attività di \gl{validazione} e \gl{verifica};
		\item fissare gli obiettivi di sviluppo;
		\item fissare i vincoli tecnologici.
\end{itemize}

\subsubsection{Descrizione}
In accordo con lo standard \iso{ISO/IEC 12207}, il processo di sviluppo è composto dalle attività di:
\begin{itemize}
		\item analisi dei requisiti;
		\item progettazione;
		\item codifica;
		\item validazione.
\end{itemize}

\subsubsection{Analisi dei requisiti}
 \paragraph{Scopo dell'attività}
  Individuare i requisiti del \gl{progetto} dalle specifiche del \gl{capitolato} e tramite incontri con il pro-
  ponente. Tale attività produrrà un documento redatto dagli analisti, i quali avranno cura di elencare i \gl{casi d'uso} e i requisiti. Tale documento permette di
 capire le scelte di progettazione effettuate.
 \paragraph{Aspettative dell'attività}
 L'attività fissa come scopo la creazione di un documento che elencherà e rappresenterà i requisiti richiesti dal \gl{proponente}.
 \paragraph{Descrizione dell'attività}
 Tutti i requisiti analizzati, utilizzando le specifiche del \gl{capitolato} e consultando i proponenti negli
incontri effettuati, vanno specificati nell'\ARdocRR. Per analizzare e trovare i
requisiti si utilizza la tecnica dei \gl{casi d'uso}. Il tracciamento dei requisiti avviene tramite l'applicativo PragmaDB.
 \paragraph{Studio di fattibilità}
 Il \RESP{} di \gl{progetto} deve organizzare delle riunioni preventive, per permettere lo scambio
di opinioni tra i membri del gruppo sui capitolati proposti. Il documento \gl{prodotto} da queste
riunioni è lo \SFdocRR , il quale viene realizzato dagli \ANP{}. Essi devono
descrivere i seguenti punti: 
\begin{itemize}
 \item \textbf{Dominio tecnologico e applicativo}: si dà una valutazione prendendo in   considerazione
 la conoscenza attuale delle tecnologie richieste dal \gl{capitolato} in analisi da parte dei membri  
del gruppo;
 \item \textbf{Interesse strategico}: si valuta l'interesse strategico del gruppo di \gl{progetto} in relazione
al \gl{capitolato} in analisi;
 \item \textbf{Individuazione dei rischi}: si analizzano i possibili rischi in cui si può incorrere nel
\gl{capitolato} in analisi.
\end{itemize}
 \paragraph{\gl{Casi d'uso}}
 Ogni caso d'uso è così composto:
 \begin{itemize}
  \item \textbf{Codice identificativo}: codice univoco del caso d'uso in esame;
  \item \textbf{Titolo}: indica il titolo del caso d'uso;
  \item \textbf{Diagramma \gl{UML}}: rappresenta graficamente il caso d'uso;
  \item \textbf{Attori primari}: indica gli attori primari coinvolti;
  \item \textbf{Descrizione}: chiara, precisa e concisa descrizione del caso d'uso;
  \item \textbf{Precondizione}: indica la situazione che deve essere vera prima dell'esecuzione del caso d'uso;
  \item \textbf{Postcondizione} indica la situazione che deve essere vera dopo l'esecuzione del caso d'uso;
  \item \textbf{Scenario principale}: descrizione composta dal flusso dei \gl{casi d'uso} figli;
  \item \textbf{Scenari alternativi}: descrizione composta dai \gl{casi d'uso} che non appartengono al flusso
principale di esecuzione.
 \end{itemize}
 \paragraph{Codice identificativo dei \gl{casi d'uso}}
Ogni caso d'uso ha un proprio codice identificativo  che rispetta il seguente formalismo:\\ \\
\centerline{UC\textbraceleft{Codice}\textbraceright{}}
\\ \\dove:
\begin{itemize}
	\item \textbf{Codice}: indica il codice identificativo del requisito, è univoco e deve essere identificato in forma gerarchica.
\end{itemize}
 \paragraph{Requisiti}
 Ogni requisito è così composto:
  \begin{itemize}
  \item \textbf{Codice identificativo}: codice univoco del requisito;
  \item \textbf{Descrizione}: una breve descrizione, deve essere meno ambigua possibile;
  \item \textbf{Fonti}: identifica la fonte dalla quale è stato identificato il requisito.
 \end{itemize}
 \paragraph{Codice identificativo dei requisiti}
 Ogni requisito individuato avrà un codice identificativo univoco così formato: \\ \\
 \centerline{R\textbraceleft{}Tipo\textbraceright{}\textbraceleft{}Importanza\textbraceright{}\textbraceleft{}Codice\textbraceright{}}
 \\ \\
 dove:
 \begin{itemize}
 	\item \textbf{Tipo}: può assumere uno di questi valori:
 	\begin{itemize}
 		\item \textbf{F}: indica un requisito funzionale;
 		\item \textbf{Q}: indica un requisito di qualità;
 		\item \textbf{P}: indica un requisito prestazionale;
 		\item \textbf{V}: indica un requisito di vincolo.
 	\end{itemize}
 	\item \textbf{Importanza}: può assumere uno di questi valori:
 	\begin{itemize}
 		\item \textbf{O}: indica un requisito obbligatorio;
 		\item \textbf{D}: indica un requisito desiderabile;
 		\item \textbf{F}: indica un requisito facoltativo.
 	\end{itemize}
 	\item \textbf{Codice}: indica il codice identificativo del requisito, è univoco e deve essere identificato in forma gerarchica.
 \end{itemize}
 \paragraph{\gl{UML}}
 Viene utilizzata la versione corrente alla stesura del documento, ovvero la 2.5.
 \subsubsection{Progettazione}
 \paragraph{Scopo dell'attività}
 L'attività di progettazione definisce le linee essenziali della struttura del \gl{prodotto} \gl{software} in
 funzione dei requisiti individuati dall'analisi. L'obiettivo del processo consiste nella stesura dei
 documenti: \textit{"Specifica Tecnica"} e \textit{"Definizione di Prodotto"}. \\
 Compresi pienamente quali siano i requisiti del problema e approfondendo la progettazione a moduli
 abbastanza semplici da essere capiti da una sola persona, si otterranno le
 istruzioni necessarie ai \PRP{} per sviluppare il prodotto.
 \paragraph{Aspettative dell'attività}
 Il processo porta alla formazione dei documenti sopracitati, i quali garantiscono affidabilità e
 coerenza.
 \paragraph{Descrizione dell'attività}
 La progettazione deve rispettare tutti i vincoli e i requisiti concordati tra i componenti del gruppo
 e i proponenti. I documenti derivati da questa attività sono:
 \begin{itemize}
 	\item \textbf{Specifica tecnica}: descrive la progettazione ad alto livello relativa all'architettura dell'applicazione
 	e dei singoli componenti. Il documento specifica i diagrammi \gl{UML} ed i design
 	pattern utilizzati per realizzare l'architettura definendo inoltre i test necessari alla \gl{verifica};
 	\item \textbf{Definizione di \gl{prodotto}}: descrive in dettaglio la progettazione di \gl{sistema}, integrando
 	quanto scritto nella Specifica Tecnica. Il documento specifica i diagrammi \gl{UML} e le
 	definizioni delle classi definendo inoltre i test necessari alla \gl{verifica}.
 \end{itemize}
 \paragraph{Specifica tecnica}
 \begin{itemize} 
 	\item \textbf{Diagrammi \gl{UML}}:
 	\begin{itemize}
 		\item diagrammi delle classi;
 		\item diagrammi dei \gl{package};
 		\item diagrammi di attività;
 		\item diagrammi di sequenza.
 	\end{itemize}
 	\item \textbf{\gl{Design pattern}}: devono essere descritti i \gl{design pattern} utilizzati per realizzare l'architettura. Ogni design
 	pattern deve essere accompagnato da una descrizione ed un diagramma, che ne esponga il
 	significato e la struttura;
 	\item \textbf{Test di integrazione}: devono essere definite delle classi di \gl{verifica}, utili a verificare che ogni componente del
 	\gl{sistema} funzioni nella maniera appropriata.
 \end{itemize}
 \paragraph{Definizione di \gl{prodotto}}
 \begin{itemize}
 	\item \textbf{Diagrammi \gl{UML}}:
 	\begin{itemize}
 		\item diagrammi delle classi;
 		\item diagrammi di attività;
 		\item diagrammi di sequenza.
 	\end{itemize}
 	\item \textbf{Definizioni delle classi}: ogni classe progettata deve essere descritta in modo da spiegarne lo scopo e definirne le
 	funzionalità ad essa associate.
 	\item \textbf{Tracciamento delle classi}: ogni requisito deve essere tracciato, in modo da poter risalire alle classi ad esso associate. 
 	\item \textbf{Test di unità}: devono essere definiti dei test di unità utili a verificare che le componenti del \gl{sistema} funzionino nel modo previsto.
 \end{itemize}
 
 
 \subsubsection{Codifica}
 \paragraph{Scopo dell'attività}
 Lo scopo dell'attività è l'implementazione del \gl{prodotto}, concretizzando la soluzione tramite la codifica.  
 \paragraph{Aspettative dell'attività}
 L'aspettativa dell'attività è un \gl{prodotto} corretto, ovvero stabile, affidabile, funzionale e che soddisfi i requisiti. 
 \paragraph{Descrizione dell'attività}
 L'attività deve rispettare i compiti e gli strumenti espressi nel \PPdocRP.
 \paragraph{Stile}
 Al fine di rendere il codice più leggibile possibile, il gruppo dovrà rispettare le seguenti regole:
 \begin{itemize}
 	\item all'inizio di ogni file dovrà essere presente un'intestazione contenete il nome dell'autore, la data di creazione, l'utilità del file e il diario delle modifiche;
 	\item ogni modifica apportata dovrà essere descritta nell'intestazione;
 	\item scrivere dei commenti in italiano qualora certe porzioni di codice possano risultare poco chiare o complesse;
 	\item evitare di commentare porzioni di codice dal funzionamento banale;
 	\item il codice deve essere identato al meglio, in particolare se uno o più statement sono interni a una porzione di codice, essi dovranno rientrare di quattro spazi, come nell'esempio che segue: \begin{verbatim}
 	if(condition)
 	{
 	    a=b;
 	    b=c;
 	}
 	\end{verbatim}
 	\item le parentesi graffe che vanno a delimitare un blocco di codice dovranno trovarsi al di sotto della sua segnatura, come negli esempi che seguono:
 	\begin{verbatim}
 	while(condition)
 	{
 	    operazioni
 	}
 	
 	void foo()
 	{
     	operazioni
 	}
 	\end{verbatim}
 	\item evitare più di una istruzione per riga;
 	\item le costanti devono essere scritte in maiuscolo;
 	\item i nomi delle variabili devono rispettare le seguenti regole:
 	\begin{itemize}
 		\item avere un nome più esplicativo possibile;
 		\item se il nome è composto da più parole, esse devono essere divise da un carattere di underscore ' \textbf{\_} ';
 		\item essere, se possibile, inizializzate ad un valore.
 	\end{itemize}
 	\item i nomi dei metodi e funzioni devono rispettare le seguenti regole:
 	\begin{itemize}
 		\item avere un nome più esplicativo possibile;
 		\item se composto da più parole, il nome deve seguire la notazione camelCase;
 		\item essere associati ad un contratto, il quale ne spiega l'utilità;
 		\item evitare un numero eccessivo di parametri.
 	\end{itemize}
 	\item evitare l'eccessivo innestarsi di statement;
 	\item linee troppo lunghe dovranno essere spezzate in più parti.
 \end{itemize}
 Il codice che segue vuole essere un esempio delle regole appena descritte.
\begin{figure}[h]
	\centering
	\includegraphics[scale=0.35]{img/esempioStileCodifica.png}
	\caption{Esempio di stile di codifica}
\end{figure} 
 \newpage
 \paragraph{Versionamento}
 La versione del codice viene inserita all’interno dell’intestazione del file e segue il seguente formalismo:\\
 \centerline{\textbf{x.y.z}} dove:
  \begin{itemize}
 	\item x: è l’indice di versione principale, un incremento di tale indice rappresenta un avanzamento
 	della versione stabile, di conseguenza i valori degli indici y e z devono essere azzerati;
 	\item y: è l’indice di verifica del file, un incremento di tale indice comporta l’azzeramento
 	dell’indice z;
 	\item z: è l’indice di modifica parziale, un incremento di tale indice rappresenta una modifica
 	rilevante, come per esempio la rimozione o l’aggiunta di una istruzione. La versione 1.0.0
 	deve rappresentare la prima versione del file completo e stabile, cioè quando le sue fun-
 	zionalità obbligatorie sono state definite e si considerano funzionanti. Solo dalla versione
 	1.0.0 è possibile testare il file, con degli appositi test definiti, per verificarne l’effettivo
 	funzionamento.
 \end{itemize}
 \paragraph{Ricorsione}
 La \gl{ricorsione} va evitata. Se non risulta accettabile convertirla in \gl{iterazione}, bisogna fornirne la prova di terminazione e l'analisi del costo in termini di spazio.
 \subsubsection{Procedure}
 \paragraph{Inserimento di un attore in PragmaDB}
 Per inserire un attore in PragmaDB è necessario applicare la seguente procedura:
 \begin{itemize}
 	\item effettuare l'accesso in PragmaDB;
 	\item selezionare la voce "Attori";
 	\item selezionare la voce "Inserisci Attore";
 	\item inserire nome e descrizione dell'attore.
 \end{itemize}
 \paragraph{Inserimento di un caso d'uso in PragmaDB}
 Per inserire un caso d'uso in PragmaDB è necessario applicare la seguente procedura:
 \begin{itemize}
 	\item effettuare l'accesso in PragmaDB;
 	\item selezionare la voce "Use Case";
 	\item selezionare la voce "Inserisci Use Case";
 	\item popolare tutti i campi richiesti;
 	\item inserire gli attori coinvolti;
 	\item inserire i requisiti correlati.
 \end{itemize}
 \paragraph{Inserimento di una fonte in PragmaDB}
 Per inserire una fonte in PragmaDB è necessario applicare la seguente procedura:
 \begin{itemize}
 	\item effettuare l'accesso in PragmaDB;
 	\item selezionare la voce "Fonti";
 	\item selezionare la voce "Inserisci Fonte";
 	\item inserire nome e descrizione della fonte.
 \end{itemize}
 \paragraph{Inserimento di un requisito in PragmaDB}
 Per inserire un requisito in PragmaDB è necessario applicare la seguente procedura:
 \begin{itemize}
 	\item effettuare l'accesso in PragmaDB;
 	\item selezionare la voce "Requisiti";
 	\item selezionare la voce "Inserisci Requisito";
 	\item popolare tutti i campi richiesti;
 	\item inserire la fonte correlata;
 	\item inserire l'use case correlato.
 \end{itemize}
 \paragraph{Inserimento di un package in PragmaDB}
 Per inserire un package in PragmaDB è necessario applicare la seguente procedura:
 \begin{itemize}
 	\item effettuare l'accesso in PragmaDB;
 	\item selezionare la voce "Package";
 	\item selezionare la voce "Inserisci Package";
 	\item popolare tutti i campi richiesti.
 \end{itemize}
 \paragraph{Inserimento di una classe in PragmaDB}
 Per inserire una classe in PragmaDB è necessario applicare la seguente procedura:
 \begin{itemize}
 	\item effettuare l'accesso in PragmaDB;
 	\item selezionare la voce "Classi";
 	\item selezionare la voce "Inserisci Classe";
 	\item popolare i campi richiesti;
 	\item inserire i requisiti correlati.
 \end{itemize}
 \paragraph{Tracciamento requisiti-fonti}
 Per tracciare una fonte su un requisito è sufficiente, al momento della creazione del requisito, inserire la fonte correlata. Se questo non è stato fatto, è necessario applicare la seguente procedura:
 \begin{itemize}
 	\item effettuare l'accesso in PragmaDB;
 	\item selezionare la voce "Requisiti";
 	\item selezionare la voce "Modifica" del requisito interessato;
 	\item selezionare e inserire la fonte correlata.
 \end{itemize}
 \paragraph{Tracciamento classi-requisiti}
 Per tracciare un requisito su una classe è sufficiente, al momento della creazione della classe, inserire il requisito correlato. Se questo non è stato fatto, è necessario applicare la seguente procedura:
 \begin{itemize}
 	\item effettuare l'accesso in PragmaDB;
 	\item selezionare la voce "Classe";
 	\item selezionare la voce "Modifica" della classe interessata;
 	\item selezionare e inserire il requisito correlato.
 \end{itemize}
 \paragraph{Tracciamento package-requisiti}
 Per realizzare questo tracciamento è sufficiente applicare il tracciamento classi-requisiti, in quanto il relativo package contenitore è dedotto automaticamente da PragmaDB.
 \paragraph{Produzione automatica del documento \DPdoc}
 Il documento \DPdoc{} è generato automaticamente da PragmaDB, il quale produrrà il relativo codice \LaTeX{} da compilare.
 Per far ciò, è necessario seguire la seguente procedura:
 \begin{itemize}
 	\item effettuare l'accesso in PragmaDB;
 	\item selezionare la voce "Classi";
 	\item selezionare le voci:
 	\begin{itemize}
 		\item "Genera Classi";
 		\item "Tracciamento Classi-Requisiti";
 		\item "Tracciamento Requisiti-Classi";
 		\item "Tracciamento Componenti-Requisiti";
 		\item "Tracciamento Requisiti-Componenti";
 		\item "Genera immagini";
 		\item "Segnali";
 	\end{itemize}
 	\item selezionare la voce "Homepage";
 	\item selezionare la voce "Package";
 	\item selezionare le voci:
 	\begin{itemize}
 		\item "Genera Package";
 		\item "Tracciamento Componenti-Requisiti";
 		\item "Tracciamento Requisiti-Componenti";
 	\end{itemize}
 	\item aprire il file \DPfile;
 	\item in quest'ultimo file includere i file .tex generati da pragmaDB con il comando \begin{verbatim}
 	input{nomeFile}
 	\end{verbatim}
 	seguendo questo ordine: 
 	\begin{itemize}
 		\item package;
 		\item classi;
 		\item tracciamentoClassiRequisiti;
 		\item tracciamentoRequisitiClassi;
 		\item tracciamentoClassiComponenti;
 		\item tracciamentoComponentiClassi;
 		\item tracciamentoComponentiRequisiti;
 		\item tracciamentoRequisitiComponenti.
 	\end{itemize}
 \end{itemize}
 Sarà cura dei \PJP{} produrre e posizionare i diagrammi di sequenza.
\paragraph{Produzione automatica del documento \ARdoc}
Il documento \ARdoc{} è generato automaticamente da PragmaDB, il quale produrrà il relativo codice \LaTeX{} da compilare.
Per far ciò, è necessario seguire la seguente procedura:
\begin{itemize}
	\item effettuare l'accesso in PragmaDB;
	\item selezionare la voce "Use Case";
	\item selezionare tutte le voci sotto "Esporta in \LaTeX";
	\item tornare nella home page;
	\item selezionare la voce "Requisiti";
	\item selezionare tutte le voci sotto "Esporta in \LaTeX";
	\item aprire il file \ARfile;
	\item in quest'ultimo file includere i file .tex generati da pragmaDB con il comando \begin{verbatim}
	input{nomeFile}
	\end{verbatim}
	seguendo questo ordine: 
	\begin{itemize}
		\item useCase;
		\item requisiti;
		\item tracciamentoFontiRequisiti;
		\item tracciamentoRequisitiFonti;
		\item riepilogoRequisiti;
	\end{itemize}
\end{itemize}

\subsubsection{Strumenti}
 \paragraph{PragmaDB} 
  PragmaDB è uno strumento \gl{open source} di tracciamento dei requisiti. Verrà quindi utilizzato per semplificare e automatizzare il più possibile l'attività di analisi dei requisiti e di progettazione. In particolare, una volta inseriti \gl{casi d'uso}, attori, requisiti, fonti, classi e package, PragmaDB genera: 
  \begin{itemize}
  \item il codice \LaTeX{} relativo a \gl{casi d'uso} e requisiti in forma tabellare;
  \item i diagrammi \gl{UML} associati ai \gl{casi d'uso}.
  
  \end{itemize}
  Essendo \gl{open source}, questo strumento è stato adattato dal team \GRUPPO{} in base alle proprie necessità.
  \paragraph{\gl{Astah}}
  \gl{Astah} è uno strumento di modellazione \gl{UML}. Qualora i diagrammi \gl{UML} generati da PragmaDB non siano soddisfacenti, si ricorrerà all'utilizzo di \gl{Astah}. Viene utilizzata la versione 7.0 o superiori.
\begin{figure}[h]
\centering
\includegraphics[scale=0.3]{img/astah.png}
\caption{Astah}\label{sec:Figura1}
\end{figure}

\paragraph{api.ai}
api.ai è lo strumento che verrà adottato per lo sviluppo dell'assistente virtuale, per gestire le interazione con gli utenti del sistema.
\begin{figure}[h]
\centering
\includegraphics[scale=0.5]{img/api_ai.png}
\caption{api.ai}\label{sec:Figura1}
\end{figure}

\newpage


  
