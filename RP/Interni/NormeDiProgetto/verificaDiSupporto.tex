\subsubsection{Scopo del processo}
Si occupa di accertare che lo svolgimento del processo in esame non introduca errori nel \gl{prodotto}.
\subsubsection{Aspettative del processo}
Una corretta implementazione di tale processo permette di individuare:
\begin{itemize}
	\item una procedura di \gl{verifica};
	\item i criteri per la \gl{verifica} del \gl{prodotto}.
\end{itemize}
\subsubsection{Documenti}
Il \RESP{} ha il compito di avviare la fase di verifica, assegnando i task ai \VERP{}. Essi dovranno controllare che:
\begin{itemize}
	\item la sintassi sia corretta, facendo utilizzo degli strumenti automatici preposti e della tecnica walkthrough;
	\item i periodi non siano troppo lunghi, facendo utilizzo degli strumenti automatici preposti e della tecnica walkthrough;
	\item la struttura sia completa, non complicata e coerente con il contenuto del documento in esame.
\end{itemize}

\subsubsection{Diagrammi UML}
I \VERP{} devono controllare tutti i diagrammi UML prodotti, sia che
venga rispettato lo standard UML, sia che siano corretti semanticamente.
\paragraph{Diagrammi dei casi d'uso}
Per tutti i diagrammi d'uso, i \VERP{} dovranno controllare che:
\begin{itemize}
	\item rispettino lo standard UML;
	\item rappresentino ciò che dovrebbero modellare, facendo attenzione a inclusione, estensioni e generalizzazione;
	\item gli attori correlati siano corretti;
	\item non ci siano difetti grafici.
\end{itemize}
\paragraph{Diagrammi di sequenza}
Per tutti i diagrammi di sequenza, i \VERP{} dovranno controllare che:
\paragraph{Diagrammi di attività}
Per tutti i diagrammi di attività, i \VERP{} dovranno controllare che:
\paragraph{Diagramma dei package}
Per tutti i diagrammi dei package, i \VERP{} dovranno controllare che:
\subsubsection{Attività}
 \paragraph{Analisi statica}
E' una tecnica di analisi del codice sorgente e della documentazione associata, prevalentemente
usata quando il \gl{sistema} non è ancora disponibile e durante tutto l'arco del suo sviluppo. Non
richiede l'esecuzione del \gl{prodotto} \gl{software} in alcuna sua parte. Può essere applicata tramite una
delle seguenti strategie:
\begin{itemize}
	\item \textbf{Walkthrough}: si legge l'intero documento (o codice) in cerca di tutte le possibili anomalie. E' una tecnica onerosa che richiede l'impegno di più persone e per questo deve essere utilizzata solo durante la prima parte del \gl{progetto}, dove non tutti i membri hanno piena padronanza e conoscenza delle \NPdoc e del \PQdoc;
	\item \textbf{Inspection}: questa tecnica dev'essere applicata quando si ha idea della
problematica che si sta cercando; consiste in una lettura mirata del
documento (o del codice), sulla base di una lista degli errori precedentemente
stilata.
\end{itemize}
 \paragraph{Analisi dinamica}
L'attività di analisi dinamica è una tecnica di \gl{verifica} applicabile solamente al \gl{software}. Tale tecnica può essere utilizzata per analizzare l'intero \gl{software} o una
porzione limitata dello stesso. L'attività consiste nell'esecuzione di test automatici realizzati
dal team. Le verifiche devono essere effettuate su un insieme finito di casi, con valori di
ingresso, uno stato iniziale e un esito decidibile. Tutti i test producono risultati automatici
che inviano notifiche sulla tipologia di problema individuato. Ogni test è ripetibile, ossia
applicabile durante l'intero \gl{ciclo di vita} del \gl{software}.
\subsubsection{Procedure}
\paragraph{Issue tracking}
L'\gl{issue} tracking è un'attività di supporto per la figura dei \VERP, ai quali permette di tenere traccia, e contemporaneamente segnalare al \RESP, la presenza di potenziali errori in un documento o nel codice sorgente.
 \subparagraph{Gestione delle \gl{issue}}
Qualora un \VER{} dovesse riscontrare delle anomalie, la procedura per la segnalazione e gestione del \gl{ticketing} di una \gl{issue} è la seguente:
\begin{enumerate}
	\item il \VER{} dovrà aprire una nuova \gl{issue} assegnandole una label che si riferisca al problema trovato;
	\item il \RESP{} di \gl{progetto} dovrà valutare la \gl{issue}; se la ritiene appropriata assegnerà ai redattori del documento (o ai \PRP) il compito di risolvere la \gl{issue};
	\item una volta risolta, e verificata, la \gl{issue} dovrà essere marcata come conlcusa da parte del \RESP{} o del \VER;
\end{enumerate}
\subsubsection{Strumenti}
\paragraph{Strumenti per l'issue tracking}
Lo strumento utilizzato per l'\gl{issue} tracking è il servizio Issues messo a disposizione da \gl{GitHub}.
\paragraph{Verifica ortografica}
Viene utilizzata la \gl{verifica} in tempo reale dell'ortografia, integrata in TexMaker. Essa marca,
sottolineando in rosso, le parole errate secondo la lingua italiana.
\paragraph{Indice di Gulpease}
Affinché un documento possa superare la fase di approvazione, è necessario che soddisfi il test di leggibilità con un indice Gulpease superiore a 40 punti.
 \subsection{Validazione}
 \subsubsection{Scopo}
 Lo scopo di questo processo è verificare il prodotto ottenuto sia coerente rispetto gli obiettivi prefissati.
 \subsubsection{Aspettative}
 Le aspettative di questo processo sono che:
 \begin{itemize}
 	\item il prodotto finale rispetti i parametri di qualità imposti;
 	\item il prodotto finale sia conforme rispetto le aspettative;
 	\item il prodotto finale sia corretto.
 \end{itemize}
 \subsubsection{Descrizione}
 Le responsabilità sono così distribuite:
 \begin{itemize}
 	\item i \VERP{} hanno il compito di eseguire i test, tracciandone i risultati;
 	\item il \RESP{} revisiona i risultati dei test, decidendo se ritenerli accettabili o meno. Inoltre, si assume la responsabilità con il committente della conformità del prodotto finale rispetto le aspettative.
 \end{itemize}