\subsubsection{Scopo del processo}
Si occupa di accertare che lo svolgimento del processo in esame non introduca errori nel \gl{prodotto}.
\subsubsection{Aspettative del processo}
Una corretta implementazione di tale processo permette di individuare:
\begin{itemize}
	\item una procedura di \gl{verifica};
	\item i criteri per la \gl{verifica} del \gl{prodotto}.
\end{itemize}
\subsubsection{Attività}
 \paragraph{Analisi statica}
E' una tecnica di analisi del codice sorgente e della documentazione associata, prevalentemente
usata quando il \gl{sistema} non è ancora disponibile e durante tutto l'arco del suo sviluppo. Non
richiede l'esecuzione del \gl{prodotto} \gl{software} in alcuna sua parte. Può essere applicata tramite una
delle seguenti strategie:
\begin{itemize}
	\item \textbf{Walkthrough}: si legge l'intero documento (o codice) in cerca di tutte le possibili anomalie. E' una tecnica onerosa che richiede l'impegno di più persone e per questo deve essere utilizzata solo durante la prima parte del \gl{progetto}, dove non tutti i membri hanno piena padronanza e conoscenza delle \NPdoc e del \PQdoc;
	\item \textbf{Inspection}: questa tecnica dev'essere applicata quando si ha idea della
problematica che si sta cercando; consiste in una lettura mirata del
documento (o del codice), sulla base di una lista degli errori precedentemente
stilata.
\end{itemize}
 \paragraph{Analisi dinamica}
L'attività di analisi dinamica è una tecnica di \gl{verifica} applicabile solamente al \gl{software}. Tale tecnica può essere utilizzata per analizzare l'intero \gl{software} o una
porzione limitata dello stesso. L'attività consiste nell'esecuzione di test automatici realizzati
dal team. Le verifiche devono essere effettuate su un insieme finito di casi, con valori di
ingresso, uno stato iniziale e un esito decidibile. Tutti i test producono risultati automatici
che inviano notifiche sulla tipologia di problema individuato. Ogni test è ripetibile, ossia
applicabile durante l'intero \gl{ciclo di vita} del \gl{software}.
\subsection{Qualità}
\subsubsection{Notazione}
\paragraph{Metriche}
Per garantire la qualità del lavoro del team gli \AMMP{} hanno definito delle metriche, riportandole
nel \PQdoc, che devono rispettare la seguente notazione:\\ \\
\centerline{\textbf{M\textbraceleft{}X\textbraceright{}\textbraceleft{}Y\textbraceright{}\textbraceleft{}Z\textbraceright{}}} \\ \\
dove:
\begin{itemize}
	\item \textbf{X} indica se la metrica si riferisce a prodotti o processi e può assumere
	i valori:
	\begin{itemize}
		\item \textbf{PC} per indicare i processi;
		\item \textbf{PD} per indicare i prodotti.
	\end{itemize}
	\item \textbf{Y} presente solo se la metrica è riferita ai prodotti, indica se il termine \gl{prodotto} si riferisce a documenti o al software e può assumere i seguenti valori:
	\begin{itemize}
		\item \textbf{D} per indicare i documenti;
		\item \textbf{S} per indicare il software;
	\end{itemize}
	\item \textbf{Z} indica il codice univoco della metrica (numero intero incrementale a partire da 1).
\end{itemize}
\paragraph{Obiettivi}
Per garantire la qualità del lavoro del team, gli \AMMP{} hanno definito degli obiettivi di qualità,
riportandoli nel \PQdoc, che devono rispettare la seguente notazione:\\ \\
\centerline{\textbf{O\textbraceleft{}X\textbraceright{}\textbraceleft{}Y\textbraceright{}\textbraceleft{}Z\textbraceright{}}} \\ \\
dove:
\begin{itemize}
	\item \textbf{X} indica se l'obiettivo si riferisce a prodotti o processi e può assumere i valori:
	\begin{itemize}
		\item \textbf{PC} per indicare i processi;
		\item \textbf{PD} per indicare i prodotti.
	\end{itemize}
	\item \textbf{Y} presente solo se l'obiettivo è riferito ai prodotti, indica se il termine \gl{prodotto} si riferisce a documenti o al software e può assumere i seguenti valori:
	\begin{itemize}
		\item \textbf{D} per indicare i documenti;
		\item \textbf{S} per indicare il software;
	\end{itemize}
	\item \textbf{Z} indica il codice univoco dell'obiettivo (numero intero incrementale a partire da 1).
\end{itemize}
\subsubsection{Definizione metriche}
Di seguito sono definite le metriche utilizzate nel documento \PQdocRP{}. Ad ogni metrica è stata assegnato un codice identificativo per facilitare il tracciamento.
\paragraph{Qualità di processo}
\subparagraph{Percentuale di accessi avvenuti correttamente a PragmaDB - MPC1}
Indica il numero  di accessi avvenuti correttamente espresso in percentuale.\\
\textbf{Misurazione:}\begin{equation}
Percentuale accessi = \frac{Accessi avvenuti}{Richieste d'accesso} * 100
\end{equation}
\subparagraph{Schedule Variance - MPC2}
Indica se si è in linea, in anticipo o in ritardo rispetto ai tempi pianificati.\\
\textbf{Misurazione:}\begin{equation}
Schedule Variance = TP - TR 
\end{equation}
\begin{itemize}
	\item \textbf{TP} è il tempo pianificato per terminare un attività;
	\item \textbf{TR} è il tempo reale che è stato impiegato.
\end{itemize}		
\subparagraph{Cost Variance in percentuale - MPC3}
Indica se alla data corrente i costi corrispondono alla pianificazione, espressi in percentuale.
\textbf{Misurazione:}\begin{equation}
Cost Variance = \frac{CP - CR}{CP} * 100 
\end{equation}
\begin{itemize}
	\item \textbf{CP} sono i costi pianificati per la data corrente;
	\item \textbf{CR} sono i costi reali sostenuti.
\end{itemize}
\subparagraph{Indice dei rischi non preventivati - MPC4}
É un indice che viene incrementato ogniqualvolta si manifesta un rischio non individuato nell'attività di analisi dei rischi.
\subparagraph{Numero di requisiti obbligatori soddisfatti - MPC5}
Indica la percentuale di requisiti obbligatori soddisfatti del prodotto.
\textbf{Misurazione:}\begin{equation}
Numero requisiti = \frac{requisiti obbligatori soddisfatti}{requisiti obbligatori totali} * 100
\end{equation}
\subparagraph{Numero di requisiti desiderabili soddisfatti - MPC6}
Indica la percentuale di requisiti desiderabili soddisfatti del prodotto.
\textbf{Misurazione:}\begin{equation}
Numero requisiti = \frac{requisiti desiderabili soddisfatti}{requisiti desiderabili totali} * 100
\end{equation}
\subparagraph{Numero di requisiti opzionali soddisfatti - MPC7}
Indica la percentuale di requisiti opzionali soddisfatti del prodotto.
\textbf{Misurazione:}\begin{equation}
Numero requisiti = \frac{requisiti opzionali soddisfatti}{requisiti opzionalii totali} * 100
\end{equation}
\subparagraph{SF-IN - MPC8}
Indice numerico che incrementa nel momento in cui viene individuato un modulo che, durante la sua esecuzione, chiama il modulo in oggetto.
\subparagraph{SF-OUT - MPC9}
Indice numerico che incrementa nel momento in cui viene individuato un modulo utilizzato dal modulo in oggetto durante la sua esecuzione.
\subparagraph{???? - MPC10}

\subparagraph{Numero di metodi per classe - MPC11}
Indica il numero di metodi definiti in ogni classe.
\subparagraph{Numero di parametri per metodo - MPC12}
Indica il numero di parametri definiti in ogni metodo.
\subparagraph{Indice di complessità ciclomatica - MPC13}
Dato un grafo non fortemente connesso che rappresenta una sezione di codice del software, indica il numero di cammini linearmente indipendenti.
\textbf{Misurazione:}\begin{equation}
Complessità ciclomatica = E - N + 2P
\end{equation}
\begin{itemize}
	\item \textbf{E} è il numero di archi del grafo;
	\item \textbf{N} è il numero di nodi del grafo;
	\item \textbf{P} è il numero di componenti connesse.
\end{itemize}
\subparagraph{Numero di livelli di annidamento - MPC14}
Indica il numero di procedure e funzioni annidate, ovvero richiamate all'interno di altre procedure o funzioni.
\subparagraph{Percentuale di linee di commento per linee di codice - MPC15}
Indica la percentuale di linee di commento rispetto alle linee di codice.
\textbf{Misurazione:}\begin{equation}
Percentuale linee di commento = \frac{Linee di commento}{Linee di codice} * 100
\end{equation}
\subparagraph{Halstead Difficulty per funzione - MPC16}
Indica il livello di complessità di una funzione.
\textbf{Misurazione:}\begin{equation}
Halstead Difficulty = \frac{UOP}{2} * \frac{OD}{UOD}
\end{equation}	
\begin{itemize}
	\item \textbf{UOP} è il numero di operatori distinti;
	\item \textbf{OD} è il numero totale di operandi;
	\item \textbf{UOD} è il numero di operandi distinti.
\end{itemize}
\subparagraph{Halstead Volume per funzione - MPC17}
Indica la dimensione dell'implementazione di un algoritmo.
\textbf{Misurazione:}\begin{equation}
Halstead Volume = (OP + OD) * \log_2(UOP + UOD)
\end{equation}	
\begin{itemize}
	\item \textbf{OP} è il numero totale di operatori;
	\item \textbf{UOP} è il numero di operatori distinti;
	\item \textbf{OD} è il numero totale di operandi;
	\item \textbf{UOD} è il numero di operandi distinti.
\end{itemize}
\subparagraph{Halstead Effort per funzione - MPC18}
Indica il costo necessario a scrivere il codice di una funzione.
\textbf{Misurazione:}\begin{equation}
Halstead Effort = Halsted Difficulty * Halstead Volume
\end{equation}	
\subparagraph{Indice di manutenibilità - MPC19}
Indica quanto sarà semplice mantenere il codice prodotto.
\textbf{Misurazione:}\begin{equation}
Manutenibilità = 171 - 5,2 * \ln(Halstead Volume) - 0.23 * (Complessità ciclomatica) - 16.2 * \ln(Linee di codice)
\end{equation}
\subparagraph{Percentuale di componenti integrate nel sistema - MPC20}
Indica la percentuale di componenti attualmente implementate e correttamente integrate nel sistema.
\textbf{Misurazione:}\begin{equation}
Componenti integrate = \frac{Numero componenti integrate}{Numero componenti totali progettate} * 100
\end{equation}
\subparagraph{Percentuale di test di unità eseguiti - MPC21}
Indica la percentuale di test di unità eseguiti.
\textbf{Misurazione:}\begin{equation}
Test di unità eseguiti = \frac{Numero test di unità eseguiti}{Numero test di unità pianificati} * 100
\end{equation}
\subparagraph{Percentuale di test di integrazione eseguiti - MPC22}
Indica la percentuale di test di integrazione eseguiti.
\textbf{Misurazione:}\begin{equation}
Test di integrazione eseguiti = \frac{Numero test di integrazione eseguiti}{Numero test di integrazione pianificati} * 100
\end{equation}
\subparagraph{Percentuale di test di sistema eseguiti - MPC23}
Indica la percentuale di test di sistema eseguiti.
\textbf{Misurazione:}\begin{equation}
Test di sistema eseguiti = \frac{Numero test di sistema eseguiti}{Numero test di sistema pianificati} * 100
\end{equation}
\subparagraph{Percentuale di test di validazione eseguiti - MPC24}
Indica la percentuale di test di validazione eseguiti.
\textbf{Misurazione:}\begin{equation}
Test di validazione eseguiti = \frac{Numero test di validazione eseguiti}{Numero test di validazione pianificati} * 100
\end{equation}
\subparagraph{Percentuale dei test superati - MPC25}
Indica la percentuale di test superati.
\textbf{Misurazione:}\begin{equation}
Test superati = \frac{Numero test superati}{Numero test eseguiti} * 100
\end{equation}
\subparagraph{Percentuale di rami decisionali percorsi - MPC26}
Indica la percentuale di rami decisionali percorsi dai test di unità utilizzati.
\textbf{Misurazione:}\begin{equation}
Rami decisionali percorsi = \frac{Numero rami decisionali percorsi}{Numero rami decisionali totali} * 100
\end{equation}
\subparagraph{Numero di funzioni chiamate nei test - MPC27}
Indica il numero di funzioni chiamate nel test di unità utilizzato.
\subparagraph{Numero di istruzioni nei test - MPC28}
Indica il numero di istruzioni eseguite nel test di unità utilizzato.


\paragraph{Qualità di prodotto}
\subparagraph{Indice Gulpease - MPDD1}
Permette di calcolare i livello di leggibilità e comprensibilià del documento.
\textbf{Misurazione:}
\begin{equation}\textit{Indice Gulpease} = 89 + \frac{300 * \textit{A} + 10 * B}{C}\end{equation} \\
\begin{itemize}
	\item \textbf{A} è il numero totale di frasi;
	\item \textbf{B} è il numero totale di lettere
	\item \textbf{C} è il numero totale di parole;
\end{itemize}
\subparagraph{Completezza dell’implementazione funzionale - MPDS1}
Permette di calcolare i livello di leggibilità e comprensibilià del documento.
\textbf{Misurazione:}
\begin{equation}
C=(1-\frac{FM}{FI})\cdot 100
\end{equation}
\begin{itemize}
	\item \textbf{FM} è il numero di funzionalità mancanti nell'implementazione;
	\item \textbf{FI} è il numero di funzionalità individuate nell'attività di analisi;
\end{itemize}
\subparagraph{Percentuale di risultati concordi alle attese - MPDS2}
Calcola quanti risultati sono concordi alle attese.
\textbf{Misurazione:}
$A=(1-\frac{N_{RD}}{N_{TE}}) \cdot 100$
\begin{itemize}
	\item \textbf{RD} è il numero di test che producono risultati discordanti rispetto alle attese;
	\item \textbf{TE}  è il numero di test-case eseguiti;
\end{itemize}
\subparagraph{Percentuale di operazioni illegali non bloccate - MPDS3}
Calcola quante operazioni illegali non sono state bloccate.
\textbf{Misurazione:}
$I=\frac{N_{IE}}{N_{II}} \cdot 100$, dove $N_{IE}$
\begin{itemize}
	\item \textbf{IE} è il numero di operazioni illegali effettuabili dai test;
	\item \textbf{II} è il numero di operazioni illegali individuate;
\end{itemize}
\subparagraph{Percentuale failure su test-case - MPDS4}
Calcola la percentuale di operazioni di testing che si sono concluse in failure.
\textbf{Misurazione:}
$F=\frac{N_{FR}}{N_{TE}} \cdot 100$
\begin{itemize}
	\item \textbf{FR} è il numero di failure rilevati durante l'attività di testing;
	\item \textbf{TE} è il numero di test-case eseguiti;
\end{itemize}
\subparagraph{Numero di failure evitati - MPDS5}
Calcola la percentuale di funzionalità in grado di gestire correttamente i fault che potrebbero verificarsi.
\textbf{Misurazione:}
$B=\frac{N_{FE}}{N_{ON}} \cdot 100$
\begin{itemize}
	\item \textbf{FE} è il numero di failure evitati durante i test effettuati;
	\item \textbf{ON} è il numero di test-case eseguiti che prevedono l'esecuzione di operazioni non corrette, causa di possibili failure;
\end{itemize}
\subparagraph{Percentuale delle funzionalità comprese  - MPDS6}
Calcola la percentuale di operazioni comprese in modo immediato dall'utente, senza la consultazione del manuale.
\textbf{Misurazione:}
$C=\frac{N_{FC}}{N_{FO}} \cdot 100$
\begin{itemize}
	\item \textbf{FC} è il numero di funzionalità comprese in modo immediato dall'utente durante l'attività di testing del prodotto;
	\item \textbf{FO} è il numero di funzionalità offerte dal sistema;
\end{itemize}
\subparagraph{Percentuale di funzionalità conformi alle aspettative - MPDS7}
Calcola la percentuale di funzionalità offerte all'utente che rispettano le sue aspettative riguardo al comportamento del software.
\textbf{Misurazione:}
$C=(1-\frac{N_{MFI}}{N_{MFO}}) \cdot 100$
\begin{itemize}
	\item \textbf{MFI} è il numero di messaggi e funzionalità che non rispettano le aspettative dell'utente;
	\item \textbf{MFO} è il numero di messaggi e funzionalità offerti dal sistema;
\end{itemize}
\subparagraph{Tempo medio di risposta - MPDS8}
Calcola il periodo temporale medio trascorso tra la richiesta al software di una determinata funzionalità e la risposta all’utente.
\textbf{Misurazione:}
$T_{RISP} = \frac{\sum_{i=1}^{n} T_{i}}{n}$
\begin{itemize}
	\item \textbf{$T_{RISP}$} espresso in \textit{secondi};
	\item \textbf{$T_{i}$} è il tempo intercorso fra la richiesta $i$ di una funzionalità ed il completamento delle operazioni necessarie a restituire un risultato a tale richiesta;
\end{itemize}
\subparagraph{Percentuale di failure con cause individuate - MPDS9}
Calcola la percentuale di failure di cui sono state individuate le cause.
\textbf{Misurazione:}
$I=\frac{N_{FI}}{N_{FR}} \cdot 100$
\begin{itemize}
	\item \textbf{FI}  è il numero di failure delle quali sono state individuate le cause;
	\item \textbf{FR} è il numero di failure rilevate;
\end{itemize}
\subparagraph{percentuale di failure introdotte con modifiche - MPDS10}
Calcola la percentuale di modifiche effettuate in risposta a failure che hanno portato all'introduzione di nuove failure in altre componenti del sistema.
\textbf{Misurazione:}
$I=\frac{N_{FRF}}{N_{FR}} \cdot 100$
\begin{itemize}
	\item \textbf{FRF}  è il numero di failure risolte con l'introduzione di nuove failuree;
	\item \textbf{FR} è il numero di failure risolte;
\end{itemize}

\subsubsection{Procedure}
\paragraph{Calcolo dell'indice di Gulpease}
Affinché un documento possa superare la fase di approvazione, è necessario che soddisfi il test di leggibilità con un indice Gulpease superiore a 40 punti. Per valutare questa metrica di qualità, è necessario seguire la seguente procedura:
\begin{itemize}
	\item dirigersi con il terminale in \GulScript;
	\item dare il comando \file{php gulpease.php};
	\item visualizzare il risultato sul terminale.
\end{itemize}
\paragraph{Controllo ortografico}
Per verificare la correttezza ortografica è necessario seguire la seguente procedura:
\begin{itemize}
	\item aprire Texmaker;
	\item aprire il documento interessato nel formato .tex;
	\item dal menù a tendina "Modifica", selezionare la voce "verifica ortografia".
\end{itemize}
Per rendere ciò possibile, è necessario installare il pacchetto relativo dizionario italiano per Texmaker.
\paragraph{Resoconto stato metriche}
Per visualizzare il resoconto corrente dello stato di ciò che le metriche indicano, è necessario seguire la seguente procedura:
\begin{itemize}
	\item effettuare l'accesso in PragmaDB;
	\item selezionare la voce "Metriche".
\end{itemize}
\subsubsection{Strumenti}
\paragraph{Script per il calcolo dell'indice di Gulpease}
In \GulScript{} si trova lo script che calcola l'indice di Gulpease per ogni documento.
\paragraph{Controllo ortografico}
Per il controllo ortografico dei documenti si farà utilizzo dello strumento integrato in Texmaker.\\
Per poterlo utilizzare, è necessario disporre del pacchetto per il dizionario italiano.

\paragraph{Requisiti obbligatori soddisfatti}
Lo strumento scelto per il calcolo del valore di questa metrica è PragmaDB, il quale permette di tracciare i requisiti ed associarli su use case e fonti.
\paragraph{Requisiti accettati soddisfatti}
Lo strumento scelto per il calcolo del valore di questa metrica è PragmaDB, il quale permette di tracciare i requisiti ed associarli su use case e fonti.
\paragraph{Requisiti non accettati soddisfatti}
Lo strumento scelto per il calcolo del valore di questa metrica è PragmaDB,
\paragraph{Requisiti obbligatori soddisfatti}
Lo strumento scelto per il calcolo del valore di questa metrica è PragmaDB, il quale permette di tracciare i requisiti ed associarli su use case e fonti.
\paragraph{Structural Fan-In}
Lo strumento scelto per il calcolo del valore di questa metrica è PragmaDB,
\paragraph{Structural Fan-Out}
Lo strumento scelto per il calcolo del valore di questa metrica è PragmaDB,
\paragraph{Metodi per classe}
Lo strumento scelto per il calcolo del valore di questa metrica è PragmaDB, il quale permette di tracciare le classi ed associarle ad altre classi correlate e requisiti.
\paragraph{Parametri per metodo}
Lo strumento scelto per il calcolo del valore di questa metrica è PragmaDB,
\paragraph{Componenti integrate}
Lo strumento scelto per il calcolo del valore di questa metrica è PragmaDB,
\paragraph{Test di unità eseguiti}
Lo strumento scelto per il calcolo del valore di questa metrica è PragmaDB,
\paragraph{Test di integrazione eseguiti}
Lo strumento scelto per il calcolo del valore di questa metrica è PragmaDB,
\paragraph{Test di sistema eseguiti}
Lo strumento scelto per il calcolo del valore di questa metrica è PragmaDB,
\paragraph{Test di validazione eseguiti}
Lo strumento scelto per il calcolo del valore di questa metrica è PragmaDB,
\paragraph{Test superati}
Lo strumento scelto per il calcolo del valore di questa metrica è PragmaDB,
\paragraph{Completezza implementazione funzionale}
Lo strumento scelto per il calcolo del valore di questa metrica è PragmaDB,
\paragraph{Densità di failure}
Lo strumento scelto per il calcolo del valore di questa metrica è PragmaDB,