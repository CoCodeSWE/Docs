\subsection{Qualità}

\subsubsection{Metriche}
Per garantire la qualità del lavoro del team gli \AMMP{} hanno definito delle metriche, riportandole
nel \PQdoc, che devono rispettare la seguente notazione:\\ \\
\centerline{\textbf{M\textbraceleft{}X\textbraceright{}\textbraceleft{}Y\textbraceright{}\textbraceleft{}Z\textbraceright{}}} \\ \\
dove:
\begin{itemize}
	\item \textbf{X} indica se la metrica si riferisce a prodotti o processi e può assumere
	i valori:
	\begin{itemize}
		\item \textbf{PC} per indicare i processi;
		\item \textbf{PD} per indicare i prodotti.
	\end{itemize}
	\item \textbf{Y} presente solo se la metrica è riferita ai prodotti, indica se il termine \gl{prodotto} si riferisce a documenti o al software e può assumere i seguenti valori:
	\begin{itemize}
		\item \textbf{D} per indicare i documenti;
		\item \textbf{S} per indicare il software;
	\end{itemize}
	\item \textbf{Z} indica il codice univoco della metrica (numero intero incrementale a partire da 1).
\end{itemize}
\subsubsection{Obiettivi}
Per garantire la qualità del lavoro del team, gli \AMMP{} hanno definito degli obiettivi di qualità,
riportandoli nel \PQdoc, che devono rispettare la seguente notazione:\\ \\
\centerline{\textbf{O\textbraceleft{}X\textbraceright{}\textbraceleft{}Y\textbraceright{}\textbraceleft{}Z\textbraceright{}}} \\ \\
dove:
\begin{itemize}
	\item \textbf{X} indica se l'obiettivo si riferisce a prodotti o processi e può assumere i valori:
	\begin{itemize}
		\item \textbf{PC} per indicare i processi;
		\item \textbf{PD} per indicare i prodotti.
	\end{itemize}
	\item \textbf{Y} presente solo se l'obiettivo è riferito ai prodotti, indica se il termine \gl{prodotto} si riferisce a documenti o al software e può assumere i seguenti valori:
	\begin{itemize}
		\item \textbf{D} per indicare i documenti;
		\item \textbf{S} per indicare il software;
	\end{itemize}
	\item \textbf{Z} indica il codice univoco dell'obiettivo (numero intero incrementale a partire da 1).
\end{itemize}
\subsubsection{Procedure}

\paragraph{Calcolo dell'indice di Gulpease}
Affinché un documento possa superare la fase di approvazione, è necessario che soddisfi il test di
leggibilità con un indice Gulpease superiore a 40 punti. Per valutare questa metrica di qualità, è necessario seguire la seguente procedura:
\begin{itemize}
	\item dirigersi con il terminale in \GulScript;
	\item dare il comando \file{php gulpease.php};
    \item visualizzare il risultato sul terminale.
\end{itemize}
\paragraph{Controllo ortografico}
Per verificare la correttezza ortografica è necessario seguire la seguente procedura:
\begin{itemize}
	\item aprire Texmaker;
	\item aprire il documento interessato nel formato .tex;
	\item dal menù a tendina "Modifica", selezionare la voce "verifica ortografia".
\end{itemize}
Per rendere ciò possibile, è necessario installare il pacchetto relativo dizionario italiano per Texmaker.
\paragraph{Resoconto stato metriche}
Per visualizzare il resoconto corrente dello stato di ciò che le metriche indicano, è necessario seguire la seguente procedura:
\begin{itemize}
	\item effettuare l'accesso in PragmaDB;
	\item selezionare la voce "Metriche".
\end{itemize}
\subsubsection{Strumenti}
\paragraph{Script per il calcolo dell'indice di Gulpease}
In \GulScript{} si trova lo script che calcola l'indice di Gulpease per ogni documento.
\paragraph{Controllo ortografico}
Per il controllo ortografico dei documenti si farà utilizzo dello strumento integrato in Texmaker.\\
Per poterlo utilizzare, è necessario disporre del pacchetto per il dizionario italiano.

\paragraph{Requisiti obbligatori soddisfatti}
Lo strumento scelto per il calcolo del valore di questa metrica è PragmaDB, il quale permette di tracciare i requisiti ed associarli su use case e fonti.
\paragraph{Requisiti accettati soddisfatti}
Lo strumento scelto per il calcolo del valore di questa metrica è PragmaDB, il quale permette di tracciare i requisiti ed associarli su use case e fonti.
\paragraph{Requisiti non accettati soddisfatti}
Lo strumento scelto per il calcolo del valore di questa metrica è PragmaDB,
\paragraph{Requisiti obbligatori soddisfatti}
Lo strumento scelto per il calcolo del valore di questa metrica è PragmaDB, il quale permette di tracciare i requisiti ed associarli su use case e fonti.
\paragraph{Structural Fan-In}
Lo strumento scelto per il calcolo del valore di questa metrica è PragmaDB,
\paragraph{Structural Fan-Out}
Lo strumento scelto per il calcolo del valore di questa metrica è PragmaDB,
\paragraph{Metodi per classe}
Lo strumento scelto per il calcolo del valore di questa metrica è PragmaDB, il quale permette di tracciare le classi ed associarle ad altre classi correlate e requisiti.
\paragraph{Parametri per metodo}
Lo strumento scelto per il calcolo del valore di questa metrica è PragmaDB,
\paragraph{Componenti integrate}
Lo strumento scelto per il calcolo del valore di questa metrica è PragmaDB,
\paragraph{Test di unità eseguiti}
Lo strumento scelto per il calcolo del valore di questa metrica è PragmaDB,
\paragraph{Test di integrazione eseguiti}
Lo strumento scelto per il calcolo del valore di questa metrica è PragmaDB,
\paragraph{Test di sistema eseguiti}
Lo strumento scelto per il calcolo del valore di questa metrica è PragmaDB,
\paragraph{Test di validazione eseguiti}
Lo strumento scelto per il calcolo del valore di questa metrica è PragmaDB,
\paragraph{Test superati}
Lo strumento scelto per il calcolo del valore di questa metrica è PragmaDB,
\paragraph{Completezza implementazione funzionale}
Lo strumento scelto per il calcolo del valore di questa metrica è PragmaDB,
\paragraph{Densità di failure}
Lo strumento scelto per il calcolo del valore di questa metrica è PragmaDB,