\subsection{Qualità}
\subsubsection{Notazione}
\paragraph{Metriche}
Per garantire la qualità del lavoro del team gli \AMMP{} hanno definito delle metriche, riportandole
nel \PQdoc, che devono rispettare la seguente notazione:\\ \\
\centerline{\textbf{M\textbraceleft{}X\textbraceright{}\textbraceleft{}Y\textbraceright{}\textbraceleft{}Z\textbraceright{}}} \\ \\
dove:
\begin{itemize}
	\item \textbf{X} indica se la metrica si riferisce a prodotti o processi e può assumere
	i valori:
	\begin{itemize}
		\item \textbf{PC} per indicare i processi;
		\item \textbf{PD} per indicare i prodotti.
	\end{itemize}
	\item \textbf{Y} presente solo se la metrica è riferita ai prodotti, indica se il termine \gl{prodotto} si riferisce a documenti o al software e può assumere i seguenti valori:
	\begin{itemize}
		\item \textbf{D} per indicare i documenti;
		\item \textbf{S} per indicare il software;
	\end{itemize}
	\item \textbf{Z} indica il codice univoco della metrica (numero intero incrementale a partire da 1).
\end{itemize}
\paragraph{Obiettivi}
Per garantire la qualità del lavoro del team, gli \AMMP{} hanno definito degli obiettivi di qualità,
riportandoli nel \PQdoc, che devono rispettare la seguente notazione:\\ \\
\centerline{\textbf{O\textbraceleft{}X\textbraceright{}\textbraceleft{}Y\textbraceright{}\textbraceleft{}Z\textbraceright{}}} \\ \\
dove:
\begin{itemize}
	\item \textbf{X} indica se l'obiettivo si riferisce a prodotti o processi e può assumere i valori:
	\begin{itemize}
		\item \textbf{PC} per indicare i processi;
		\item \textbf{PD} per indicare i prodotti.
	\end{itemize}
	\item \textbf{Y} presente solo se l'obiettivo è riferito ai prodotti, indica se il termine \gl{prodotto} si riferisce a documenti o al software e può assumere i seguenti valori:
	\begin{itemize}
		\item \textbf{D} per indicare i documenti;
		\item \textbf{S} per indicare il software;
	\end{itemize}
	\item \textbf{Z} indica il codice univoco dell'obiettivo (numero intero incrementale a partire da 1).
\end{itemize}
\subsubsection{Definizione metriche}
Di seguito sono definite le metriche utilizzate nel documento \PQdocRP{}. Ad ogni metrica è stata assegnato un codice identificativo per facilitare il tracciamento.
\paragraph{Qualità di processo}
\subparagraph{Percentuale di accessi avvenuti correttamente a PragmaDB - MPC1}
Indica il numero  di accessi avvenuti correttamente espresso in percentuale.\\
\textbf{Misurazione:}\begin{equation}

Percentuale accessi = \frac{Accessi avvenuti}{Richieste d'accesso} * 100
\end{equation}
\subparagraph{Schedule Variance - MPC2}
Indica se si è in linea, in anticipo o in ritardo rispetto ai tempi pianificati.\\
\textbf{Misurazione:}\begin{equation}
Schedule Variance = TP - TR 
\end{equation}
<<<<<<< HEAD
\begin{itemize}
	\item \textbf{TP} è il tempo pianificato per terminare un attività;
	\item \textbf{TR} è il tempo reale che è stato impiegato.
\end{itemize}		
\subparagraph{Cost Variance in percentuale - MPC3}
Indica se alla data corrente i costi corrispondono alla pianificazione, espressi in percentuale.
\textbf{Misurazione:}\begin{equation}
Cost Variance = \frac{CP - CR}{CP} * 100 
\end{equation}
\begin{itemize}
	\item \textbf{CP} sono i costi pianificati per la data corrente;
	\item \textbf{CR} sono i costi reali sostenuti.
\end{itemize}
\subparagraph{Indice dei rischi non preventivati - MPC4}
É un indice che viene incrementato ogniqualvolta si manifesta un rischio non individuato nell'attività di analisi dei rischi.
\subparagraph{Numero di requisiti obbligatori soddisfatti - MPC5}
Indica la percentuale di requisiti obbligatori soddisfatti del prodotto.
\textbf{Misurazione:}\begin{equation}
Numero requisiti = \frac{requisiti obbligatori soddisfatti}{requisiti obbligatori totali} * 100
\end{equation}
\subparagraph{Numero di requisiti desiderabili soddisfatti - MPC6}
Indica la percentuale di requisiti desiderabili soddisfatti del prodotto.
\textbf{Misurazione:}\begin{equation}
Numero requisiti = \frac{requisiti desiderabili soddisfatti}{requisiti desiderabili totali} * 100
\end{equation}
\subparagraph{Numero di requisiti opzionali soddisfatti - MPC7}
Indica la percentuale di requisiti opzionali soddisfatti del prodotto.
\textbf{Misurazione:}\begin{equation}
Numero requisiti = \frac{requisiti opzionali soddisfatti}{requisiti opzionalii totali} * 100
\end{equation}
\subparagraph{SF-IN - MPC8}
Indice numerico che incrementa nel momento in cui viene individuato un modulo che, durante la sua esecuzione, chiama il modulo in oggetto.
\subparagraph{SF-OUT - MPC9}
Indice numerico che incrementa nel momento in cui viene individuato un modulo utilizzato dal modulo in oggetto durante la sua esecuzione.
\subparagraph{???? - MPC10}

\subparagraph{Numero di metodi per classe - MPC11}
Indica il numero di metodi definiti in ogni classe.
\subparagraph{Numero di parametri per metodo - MPC12}
Indica il numero di parametri definiti in ogni metodo.
\subparagraph{Indice di complessità ciclomatica - MPC13}
Dato un grafo non fortemente connesso che rappresenta una sezione di codice del software, indica il numero di cammini linearmente indipendenti.
\textbf{Misurazione:}\begin{equation}
Complessità ciclomatica = E - N + 2P
\end{equation}
\begin{itemize}
	\item \textbf{E} è il numero di archi del grafo;
	\item \textbf{N} è il numero di nodi del grafo;
	\item \textbf{P} è il numero di componenti connesse.
\end{itemize}
\subparagraph{Numero di livelli di annidamento - MPC14}
Indica il numero di procedure e funzioni annidate, ovvero richiamate all'interno di altre procedure o funzioni.
\subparagraph{Percentuale di linee di commento per linee di codice - MPC15}
Indica la percentuale di linee di commento rispetto alle linee di codice.
\textbf{Misurazione:}\begin{equation}
Percentuale linee di commento = \frac{Linee di commento}{Linee di codice} * 100
\end{equation}
\subparagraph{Halstead Difficulty per funzione - MPC16}
Indica il livello di complessità di una funzione.
\textbf{Misurazione:}\begin{equation}
Halstead Difficulty = \frac{UOP}{2} * \frac{OD}{UOD}
\end{equation}	
\begin{itemize}
	\item \textbf{UOP} è il numero di operatori distinti;
	\item \textbf{OD} è il numero totale di operandi;
	\item \textbf{UOD} è il numero di operandi distinti.
\end{itemize}
\subparagraph{Halstead Volume per funzione - MPC17}
Indica la dimensione dell'implementazione di un algoritmo.
\textbf{Misurazione:}\begin{equation}
Halstead Volume = (OP + OD) * \log_2(UOP + UOD)
\end{equation}	
\begin{itemize}
	\item \textbf{OP} è il numero totale di operatori;
	\item \textbf{UOP} è il numero di operatori distinti;
	\item \textbf{OD} è il numero totale di operandi;
	\item \textbf{UOD} è il numero di operandi distinti.
\end{itemize}
\subparagraph{Halstead Effort per funzione - MPC18}
Indica il costo necessario a scrivere il codice di una funzione.
\textbf{Misurazione:}\begin{equation}
Halstead Effort = Halsted Difficulty * Halstead Volume
\end{equation}	
\subparagraph{Indice di manutenibilità - MPC19}
Indica quanto sarà semplice mantenere il codice prodotto.
\textbf{Misurazione:}\begin{equation}
Manutenibilità = 171 - 5,2 * \ln(Halstead Volume) - 0.23 * (Complessità ciclomatica) - 16.2 * \ln(Linee di codice)
\end{equation}
\subparagraph{Percentuale di componenti integrate nel sistema - MPC20}
Indica la percentuale di componenti attualmente implementate e correttamente integrate nel sistema.
\textbf{Misurazione:}\begin{equation}
Componenti integrate = \frac{Numero componenti integrate}{Numero componenti totali progettate} * 100
\end{equation}
\subparagraph{Percentuale di test di unità eseguiti - MPC21}
Indica la percentuale di test di unità eseguiti.
\textbf{Misurazione:}\begin{equation}
Test di unità eseguiti = \frac{Numero test di unità eseguiti}{Numero test di unità pianificati} * 100
\end{equation}
\subparagraph{Percentuale di test di integrazione eseguiti - MCP22}
Indica la percentuale di test di integrazione eseguiti.
\textbf{Misurazione:}\begin{equation}
Test di integrazione eseguiti = \frac{Numero test di integrazione eseguiti}{Numero test di integrazione pianificati} * 100
\end{equation}
\subparagraph{Percentuale di test di sistema eseguiti - MCP23}
Indica la percentuale di test di sistema eseguiti.
\textbf{Misurazione:}\begin{equation}
Test di sistema eseguiti = \frac{Numero test di sistema eseguiti}{Numero test di sistema pianificati} * 100
\end{equation}
\subparagraph{Percentuale di test di validazione eseguiti - MCP24}
Indica la percentuale di test di validazione eseguiti.
\textbf{Misurazione:}\begin{equation}
Test di validazione eseguiti = \frac{Numero test di validazione eseguiti}{Numero test di validazione pianificati} * 100
\end{equation}
\subparagraph{Percentuale dei test superati - MCP25}
Indica la percentuale di test superati.
\textbf{Misurazione:}\begin{equation}
Test superati = \frac{Numero test superati}{Numero test eseguiti} * 100
\end{equation}
\subparagraph{Percentuale di rami decisionali percorsi - MPC26}
Indica la percentuale di rami decisionali percorsi dai test di unità utilizzati.
\textbf{Misurazione:}\begin{equation}
Rami decisionali percorsi = \frac{Numero rami decisionali percorsi}{Numero rami decisionali totali} * 100
\end{equation}
\subparagraph{Numero di funzioni chiamate nei test - MPC27}
Indica il numero di funzioni chiamate nel test di unità utilizzato.
\subparagraph{Numero di istruzioni nei test - MPC28}
Indica il numero di istruzioni eseguite nel test di unità utilizzato.
=======
Dove TP è il tempo pianificato per terminare un attività e TR è il tempo reale che è stato impiegato.
\subparagraph{Cost Variance in percentuale - MPC3}
Indica se alla data corrente i costi corrispondono alla pianificazione, espressi in percentuale.
\textbf{Misurazione:}\begin{equation}
Cost Variance = \frac{CP - CR}{CP}*100 
\end{equation}
Dove CP sono i costi pianificati per la data corrente e CR sono i costi reali sostenuti.
\subparagraph{Indice rischi non preventivati - MPC4}
Indicatore che evidenzia i rischi non preventivati.
\textbf{Misurazione:} è un indice che viene incrementato ogniqualvolta si manifesta un rischio non individuato nell'attività di analisi dei rischi.
\subparagraph{Numero requisiti obbligatori soddisfatti - MPC5}
Indica la percentuale di requisiti obbligatori soddisfatti del prodotto.
\textbf{Misurazione:}\begin{equation}
Numero requisiti = \frac{requisiti obbligatori soddisfatti}{requisiti obbligatori totali}*100
\end{equation}
\subparagraph{Numero requisiti desiderabili soddisfatti - MPC6}
Indica la percentuale di requisiti desiderabili soddisfatti del prodotto.
\textbf{Misurazione:}\begin{equation}
Numero requisiti = \frac{requisiti desiderabili soddisfatti}{requisiti desiderabili totali}*100
\end{equation}
\subparagraph{Numero requisiti opzionali soddisfatti - MPC7}
Indica la percentuale di requisiti opzionali soddisfatti del prodotto.
\textbf{Misurazione:}\begin{equation}
Numero requisiti = \frac{requisiti opzionali soddisfatti}{requisiti opzionalii totali}*100
\end{equation}


>>>>>>> 514256ed2a6a91d825a6a2fa884eeb865faceff9
\paragraph{Qualità di prodotto}
\subsubsection{Procedure}

\paragraph{Calcolo dell'indice di Gulpease}
Affinché un documento possa superare la fase di approvazione, è necessario che soddisfi il test di
leggibilità con un indice Gulpease superiore a 40 punti. Per valutare questa metrica di qualità, è necessario seguire la seguente procedura:
\begin{itemize}
	\item dirigersi con il terminale in \GulScript;
	\item dare il comando \file{php gulpease.php};
    \item visualizzare il risultato sul terminale.
\end{itemize}
\paragraph{Controllo ortografico}
Per verificare la correttezza ortografica è necessario seguire la seguente procedura:
\begin{itemize}
	\item aprire Texmaker;
	\item aprire il documento interessato nel formato .tex;
	\item dal menù a tendina "Modifica", selezionare la voce "verifica ortografia".
\end{itemize}
Per rendere ciò possibile, è necessario installare il pacchetto relativo dizionario italiano per Texmaker.
\paragraph{Resoconto stato metriche}
Per visualizzare il resoconto corrente dello stato di ciò che le metriche indicano, è necessario seguire la seguente procedura:
\begin{itemize}
	\item effettuare l'accesso in PragmaDB;
	\item selezionare la voce "Metriche".
\end{itemize}
\subsubsection{Strumenti}
\paragraph{Script per il calcolo dell'indice di Gulpease}
In \GulScript{} si trova lo script che calcola l'indice di Gulpease per ogni documento.
\paragraph{Controllo ortografico}
Per il controllo ortografico dei documenti si farà utilizzo dello strumento integrato in Texmaker.\\
Per poterlo utilizzare, è necessario disporre del pacchetto per il dizionario italiano.

\paragraph{Requisiti obbligatori soddisfatti}
Lo strumento scelto per il calcolo del valore di questa metrica è PragmaDB, il quale permette di tracciare i requisiti ed associarli su use case e fonti.
\paragraph{Requisiti accettati soddisfatti}
Lo strumento scelto per il calcolo del valore di questa metrica è PragmaDB, il quale permette di tracciare i requisiti ed associarli su use case e fonti.
\paragraph{Requisiti non accettati soddisfatti}
Lo strumento scelto per il calcolo del valore di questa metrica è PragmaDB,
\paragraph{Requisiti obbligatori soddisfatti}
Lo strumento scelto per il calcolo del valore di questa metrica è PragmaDB, il quale permette di tracciare i requisiti ed associarli su use case e fonti.
\paragraph{Structural Fan-In}
Lo strumento scelto per il calcolo del valore di questa metrica è PragmaDB,
\paragraph{Structural Fan-Out}
Lo strumento scelto per il calcolo del valore di questa metrica è PragmaDB,
\paragraph{Metodi per classe}
Lo strumento scelto per il calcolo del valore di questa metrica è PragmaDB, il quale permette di tracciare le classi ed associarle ad altre classi correlate e requisiti.
\paragraph{Parametri per metodo}
Lo strumento scelto per il calcolo del valore di questa metrica è PragmaDB,
\paragraph{Componenti integrate}
Lo strumento scelto per il calcolo del valore di questa metrica è PragmaDB,
\paragraph{Test di unità eseguiti}
Lo strumento scelto per il calcolo del valore di questa metrica è PragmaDB,
\paragraph{Test di integrazione eseguiti}
Lo strumento scelto per il calcolo del valore di questa metrica è PragmaDB,
\paragraph{Test di sistema eseguiti}
Lo strumento scelto per il calcolo del valore di questa metrica è PragmaDB,
\paragraph{Test di validazione eseguiti}
Lo strumento scelto per il calcolo del valore di questa metrica è PragmaDB,
\paragraph{Test superati}
Lo strumento scelto per il calcolo del valore di questa metrica è PragmaDB,
\paragraph{Completezza implementazione funzionale}
Lo strumento scelto per il calcolo del valore di questa metrica è PragmaDB,
\paragraph{Densità di failure}
Lo strumento scelto per il calcolo del valore di questa metrica è PragmaDB,