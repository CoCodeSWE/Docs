\subsubsection{Scopo}
Lo scopo di questo processo consiste nell'illustrazione di come deve essere redatta e mantenuta
la documentazione, durante il ciclo di vita del software.
\subsubsection{Aspettative}
Le aspettative della corretta implementazione di tale processo sono:
\begin{itemize}
	\item una chiara visione della documentazione prodotta durante il ciclo di vita
del software;
	\item una serie di norme per la stesura di documenti coerenti e validi;
	\item una documentazione formale e coerente.
\end{itemize}
\subsubsection{Descrizione}
In questo documento devono essere redatte tutte le norme e le convenzioni adottate dal gruppo,
in modo da produrre una documentazione valida e coerente.
\subsubsection{Procedure}
Per la stesura della documentazione si è utilizzato il linguaggio \LaTeX, si veda 4.1.11.
 \paragraph{Approvazione dei documenti}
La formalizzazione di un documento segue la seguente procedura:
\begin{enumerate}
	\item il documento viene redatto da coloro che sono incaricati della sua stesura ed eventuale correzione di errori;
	\item il \RESP{} di progetto assegna i \VERP al documento, i quali dovranno occuparsi di controllare la correttezza dello stesso;
	\item se i \VERP riscontrano degli errori si ritorna al punto 1, altrimenti il documento viene consegnato al \RESP ;
	\item il \RESP{} di progetto decide se approvare, e quindi formalizzare il documento, oppure se rifiutarlo comunicando la motivazione e le modifiche da apportare, tornando così al punto 1.
\end{enumerate}

\subsubsection{Template}
Per garantire omogeneità tra i documenti è stato creato un template \LaTeX, dove sono state definite tutte le regole di formattazione da applicare al documento. Questo permette a tutti i componenti del gruppo di concentrarsi solo nella stesura del contenuto, senza doversi preoccupare dell'aspetto. 
\subsubsection{Struttura dei documenti}
 \paragraph{Frontespizio} 
La prima pagina di ogni documento dovrà contenere:
\begin{itemize}
	\item logo del gruppo;
	\item nome del \gl{progetto};
	\item nome del documento e la relativa versione;
	\item sommario;
	\item data di redazione;
	\item nome e cognome dei redattori del documento;
	\item nome e cognome dei verificatori del documento;
	\item nome e cognome del responsabile per l'approvazione del documento;
	\item destinazione d'uso del documento;
	\item lista di distribuzione del documento.
\end{itemize}
 \paragraph{Diario delle modifiche}
 La seconda pagina dovrà contenere i diario delle modifiche di quel determinato documento.\\
 Il diario è costituito di una tabella ordinata in modo decrescente seconda la data di modifica e il numero di versione.\\
 Gli \gl{attributi} della tabella rappresentano:
 \begin{itemize}
 	\item numero di versione;
 	\item breve riepilogo delle modifiche apportate;
 	\item autore delle modifiche;
 	\item ruolo ricoperto dall'attore all'interno del progetto;
 	\item data di modifica.
 \end{itemize}
 \paragraph{Indice}
 In ogni documento è presente un indice delle sezioni, utile a fornire una visione macroscopica della struttura del documento. Sono previsti, se necessari, gli indici relativi alle tabelle e alle figure presenti nel documento in questo ordine.
 \paragraph{Contenuto principale}
 \paragraph{Note a piè di pagina}
 
\subsubsection{Versionamento}

\subsubsection{Norme tipografiche}
 \paragraph{Stile del testo} 
 \paragraph{Elenchi puntati}
 \paragraph{Formati comuni}
 \paragraph{Sigle}
 
\subsubsection{Elementi grafici}
 \paragraph{Tabelle} 
 \paragraph{Immagini}

\subsubsection{Classificazione dei documenti}
 \paragraph{Documenti informali}
 \paragraph{Documenti formali}
 \paragraph{Verbali}

\subsubsection{Strumenti}
 \paragraph{\LaTeX}
 \paragraph{TexMaker}