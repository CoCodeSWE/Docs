\subsubsection{Scopo}
Include le attività e i compiti svolti per produrre il prodotto.
\subsubsection{Aspettative}
Le aspettative della corretta implementazione del processo sono:
\begin{itemize}
		\item realizzare un prodotto finale conforme alle richieste del \gl{proponente} e che soddisfi le attività di \gl{validazione} e \gl{verifica};
		\item fissare gli obiettivi di sviluppo;
		\item fissare i vincoli tecnologici.
\end{itemize}

\subsubsection{Descrizione}
In accordo con lo \gl{standard} \iso{ISO/IEC 12207}, il processo di sviluppo è composto dalle attività di:
\begin{itemize}
		\item analisi dei requisiti;
		\item progettazione;
		\item codifica.
\end{itemize}

\subsubsection{Analisi dei requisiti}
 \paragraph{Scopo dell'attività}
  Individuare i requisiti del progetto dalle specifiche del \gl{capitolato} e tramite incontri con il pro-
  ponente. Tale attività produrrà un documento redatto dagli analisti, i quali avranno cura di elencare i casi d'uso e i requisiti. Tale documento permette di
 capire le scelte di progettazione effettuate.
 \paragraph{Aspettative dell'attività}
 L'attività fissa come scopo la creazione di un documento, il quale conterrà e rappresenterà i requisiti richiesti dal proponente.
 \paragraph{Descrizione dell'attività}
 bla bla... \ARdocRR
 Il tracciamento avviene tramite boh.
 \paragraph{Studio di fattibilità}
 Il \RESP \SFdocRR
 \paragraph{Casi d'uso}

 \paragraph{Codice identificativo dei casi d'uso}

 \paragraph{Requisiti}

 \paragraph{Codice identificativo dei requisiti}

 \paragraph{UML}

 \paragraph{Progettazione}

\subsubsection{Progettazione}
 \paragraph{Scopo dell'attività}
L'attività di progettazione definisce le linee essenziali della struttura del prodotto software in
funzione dei requisiti individuati dall'analisi. L'obiettivo del processo consiste nella stesura dei
documenti: Specifica Tecnica e Definizione di Prodotto.
 \paragraph{Aspettative dell'attività}
Il processo porta alla formazione dei documenti sopra citati, i quali garantiscono affidabilità e
coerenza.
 \paragraph{Descrizione dell'attività}
La progettazione deve rispettare tutti i vincoli e i requisiti concordati tra i componenti del gruppo
e i proponenti. I documenti derivati da questa attività sono:
\begin{itemize}
	\item \textbf{Specifica tecnica}: descrive la progettazione ad alto livello relativa all'architettura dell'applicazione
e dei singoli componenti. Il documento specifica i diagrammi UML ed i \gl{design
pattern} utilizzati per realizzare l'architettura definendo inoltre i test necessari alla verifica;
	\item \textbf{Definizione di Prodotto}: descrive in dettaglio la progettazione di \gl{sistema}, integrando
quanto scritto nella Specifica Tecnica. Il documento specifica i diagrammi UML e le
definizioni delle classi definendo inoltre i test necessari alla verifica.
\end{itemize}
 \paragraph{Specifica tecnica}
\begin{itemize} 
	\item \textbf{Diagrammi UML}:
	\begin{itemize}
	\item[--] Diagrammi delle classi
	\item[--] Diagrammi dei \gl{package}
	\item[--] Diagrammi di attività
	\item[--] Diagrammi di sequenza
	\end{itemize}
	\item \textbf{Design pattern}:\\	Devono essere descritti i design pattern utilizzati per realizzare l'architettura. Ogni design
pattern deve essere accompagnato da una descrizione ed un diagramma, che ne esponga il
significato e la struttura;
	\item \textbf{Tracciamento delle componenti}:
	\item \textbf{Test di integrazione}:\\	Devono essere definite delle classi di verifica, utili a verificare che ogni componente del
sistema funzioni nella maniera appropriata.
\end{itemize}
 \paragraph{Definizione di prodotto}
\begin{itemize}
	\item \textbf{Diagrammi UML}:
	\begin{itemize}
		\item[--] Diagrammi delle classi
		\item[--] Diagrammi di attività
		\item[--] Diagrammi di sequenza
	\end{itemize}
	\item \textbf{Definizioni delle classi}:\\Ogni classe progettata deve essere descritta in modo da spiegarne lo scopo e definirne le
funzionalità ad essa associate;
	\item \textbf{Tracciamento delle classi}:\\Ogni requisito deve essere tracciato, in modo da poter risalire alle classi ad esso associate; Trender????
	\item \textbf{Test di unità}:\\Devono essere definiti dei test di \gl{unità} utili a verificare che le componenti del sistema
funzionino nel modo previsto.
\end{itemize}

\subsubsection{Codifica}
 \paragraph{Scopo dell'attività}
 
 \paragraph{Aspettative dell'attività}

 \paragraph{Descrizione dell'attività}

 \paragraph{Stile}

 \paragraph{Versionamento}

 \paragraph{Ricorsione}

\subsubsection{Strumenti}
  \paragraph{Strumento uno}

 \paragraph{Strumento due}



  
