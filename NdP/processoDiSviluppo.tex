\subsubsection{Scopo}
Include le attività e i compiti svolti per produrre il prodotto.
\subsubsection{Aspettative}
Le aspettative della corretta implementazione del processo sono:
\begin{itemize}
		\item realizzare un prodotto finale conforme alle richieste del \gl{proponente} e che soddisfi le attività di \gl{validazione} e \gl{verifica};
		\item fissare gli obiettivi di sviluppo;
		\item fissare i vincoli tecnologici.
\end{itemize}

\subsubsection{Descrizione}
In accordo con lo \gl{standard} \iso{ISO/IEC 12207}, il processo di sviluppo è composto dalle attività di:
\begin{itemize}
		\item analisi dei requisiti;
		\item progettazione;
		\item codifica.
\end{itemize}

\subsubsection{Analisi dei requisiti}
 \paragraph{Scopo dell'attività}
  Individuare i requisiti del progetto dalle specifiche del \gl{capitolato} e tramite incontri con il pro-
  ponente. Tale attività produrra un documento redatto dagli analisti, i quali avranno cura di elencare i casi d'uso e i requisiti. Tale documento permette di
 capire le scelte di progettazione effettuate.
 \paragraph{Aspettative dell'attività}
 L'attività fissa come scopo la creazione di un documento, il quale conterrà e rappresenterà i requisiti richiesti dal proponente.
 \paragraph{Descrizione dell'attività}
 bla bla... \ARdocRR
 Il tracciamento avviene tramite boh.
 \paragraph{Studio di fattibilità}
 Il \RESP \SFdocRR
 \paragraph{Casi d'uso}

 \paragraph{Codice identificativo dei casi d'uso}

 \paragraph{Requisiti}

 \paragraph{Codice identificativo dei requisiti}

 \paragraph{UML}

 \paragraph{Progettazione}

\subsubsection{Progettazione}
 \paragraph{Scopo dell'attività}

 \paragraph{Aspettative dell'attività}

 \paragraph{Descrizione dell'attività}

 \paragraph{Specifica tecnina}

 \paragraph{Definizione di prodotto}

\subsubsection{Codifica}
 \paragraph{Scopo dell'attività}

 \paragraph{Aspettative dell'attività}

 \paragraph{Descrizione dell'attività}

 \paragraph{Stile}

 \paragraph{Versionamento}

 \paragraph{Ricorsione}

\subsubsection{Strumenti}
  \paragraph{Strumento uno}

 \paragraph{Strumento due}



  
