\subsubsection{Scopo}
Include le attività e i compiti svolti per produrre il prodotto.
\subsubsection{Aspettative}
Le aspettative della corretta implementazione del processo sono:
\begin{itemize}
		\item realizzare un prodotto finale conforme alle richieste del \gl{proponente} e che soddisfi le attività di \gl{validazione} e \gl{verifica};
		\item fissare gli obiettivi di sviluppo;
		\item fissare i vincoli tecnologici.
\end{itemize}

\subsubsection{Descrizione}
In accordo con lo \gl{standard} \iso{ISO/IEC 12207}, il processo di sviluppo è composto dalle attività di:
\begin{itemize}
		\item analisi dei requisiti;
		\item progettazione;
		\item codifica.
\end{itemize}

\subsubsection{Analisi dei requisiti}
 \paragraph{Scopo dell'attività}
  Individuare i requisiti del progetto dalle specifiche del \gl{capitolato} e tramite incontri con il pro-
  ponente. Tale attività produrra un documento redatto dagli analisti, i quali avranno cura di elencare i casi d'uso e i requisiti. Tale documento permette di
 capire le scelte di progettazione effettuate.
 \paragraph{Aspettative dell'attività}
 L'attività fissa come scopo la creazione di un documento, il quale conterrà e rappresenterà i requisiti richiesti dal proponente.
 \paragraph{Descrizione dell'attività}
 Tutti i requisiti analizzati, utilizzando le specifiche del capitolato e consultando i proponenti negli
incontri effettuati, vanno specificati nell’\ARdocRR. Per analizzare e trovare i
requisiti (si utilizza la tecnica dei casi d’uso). Il tracciamento dei requisiti avviene tramite ... 
 \paragraph{Studio di fattibilità}
 Il \RESP{} di progetto deve organizzare delle riunioni preventive, per permettere lo scambio
di opinioni tra i membri del gruppo sui capitolati proposti. Il documento prodotto da queste
riunioni è lo \SFdocRR , il quale viene realizzato dagli Analisti. Essi devono
descrivere i seguenti punti: 
\begin{itemize}
 \item \textbf{Dominio tecnologico e applicativo}: si dà una valutazione prendendo in   considerazione
 la conoscenza attuale delle tecnologie richieste dal capitolato in analisi da parte dei membri  
del gruppo;
 \item \textbf{Interesse strategico}: si valuta l’interesse strategico del gruppo di progetto in relazione
al capitolato in analisi;
 \item \textbf{Individuazione dei rischi}: si analizzano i possibili rischi a cui si può incorrere nel
capitolato in analisi.
\end{itemize}
 \paragraph{Casi d'uso}
 Ogni \gl{caso d'uso} è così composto:
 \begin{itemize}
  \item codice identificativo;
  \item titolo;
  \item diagramma \gl{UML};
  \item attori primari;
  \item attori secondari;
  \item scopo;
  \item descrizione;
  \item precondizione;
  \item scenario principale;
  \item scenari alternativi.
 \end{itemize}
 \paragraph{Codice identificativo dei casi d'uso}
da decidere
 \paragraph{Requisiti}
 Ogni requisito è così composto:
  \begin{itemize}
  \item codice identificativo;
  \item tipologia;
  \item descrizione;
  \item fonti.
 \end{itemize}
 \paragraph{Codice identificativo dei requisiti}
 da decidere
 \paragraph{UML}
 che versione di uml?
 \paragraph{Progettazione}

\subsubsection{Progettazione}
 \paragraph{Scopo dell'attività}

 \paragraph{Aspettative dell'attività}

 \paragraph{Descrizione dell'attività}

 \paragraph{Specifica tecnina}

 \paragraph{Definizione di prodotto}

\subsubsection{Codifica}
 \paragraph{Scopo dell'attività}
 Lo scopo dell'attività è l'implementazione del prodotto, concretizzando la soluzione tramite la codifica.  
 \paragraph{Aspettative dell'attività}
 L'aspettativa dell'attività è un prodotto corretto, ovvero stabile, affidabile, funzionale e che soddisfi i requisiti. 
 \paragraph{Descrizione dell'attività}
 L'attività deve rispettare i compiti e gli strumenti espressi nel \PPdocRR.
 \paragraph{Stile}
 Lo stile di codifica verrà trattato e descritto in versioni successive di questo documento.
 \paragraph{Versionamento}
 Lo stile di rappresentazione della versione del codice verr trattato e descritto in versioni successive di questo documento.
 \paragraph{Ricorsione}
 La \gl{ricorsione} va evitata. Se non risulta accettabile convertirla in \gl{iterazione}, bisogna fornirne la prova di terminazione e l'analisi del costo in termini di spazio.
\subsubsection{Strumenti}
  \paragraph{Strumento uno}

 \paragraph{Strumento due}



  
