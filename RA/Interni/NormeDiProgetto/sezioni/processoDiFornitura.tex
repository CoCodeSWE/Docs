\subsubsection{Studio di fattibilità}\label{PF}
 Il \RESP{} di \gl{progetto} deve organizzare delle riunioni preventive, per permettere lo scambio
di opinioni tra i membri del gruppo sui capitolati proposti. Il documento \gl{prodotto} da queste
riunioni è lo \SFdoc , il quale viene realizzato dagli \ANP{}. Essi devono
descrivere i seguenti punti:
\begin{itemize}
 \item \textbf{Dominio tecnologico e applicativo}: si dà una valutazione prendendo in   considerazione
 la conoscenza attuale delle tecnologie richieste dal \gl{capitolato} in analisi da parte dei membri
del gruppo;
 \item \textbf{Interesse strategico}: si valuta l'interesse strategico del gruppo di progetto in relazione
al capitolato in analisi;
 \item \textbf{Individuazione dei rischi}: si analizzano i possibili rischi in cui si può incorrere nel
capitolato in analisi.
\end{itemize}