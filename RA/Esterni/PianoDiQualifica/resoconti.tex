\documentclass[PdQ.tex]{subfiles}

\begin{document}


\section{Resoconto delle attività di verifica}
All'interno di questa sezione sono riportati gli esiti di tutte le attività di \gl{verifica} effettuate. Ove necessario sono state tratte conclusioni sui risultati e su come essi possano essere migliorati.

\subsection{Qualità di processo}
	\subsubsection{Miglioramento continuo tramite CMM}
		\paragraph{Periodo RR}
		Per rendere le performance dei processi costantemente migliorabili e perseguire gli obiettivi quantitativi di miglioramento viene utilizzato il modello Capability Maturity Model (\gl{CMM}).\\
		All'inizio del periodo i processi si trovavano al livello 1 della scala CMM. In seguito, grazie alla stesura del documento \NPdocRR{} sono state definite regole per ogni tipo di documentazione, strumenti da utilizzare e procedure da seguire. Questo ha permesso un maggiore controllo dei processi, che hanno ottenuto la ripetibilità, proprietà che caratterizza il livello 2 della scala CMM. Si può quindi affermare che i processi hanno raggiunto tale livello. Non si può ancora affermare di aver raggiunto il livello 3 del modello perchè al processo manca ancora la sua caratteristica principale, la proattività.\\

\paragraph{Periodo RP}

		All'inizio del periodo i processi si trovavano al livello 2 della scala CMM. In seguito, grazie alla riorganizzazione del documento \NPdocRP{} e alla maggiore esperienza dei membri del gruppo i processi e la loro organizzazione sono migliorati. Questo ci ha permesso di raggiungere il livello 3 della scala CMM.

\paragraph{Periodo RQ}
	All'inizio del periodo i processi si trovavano al livello 3 della scala CMM. Nonostante la definizione di nuove norme e l'aggiunta di nuovi strumenti, il gruppo non è riuscito a raggiungere il livello 4 della scala CMM, e rimane quindi al livello 3.
\subsubsection{Stato di implementazione dei test}
\paragraph{Periodo RQ}
	\begin{table}[h]
				\centering
				\begin{tabular}{l c c c}
					\hline
					\rule[-0.3cm]{0cm}{0.8cm}
					\textbf{Documento} & \textbf{Test implementati} & \textbf{Test definiti} & \textbf{Percentuale}\\
					\hline
					\rule[0cm]{0cm}{0.4cm}
					Test di Unità & 219 & 223 & 98\% \\
					\rule[0cm]{0cm}{0.4cm}
					Test di integrazione & 16 & 18 & 88.8\% \\
					\rule[0cm]{0cm}{0.4cm}
					Test di sistema & 0 & 10 & 0\% \\
					\rule[0cm]{0cm}{0.4cm}
					Totale & 235 & 251 & 92\% \\
					\hline
				\end{tabular}
				\caption{Numero di test implementati nel periodo RQ}
			\end{table}
			\clearpage
\paragraph{Periodo RA}
	\begin{table}[h]
				\centering
				\begin{tabular}{l c c c}
					\hline
					\rule[-0.3cm]{0cm}{0.8cm}
					\textbf{Documento} & \textbf{Test implementati} & \textbf{Test definiti} & \textbf{Percentuale}\\
					\hline
					\rule[0cm]{0cm}{0.4cm}
					Test di Unità(TO DO) & 219 & 223 & 98\% \\
					\rule[0cm]{0cm}{0.4cm}
					Test di integrazione (TO DO) & 16 & 18 & 88.8\% \\
					\rule[0cm]{0cm}{0.4cm}
					Test di sistema (TO DO) & 0 & 10 & 0\% \\
					\rule[0cm]{0cm}{0.4cm}
					Totale (TO DO) & 235 & 251 & 92\% \\
					\hline
				\end{tabular}
				\caption{Numero di test implementati nel periodo RA}
			\end{table}
\subsubsection{Soddisfacimento obiettivi di qualità}
			Di seguito sono riportati i valori ottenuti utilizzando le metriche definite sui seguenti obiettivi di qualità, i cui esiti sono suddivisi in \textcolor{red}{non accettabile}, \textcolor{orange}{accettabile} e \textcolor{green}{ottimale}:
			\begin{table}[h]
				\centering
				\begin{tabular}{l c c c}
					\hline
					\rule[-0.3cm]{0cm}{0.8cm}
					\textbf{Obiettivo} & \textbf{RR} & \textbf{RP} & \textbf{RQ} & \textbf{RA}\\
					\hline
					\rule[0cm]{0cm}{0.4cm}
					Disponibilità PragmaDB - OPC1 & & \textcolor{green}{99\%} & \textcolor{green}{97\%} & x\\
					\rule[0cm]{0cm}{0.4cm}
					Rispetto dei tempi - OPC2 & \textcolor{green}{10\%} & \textcolor{green}{2\%} & \textcolor{green}{5\%} & x \\
					\rule[0cm]{0cm}{0.4cm}
					Rispetto dei costi - OPC3 & \textcolor{orange}{-5\%} & \textcolor{green}{13\%} & \textcolor{green}{0\%} & x\\
					\rule[0cm]{0cm}{0.4cm}
					Rischi non preventivati - OPC4 & & \textcolor{green}{0} & \textcolor{green}{0} & x\\
					\rule[0cm]{0cm}{0.4cm}
					Numero di metodi per classe - OPC10 & & \textcolor{green}{2.13} & \textcolor{green}{2.13} & x \\
					\rule[0cm]{0cm}{0.4cm}
					Numero di parametri per metodo - OPC11 & & \textcolor{green}{1.03} & \textcolor{green}{1.03} & x\\
					\rule[0cm]{0cm}{0.4cm}
					Requisiti obbligatori soddisfatti - OPC5 & & & \textcolor{green}{100\%} & x \\
					\rule[0cm]{0cm}{0.4cm}
					Requisiti desiderabili soddisfatti - OPC6 & & & \textcolor{red}{66\%} & x \\
					\rule[0cm]{0cm}{0.4cm}
					Requisiti facoltativi soddisfatti - OPC7 & & & \textcolor{orange}{1\%} & x \\
					\rule[0cm]{0cm}{0.4cm}
					Complessità ciclomatica - OPC12 & & & \textcolor{orange}{11.5} & x \\
					\rule[0cm]{0cm}{0.4cm}
					Livelli di annidamento - OPC13 & & & \textcolor{orange}{5.8} & x\\
					\rule[0cm]{0cm}{0.4cm}
					Linee di commento per linee di codice - OPC14 & & & \textcolor{orange}{12\%} & x \\
					\rule[0cm]{0cm}{0.4cm}
					Manutenibilità - OPC15 & & & \textcolor{green}{32} & x \\
					\rule[0cm]{0cm}{0.4cm}
					Componenti integrate - OPC16 & & & \textcolor{red}{95\%} & x \\
					\rule[0cm]{0cm}{0.4cm}
					Test di unità eseguiti - OPC17 & & & \textcolor{orange}{99\%} & x \\
					\rule[0cm]{0cm}{0.4cm}
					Test di integrazione eseguiti - OPC18 & & & \textcolor{green}{89\%} & x \\
					\rule[0cm]{0cm}{0.4cm}
					Test di \gl{sistema} eseguiti - OPC19 & & & \textcolor{red}{0\%} & x \\
					\rule[0cm]{0cm}{0.4cm}
					Test di \gl{validazione} eseguiti - OPC20 & & & \textcolor{red}{0\%} & x \\
					\rule[0cm]{0cm}{0.4cm}
					Test superati - OPC21 & & & \textcolor{green}{100\%} & x \\
					\rule[0cm]{0cm}{0.4cm}
					Branch coverage - OPC22 & & & \textcolor{orange}{76.08\%} & x \\
					\rule[0cm]{0cm}{0.4cm}
					Function coverage - OPC23 & & & \textcolor{green}{84.13\%} & x \\
					\rule[0cm]{0cm}{0.4cm}
					Statement coverage - OPC24 & & & \textcolor{green}{83.19\%} & x \\

					\hline
				\end{tabular}
				\caption{Esiti del calcolo delle metriche sui processi}
			\end{table}
		\newpage
		\paragraph{Structural Fan-In - OPC8 e Structural Fan-Out - OPC9}
		\subparagraph{Structural Fan-In}
		\normalsize
\begin{longtable}{|c|c|c|}
\hline Classe & SFIN & Esito \\
\hline Client::Recorder::RecorderWorker & 1 & Accettabile \\
\hline Client::TTS::Player & 1 & Accettabile \\
\hline Client::ApplicationManager::Application & 1 & Accettabile \\
\hline Client::ApplicationManager::State & 1 & Accettabile \\
\hline Client::TTS::PlayerObserver & 1 & Accettabile \\
\hline Client::ApplicationManager::ApplicationManagerObserver & 2 & Accettabile \\
\hline Client::Logic::DataArrivedSubject & 1 & Accettabile \\
\hline Client::Logic::LogicObserver & 2 & Accettabile \\
\hline Client::ApplicationManager::ApplicationLocalRegistry & 1 & Accettabile \\
\hline Client::ApplicationManager::$<$$<$interface$>$$>$ ApplicationRegistryClient & 1 & Accettabile \\
\hline Client::ApplicationManager::ApplicationPackage & 3 & Accettabile \\
\hline Client::TTS::TTSConfig & 1 & Accettabile \\
\hline Client::Utility::BoolObservable & 1 & Accettabile \\
\hline Client::Utility::BoolSubject & 1 & Accettabile \\
\hline Client::Utility::BoolObserver & 3 & Accettabile \\
\hline Client::Logic::HttpPromise & 1 & Accettabile \\
\hline Client::Logic::HttpError & 2 & Accettabile \\
\hline Client::Recorder::RecorderWorkerConfig & 2 & Accettabile \\
\hline Client::Recorder::RecorderMsg & 2 & Accettabile \\
\hline Client::Recorder::RecorderWorkerMsg & 2 & Accettabile \\
\hline Client::Logic::DataArrivedObservable & 1 & Accettabile \\
\hline Back-end::VirtualAssistant::$<$$<$interface$>$$>$ AgentsDAO & 1 & Accettabile \\
\hline Back-end::VirtualAssistant::Agent & 4 & Ottimale \\
\hline Back-end::VirtualAssistant::$<$$<$interface$>$$>$ VAModule & 1 & Accettabile \\
\hline Libs::ResponseBody & 1 & Accettabile \\
\hline Back-end::VirtualAssistant::VAQuery & 2 & Accettabile \\
\hline Back-end::VirtualAssistant::Context & 1 & Accettabile \\
\hline Back-end::VirtualAssistant::ButtonObject & 1 & Accettabile \\
\hline Back-end::VirtualAssistant::Metadata & 1 & Accettabile \\
\hline Back-end::VirtualAssistant::MsgObject & 1 & Accettabile \\
\hline Back-end::VirtualAssistant::Fulfillment & 1 & Accettabile \\
\hline Back-end::Notifications::NotificationChannel & 1 & Accettabile \\
\hline Back-end::Users::User & 4 & Ottimale \\
\hline Back-end::Notifications::NotificationMessage & 1 & Accettabile \\
\hline Back-end::Notifications::Attachment & 1 & Accettabile \\
\hline Back-end::Notifications::Action & 1 & Accettabile \\
\hline Back-end::Users::$<$$<$interface$>$$>$ UsersDAO & 3 & Accettabile \\
\hline Back-end::Rules::RuleTarget & 1 & Accettabile \\
\hline Back-end::Users::SRUser & 1 & Accettabile \\
\hline Back-end::Rules::RuleTaskInstance & 1 & Accettabile \\
\hline Back-end::Rules::Rule & 4 & Ottimale \\
\hline Back-end::Rules::$<$$<$interface$>$$>$ RulesDAO & 2 & Accettabile \\
\hline Back-end::Rules::Task & 3 & Accettabile \\
\hline Back-end::Rules::$<$$<$interface$>$$>$ TasksDAO & 1 & Accettabile \\
\hline Back-end::APIGateway::Enrollment & 1 & Accettabile \\
\hline Back-end::Guests::$<$$<$interface$>$$>$ GuestsDAO & 2 & Accettabile \\
\hline Back-end::Conversations::$<$$<$interface$>$$>$ ConversationsDAO & 2 & Accettabile \\
\hline Back-end::Utility::LambdaContext & 7 & Ottimale \\
\hline Back-end::Utility::LambdaResponse & 1 & Accettabile \\
\hline Back-end::Guests::Guest & 3 & Accettabile \\
\hline Back-end::Conversations::Conversation & 3 & Accettabile \\
\hline Back-end::Conversations::ConversationMsg & 3 & Accettabile \\
\hline Back-end::Events::SNSMessage & 1 & Accettabile \\
\hline Libs::VARequestAPIBody & 1 & Accettabile \\
\hline Back-end::STT::$<$$<$interface$>$$>$ STTModule & 1 & Accettabile \\
\hline Back-end::APIGateway::VocalLoginModuleConfig & 1 & Accettabile \\
\hline Back-end::Utility::LambdaEvent & 7 & Ottimale \\
\hline Back-end::Utility::LambdaIdEvent & 2 & Accettabile \\
\hline Back-end::Utility::PathIdParam & 1 & Accettabile \\
\hline Back-end::Notifications::ConfirmationFields & 1 & Accettabile \\
\hline Back-end::VirtualAssistant::VAEventObject & 1 & Accettabile \\
\hline Back-end::SNSEvent & 1 & Accettabile \\
\hline Back-end::Events::SNSRecord & 1 & Accettabile \\
\hline Back-end::Users::UserObservable & 2 & Accettabile \\
\hline Back-end::Users::UserObserver & 1 & Accettabile \\
\hline Client::Recorder::RecorderConfig & 1 & Accettabile \\
\hline Back-end::Conversations::ConversationObservable & 2 & Accettabile \\
\hline Back-end::Conversations::ConversationObserver & 1 & Accettabile \\
\hline Back-end::Guests::GuestObserver & 1 & Accettabile \\
\hline Back-end::Guests::GuestObservable & 2 & Accettabile \\
\hline Back-end::Rules::TaskObserver & 1 & Accettabile \\
\hline Back-end::Rules::TaskObservable & 2 & Accettabile \\
\hline Back-end::Rules::RuleObserver & 1 & Accettabile \\
\hline Back-end::Rules::RuleObservable & 2 & Accettabile \\
\hline Back-end::VirtualAssistant::AgentObserver & 1 & Accettabile \\
\hline Back-end::VirtualAssistant::AgentObservable & 2 & Accettabile \\
\hline Libs::ErrorObserver & 3 & Accettabile \\
\hline Libs::ErrorObservable & 14 & Ottimale \\
\hline Back-end::Members::Member & 3 & Accettabile \\
\hline Back-end::Members::MemberObservable & 1 & Accettabile \\
\hline Back-end::Members::MemberObserver & 1 & Accettabile \\
\hline Back-end::Users::$<$$<$interface$>$$>$VocalLoginModule & 1 & Accettabile \\
\hline Client::ConversationApp::ConversationDispatcher & 1 & Accettabile \\
\hline Client::ConversationApp::ConversationView & 1 & Accettabile \\
\hline Client::ConversationApp::MessageStore & 2 & Accettabile \\
\hline Client::ConversationApp::ConversationAction & 5 & Ottimale \\
\hline Libs::ErrorSubject & 1 & Accettabile \\
\hline Client::ConversationApp::ConversationActionObserver & 3 & Accettabile \\
\hline Client::ConversationApp::ConversationActionSubject & 1 & Accettabile \\
\hline Client::IndexView & 0 & Accettabile \\
\hline Client::ApplicationManager::Manager & 0 & Accettabile \\
\hline Client::ApplicationManager::ApplicationRegistryLocalClient & 0 & Accettabile \\
\hline Client::Logic::Logic & 0 & Accettabile \\
\hline Back-end::ConversationWebhookService & 0 & Accettabile \\
\hline Back-end::AdministrationWebhookService & 0 & Accettabile \\
\hline Back-end::APIGateway::VocalAPI & 0 & Accettabile \\
\hline Back-end::Users::UsersService & 0 & Accettabile \\
\hline Back-end::Users::VocalLoginMicrosoftModule & 0 & Accettabile \\
\hline Back-end::Users::UsersDAODynamoDB & 0 & Accettabile \\
\hline Back-end::VirtualAssistant::ApiAiVAAdapter & 0 & Accettabile \\
\hline Back-end::VirtualAssistant::VAService & 0 & Accettabile \\
\hline Back-end::VirtualAssistant::$<$$<$interface$>$$>$ WebhookService & 0 & Accettabile \\
\hline Back-end::VirtualAssistant::AgentsDAODynamoDB & 0 & Accettabile \\
\hline Back-end::Rules::RulesService & 0 & Accettabile \\
\hline Back-end::Rules::RulesDAODynamoDB & 0 & Accettabile \\
\hline Back-end::Rules::TasksDAODynamoDB & 0 & Accettabile \\
\hline Back-end::Notifications::NotificationService & 0 & Accettabile \\
\hline Client::Recorder::Recorder & 0 & Accettabile \\
\hline Client::Recorder::SpeechEndSubject & 0 & Accettabile \\
\hline Client::Recorder::SpeechEndObservable & 0 & Accettabile \\
\hline RxJS 5::Subject & 0 & Accettabile \\
\hline RxJS 5::Observable & 0 & Accettabile \\
\hline RxJS 5::Observer & 0 & Accettabile \\
\hline Back-end::Utility::StatusObject & 0 & Accettabile \\
\hline Back-end::Utility::ProcessingResult & 0 & Accettabile \\
\hline Back-end::STT::STTWatsonAdapter & 0 & Accettabile \\
\hline Slack::WebClient & 0 & Accettabile \\
\hline Back-end::Members::$<$$<$interface$>$$>$ MembersDAO & 0 & Accettabile \\
\hline Back-end::Members::MembersDAOSlack & 0 & Accettabile \\
\hline Back-end::Events::VAMessageListener & 0 & Accettabile \\
\hline Back-end::Guests::GuestsDAODynamoDB & 0 & Accettabile \\
\hline Back-end::Conversations::ConversationsDAODynamoDB & 0 & Accettabile \\
\hline React::Component {abstract} & 0 & Accettabile \\
\hline Client::ConversationApp::ConversationApp & 0 & Accettabile \\
\hline Client::ConversationApp::ConversationActionObservable & 0 & Accettabile \\
\hline Libs::ObserverAdapter & 0 & Accettabile \\
\hline Libs::VAResponse & 0 & Accettabile \\
\hline Libs::Exception & 0 & Accettabile \\
\hline IBMWatson::STTParams & 0 & Accettabile \\
\hline IBMWatson::SpeechToTextV1 & 0 & Accettabile \\
\hline WebSpeechApi::SpeechSynthesisVoice & 0 & Accettabile \\
\hline WebSpeechApi::SpeechSyntesis & 0 & Accettabile \\
\hline \end{longtable}
Nella seguente tabella sono riportati i valori ottenuti calcolando lo SFIN, i quali indicano il numero di classi che superano un certo esito.
\begin{table}[h]
	\centering
	\begin{tabular}{l r}
		\hline
		\rule[-0.3cm]{0cm}{0.8cm}
		\textbf{Esito} & \textbf{Numero} \\
		\hline
		\rule[0cm]{0cm}{0.4cm}
		Non accettabile & 0 \\
		\rule[0cm]{0cm}{0.4cm}
		Accettabile & 125 \\
		\rule[0cm]{0cm}{0.4cm}
		Ottimale & 7 \\
		\hline
  \end{tabular}
	\caption{Esiti SFIN}
\end{table}

		\subparagraph{Structural Fan-Out}
		\normalsize
\begin{longtable}{|c|c|c|}
\hline Classe & SFIN & Esito \\
\hline Client::Recorder::Recorder & 5 & Accettabile \\
\hline Client::Recorder::RecorderWorker & 3 & Ottimale \\
\hline Client::TTS::Player & 3 & Ottimale \\
\hline Client::ApplicationManager::Application & 1 & Ottimale \\
\hline Client::ApplicationManager::State & 1 & Ottimale \\
\hline Client::ApplicationManager::Manager & 3 & Ottimale \\
\hline Client::TTS::Player\gl{Observer} & 1 & Ottimale \\
\hline Client::Logic::Logic & 5 & Accettabile \\
\hline Client::ApplicationManager::ApplicationLocalRegistry & 1 & Ottimale \\
\hline Client::ApplicationManager::$<$$<$interface$>$$>$ ApplicationRegistryClient & 1 & Ottimale \\
\hline Client::ApplicationManager::ApplicationRegistryLocalClient & 1 & Ottimale \\
\hline Client::Utility::BoolObservable & 1 & Ottimale \\
\hline Client::Utility::BoolSubject & 1 & Ottimale \\
\hline Client::Logic::HttpPromise & 1 & Ottimale \\
\hline Client::Recorder::RecorderMsg & 1 & Ottimale \\
\hline Client::Logic::DataArrivedObservable & 2 & Ottimale \\
\hline Client::Recorder::SpeechEndObservable & 1 & Ottimale \\
\hline Back-end::VirtualAssistant::$<$$<$interface$>$$>$ AgentsDAO & 3 & Ottimale \\
\hline Back-end::VirtualAssistant::$<$$<$interface$>$$>$ VAModule & 1 & Ottimale \\
\hline Back-end::VirtualAssistant::ApiAiVAAdapter & 2 & Ottimale \\
\hline Libs::VAResponse & 1 & Ottimale \\
\hline Back-end::VirtualAssistant::VAQuery & 1 & Ottimale \\
\hline Back-end::VirtualAssistant::VAService & 4 & Accettabile \\
\hline Back-end::VirtualAssistant::$<$$<$interface$>$$>$ WebhookService & 1 & Ottimale \\
\hline Back-end::Utility::ProcessingResult & 3 & Ottimale \\
\hline Back-end::VirtualAssistant::MsgObject & 1 & Ottimale \\
\hline Back-end::VirtualAssistant::Fulfillment & 1 & Ottimale \\
\hline Back-end::Notifications::NotificationService & 4 & Accettabile \\
\hline Back-end::Notifications::NotificationMessage & 1 & Ottimale \\
\hline Back-end::Notifications::Attachment & 1 & Ottimale \\
\hline Back-end::Notifications::Action & 1 & Ottimale \\
\hline Libs::Exception & 1 & Ottimale \\
\hline Back-end::Users::$<$$<$interface$>$$>$ UsersDAO & 3 & Ottimale \\
\hline Back-end::\gl{Rule}s::Rule & 2 & Ottimale \\
\hline Back-end::Users::UsersService & 4 & Accettabile \\
\hline Back-end::Users::VocalLoginMicrosoftModule & 1 & Ottimale \\
\hline Back-end::Rules::$<$$<$interface$>$$>$ RulesDAO & 3 & Ottimale \\
\hline Back-end::APIGateway::VocalAPI & 8 & Accettabile \\
\hline Back-end::Rules::$<$$<$interface$>$$>$ \gl{Task}sDAO & 3 & Ottimale \\
\hline Back-end::Rules::RulesService & 5 & Accettabile \\
\hline Back-end::Events::VAMessageListener & 3 & Ottimale \\
\hline Client::ConversationApp::ConversationApp & 4 & Accettabile \\
\hline Back-end::Guests::$<$$<$interface$>$$>$ GuestsDAO & 3 & Ottimale \\
\hline Back-end::ConversationWebhookService & 5 & Accettabile \\
\hline Back-end::Conversations::$<$$<$interface$>$$>$ ConversationsDAO & 4 & Accettabile \\
\hline Back-end::Utility::LambdaContext & 1 & Ottimale \\
\hline Back-end::Conversations::Conversation & 1 & Ottimale \\
\hline Back-end::AdministrationWebhookService & 3 & Ottimale \\
\hline Back-end::Utility::LambdaIdEvent & 1 & Ottimale \\
\hline Back-end::Users::UsersDAODynamoDB & 3 & Ottimale \\
\hline Back-end::SNSEvent & 1 & Ottimale \\
\hline Back-end::Events::SNSRecord & 1 & Ottimale \\
\hline Back-end::Users::UserObservable & 1 & Ottimale \\
\hline Back-end::Users::UserObserver & 1 & Ottimale \\
\hline Back-end::Conversations::ConversationsDAODynamoDB & 4 & Accettabile \\
\hline Back-end::Rules::RulesDAODynamoDB & 3 & Ottimale \\
\hline Back-end::Guests::GuestsDAODynamoDB & 3 & Ottimale \\
\hline Back-end::Rules::TasksDAODynamoDB & 3 & Ottimale \\
\hline Back-end::VirtualAssistant::AgentsDAODynamoDB & 3 & Ottimale \\
\hline Back-end::Conversations::ConversationObservable & 1 & Ottimale \\
\hline Back-end::Conversations::ConversationObserver & 1 & Ottimale \\
\hline Back-end::Guests::GuestObserver & 1 & Ottimale \\
\hline Back-end::Guests::GuestObservable & 1 & Ottimale \\
\hline Back-end::Rules::TaskObserver & 1 & Ottimale \\
\hline Back-end::Rules::TaskObservable & 1 & Ottimale \\
\hline Back-end::Rules::RuleObserver & 1 & Ottimale \\
\hline Back-end::Rules::RuleObservable & 1 & Ottimale \\
\hline Back-end::VirtualAssistant::AgentObserver & 1 & Ottimale \\
\hline Back-end::VirtualAssistant::AgentObservable & 1 & Ottimale \\
\hline Libs::ErrorObservable & 1 & Ottimale \\
\hline Back-end::Members::$<$$<$interface$>$$>$ MembersDAO & 3 & Ottimale \\
\hline Back-end::Members::MembersDAO\gl{Slack} & 2 & Ottimale \\
\hline Back-end::Members::MemberObservable & 1 & Ottimale \\
\hline Back-end::Members::MemberObserver & 1 & Ottimale \\
\hline Back-end::Users::$<$$<$interface$>$$>$VocalLoginModule & 1 & Ottimale \\
\hline Client::ConversationApp::ConversationDispatcher & 2 & Ottimale \\
\hline Client::ConversationApp::ConversationView & 1 & Ottimale \\
\hline Client::ConversationApp::MessageStore & 2 & Ottimale \\
\hline Libs::ErrorSubject & 1 & Ottimale \\
\hline Client::ConversationApp::ConversationActionObserver & 1 & Ottimale \\
\hline Client::ConversationApp::ConversationActionObservable & 2 & Ottimale \\
\hline Client::ConversationApp::ConversationActionSubject & 2 & Ottimale \\
\hline Client::IndexView & 0 & Ottimale \\
\hline Client::ApplicationManager::ApplicationManagerObserver & 0 & Ottimale \\
\hline Client::ApplicationManager::ApplicationPackage & 0 & Ottimale \\
\hline Client::Logic::DataArrivedSubject & 0 & Ottimale \\
\hline Client::Logic::LogicObserver & 0 & Ottimale \\
\hline Client::Logic::HttpError & 0 & Ottimale \\
\hline Back-end::APIGateway::\gl{Enrollment} & 0 & Ottimale \\
\hline Back-end::APIGateway::VocalLoginModuleConfig & 0 & Ottimale \\
\hline Back-end::Users::User & 0 & Ottimale \\
\hline Back-end::Users::SRUser & 0 & Ottimale \\
\hline Back-end::VirtualAssistant::Agent & 0 & Ottimale \\
\hline Back-end::VirtualAssistant::Context & 0 & Ottimale \\
\hline Back-end::VirtualAssistant::ButtonObject & 0 & Ottimale \\
\hline Back-end::VirtualAssistant::Metadata & 0 & Ottimale \\
\hline Back-end::VirtualAssistant::VAEventObject & 0 & Ottimale \\
\hline Back-end::Rules::RuleTarget & 0 & Ottimale \\
\hline Back-end::Rules::RuleTaskInstance & 0 & Ottimale \\
\hline Back-end::Rules::Task & 0 & Ottimale \\
\hline Back-end::Notifications::NotificationChannel & 0 & Ottimale \\
\hline Back-end::Notifications::ConfirmationFields & 0 & Ottimale \\
\hline Client::TTS::TTSConfig & 0 & Ottimale \\
\hline Client::Recorder::SpeechEndSubject & 0 & Ottimale \\
\hline Client::Recorder::RecorderWorkerConfig & 0 & Ottimale \\
\hline Client::Recorder::RecorderWorkerMsg & 0 & Ottimale \\
\hline Client::Recorder::RecorderConfig & 0 & Ottimale \\
\hline RxJS 5::Subject & 0 & Ottimale \\
\hline RxJS 5::Observable & 0 & Ottimale \\
\hline RxJS 5::Observer & 0 & Ottimale \\
\hline Client::Utility::BoolObserver & 0 & Ottimale \\
\hline Back-end::Utility::StatusObject & 0 & Ottimale \\
\hline Back-end::Utility::LambdaResponse & 0 & Ottimale \\
\hline Back-end::Utility::LambdaEvent & 0 & Ottimale \\
\hline Back-end::Utility::PathIdParam & 0 & Ottimale \\
\hline Back-end::STT::STTWatsonAdapter & 0 & Ottimale \\
\hline Back-end::STT::$<$$<$interface$>$$>$ STTModule & 0 & Ottimale \\
\hline Slack::WebClient & 0 & Ottimale \\
\hline Back-end::Members::Member & 0 & Ottimale \\
\hline Back-end::Events::SNSMessage & 0 & Ottimale \\
\hline Back-end::Guests::Guest & 0 & Ottimale \\
\hline Back-end::Conversations::ConversationMsg & 0 & Ottimale \\
\hline React::Component {abstract} & 0 & Ottimale \\
\hline Client::ConversationApp::ConversationAction & 0 & Ottimale \\
\hline Libs::ObserverAdapter & 0 & Ottimale \\
\hline Libs::ResponseBody & 0 & Ottimale \\
\hline Libs::VARequestAPIBody & 0 & Ottimale \\
\hline Libs::ErrorObserver & 0 & Ottimale \\
\hline IBMWatson::STTParams & 0 & Ottimale \\
\hline IBMWatson::SpeechToTextV1 & 0 & Ottimale \\
\hline WebSpeechApi::SpeechSynthesisVoice & 0 & Ottimale \\
\hline WebSpeechApi::SpeechSyntesis & 0 & Ottimale \\
\hline \end{longtable}
\newpage
Nella seguente tabella sono riportati i valori ottenuti calcolando lo SFOUT, i quali indicano il numero di classi che superano un certo esito.
\begin{table}[h]

\centering
\begin{tabular}{l r}
	\hline
	\rule[-0.3cm]{0cm}{0.8cm}
	\textbf{Esito} & \textbf{Numero} \\
	\hline
	\rule[0cm]{0cm}{0.4cm}
	Non accettabile & 0 \\
	\rule[0cm]{0cm}{0.4cm}
	Accettabile & 11 \\
	\rule[0cm]{0cm}{0.4cm}
	Ottimale & 121 \\
	\hline
\end{tabular}
\caption{Esiti SFOUT}
\end{table}

\newpage
\subsection{Qualità di prodotto}
	\subsubsection{Documenti}
		\paragraph{Leggibilità e comprensibilità - OPDD1}
				Di seguito sono riportati i valori ottenuti calcolando l'indice Gulpease sui documenti, i cui esiti sono suddivisi in \textcolor{red}{non accettabile}, \textcolor{orange}{accettabile} e \textcolor{green}{ottimale}:
				\begin{table}[h]
				\centering
				\begin{tabular}{l c c c}
					\hline
					\rule[-0.3cm]{0cm}{0.8cm}
					\textbf{Documento} & \textbf{RR} & \textbf{RP} & \textbf{RQ} & \textbf{RA}\\
					\hline
					\rule[0cm]{0cm}{0.4cm}
					\textit{Piano di Progetto} & \textcolor{orange}{49} & \textcolor{orange}{50} & \textcolor{orange}{51} & x \\
					\rule[0cm]{0cm}{0.4cm}
					\textit{Norme di Progetto}& \textcolor{orange}{58} & \textcolor{green}{60} & \textcolor{green}{62} & x \\
					\rule[0cm]{0cm}{0.4cm}
					\textit{Analisi dei Requisiti} & \textcolor{green}{66} & \textcolor{green}{65} & \textcolor{green}{65} & x \\
					\rule[0cm]{0cm}{0.4cm}
					\textit{Piano di Qualifica} & \textcolor{orange}{54} & \textcolor{green}{63} & \textcolor{green}{61} & x \\
					\rule[0cm]{0cm}{0.4cm}
					\textit{Glossario} & \textcolor{orange}{50} & \textcolor{orange}{48} & \textcolor{orange}{51} & x\\
					\rule[0cm]{0cm}{0.4cm}
					\textit{Definizione di Prodotto} & & \textcolor{green}{61} & \textcolor{green}{62} & x \\
					\rule[0cm]{0cm}{0.4cm}
					\textit{Manuale Utente} & & & \textcolor{green}{60} & x \\
					\rule[0cm]{0cm}{0.4cm}
					\textit{Analisi degli SDK} & \textcolor{green}{67} & & & \\
					\rule[0cm]{0cm}{0.4cm}
					\textit{Verbale esterno 2016-12-17} & \textcolor{green}{66} & & &\\
					\rule[0cm]{0cm}{0.4cm}
					\textit{Verbale interno 2016-12-10} & \textcolor{green}{61} & & &\\
					\rule[0cm]{0cm}{0.4cm}
					\textit{Verbale interno 2016-12-19} & \textcolor{green}{62} & & &\\
					\rule[0cm]{0cm}{0.4cm}
					\textit{Verbale esterno 2017-02-07} & & \textcolor{orange}{56} & & \\
					\rule[0cm]{0cm}{0.4cm}
					\textit{Verbale esterno 2017-02-28} & & \textcolor{green}{60} & &\\
					\rule[0cm]{0cm}{0.4cm}
					\textit{Verbale interno 2017-01-24} & & \textcolor{orange}{58} & &\\
					\rule[0cm]{0cm}{0.4cm}
					\textit{Verbale interno 2017-03-05} & & \textcolor{orange}{54} & &\\
					\rule[0cm]{0cm}{0.4cm}
					\textit{Verbale esterno 2017-04-20} & & & \textcolor{orange}{56} & x\\
					\rule[0cm]{0cm}{0.4cm}
					\textit{Verbale interno 2017-04-18} & & & \textcolor{orange}{58} & x\\
					\rule[0cm]{0cm}{0.4cm}
					\textit{Altri verbali} & & & & x\\
					\hline
				\end{tabular}
				\caption{Esiti del calcolo dell'indice Gulpease sui documenti}
			\end{table}
	\subsubsection{Software}
	Di seguito sono riportati i valori ottenuti utilizzando le metriche definite sui seguenti obiettivi di qualità, i cui esiti sono suddivisi in \textcolor{red}{non accettabile}, \textcolor{orange}{accettabile} e \textcolor{green}{ottimale}:
	\begin{table}[h]
				\centering
				\begin{tabular}{l c c c}
					\hline
					\rule[-0.3cm]{0cm}{0.8cm}
					\textbf{Obiettivo} & \textbf{RR} & \textbf{RP} & \textbf{RQ}& \textbf{RA}\\
					\hline
					\rule[0cm]{0cm}{0.4cm}
					Implementazione funzionale - OPDS1 & & & & x \\
					\rule[0cm]{0cm}{0.4cm}
					Accuratezza rispetto alle attese - OPDS2 & & & & x \\
					\rule[0cm]{0cm}{0.4cm}
					Controllo degli accessi - OPDS3 & & & & x \\
					\rule[0cm]{0cm}{0.4cm}
					Densità di failure - OPDS4 & & & \textcolor{green}{0\%} & x \\
					\rule[0cm]{0cm}{0.4cm}
					Blocco di operazioni non corrette - OPDS5 & & & \textcolor{green}{94\%} & x\\
					\rule[0cm]{0cm}{0.4cm}
					Comprensibilità delle funzioni offerte - OPDS6 & & & & x \\
					\rule[0cm]{0cm}{0.4cm}
					Consistenza operazionale in uso - OPDS7 & & & & x \\
					\rule[0cm]{0cm}{0.4cm}
					Tempo di risposta - OPDS8 & & & \textcolor{green}{6.9} & x\\
					\rule[0cm]{0cm}{0.4cm}
					Capacità analisi di failure - OPDS9 & & & \textcolor{green}{100\%} & x\\
					\rule[0cm]{0cm}{0.4cm}
					Impatto delle modifiche - OPDS10 & & & \textcolor{green}{0\%} & x\\
					\hline
				\end{tabular}
				\caption{Esiti del calcolo delle metriche sul prodotto software}
			\end{table}
\end{document}
