\section{Glossario}
\subsection{Agent}
In api.ai, un agent è ciò che converte il linguagio naturale dell'utente in dati processabili. Nel contesto del progetto \PROGETTO, esiste un agent per ogni tipo di applicazione.\\
Gli agent esistenti sono:
\begin{itemize}
	\item ConversationAppGuest, ovvero l'agent che gestisce la conversazione con l'ospite;
	\item ConversationAppAdmin, ovvero l'agent che gestisce la conversazione con l'amministratore.
\end{itemize}
\subsection{Enrollment}
L'enrollment è la fase iniziale del riconoscimento vocale, nella quale la voce dell’utente viene registrata durante la ripetizione di frasi specifiche per estrarne la sua impronta vocale.\\
Questa impronta verrà utilizzata nella fase di verifica vocale. Tutto ciò è reso possibile dal servizio esterno descritto in \ref{speakerRec}.
\subsection{Member}
Il member del target di una rule è un membro dell'azienda alla quale indirizzare una rule. \\
Se, per esempio, il tipo del task della rule è \file{send\_to\_slack}, il member della rule sarà quella persona che dovrà essere notificata dell'arrivo del target.
\subsection{Pannello d'amministrazione}
In caso l'utente sia un (super) amministratore, per pannello d'amministrazione si intende quella sezione principale nella quale è possibile dare i comandi per godere delle funzionalità esposte in \ref{admin} e \ref{superAdmin}
\subsection{Rule (direttiva)}
Una rule (direttiva) è una regola impostata da un amministratore per l'assistente virtuale.
\subsection{Target}
Il target di una rule è la persona alla quale applicare la rule. \\
Se, per esempio, il tipo del task della rule è \file{send\_to\_slack}, il target della rule sarà la persona il quale arrivo deve essere segnalato al member. In questo contesto, quindi, il target potrebbe essere un ospite che \PROPONENTE{} conosce o prevede che arriverà.\\
Una direttiva può contenere più di un target.
\subsection{Task}\label{task}
Il task di una direttiva specifica l'effettivo compito che la direttiva deve portare a termine.\\
I task supportati al momento sono:
\begin{itemize}
	\item \file{send\_to\_slack}: il compito della direttiva è notificare il member dell'arrivo del target.
\end{itemize}