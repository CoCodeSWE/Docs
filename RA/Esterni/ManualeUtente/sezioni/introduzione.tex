\section{Introduzione}
\subsection{Scopo del documento}
Lo scopo del documento è quello di guidare gli amministratori del sistema nell'installazione, configurazione e utilizzo del prodotto.\\
Inoltre, come richiesto dal proponente, sono elencate tutte le interazioni possibili che ogni tipo di utente (vedere \ref{funz}) (ospite, amministratore, super amministratore) può avere con \PROGETTO.\\
\subsection{Scopo del progetto}
\SCOPO\\
Inoltre, il prodotto realizzato prevede anche un utilizzo lato amministratore, il quale consente di modificare il comportamento del sistema.
\subsection{Prerequisiti}
Per il funzionamento del prodotto sono necessarie i seguenti requisiti:
\begin{itemize}
	\item avere una connessione ad internet;
	\item utilizzare uno tra i seguenti browser:
	\begin{itemize}
		\item Google Chrome 53+;
	\end{itemize}
	\item configurare alcune piattaforme, spiegate in \ref{configurazione}.
\end{itemize}
\subsection{Segnalazione dei problemi}
In caso si vogliano segnalare errori o malfunzionamenti del sistema, è possibile contattare il team di sviluppo \GRUPPO{} all'indirizzo email swe.co.code@gmail.com. \\
L'email inviata deve:
\begin{itemize}
	\item avere come oggetto "Segnalazione problemi \PROGETTO";
	\item contenere una descrizione dettagliata del problema verificatosi;
	\item contenere una descrizione dettagliata delle azioni che hanno portato al verificarsi del problema.
\end{itemize}