\documentclass[PdQ.tex]{subfiles}

\begin{document}

\section{Resoconto delle attività di verifica - RR}
All'interno di questa sezione sono riportati gli esiti di tutte le attività di verifica effettuate sui documenti consegnati per la \RR{}. Ove necessario sono state tratte conclusioni sui risultati e su come essi possano essere migliorati.

\subsection{Qualità di processo}
		\subsubsection{Miglioramento continuo tramite CMM}
		Per rendere le performance dei processi costantemente migliorabili e perseguire gli obiettivi quantitativi di miglioramento viene utilizzato il modello Capability Maturity Model (CMM).\\
		All'inizio del periodo i processi si trovavano al livello 1 della scala CMM. In seguito, grazie alla stesura del documento \NPdocRR{} sono state definite regole per ogni tipo di documentazione, strumenti da utilizzare e procedure da seguire. Questo ha permesso un maggiore controllo dei processi, che hanno ottenuto la ripetibilità, proprietà che caratterizza il livello 2 della scala CMM. Si può quindi affermare che i processi hanno raggiunto tale livello. Non si può ancora affermare di aver raggiunto il livello 3 del modello perchè al processo manca ancora la sua caratteristica principale, la proattività.\\

		\paragraph{Soddisfacimento obiettivi di qualità}
			Di seguito sono riportati i valori ottenuti utilizzando le metriche definite sui seguenti obiettivi di qualità:
			\begin{table}[h]
				\centering
				\begin{tabular}{l c c}
					\hline
					\rule[-0.3cm]{0cm}{0.8cm}
					\textbf{Obiettivo} & \textbf{Valore} & \textbf{Esito} \\
					\hline
					\rule[0cm]{0cm}{0.4cm}
					Rispetto dei tempi - OPC2 & 10 & ottimale \\
					\rule[0cm]{0cm}{0.4cm}
					Rispetto dei costi - OPC3 & -5\% & accettabile\\ 
					
					\hline
				\end{tabular}
				\caption{Esiti del calcolo delle metriche sui processi}
			\end{table}
		
			Il valore di OPC3 è dovuto al fatto che l’attività degli \AMMP{} ha richiesto più tempo del previsto in quanto è stato necessario modificare alcune funzioni del \gl{software} utilizzato per il tracciamento dei requisiti e dei casi
d’uso, inoltre l’attività degli \ANP{} ha richiesto più tempo del previsto, in quanto si è dovuta fare un’analisi più approfondita rispetto a quella prefissata per una corretta stesura dei requisiti e dei casi d’uso. Questo è dovuto, in parte, all’interfaccia vocale da progettare, non convenzionale.
		
\newpage		
\subsection{Qualità di prodotto}
	\subsubsection{Documenti}
		\paragraph{Leggibilità e comprensibilità - OPDD1}
				Di seguito sono riportati i valori ottenuti calcolando l'indice Gulpease sui documenti:
				\begin{table}[h]
				\centering
				\begin{tabular}{l c c}
					\hline
					\rule[-0.3cm]{0cm}{0.8cm}
					\textbf{Documento} & \textbf{Gulpease} & \textbf{Esito} \\
					\hline
					\rule[0cm]{0cm}{0.4cm}
					\PPdocRR & 49 & accettabile \\
					\rule[0cm]{0cm}{0.4cm}
					\NPdocRR & 58 & accettabile \\ 
					\rule[0cm]{0cm}{0.4cm}
					\ARdocRR & 66 & ottimale \\ 
					\rule[0cm]{0cm}{0.4cm}
					\PQdocRR & 54 & accettabile \\ 
					\rule[0cm]{0cm}{0.4cm}
					\GldocRR & 50 & accettabile\\ 
					\rule[0cm]{0cm}{0.4cm}
					\SDKdoc & 67 & ottimale \\ 
					\rule[0cm]{0cm}{0.4cm}
					Verbale esterno 2016-12-17 & 66 & ottimale\\ 
					\rule[0cm]{0cm}{0.4cm}
					Verbale interno 2016-12-10 & 61 & ottimale\\ 
					\rule[0cm]{0cm}{0.4cm}
					Verbale interno 2016-12-19 & 62 & ottimale\\ 
					
					\hline
				\end{tabular}
				\caption{Esiti del calcolo dell'indice Gulpease sui documenti}
			\end{table}		

\end{document}