\documentclass[a4paper,titlepage]{article}

\makeatletter
\def\input@path{{../../../template/}{./img}}
\makeatother

\usepackage{Comandi}
\usepackage{Riferimenti}
\usepackage{Stile}

\def\NOME{Verbale 2017-04-20}
\def\VERSIONE{1.0.0}
\def\DATA{2017-04-20}
\def\REDATTORE{Mauro Carlin}
	\def\VERIFICATORE{Pier Paolo Tricomi}
	\def\RESPONSABILE{Mauro Carlin}
	\def\USO{Esterno}
	\def\DESTINATARI{\COMMITTENTE \\ & \CARDIN \\ & \GRUPPO \\ & \PROPONENTE} % Se esterno va anche il \gl{proponente}
	\def\SOMMARIO{Verbale dell'incontro esterno in data 2017-04-20 per il \gl{capitolato} \quotes{\CAPITOLATO{}}  del gruppo \GRUPPO.}
	
	
	\begin{document}
		
		\maketitle
		\begin{diario}
			\modifica{Mauro Carlin}{\RESP}{Approvazione del documento}{2017-04-22}{1.0.0}
			\modifica{Pier Paolo Tricomi}{\VER}{Verifica del documento}{2017-04-21}{0.1.0}
			\modifica{Mauro Carlin}{\RESP}{Stesura documento}{2017-04-20}{0.0.1}
		\end{diario}
		\newpage
		\tableofcontents
		
		\newpage
		\section{Informazioni generali}
		\label{sec:Informazioni}
		
		\begin{itemize}
			\item \textbf{Luogo}: aula Lum250, via Luzzati, Padova.
			\item \textbf{Data}: 2017-04-20.
			\item \textbf{Orario di inizio}: 14:30.
			\item \textbf{Orario di fine}: 14:45.
			\item \textbf{Durata}: 15m.
			\item \textbf{Oggetto}: chiarimento dubbi progettazione di dettaglio.
			\item \textbf{Partecipanti}: Mattia Bottaro, Pier Paolo Tricomi, Mauro Carlin, \PROPONENTE.
			\item \textbf{Segretario}: Mauro Carlin.
			
		\end{itemize}
		\section{Riassunto della riunione}
		\label{sec:RiassuntoRiunione}
		\subsection{Descrizione}
		La riunione è avvenuta presso via Luzzati a Padova, in particolare nell'aula Lum250. Erano presenti tre membri dell'azienda \gl{proponente} e tre componenti del gruppo \GRUPPO{}. Sono state poste diverse domande al proponente in merito alla progettazione di dettaglio.
		\subsection{Decisioni}
		\begin{itemize}
			\item DE1.1 - Alla richiesta del gruppo per l'autorizzazione all'utilizzo di IBM Watson come strumento di Speech To Text e Microsoft Speaker Recognition come strumento per il riconoscimento vocale, il proponente \PROPONENTE{} ha acconsentito dopo aver valutato insieme al gruppo \GRUPPO{} un possibile preventivo e averlo giudicato accettabile, dichiarando di poter stanziare fondi per l'utilizzo.  
			\item DE1.2 - Il proponente ha richiesto la stesura di un manuale con tutte le possibili interazioni tra i vari tipi di utente (admin, super admin, ospite) e \PROGETTO{} per la Revisione di Qualifica. A tal scopo, verrà prodotto un Manuale Utente che, oltre ad esporre quanto appena detto, spiegherà nel dettaglio le funzionalità (ovvero interazioni) disponibili agli amministratori del sistema.
			\item DE1.3 - Un manuale manutentore è eventualmente richiesto per la Revisione di Accettazione e non per la Revisione di Qualifica.
		\end{itemize}
		
		
	\end{document}
