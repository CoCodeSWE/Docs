\documentclass[AdR.tex]{subfiles}
\begin{document}
In questa sezione verranno presentati i requisiti individuati dal \gl{team} durante l'analisi del \gl{capitolato}
e dei \gl{casi d'uso}, discussi con il \gl{proponente} durante le riunioni esterne e decisi dai componenti
nelle riunioni interne.
Ogni requisito individuato avrà un codice identificativo univoco così formato: \\ \\
\centerline{R\textbraceleft{}Tipo\textbraceright{}\textbraceleft{}Importanza\textbraceright{}\textbraceleft{}Codice\textbraceright{}}
 \\ \\
dove:
\begin{itemize}
 	\item \textbf{Tipo}: può assumere uno di questi valori:
 	\begin{itemize}
 		\item \textbf{F}: indica un requisito funzionale;
 		\item \textbf{Q}: indica un requisito di qualità;
 		\item \textbf{P}: indica un requisito prestazionale;
 		\item \textbf{V}: indica un requisito di vincolo.
 	\end{itemize}
 	\item \textbf{Importanza}: può assumere uno di questi valori:
 	\begin{itemize}
 		\item \textbf{O}: indica un requisito obbligatorio;
 		\item \textbf{D}: indica un requisito desiderabile;
 		\item \textbf{F}: indica un requisito facoltativo.
 	\end{itemize}
 	\item \textbf{Codice}: indica il codice identificativo del requisito, è univoco e deve essere identificato in forma gerarchica.
 \end{itemize}
Per ogni requisito inoltre verranno riportate:
\begin{itemize}
	\item \textbf{Descrizione}: breve testo ma completo che andrà a descrivere il requisito in esame;
	\item \textbf{Fonte}: che potrà essere una tra le seguenti:
	\begin{itemize}
		\item \textbf{Capitolato}: requisito dedotto direttamente dall'analisi del capitolato;
		\item \textbf{Verbale Esterno 1}: requisito derivato dal verbale esterno \textit{Verbale\textunderscore{}E\textunderscore{}2016-12-17};
		\item \textbf{Interno}: requisito identificato dagli \ANP;
		\item \textbf{Caso d'uso}: si tratta di un requisito emerso da un caso d'uso; viene riportato l'identificativo del caso d'uso associato.
	\end{itemize}
\end{itemize}
\end{document}