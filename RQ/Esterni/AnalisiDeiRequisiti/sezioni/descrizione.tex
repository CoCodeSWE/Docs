\subsection{Obiettivo del prodotto}
Il \gl{prodotto} vuole essere un assistente virtuale in grado di accogliere un cliente in visita all'ufficio di \PROPONENTE{} e, allo stesso tempo, informare la persona desiderata dell'arrivo dell'ospite. 
\subsection{Funzioni del prodotto}
Il \gl{sistema}, dopo aver dato il benvenuto all'ospite, richiede alcune informazioni riguardanti la sua visita all'ufficio di \PROPONENTE. I dati di interesse riguardanti l'incontro, che l'assistente ha il compito di raccogliere tramite delle domande mirate, sono: l'identità del visitatore, l'azienda di provenienza e la persona desiderata.\\
L'utente sarà in grado di visualizzare l'intera conversazione con il sistema tramite un'interfaccia grafica, che mostra sia la domanda dell' assistente che la risposta data dall'utente stesso.
A seguito delle richieste iniziali, il nostro prodotto richiede all'ospite tutte le possibili esigenze (caffè, stanze con apparecchiature specifiche, materiale per l'incontro) e invia tutti i dati ottenuti alla persona desiderata tramite l'applicazione di messaggistica aziendale \gl{Slack}.\\
Durante l'attesa dell'arrivo del membro di \PROPONENTE{}, l'assistente offre all'ospite una serie di intrattenimenti di carattere ludico (tris, mankala, sasso carta forbice) o informativo (curiosità di vario genere o indovinelli).\\
Il sistema è in grado di riconoscere un ospite che è già stato in visita all'azienda, grazie alle informazioni memorizzate durante le interazioni precedenti. Questo permette all'assistente di migliorare le proprie domande, ad esempio una volta ottenuti nome e cognome può chiedere di confermare l'azienda di provenienza e la persona desiderata senza che l'utente le pronunci.\\
Il prodotto mette a disposizione anche un'area di amministrazione dove è possibile modificare il comportamento dell'assistente per determinate categorie di persone, come ad esempio ospiti di un'azienda specifica, oppure per coloro che cercano una persona specifica di \PROPONENTE{}. \\
Gli amministratori possono effettuare il login utilizzando semplicemente la propria voce. Il sistema infatti si appoggia ad un servizio esterno di Speaker Recognition di Microsoft dove ogni amministratore associa una determinata frase al proprio profilo; nel momento in cui un utente, dopo aver fornito il proprio nome e cognome, voglia usufruire dei privilegi di amministratore, basterà che pronunci la propria frase e il sistema confronterà il suo timbro vocale con quello precedentemente memorizzato, approvando o meno il tentativo di login.\\
Una volta finita l'interazione il sistema si metterà in pausa in attesa del prossimo ospite.
\subsection{Caratteristiche degli utenti}
Non sono richieste competenze particolari per poter utilizzare questo prodotto, che deve risultare
quindi accessibile ad un'ampia categoria di utenti. Questo sarà garantito dal fatto che l'interazione con l'assistente sarà quasi completamente di carattere vocale.
\subsection{Vincoli generali}
Essendo un applicativo Web dovrà funzionare correttamente su PC, Mac o tablet, senza alcuna limitazione sul sistema operativo.\\
Il \gl{browser} che verrà utilizzato dovrà essere compatibile con \gl{JavaScript} e gli standard \gl{HTML5}, \gl{CSS3}.
I browser supportati saranno:
\begin{itemize}
	\item Google Chrome 53+ (obbligatorio);
	\item Mozilla Firefox 51+ (facoltativo);
	\item Safari 10.0.3+ (facoltativo);
	\item Opera 41+ (facoltativo);
	\item Internet Explorer 9+ (facoltativo).
\end{itemize}
