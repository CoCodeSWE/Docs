\section{Standard di progetto}
\subsection{Progettazione architetturale}
È stato deciso di utilizzare UML 2.5 per la realizzazione dei diagrammi. \\
Inoltre è stata applicata una modifica alla notazione, in modo da rappresentare più chiaramente il passaggio di una funzione come parametro di un'altra.
Tale modifica consiste nell’uso della notazione \file{function(<parameters>): <T> }, nella quale:
\begin{itemize}
	\item \file{<parameters>} indica i parametri accettati dalla funzione secondo la notazione UML adottata;
	\item \file{<T>} indica il tipo di ritorno della funzione.
\end{itemize}
\subsection{Documentazione del codice}
Per gli standard di documentazione del codice si fa riferimento al documento \NPdoc.
\subsection{Programmazione}
Per gli standard di programmazione si fa riferimento al documento \NPdoc.
\subsection{Strumenti e procedure}
Per gli strumenti di lavoro e le procedure per la realizzazione del progetto si fa riferimento al documento \NPdoc.
\subsection{Denominazione relazioni ed entità}
Per tutte le entità e le relazioni valgono gli standard di denominazione seguenti:
\begin{itemize}
	\item per le entità definite come package, classi, attributi, metodi e parametri è necessario fornire denominazioni chiare, concise e autoesplicative;
	\item per la denominazione di package, classi e attributi sono da preferire i sostantivi, mentre per i metodi e le relazioni i verbi;
	\item per le regole tipografiche sui nomi si fa riferimento al documento \NPdoc.
\end{itemize}
