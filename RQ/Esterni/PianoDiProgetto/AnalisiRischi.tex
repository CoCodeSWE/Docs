\documentclass[PianoDiProgetto.tex]{subfiles}
\renewcommand{\arraystretch}{1.5}

\begin{document}
\section{Analisi dei rischi}
Al fine di evitare rallentamenti dei periodi di lavoro è stata effettuata una dettagliata analisi dei rischi, in modo da poter evitare le situazioni che portano alla creazione di eventi non pianificati, ove possibile. L'analisi si suddivide in quattro attività:
	\begin{itemize}
		\item \textbf{Identificazione}: individuare quali rischi possono incorrere durante lo sviluppo del \gl{progetto}, analizzarli e provare a intuire quali saranno le conseguenze se questi si verificano;
		\item \textbf{Analisi di periodo}: valutare la probabilità di occorrenza di un rischio e analizzare la criticità delle conseguenze di questo rispetto all'andamento del progetto nel periodo in corso. Tutte i periodi sono illustrati nella sezione 4 riguardante la Pianificazione. Il \RESP{} di progetto è incaricato  di fare una nuova analisi dei rischi ogni qualvolta si cambi periodo;
		\item \textbf{Pianificazione e mitigazione}: identificare un metodo di controllo dei rischi così da renderli evitabili, nel qual caso non lo siano, pianificare delle contromisure per ridurre al minimo i danni;
		\item \textbf{Controllo}: monitorare nel tempo ogni rischio, e nel caso si verifichi descriverne il riscontro effettivo e come il \gl{team} ha reagito per ridurre al minimo i danni.
	\end{itemize}
I rischi trattati vengono suddivisi in 5 sottosezioni per essere analizzati al meglio:
	\begin{itemize}
		\item livello tecnologico;
		\item livello degli strumenti;
		\item livello del personale;
		\item livello organizzativo;
		\item livello dei requisiti.
	\end{itemize}
Ogni rischio trattato ha una serie di caratteristiche necessarie per comprenderne la natura:
	\begin{itemize}
		\item nome;
		\item descrizione;
		\item risultati analisi:
			\begin{itemize}
				\item probabilità di occorrenza;
				\item grado di pericolosità;
				\item possibili conseguenze;
				\item riconoscimento;
				\item trattamento;
				\item attualizzazione nel periodo.
			\end{itemize}
	\end{itemize}
Ogni rischio verrà monitorato nel tempo e ne verrà indicato l'effettivo riscontro nel periodo in corso.

\newpage

\begin{table}[h]
\centering
\label{Analisi dei Rischi}
\begin{tabular}{cc}
\hline
\textbf{Livello}       & \textbf{Tipologia} \\ \hline
Tecnologico   & \begin{tabular}[cl]{ll} Tecnologie adottate sconosciute \ref{sec:tas}
 \end{tabular}\\  \cline{2-2}
              & \begin{tabular}[cl]{ll} Guasti hardware e malfunzionamenti \gl{software} \ref{sec:ghs}\end{tabular} \\ \hline
Organizzativo & \begin{tabular}[cl]{ll} Valutazioni delle risorse \ref{sec:lo}\end{tabular}                   \\ \hline
Personale     & \begin{tabular}[cl]{ll} Problemi personali dei membri del team \ref{sec:ppdct}\end{tabular}     \\ \cline{2-2}
              & \begin{tabular}[cl]{ll} Problemi personali tra i membri del team \ref{sec:pptc}\end{tabular}     \\ \hline
Requisiti     &   Mancata comprensione \ref{sec:lr}\\ \hline
\end{tabular}
\caption{Registro dei rischi}
\end{table}

\newpage
\subsection{Livello tecnologico}
		\subsubsection{Tecnologie adottate sconosciute}
			\label{sec:tas}
	\begin{table}[h]
		\begin{center}
			\begin{tabular}{ | c | p{10cm} |}
				\hline
					\textbf{Descrizione}& per lo svolgimento e l'implementazione del progetto, il team dovrà utilizzare una serie di tecnologie che nessun membro ha mai utilizzato; \\ \hline
				\textbf{Probabilità di occorrenza} & media  \\ \hline
				\textbf{Grado di pericolosità} & alto \\ \hline
				\textbf{Possibili conseguenze} & l'utilizzo di tecnologie sconosciute richiede tempo per la scelta e l'apprendimento, il che può portare ad un ritardo sulle date di consegna \\ \hline
				\textbf{Riconoscimento} & il \RESP{} deve verificare il grado di preparazione di ogni membro del gruppo relativo alle tecnologie utilizzate; \\ \hline
				\textbf{Trattamento} &  ogni membro del team deve studiare e approfondire autonomamente i documenti forniti dall'\AMM{} che spiegano come utilizzare propriamente le nuove tecnologie adottate; \\ \hline
				\textbf{Attualizzazione nel periodo} &
				\\ \hline \textbf{\PerAR}& il rischio non si è presentato, in quanto le nuove tecnologie non sono ancora state prese in carico
				\\ \hline \textbf{\PerAD}& il rischio non si è presentato, in quanto le nuove tecnologie non sono ancora state prese in carico
				\\ \hline \textbf{\PerPA}& lo studio delle tecnologie necessarie ha causato ritardi sulle date previste per la stesura della \DPdocRP{}
				\\ \hline \textbf{\PerPD}& il tempo necessario allo studio delle nuove tecnologie è stato sottostimato, causando ulteriori ritardi sulla pianificazione. Per ovviare a problemi simili, il gruppo si è organizzato per fare in
        modo che in futuro un solo membro debba studiarsi una determinata tecnologia, per poi spiegarla agli altri, ottimizzando in questo modo le risorse \\ \hline 
        \textbf{\PerC}& in seguito alle strategie adottate nei periodi precedenti, il rischio non si è presentato \\
				\hline

			\end{tabular}
		\caption{Tecnologie adottate sconosciute}
		\end{center}
	\end{table}

\clearpage

	\subsubsection{Guasti hardware e malfunzionamenti software}
		\label{sec:ghs}

		\begin{table}[h]
		\begin{center}
			\begin{tabular}{ | c | p{10cm} |}
				\hline
		 \textbf{Descrizione} & durante tutto il progetto è possibile che si verifichino guasti hardware e/o malfunzionamenti software ai computer usati dal team per sviluppare \\ \hline
		 \textbf{Probabilità di occorrenza} & basso \\ \hline
		 \textbf{Grado di pericolosità} & medio \\ \hline
	 \textbf{Possibili conseguenze} & il malfunzionamento di uno dei dispositivi del team può portare a perdita di dati e di conseguenza perdita di tempo in quanto si va a svolgere nuovamente un lavoro già effettuato \\ \hline
		 \textbf{Riconoscimento} & tutti i membri del team devono mantenere un'elevata attenzione sulle condizioni dei propri dispositivi \\ \hline
		 \textbf{Trattamento} & ogni componente del team si impegnerà a fare un backup giornaliero del lavoro effettuato su un dispositivo esterno al computer utilizzato per sviluppare; in caso di rotture hardware ogni membro possiede un altro dispositivo che gli permette di continuare il lavoro \\ \hline
		 \textbf{Attualizzazione nel periodo}&  \\
		 \hline
				 \textbf{\PerAR}& il rischio non si è presentato\\
				 \hline
				\textbf {\PerAD} & il rischio non si è presentato\\
				\hline
				\textbf{\PerPA} & il rischio non si è presentato \\
				\hline
				\textbf{ \PerPD} & il rischio non si è presentato\\
				\hline
				\textbf{\PerC} & il rischio si è presentato: uno dei membri del team ha avuto un guasto al proprio dispositivo; grazie alle strategie adottate però, questo non ha portato a nessun rallentamento \\
			\hline
			\end{tabular}
		\caption{Guasti hardware e malfunzionamenti software}
		\end{center}
	\end{table}

\clearpage
	\subsection{Livello organizzativo}
	\subsubsection{Valutazione delle risorse}
		\label{sec:lo}


\begin{table}[h]
		\begin{center}
			\begin{tabular}{ | c | p{10cm} |}
				\hline

		 \textbf{Descrizione} & data la mancanza di esperienza con progetti di questa dimensione il team potrebbe incorrere in stime errate di valutazione delle risorse \\ \hline
		 \textbf{Probabilità di occorrenza} & alta \\ \hline
		 \textbf{Grado di pericolosità} & alto \\ \hline
		 \textbf{Possibili conseguenze} & un'errata stima delle risorse può portare ad uno spreco di queste o a ritardi nelle date di consegna \\ \hline
		 \textbf{Riconoscimento} & il rischio in questo caso è dinamico, per questo è necessario controllare lo sviluppo delle attività di progettazione periodicamente, tramite  verifica da parte del \RESP{}, così da prendere atto di eventuali ritardi \\ \hline
		 \textbf{Trattamento} & ogni attività ha un periodo di slack, tale che l'eventuale ritardo di un'attività non condizioni le tempistiche delle altre \\ \hline
		 \textbf{Attualizzazione nel periodo} &

				\\ \hline \textbf{\PerAR} & il rischio non si è presentato
				\\ \hline \textbf{\PerAD}& il rischio non si è presentato
				\\ \hline \textbf{\PerPA}& il rischio si è presentato: a causa di una allocazione delle risorse poco corretta, dovuta alla poca esperienza dei membri del gruppo, ed ad un'analisi superficiale dei rischi legati ad imprevisti ed impegni non prorogabili, si sono verificati diversi ritardi nell'attività dei \PJP{}
				\\ \hline \textbf{\PerPD}& nonostante i tentativi del gruppo di limitare i danni causati dagli errori di pianificazione nel periodo precedente, il gruppo è stato costretto a posticipare la consegna della \RP.
        Per evitare che problemi simili si ripresentino in futuro, il gruppo ha deciso di effettuare una più attenta analisi dei rischi relativi a possibili imprevisti.
        Inoltre sono state ridotte le ore dedicate all'attività degli \ANP{} e \VERP{} in ragione di un incremento delle ore dedicate alle attività dei \PJP{} \\ \hline
        		\textbf{\PerC}& il rischio non si è presentato, grazie alla riorganizzazione effettuata nei periodi precedenti \\
				\hline
		\end{tabular}
		\caption{Valutazione delle risorse}
		\end{center}
	\end{table}

	\clearpage
	\subsection{Livello personale}
		\subsubsection{Problemi personali dei componenti del team}
			\label{sec:ppdct}

\begin{table}[h]
		\begin{center}
			\begin{tabular}{ | c | p{10cm} |}
				\hline


		 \textbf{Descrizione} & ogni membro del team avrà le sue necessità e i suoi impegni personali lungo la durata del progetto. Risulta inevitabile il verificarsi di problemi organizzativi in seguito a sovrapposizioni di tali impegni \\ \hline
		 \textbf{Probabilità di occorrenza} & media \\ \hline
		 \textbf{Grado di pericolosità} & alto \\ \hline
		 \textbf{Possibili conseguenze} & ritardo nello svolgimento delle attività \\ \hline
		 \textbf{Riconoscimento} &  per creare un calendario sincronizzato e condiviso tra i membri del gruppo è necessario che vengano notificati al \RESP{} in maniera preventiva e tempestiva gli impegni di ognuno. Grazie a questa pratica è possibile ridurre al minimo tale rischio \\ \hline
		 \textbf{Trattamento} & nel caso un membro del team abbia un impegno che non gli permetta di proseguire il lavoro, il \RESP{} andrà a modificare la pianificazione prevista in modo da coprire l'assenza creatasi \\ \hline
		 \textbf{Attualizzazione nel periodo}&

				\\ \hline \textbf{\PerAR} & i membri si sono impegnati per comunicare anticipatamente gli impegni personali, in questo modo il \RESP{} è riuscito a pianificare al meglio le attività da assegnare. Grazie a questa collaborazione il rischio non si è presentato
				\\ \hline \textbf{\PerAD}& il rischio non si è presentato
				\\ \hline \textbf{\PerPA}& il rischio non si è presentato
				\\ \hline \textbf{\PerPD}& il rischio non si è presentato \\ \hline
				\textbf{\PerC}& il rischio non si è presentato \\ \hline
\end{tabular}
		\caption{Problemi personali dei componenti del team}
		\end{center}
	\end{table}

\clearpage

		\subsubsection{Problemi personali tra i componenti del team}
			\label{sec:pptc}
\begin{table}[h]
		\begin{center}
			\begin{tabular}{ | c | p{10cm} |}
				\hline


		 \textbf{Descrizione} & essendo la prima volta che i membri del team collaborano, potrebbero sorgere attriti o squilibri interni che andrebbero a danneggiare il clima lavorativo e porterebbero a ritardi nelle consegne \\ \hline
		 \textbf{Probabilità di occorrenza} & bassa \\ \hline
		 \textbf{Grado di pericolosità} & alto \\ \hline
		 \textbf{Possibili conseguenze} & ritardo nello svolgimento delle attività \\ \hline
		 \textbf{Riconoscimento} & tutti i membri del gruppo devono avere una comunicazione costante con il \RESP{} il quale si occuperà di monitorare i rapporti tra i collaboratori \\ \hline
		 \textbf{Trattamento} & nel caso di contrasti tra membri del gruppo, il \RESP{} provvederà ad assegnare a tali membri attività differenti con il minimo contatto (nel limite del possibile) \\ \hline
		 \textbf{Attualizzazione nel periodo} &

				\\ \hline \textbf{\PerAR} & il rischio non si è presentato
				\\ \hline \textbf{\PerAD}& il rischio non si è presentato
				\\ \hline \textbf{\PerPA}& il rischio non si è presentato
				\\ \hline \textbf{\PerPD}& il rischio non si è presentato
				\\ \hline \textbf{\PerC}& il rischio non si è presentato \\
		 \hline

\end{tabular}
		\caption{Problemi personali tra componenti del team}
		\end{center}
	\end{table}

\clearpage
	\subsection{Livello dei Requisiti}
		\subsubsection{Incomprensione e scelte non ottimali}
			\label{sec:lr}
\begin{table}[h]
		\begin{center}
			\begin{tabular}{ | c | p{10cm} |}
				\hline


		 \textbf{Descrizione} & è possibile che alcuni requisiti individuati dagli \ANP{} siano fraintesi, superficiali o errati rispetto alle aspettative del \gl{proponente} \PROPONENTE{}. Inoltre esiste la probabilità che qualche requisito venga modificato, eliminato o aggiunto durante il corso del progetto \\ \hline
		 \textbf{Probabilità di occorrenza} & alta \\ \hline
		 \textbf{Grado di pericolosità} & alto \\ \hline
		 \textbf{Possibili conseguenze} & sviluppo di un \gl{prodotto} non consono alle aspettative del proponente \\ \hline
		 \textbf{Riconoscimento} & avere una costante comunicazione con il proponente \PROPONENTE{} durante il periodo di \ARdoc{} in modo da chiarire tutte le incomprensioni e assicurare la concordanza sui requisiti del prodotto \\ \hline
		 \textbf{Trattamento} & si dovranno effettuare incontri con il proponente \PROPONENTE{} così da poter correggere eventuali errori indicati dal committente durante la revisione \\ \hline
		 \textbf{Attualizzazione nel periodo}&

				\\ \hline \textbf{\PerAR} & il rischio non si è presentato
				\\ \hline \textbf{\PerAD}& il rischio si è presentato in minima parte, in quanto dopo la  \RR{} i requisiti sono stati corretti in base alle richieste del committente, per poi essere accettati dal proponente
				\\ \hline \textbf{\PerPA}& il rischio non si è presentato
				\\ \hline \textbf{\PerPD}& il rischio non si è presentato
				\\ \hline \textbf{\PerC}& il rischio non si è presentato \\ \hline

\end{tabular}
		\caption{Incomprensione e scelte non ottimali}
		\end{center}
	\end{table}

\end{document}
