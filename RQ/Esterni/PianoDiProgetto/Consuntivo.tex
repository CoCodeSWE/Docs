\documentclass[./PianoDiProgetto.tex]{subfiles}
\begin{document}
\section{Consuntivo}
\subsection{Consuntivi di periodo}
In questa sezione verranno indicate le spese effettivamente sostenute, divise per ruolo. Il bilancio potrà risultare:
\begin{itemize}
  \item \textbf{Positivo}: se il preventivo supera il consuntivo;
  \item \textbf{In pari}: se il preventivo ed il consuntivo coincidono;
  \item \textbf{In negativo}: se il consuntivo supera il preventivo.
\end{itemize}
\subsubsection{\PerAR}
	\paragraph{Consuntivo}
Le ore di lavoro sostenute in questo periodo sono da considerarsi come ore di approfondimento personale, in quanto il gruppo \GRUPPO{} non è ancora stato scelto come fornitore ufficiale per il \gl{progetto} \PROGETTO.

		Tali dati riguardano quindi le ore non rendicontate.

\begin{table}[h]
		\centering
		\begin{tabular}{l * {2}{c}}
			\toprule
			\textbf{Ruolo} & \textbf{Ore} & \textbf{Costo (\euro{})} \\
			\midrule
			Responsabile &	22 & 660,00 \\
			%\midrule
			Amministratore & 57 (+7) & 1.140,00 (+140,00)\\
			%\midrule
			Progettista & 0 & 0,00 \\
			%\midrule
			Analista & 59 (+6) & 1.475,00 (+150,00)\\
			%\midrule
			Programmatore & 0 & 0,00 \\
			%\midrule
			Verificatore & 51 (-5) & 765,00 (-75,00)\\
			\midrule
			\textbf{Totale Preventivo} & 189
 & 4.040,00
 \\
			\textbf{Totale Consuntivo} & 197 & 4.255,00
 \\
			\midrule
			\textbf{Differenza} & +8 & +215,00 \\
			\bottomrule
		\end{tabular}
		\caption{\PerAR{} - Consuntivo}
		\label{tab:consuntivoA}

	\end{table}

  \paragraph{Conclusioni}
		Come si può notare dalla tabella \ref{tab:consuntivoA}, che presenta i dati relativi al consuntivo del periodo AR, è stato necessario investire più tempo del previsto nei ruoli di  \AMM{} e \AN, di conseguenza il bilancio risultante è \textbf{negativo}.

		L'attività degli \AMMP{} ha richiesto più tempo del previsto in quanto è stato necessario modificare alcune funzioni del \gl{software} utilizzato per il tracciamento dei requisiti e dei \gl{casi d'uso}. Poichè gli strumenti ora sono completamente funzionanti si prevede di non incorrere in ore aggiuntive di amministrazione nei prossimi periodi.

		L'attività degli \ANP{} ha richiesto più tempo del previsto, in quanto si è dovuta fare un'analisi più approfondita rispetto a quella prefissata per una corretta stesura dei requisiti e dei casi d'uso.
		Questo è dovuto, in parte, all'interfaccia vocale da progettare, non convenzionale.
		Poichè dopo questo periodo di analisi il gruppo \GRUPPO\ ha più confidenza sui requisiti di una corretta interfaccia vocale, non si prevedono ulteriori costi da attribuire a ore di analisi aggiuntive nei prossimi periodi.

\subsubsection{\PerAD}
	\paragraph{Consuntivo}
	\begin{table}[h]
		\centering
		\begin{tabular}{l * {2}{c}}
			\toprule
			\textbf{Ruolo} & \textbf{Ore} & \textbf{Costo (\euro{})} \\
			\midrule
			Responsabile &	6 & 180,00 \\
			%\midrule
			Amministratore & 22 (+3)  & 500,00 (+60,00)\\
			%\midrule
			Progettista & 0 & 0,00 \\
			%\midrule
			Analista & 30 & 750,00\\
			%\midrule
			Programmatore & 0 & 0,00 \\
			%\midrule
			Verificatore & 20 (-6) & 300,00 (-90,00)\\
			\midrule
			\textbf{Totale Preventivo} & 84
 & 1760,00
 \\
			\textbf{Totale Consuntivo} & 81 & 1730,00 \\
			\midrule
			\textbf{Differenza} & -3 & -30,00 \\
			\bottomrule
		\end{tabular}
		\caption{\PerAD{} - Consuntivo}

	\end{table}
	\paragraph{Conclusioni}
	In questo periodo il consuntivo ha ottenuto esito \textbf{positivo}. Questo è dovuto al fatto che siano state spese un minor numero di ore  per l'attività di \gl{verifica} rispetto a quanto pianificato, nonostante sia stato impiegato un maggior numero di ore rispetto alla pianificazione dagli \AMMP{}. Questo ore sono state utilizzate per rivedere i documenti \textit{Norme di Progetto} e \textit{Piano di Qualifica} in base alle correzioni del committente.

\subsubsection{\PerPA}
\paragraph{Consuntivo}
	\begin{table}[h]
		\centering
		\begin{tabular}{l * {2}{c}}
			\toprule
			\textbf{Ruolo} & \textbf{Ore} & \textbf{Costo (\euro{})} \\
			\midrule
			Responsabile &	7 (-1) & 180,00 (-30,00) \\
			%\midrule
			Amministratore & 28 & 560,00\\
			%\midrule
			Progettista & 40 (+25) & 1430,00 (+550,00)\\
			%\midrule
			Analista & 54 (-15)  & 975,00 (-375,00)\\
			%\midrule
			Programmatore & 0 & 0,00 \\
			%\midrule
			Verificatore & 50 & 750,00\\
			\midrule
			\textbf{Totale Preventivo} & 179
 & 3.750,00
 \\
			\textbf{Totale Consuntivo} & 188 & 3895,00
 \\
			\midrule
			\textbf{Differenza} & +9 & +145,00 \\
			\bottomrule
		\end{tabular}
		\caption{\PerPA{} - Consuntivo}
	\end{table}
	\paragraph{Conclusioni}
			In questo periodo, l'esito del consuntivo è \textbf{negativo}. Questo è dovuto ad una cattiva pianificazione, che ha sovrastimato le ore degli \ANP{} e sottostimato quelle dei \PJP{}. I \PJP{} hanno infatti utilizzato molte più ore rispetto a quanto pianificato per la definizione dell'architettura generale e lo studio delle tecnologie necessarie. Questa è stata inoltre la causa principale che ha portato il gruppo a posticipare le date di consegna. In seguito a quanto emerso dall'analisi dei rischi, il gruppo ha deciso di modificare il \PPdoc{} e di ridurre le ore degli \ANP{} e dei \VERP{}, ed incrementare le ore dei \PJP, al fine di mitigare i rischi.
\subsubsection{\PerPD}
\paragraph{Consuntivo}
		\begin{table}[h]
		\centering
		\begin{tabular}{l * {2}{c}}
			\toprule
			\textbf{Ruolo} & \textbf{Ore} & \textbf{Costo (\euro{})} \\
			\midrule
			Responsabile &	9 & 270,00 \\
			%\midrule
			Amministratore & 19 & 380,00\\
			%\midrule
			Progettista & 40  & 1100,00 \\
			%\midrule
			Analista & 33 (-6) & 675,00 (-150,00) \\
			%\midrule
			Programmatore & 0 & 0,00 \\
			%\midrule
			Verificatore & 57 (-18) & 585,00 (-270,00) \\
			\midrule
			\textbf{Totale Preventivo} & 168
 & 3.460,00
 \\
			\textbf{Totale Consuntivo} & 144 & 3040,00
 \\
			\midrule
			\textbf{Differenza} & -24 & -420,00 \\
			\bottomrule
		\end{tabular}
		\caption{\PerPD{} - Consuntivo}

	\end{table}
	\paragraph{Conclusioni}
	In questo periodo l'esito del consuntivo è \textbf{positivo}. Questo è dovuto ad una sovrastima delle ore necessarie a determinati ruoli, come \AN{} e \VER{}, oltre al fatto che avendo utilizzato un maggior numero di ore per la progettazione nello scorso periodo, in questo sono 
state necessarie un minor numero di ore, corrispondente a quanto pianificato.\\
L'impegno impiegato per le attività progettuali negli ultimi due periodi, permetterà l'utilizzo di un minor numero di ore da parte dei \PJP{} in futuro, a favore di ruoli come \PRP{} e \VERP{}. Sarà quindi possibile ridurre le ore dedicate alla progettazione per i periodi rimanenti.
\subsubsection{\PerC}
\paragraph{Consuntivo}	
\begin{table}[H]
		\centering

		\begin{tabular}{l * {2}{c}}
			\toprule
			\textbf{Ruolo} & \textbf{Ore} & \textbf{Costo (\euro{})} \\
			\midrule
			Responsabile & 9    &  2700,00 \\
			Amministratore  & 9    &  180,00 \\
			Progettista  & 5(-8)   &  110,00(-176,00) \\
			Analista & 1(+1)    &  25,00(+25,00) \\
			Programmatore  & 134(+8)    &  2.130,00(+120,00) \\
			Verificatore  & 58(+5)    &  795,00(+75,00) \\
			\midrule
			\textbf{Totale preventivo}  & 218   &  3.541,00 \\
			\textbf{Totale consuntivo}  & 224   &  3.560,00 \\
			\textbf{Differenza}  & 6  &  +19,00 \\
			\bottomrule
		\end{tabular}
		\caption{\PerC{} - Consuntivo}
	\end{table}
	\paragraph{Conclusioni}
	In questo periodo l'esito del consuntivo è \textbf{negativo}, in quanto sono state utilizzate ore in più rispetto a quanto pianificato, anche se il costo si avvicina molto al preventivo. In seguito alle attività dei periodi precedenti, i \PJP{} hanno utilizzato meno ore rispetto a quanto pianificato; queste ore sono state usate per le attività di codifica dai \PRP{}. Inoltre è stata utilizzata un'ora di \AN{} per la correzione del documento \ARdoc{} come segnalato dal committente. Oltre a ciò sono state utilizzate più ore del previsto per le attività di verifica in quanto sono sorti alcuni problemi durante il testing del software.\\
L'impegno impiegato per le attività di codifica in questo periodo permetterà un minore utilizzo  di risorse da parte dei \PRP{} nel prossimo periodo; inoltre la risoluzione dei problemi durante il testing permetterà un miglior utilizzo di risorse da parte dei \VERP{} durante tutto il prossimo periodo.
\clearpage
\end{document}
