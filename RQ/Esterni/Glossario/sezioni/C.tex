\lettera{C}
% questa riga non deve contenere istruzioni

\parola{C\#}{È un linguaggio di programmazione orientato agli oggetti sviluppato da Microsoft.}

\parola{C++}{ E' un linguaggio di programmazione orientato agli oggetti.}

\parola{Callback}{È una funzione che viene passata come parametro ad un'altra funzione. In particolare, quando ci si riferisce alla callback richiamata da una funzione, la callback viene passata come parametro alla funzione chiamante. In questo modo la chiamante può realizzare un compito specifico (quello svolto dalla callback) che non è, molto spesso, noto al momento della scrittura del codice.\\ Se invece ci si riferisce alla callback come funzione richiamata dal sistema operativo, di norma ciò si utilizza allo scopo di gestire particolari eventi: dal premere un bottone con il mouse, allo scrivere caratteri in un campo di testo. Ciò consente, quindi, a un programma di livello più basso, di richiamare una funzione (o servizio) definita a un livello più alto.}

\parola{Capitolato}{Atto allegato ad un contratto d’appalto che intercorre tra committente e l'appaltatore, nel quale vengono descritte le attese del primo nei confronti del secondo.}

\parola{CarPlay}{È un sistema integrato in alcune automobili che permette l'interazione con un iPhone tramite il display del veicolo.}

\parola{Caso d'uso}{È una tecnica usata nei processi di ingegneria del \gl{software} per effettuare in maniera esaustiva e non ambigua la raccolta dei requisiti al fine di produrre software di qualità.}

\parola{Ciclo di vita}{Scomposizione dell'attività di realizzazione di prodotti software in sottoattività fra loro coordinate, il cui risultato finale è il prodotto stesso e tutta la documentazione ad esso associata.}

\parola{Client-side discovery}{Pattern utilizzato per localizzare microservizi. Il client richiede la posizione di uno specifico microservizio a un registro che conosce le posizioni di tutte le istanze dei microservizi.}

\parola{Cloud}{Paradigma di erogazione di risorse informatiche, come l'archiviazione, l'elaborazione o la trasmissione di dati, caratterizzato dalla disponibilità \gl{on demand} attraverso Internet a partire da un insieme di risorse preesistenti e configurabili.}

\parola{CMM}{"Capability Maturity Model". È un approccio al miglioramento dei processi il cui obiettivo è di aiutare un'organizzazione a migliorare le sue prestazioni. Il CMM può essere usato per guidare il miglioramento dei processi all'interno di un progetto, una divisione o un'intera organizzazione.}

\parola{Continuous Integration}{Pratica che si applica qualora lo sviluppo Software avvenga tramite sistema di versionamento. Consiste nell'allineamento (merge) frenquente, ovvero più volte al giorno, degli ambienti di lavoro degli sviluppatori verso l'ambiente condiviso (mainline).}

\parola{Cordova}{È un \gl{framework} per lo sviluppo di applicativi per dispositivi mobili.}

\parola{Crash}{Il termine crash nel gergo informatico indica il blocco o la terminazione improvvisa, non richiesta e inaspettata di un programma in esecuzione, oppure il blocco completo dell'intero computer.}

\parola{CSS3}{Il CSS è un linguaggio usato per definire la formattazione di documenti HTML, XHTML e XML come i siti web e relative pagine web. Le regole per comporre il CSS sono contenute in un insieme di direttive (Recommendations) emanate a partire dal 1996 dal W3C. CSS3 è lo standard più recente per il CSS.}