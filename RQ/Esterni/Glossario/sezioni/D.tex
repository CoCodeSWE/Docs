\lettera{D}

\parola{Data Access Object}{Pattern architetturale per la gestione della persistenza. Si basa sull'utilizzo di un oggetto che fornisce un'interfaccia astratta per la gestione di un database.}

\parola{DBMS}{Acronimo di Database Management System, è un sistema \gl{software} progettato per consentire la creazione, la manipolazione e l'interrogazione efficiente di un database.}

\parola{Dependency Injection}{Design pattern della Programmazione orientata agli oggetti il cui scopo è quello di semplificare lo sviluppo e migliorare la testabilità di software di grandi dimensioni.}

\parola{Deploy}{È la consegna o rilascio al cliente, con relativa installazione e messa in funzione o esercizio, di una applicazione o di un sistema software tipicamente all'interno di un sistema informatico aziendale.}

\parola{Design pattern}{In informatica, nell'ambito dell'ingegneria del software, un design pattern (traducibile in lingua italiana come schema progettuale, schema di progettazione, schema architetturale), è un concetto che può essere definito "una soluzione progettuale generale ad un problema ricorrente". Si tratta di una descrizione o modello logico da applicare per la risoluzione di un problema che può presentarsi in diverse situazioni durante le fasi di progettazione e sviluppo del software, ancor prima della definizione dell'algoritmo risolutivo della parte computazionale.}

\parola{Diagramma di Gantt}{Il diagramma di Gantt è uno strumento che permette di modellare la pianificazione dei compiti necessari alla realizzazione di un progetto. È costruito partendo da un asse orizzontale, usato per rappresentare l’arco temporale totale del progetto, e da un asse verticale, usato per rappresentare delle mansioni o attività che costituiscono il \gl{progetto}.}

\parola{Direttiva}{Nel contesto del progetto \PROGETTO{} sviluppato dal team di \GRUPPO{}, si intende un comando definito da un amministratore per l'assistente virtuale.}

\parola{DOM}{Acronimo di Document Object Model, è lo standard del W3C per la rappresentazione di documenti come modello orientato agli oggetti. Ogni documento viene trattato come un albero dove ogni nodo è un oggetto che rappresenta parte del documento.}

\parola{Double token}{È un metodo di autenticazione che si basa sull'utilizzo congiunto di due metodi di autenticazione individuale.}

\parola{DynamoDB}{Servizio di database \gl{NoSQL} offerto da \gl{Amazon Web Services}.}