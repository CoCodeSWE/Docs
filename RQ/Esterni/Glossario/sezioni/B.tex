\lettera{B}

\parola{Babel}{È uno strumento utilizzato per compilare il codice JavaScript.}

\parola{Base64}{È un sistema di numerazione posizionale che usa 64 simboli. Viene usato principalmente come codifica di dati binari.}

\parola{Bitbucket}{Servizio di hosting per progetti che usano i sistemi di controllo versione Mercurial (sin dal lancio) o \gl{Git} (dall'ottobre 2011).}

\parola{BLOB}{Acronimo per Binary Large Object, è un tipo di dato utilizzato per la memorizzazione di dati di grandi dimensioni in formato binario non direttamente interpretabili dai database.}

\parola{Bolle interattive}{Strumento utilizzato all'interno di una chat per diversi scopi come proporre sondaggi, condividere informazioni o contenuti che possono cambiare nel tempo.}

\parola{Bootstrap}{Raccolta di strumenti liberi per la creazione di siti e applicazioni per il Web.}

\parola{Bot}{È un programma che, utilizzando lo stesso tipo di canali usati da utenti umani, cerca di far credere all'utente umano di interagire con un altro utente umano. Telegram, però, si riferisce ai \gl{bot} in maniera differente: essi sono a tutti gli effetti account che interagiscono con dei messaggi preimpostati, decisi dal creatore del bot, e di ricevere risposte in merito. \gl{VotePoll} è un bot di telegram utilizzato dal gruppo \GRUPPO.}

\parola{Botkit}{È un insieme di strumenti che fornisce i blocchi fondamentali per la creazione di bot per piattaforme di messaggistica, tra le quali figura anche \gl{Slack.}}

\parola{Broadband}{Indica generalmente la trasmissione e ricezione di dati informativi, inviati e ricevuti simultaneamente in maggiore quantità, sullo stesso cavo o mezzo radio grazie all'uso di mezzi trasmissivi e tecniche di trasmissione che supportino e sfruttino un'ampiezza di banda superiore ai precedenti sistemi di telecomunicazioni detti invece a banda stretta (narrowband).}

\parola{Browser}{È un'applicazione per il recupero, la presentazione e la navigazione di risorse sul web.}