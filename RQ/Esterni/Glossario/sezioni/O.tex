\lettera{O}

\parola{OAuth2}{È l'evoluzione di OAuth, un protocollo per l'autorizzazione creato originariamente nel 2006.}

\parola{Observer}{Design Pattern che si basa su uno o più oggetti, chiamati osservatori o observer, che vengono registrati per osservare delle classi chiamate "Soggetti" e ne gestiscono gli eventi associati.} 

\parola{On demand}{È un termine generale che si riferisce a risorse fornite all'utente dopo un'esplicita richiesta.}

\parola{Open source}{Un software è open source se  gli autori (più precisamente, i detentori dei diritti) rendono pubblico il codice sorgente, favorendone il libero studio e permettendo a programmatori indipendenti di apportarvi modifiche ed estensioni. Questa possibilità è regolata tramite l'applicazione di apposite licenze d'uso. Il fenomeno ha tratto grande beneficio da Internet, perché esso permette a programmatori distanti di coordinarsi e lavorare allo stesso progetto.}