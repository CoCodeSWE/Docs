\lettera{S}

\parola{Sample}{È un campione estratto da un insieme.\\Nel contesto del \gl{progetto} \PROGETTO, si intende un campione sonoro relativo alla registrazione di un file audio.}

\parola{SDK}{Un software development kit è un insieme di strumenti per lo sviluppo e la documentazione di software.}

\parola{Serverless}{Letteralmente senza server, un'architetture serverless consente di creare ed eseguire applicazioni e servizi senza doverne gestire l'infrastruttura. Le applicazioni saranno comunque eseguite su server, ma la gestione di questi ultimi sarà a carico di terzi.\\ Nel contesto del progetto \PROGETTO, è anche il nome del framework utilizzato per lo sviluppo di lambda function e API Gateway di AWS}.

\parola{Single responsibility principle}{Nella programmazione orientata agli oggetti, il principio di singola responsabilità (single responsibility principle) afferma che ogni elemento di un programma (classe, metodo, variabile) deve avere una sola responsabilità, e che tale responsabilità debba essere interamente incapsulata dall'elemento stesso. Tutti i servizi offerti dall'elemento dovrebbero essere strettamente allineati a tale responsabilità.}

\parola{Single sign-on}{È la proprietà di un sistema di controllo d'accesso che consente ad un utente di effettuare un'unica autenticazione valida per più sistemi software o risorse informatiche alle quali è abilitato.}

\parola{Sinon}{È un framework di test indipendente utilizzato per creare test per codice JavaScript.}

\parola{Service Level Agreement}{Sono strumenti contrattuali attraverso i quali si definiscono le metriche di servizio (es. qualità di servizio) che devono essere rispettate da un fornitore di servizi (provider) nei confronti dei propri clienti/utenti.}


\parola{Sistema}{È un insieme di componenti, elementari o meno, che interagiscono tra loro.}

\parola{Slack}{È uno strumento basato su cloud per agevolare la collaborazione in un team, fornendo un canale di comunicazione per metterne in contatto i membri. È utilizzato dall'azienda \PROPONENTE{} come strumento di comunicazione interno.}

\parola{Software}{Il software in informatica è l'informazione o le informazioni utilizzate da uno o più sistemi informatici e memorizzate su uno o più supporti informatici. Tali informazioni possono essere quindi rappresentate da uno o più programmi, da uno o più dati, oppure da una combinazione delle due.}

\parola{Speaker recognition}{Consiste nel processo di validazione dell'identità che un utente dichiara, utilizzando le caratteristiche estratte dalla sua voce.}

\parola{Speech to text}{Consiste nella traduzione di un input vocale in testo.}

\parola{SQL}{È un linguaggio standardizzato per database basati sul modello relazionale.}

\parola{Stack}{È un tipo di dato astratto che viene usato in diversi contesti per riferirsi a strutture dati, le cui modalità d'accesso ai dati in essa contenuti seguono una modalità LIFO (Last In First Out), ovvero tale per cui i dati vengono estratti (letti) in ordine rigorosamente inverso rispetto a quello in cui sono stati inseriti (scritti).\\
Con \gl{stack} tecnologico si intende una cosa diversa: l'insieme delle tecnologie da utilizzare per realizzare un progetto.}

\parola{Swift}{È un linguaggio di programmazione object-oriented per sistemi macOS, iOS, watchOS, TvOS e Linux.}