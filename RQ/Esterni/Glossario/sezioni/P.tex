
\lettera{P}

\parola{Package}{È un meccanismo per organizzare classi in gruppi logici, principalmente in modo da definire namespace distinti per diversi contesti. Il package ha lo scopo di riunire classi logicamente correlate.}

\parola{Payload}{Letteralmente carico utile, si intente la parte di un flusso di dati che rappresenta il contenuto informativo.}

\parola{PDCA}{Acronimo di Plan-Do-Check-Act, detto anche ciclo di Deming o ciclo di miglioramento Continuo, è un metodo che permette di perseguire un continuo miglioramento della qualità nei processi.}

\parola{PDF}{Acronimo di Portable Document Format, è un formato sviluppato da Adobe Systems per la rappresentazione di documenti in modo indipendente dall’hardware e dal software utilizzati per la visualizzazione o generazione.}

\parola{PhantomJS}{È un browser senza interfaccia grafica utilizzato per interazione automatiche con pagine web.}

\parola{PHP}{E' un linguaggio di scripting interpretato, (acronimo di Hypertext Preprocessor) originariamente concepito per la programmazione di pagine web. Attualmente è principalmente utilizzato per sviluppare apllicazioni web lato server.}

\parola{Phython}{È un linguaggio di programmazione ad alto livello, orientato agli oggetti.}

\parola{Policy}{Con questo termine si indica un piano di azione o condotta da seguire.}

\parola{Prodotto}{Indica il risultato verificabile di un'attività, che può essere un documento o del codice sorgente.}

\parola{Progetto}{Esso consiste nell'organizzazione di azioni nel tempo per il perseguimento di uno scopo predefinito, attraverso le varie fasi di progettazione da parte di uno o più progettisti.}

\parola{Proponente}{Giuridicamente è colui che formula una proposta di contratto. Nell'ambito del progetto \PROGETTO, ci si riferisce all'azienda \PROPONENTE.}

